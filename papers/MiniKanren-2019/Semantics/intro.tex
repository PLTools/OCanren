\section{Introduction}

\textsc{miniKanren}~\cite{TRS} is a minimalistic relational DSL



The introductory book~\cite{TRS} describes the language by mean of an evolving set of examples; various flavours of the language are presented
in a number of papers~\cite{MicroKanren,CKanren,CKanren1,AlphaKanren,2016,Guided} in terms of \textsc{Scheme} implemetation. In~\cite{RelConversion}
a variant of nondeterministic operational semantics was presented, and in~\cite{DivTest} another variant of finite-set semantics was
used. None of these semantics reflect the major property of \textsc{miniKanren} search~--- \emph{interleaving}~\cite{Search}, thus
deviating from the conventional understanding of the language.

In this paper we present a formal semantics for core \textsc{miniKanren} and prove some its basic properties. First,
we define denotational semantics similar to the least Herbrand model for definite clauses~\cite{LHM}; then
we describe operational semantics with interleaving in terms of labeled transition system. Finally, we prove the soundness and
completeness of the operational semantics w.r.t the denotational one. 

The paper orginized as follows. In Section~\ref{syntax} we give the syntax of the language, describe its semantics
informally and discuss some examples. Section~\ref{denotational} contains the description of denotational semantics for
the language, and Section~\ref{operational}~--- the operational semantics.


