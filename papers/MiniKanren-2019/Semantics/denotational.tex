\begin{figure}[t]
\[
\begin{array}{rcll}
  x\,[t/x] &=& t &\\
  y\,[t/x] &=& y & y\ne x\\
  C_i^{k_i}\,(t_1,\dots,t_{k_i})\,[t/x]&=&C_i^{k_i}\,(t_1\,[t/x],\dots,t_{k_i}\,[t/x])&\\[2mm]
  (t_1 \equiv t_2)\,[t/x]&=&t_1\,[t/x] \equiv t_2\,[t/x]&\\
  (g_1 \wedge g_2)\,[t/x]&=&g_1\,[t/x] \wedge g_2\,[t/x]&\\
  (g_1 \vee g_2)\,[t/x]&=&g_1\,[t/x] \vee g_2\,[t/x]&\\
  (\mbox{\lstinline|fresh|}\;x\,.\,g)\,[t/x]&=&\mbox{\lstinline|fresh|}\;x\,.\,g&\\
  (\mbox{\lstinline|fresh|}\;y\,.\,g)\,[t/x]&=&\mbox{\lstinline|fresh|}\;y\,.\,(g\,[t/x])&y\ne x\\
  (R_i^{k_i}\,(t_1,\dots,t_{k_i})\,[t/x]&=&R_i^{k_i}\,(t_1\,[t/x],\dots,t_{k_i}\,[t/x])&
\end{array}
\]
  \caption{Substitutions for terms and goals}
  \label{substitution}
\end{figure}

\section{Denotational Semantics}
\label{denotational}

To motivate further development, we first consider the following example. Let us have the following goal:

\begin{lstlisting}
   x === Cons (y, z)
\end{lstlisting}

There are three free variables, and solving the goal delivers us the following single answer:

\begin{lstlisting}
   $\alpha\mapsto\;$ Cons ($\beta$, $\gamma$)
\end{lstlisting}

where semantic variables $\alpha$, $\beta$ and $\gamma$ correspond to the syntactic ones ``\lstinline|x|'', ``\lstinline|y|'', ``\lstinline|z|''. The
goal does not put any constraints on ``\lstinline|y|'' and ``\lstinline|z|'', so there are no bindings for ``$\beta$'' and ``$\gamma$'' in the answer.
This answer can be seen as the following ternary relation over the set of all ground terms:

\[
\{(\mbox{\lstinline|Cons ($\beta$, $\,\gamma$)|}, \beta, \gamma) \mid \beta\in\mathcal{D},\,\gamma\in\mathcal{D}\}\subset\mathcal{D}^3
\]

The order of ``dimensions'' is important, since each dimension corresponds to a certain free variable. Our main idea is to represent this relation as a set of functions 

\[
\mathfrak{f}:\mathcal{A}\to\mathcal{D}
\]

from semantic variables to ground terms. We call these functions \emph{representing functions}. Thus, we may reformulate the same relation as

\[
\{(\mathfrak{f}\,(\alpha),\mathfrak{f}\,(\beta),\mathfrak{f}\,(\gamma))\mid\mathfrak{f}\in\sembr{\mbox{\lstinline|$\alpha$ === Cons ($\beta$, $\,\gamma$)|}}\}
\]

where we use conventional semantic brackets ``$\sembr{\bullet}$'' to denote the semantics. Now we implement this idea.

First, for a representing function

\[
\mathfrak{f} : \mathcal{A}\to\mathcal{D}
\]

we introduce its homomorphic extension 

\[
  \overline{\mathfrak{f}}:\mathcal{T_A}\to\mathcal{D}
\]

which maps terms to terms:

\[
\begin{array}{rcl}

  \overline{\mathfrak f}\,(\alpha) & = & \mathfrak f\,(\alpha)\\
  \overline{\mathfrak f}\,(C_i^{k_i}\,(t_1,\dots.t_{k_i})) & = & C_i^{k_i}\,(\overline{\mathfrak f}\,(t_1),\dots \overline{\mathfrak f}\,(t_{k_i}))
\end{array}
\]

Let us have two terms $t_1, t_2\in\mathcal{T_A}$. If there is a unifier for $t_1$ and $t_2$ then, clearly, there is a substitution $\theta$ which
turns both $t_1$ and $t_2$ into the same \emph{ground} term (we do not require $\theta$ to be the most general). Thus, $\theta$ maps
(some) ground variables into ground terms, and its application to $t_{1(2)}$ is exactly $\overline{\theta}(t_{1(2)})$. This reasoning can be
performed in the opposite direction: a unification $t_1\equiv t_2$ defines the set of all representing functions $\mathfrak{f}$ for which
$\overline{\mathfrak{f}}(t_1)=\overline{\mathfrak{f}}(t_2)$. 

Then, the semantic function for goals is parameterized over environments which prescribe semantic functions to relational symbols:

\[
  \Gamma : \mathcal{R} \to (\mathcal{T_A}^*\to 2^{\mathcal{A}\to\mathcal{D}})
\]

An environment associates with relational symbol a function, which takes a string of terms (the arguments of the relation) and returns a set of
representing functions. The signature for semantic brackets for goals is as follows:

\[
\sembr{\bullet}_{\Gamma} : \mathcal{G}\to 2^{\mathcal{A}\to\mathcal{D}}
\]

It maps a goal into the set of representing functions w.r.t. an environment $\Gamma$.

We formulate the following important \emph{coverage condition} for the semantics of a goal $g$:

\[
\forall\alpha\not\in FV\,(g)\; :\; \forall d \in \mathcal{D}:\; \forall\mathfrak{f} \in \sembr{g}:\; \exists \mathfrak{f'} \in \sembr{g} :\; \mathfrak{f'}\,(\alpha)\; = d \wedge \forall \beta \neq \alpha:\; \mathfrak{f'}\,(\beta)\; = \mathfrak{f}\,(\beta)\; 
\]

Informally, it means that the values of all representing functions on each non-free variable of $g$ covers $\mathcal{D}$. In other words, representing
functions for a goal $g$ restrict only the values of free variables of $g$. This condition guarantees that our semantics is complete in the sense that
it does not introduce artificial restrictions for the relation it defines.

We remind conventional notions of pointwise modification of a function

\[
f\,[x\gets v]\,(z)=\left\{
\begin{array}{rcl}
  f\,(z) &,& z \ne x \\
  v      &,& z = x
\end{array}
\right.
\]

and substitution of a free variable with a term in terms and goals (see Figure~\ref{substitution}).

For a representing function $\mathfrak{f}:\mathcal{A}\to\mathcal{D}$ and a semantic variable $\alpha$ we define
the following \emph{generalization} operation:

\[
\mathfrak{f}\uparrow\alpha = \{ \mathfrak{f}\,[\alpha\gets d] \mid d\in\mathcal D\}
\]

Informally, this operation generalizes a representing function into a set of representing functions in such a way that the
values of these functions for a given variable cover the whole $\mathcal{D}$. We extend the generalization operation for sets of
representing functions $\mathfrak{F}\subseteq\mathcal{A}\to\mathcal{D}$:

\[
  \mathfrak{F}\uparrow\alpha = \bigcup_{\mathfrak{f}\in\mathfrak{F}}(\mathfrak{f}\uparrow\alpha)
\]

Now we are ready to specify the sematics for goals (see Figure~\ref{denotational_semantics_of_goals}). We've already given the motivation for
the semantics of unification: the condition $\overline{\mathfrak{f}}(t_1)=\overline{\mathfrak{f}}(t_2)$ gives us the set of all (otherwise
unrestricted) representing functions which ``equate'' terms $t_1$ and $t_2$. Set union and intersection provide a conventional interpretation
for disjunction and conjunction of goals, and the semantics of relational invocation reduces to the application of corresponding
function from the environment. The only interesting case is ``\lstinline|fresh $x$ . $g$|''. First, we take an arbitrary semantic variable $\alpha$,
not free in $g$, and substitute $x$ with $\alpha$. Then we calculate the semantics of $g\,[\alpha/x]$. The interesting part is the next step:
as $x$ can not be free in ``\lstinline|fresh $x$ . $g$|'', we need to generalize the result over $\alpha$ since in our model the semantics of a
goal specifies a relation over its free variables. Consider the following example:

\begin{lstlisting}
   fresh y . ($\alpha$ === y) /\ (y === Zero)
\end{lstlisting}

As there is no invocations involved, we can safely omit the environment. Then:

\[
\begin{array}{lcr}
  \sembr{\mbox{\lstinline|fresh y . ($\alpha$ === y) $\,\wedge\,$ (y === Zero)|}}&=&\mbox{(by \textsc{Fresh$_D$})}\\[1mm]
  (\sembr{\mbox{\lstinline|($\alpha$ === $\beta$) $\,\wedge\,$ ($\beta$ === Zero)|}})\uparrow\beta&=&\mbox{(by \textsc{Conj$_D$})}\\[1mm]
  (\sembr{\mbox{\lstinline|$\alpha$ === $\beta$|}} \,\cap\, \sembr{\mbox{\lstinline|$\beta$ === Zero)|}})\uparrow\beta&=&\mbox{(by \textsc{Unify$_D$})}\\[1mm]
  (\{\mathfrak{f}\mid \overline{\mathfrak{f}}\,(\alpha)=\overline{\mathfrak{f}}\,(\beta)\} \,\cap\, \{\mathfrak{f}\mid \overline{\mathfrak{f}}\,(\beta)=\overline{\mathfrak{f}}\,(\mbox{\lstinline|Zero|})\})\uparrow\beta&=&\mbox{(by the definition of ``$\overline{\mathfrak{f}}$'')}\\[1mm]
  (\{\mathfrak{f}\mid \mathfrak{f}\,(\alpha)=\mathfrak{f}\,(\beta)\} \,\cap\, \{\mathfrak{f}\mid \mathfrak{f}\,(\beta)=\mbox{\lstinline|Zero|}\})\uparrow\beta&=&\mbox{(by the definition of ``$\cap$'')}\\[1mm]
  (\{\mathfrak{f}\mid \mathfrak{f}\,(\alpha)=\mathfrak{f}\,(\beta)=\mbox{\lstinline|Zero|}\})\uparrow\beta&=&\mbox{(by the definition of ``$\uparrow$'')}\\[1mm]
  \{\mathfrak{f}\mid \mathfrak{f}\,(\alpha)=\mbox{\lstinline|Zero|}, \mathfrak{f}\,(\beta)=d, d\in\mathcal{D}\}&&
\end{array}
\]

In the end we've got the set of representing functions, each of which restricts only the value of free variable $\alpha$. 

\begin{figure}[t]
  \[
  \begin{array}{cclr}
    \sembr{t_1\equiv t_2}_\Gamma&=&\{\mathfrak f : \mathcal{A}\to\mathcal{D}\mid \overline{\mathfrak{f}}\,(t_1)=\overline{\mathfrak{f}}\,(t_2)\}& \mbox{[\textsc{Unify$_D$}]}\\
    \sembr{g_1\wedge g_2}_\Gamma&=&\sembr{g_1}_\Gamma\cap\sembr{g_1}_\Gamma&\mbox{[\textsc{Conj$_D$}]}\\
    \sembr{g_1\vee g_2}_\Gamma&=&\sembr{g_1}_\Gamma\cup\sembr{g_1}_\Gamma&\mbox{[\textsc{Disj$_D$}]}\\
    \sembr{\mbox{\lstinline|fresh|}\,x\,.\,g}_\Gamma&=&(\sembr{g\,[\alpha/x]}_\Gamma)\uparrow\alpha,\;\alpha\not\in FV(g)& \mbox{[\textsc{Fresh$_D$}]}\\
    \sembr{R\,(t_1,\dots,t_k)}_\Gamma&=&(\Gamma\,R)\,t_1\dots t_k & \mbox{[\textsc{Invoke$_D$}]}
  \end{array}
  \]
  \caption{Denotational semantics of goals}
  \label{denotational_semantics_of_goals}
\end{figure}

The final component is the semantics of specifications. Given a specification

\[
\{R_i=\lambda\,x_1^i\dots x_{k_i}^i\,.\,g_i;\}_{i=1}^n\;g
\]

we have to construct a correct environment $\Gamma_O$ and then take the semantics of the top-level goal:

\[
\sembr{\{R_i=\lambda\,x_1^i\dots x_{k_i}^i\,.\,g_i;\}_{i=1}^n\;g}=\sembr{g}_{\Gamma_0}
\]

As the set of definitions can be mutually recursive, we apply the fixed point approach. We consider the following
function

\[
\mathcal{F} : (\mathcal{T_A}^*\to 2^{\mathcal{A}\to\mathcal{D}})^n\to (\mathcal{T_A}^*\to 2^{\mathcal{A}\to\mathcal{D}})^n
\]

which represents a semantic for the set of definitions abstracted over themselves. The definition of this function is
rather standard:

\begin{gather*}
    \begin{array}{rcl}
      \mathcal{F}\,(p_1,\dots,p_n)& = &(t^1_1\dots t^1_{k_1}\mapsto\sembr{g^1\,[t^1_1/x^1_1,\dots,t^1_{k_1}/x^1_{k_1}]}_\Gamma,\\
                                  &  &\phantom{(}\dots\\
                                  &  &\phantom{(}t^n_1\dots t^n_{k_n}\mapsto\sembr{g^n\,[t^n_1/x^n_1,\dots,t^n_{k_n}/x^n_{k_n}]}_\Gamma)
    \end{array}\\
    \mbox{where}\;\Gamma\, R_i=p_i
\end{gather*}

Here $p_i$ is a semantic function for $i$-th definition; we build an environment $\Gamma$, which associates each relational symbol
$R_i$ with $p_i$ and construct a $n$-dimensional vector-function, where $i$-th component corresponds to a function which
calculates the semantics of $i$-th relational definition application to terms w.r.t. the environment $\Gamma$. Finally,
we take the least fixed point of $\mathcal{F}$ and define the top-level environment as follows:

\[
\Gamma_0\,R_i=(fix\;\mathcal{F})\,[i]
\]

where ``$[i]$'' denotes the $i$-th component of a vector-function.

The least fixed point exists by Tarski-Knaster theorem:

\begin{itemize}
\item Functions $\mathfrak{f}:\mathcal{A}\to\mathcal{D}$ form a complete lattice.
\item Semantic function is monothonous, since fresh, conjunction and disjunction are monothonous.
\end{itemize}