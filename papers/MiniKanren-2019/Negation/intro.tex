\section{Introduction}

\label{sec:intro}

\textsc{MiniKanren}~\cite{TheReasonedSchemer, byrd2010relational} is a minimalistic domain-specific language which brings 
features of logic programming into a host language.
Typical \textsc{MiniKanren} implementation introduces only a few operators:
conjunction, disjunction, unification (which can be seen as equality constraint) 
and fresh variable introduction (existential quantification).
Although this basis constitutes a Turing-complete language,
in practice there are some cases when 
the availability of negative reasoning is desirable.

For example, consider the following problem.
Suppose we want to write a program 
which removes the first occurrence of a given element from a list.
In order to do that in \textsc{MiniKanren} 
we have to first define a ternary relation \code{remove} 
which binds the desired element, original list, 
and the same list after the deletion. 
Substituting the first argument of the relation with some element \code{e}, 
the second argument with some list \code{xs} and 
the third argument with a free variable \code{q}
gives us a \emph{goal} which is a proof search procedure.
When passed to the \lstinline{run} function, 
the goal will produce a lazy stream of \emph{answers}.
Each answer is represented by substitution 
which binds free variables to some terms.
In the case of \lstinline{remove}, 
we would expect a single answer,
which binds \lstinline{q} to the list, 
equal to \lstinline{xs}, 
except that the first occurrence of \lstinline{e} 
is removed.

The code on Listing~\ref{lst:remove-buggy} demonstrates
a possible implementation of \code{remove}.
It consists of three disjuncts, which represent three different cases.
First, if the original list is empty, 
then the resulting list should also be empty.
If the original list is not empty and its head 
is equal to the given element, 
then the resulting list should be equal to the tail.
Finally, in the case when the head is not equal to the given element,
the resulting list should be equal to the original, 
from the tail of which the occurrence of the element \code{e} is deleted.

Given the definition of \code{remove} from Listing~\ref{lst:remove-buggy},
the following query \lstinline{run (remove 2 [1;2;3] q)} will
wrongly return two answers: \lstinline{q=[1;3]} and \lstinline{q=[1;2;3]}.
The redundant (and, indeed, incorrect) answer \lstinline{q=[1;2;3]}
arose because of the third disjunct from \lstinline{remove} definition,
which always succeeds and thus generates a copy of the original list.
We can prevent this behavior using \emph{disequality constraint},
yet another primitive, which some \textsc{MiniKanren} implementations provide.
Adding constraint \lstinline{x =/= e} to the third case 
makes all disjuncts mutually disjoint. 
With the fixed definition of \lstinline{remove} (Listing~\ref{lst:remove-correct}), 
the query given above returns the single answer \lstinline{q=[1;3]} as expected. 

\begin{minipage}[t]{.3\textwidth}
\begin{lstlisting}[
  caption={A flawed definition of \code{remove} relation},
  label={lst:remove-buggy}
]
let remove e xs ys =
  (xs === [] /\ ys === [])
  \/
  fresh (x xs') (
    x === e /\
    xs === x::xs' /\ 
    ys === xs'
  ) \/
  fresh (x xs' ys') (
    xs === x::xs' /\
    ys === x::ys' /\
    remove e xs' ys'
  )
\end{lstlisting}
\end{minipage}\hfill
\begin{minipage}[t]{.3\textwidth}
\begin{lstlisting}[
  caption={The correct definition of the \code{remove} relation},
  label={lst:remove-correct}
]
let remove e xs ys =
  (xs === [] /\ ys === [])
  \/
  fresh (x xs') (
    x === e /\
    xs === x::xs' /\ 
    ys === xs'
  ) \/
  fresh (x xs' ys') (
    x =/= e /\
    xs === x::xs' /\
    ys === x::ys' /\
    remove e xs' ys'
  )
\end{lstlisting}
\end{minipage}\hfill
\begin{minipage}[t]{.3\textwidth}
\begin{lstlisting}[
  caption={A generalized version of the \code{remove} relation},
  label={lst:remove-gen}
]
let remove p xs ys =
  (xs === [] /\ ys === [])
  \/
  fresh (x xs') (
    p x /\
    xs === x::xs' /\ 
    ys === xs'
  ) \/
  fresh (x xs' ys') (
    ~(p x) /\
    xs === x::xs' /\
    ys === x::ys' /\
    remove p xs' ys'
  )
\end{lstlisting}
\end{minipage}

A disequality constraint represents a very limited form of negation,
which is often not sufficient.
Imagine that we want to generalize \lstinline{remove}, 
so that instead of taking an element, 
it takes a predicate \lstinline{p} and removes
the first element of the list which satisfies the predicate (Listing~\ref{lst:remove-gen}).
In order to do that and avoid similar pitfalls as 
in our first attempt to define \lstinline{remove},
we need to ensure that the third disjunct succeeds 
only when the predicate \lstinline{p} fails on the head element of \lstinline{xs}.
Thus we need a general \emph{negation operator}. 
Unfortunately, none of the existing \textsc{MiniKanren} implementations
provide this feature.

In the world of Prolog, negative reasoning is usually implemented 
using so-called \emph{negation as failure} approach.
Under this rule, a goal \lstinline{~p} succeeds whenever \lstinline{p} fails, 
and it fails whenever \lstinline{p} succeeds.
Unfortunately, negation as failure is unsound 
if the negated goal \lstinline{p} contains free variables.
For example, the query \lstinline{run (q === 0) /\ ~(q === 1)} succeeds, 
although \lstinline{run ~(q === 1) /\ (q === 0)} fails.
The failure in the last case is due to the fact that 
at the time of negation execution the variable \lstinline{q} is free, 
thus \lstinline{q === 1} succeeds, and conversely \lstinline{~(q === 1)} fails.
This example demonstrates that
negation as failure goes against the philosophy of \textsc{MiniKanren}, 
which positions itself as a pure relational language
without non-logical features.

In this work we present an implementation of the negation operator
based on the method of \emph{constructive negation}.
It overcomes several shortcomings of negation as failure 
and provides a more expressive form of 
negative reasoning than disequality constraints. 
The idea of this method is to constructively 
build a stream of answers for the negated goal.
It is achieved by first collecting all answers to 
the positive version of the goal and then negating them. 
The constructive negation approach 
while being an improvement over negation as failure 
still has its own drawbacks. 
For example, applying the negation operator 
to the goal that has infinitely many answers 
results in a non-terminating computation. 

Although constructive negation was first 
proposed as an extension of Prolog, 
to the best of our knowledge 
our work is the first attempt to adapt it for \textsc{MiniKanren}. 
We have developed a prototype implementation for \textsc{OCanren}~\cite{kosarev2018typed}, 
a \textsc{MiniKanren} dialect embedded in \textsc{OCaml}. 
Like the rest of \textsc{MiniKanren}, 
our version of constructive negation is developed 
in a pure functional style using
persistent data structures.
This makes it different from the earlier implementations for Prolog.

Our paper is structured as follows.
In Section~\ref{sec:motivation} we give 
more examples of constructive negation usage. 
Section~\ref{sec:implementation} describes our implementation.
Next, Section~\ref{sec:evaluation} presents the evaluation
of the negation on a series of examples.
In Section~\ref{sec:limitations} 
we discuss the limitations of our approach and 
directions for future work. 
Section~\ref{sec:related-works} reviews related works. 
The final Section~\ref{sec:conclusion} concludes.

