\section{Related Works}

\label{sec:related-works}

There are two directions of work in the process of 
incorporating negative reasoning in the logic programming:
the first considers the semantics of negation,
and the second is focused mainly on implementation aspects.

The first attempt to give a semantics 
for negation in logic programming 
was done by Clark~\cite{clark1978negation, chan1988constructive}
with his completion semantics.
It was then realized, that 
Clark's semantics has various drawbacks~\cite{van1991well}.

Przymusinski~\cite{przymusinski1989constructive} 
has studied the semantics of stratified logic programs.
He introduced the notion of \emph{perfect model semantics} for such programs.
Stratified logic programs have a variety of good properties,
including the property that each stratified program 
has a unique minimal model.

In an attempt to extend the semantics of negation
to non-stratified programs the 
\emph{well-founded semantics} was proposed~\cite{van1991well}.
However, this semantics is three-valued,
meaning that for some queries it 
can return answer \lstinline{unknown}.
For example, given the relation \lstinline{winning}
(section~\ref{sec:strat}, listing~\ref{lst:game}),
queries \lstinline{winning 'a'} and \lstinline{winning 'b'}
would return \lstinline{unknown}.

An alternative approach is 
\emph{stable model semantics}~\cite{gelfond1988stable}.
Under this semantics, non-stratified logic program
can have several stable models.
Program, that defines \lstinline{winning}, 
has two stable models, 
in one of these models goal 
\lstinline{winning 'a'} succeeds and 
\lstinline{winning 'b'} fails,
in the other \lstinline{winning 'a'} fails
and \lstinline{winning 'b'} succeeds.
Logic programming under stable model semantics
is also known under the name 
answer set programming (ASP).

The works~\cite{stuckey1991constructive, dovier2000necessary}
are theoretical studies of constructive negation
in the context of constraint logic programming.
They give a necessary and sufficient condition for the
constraint structures that are compatible with constructive negation.
Namely, the constraint structure should be \emph{admissible closed}.

From an implementation side,
Chen et al.~\cite{chen1995efficient, chen1996tabled}
developed a \textsc{Prolog} system
based on SLG resolution,
which is sound with respect to well-founded semantics.
However, they used negation as failure
with delaying of non-ground negative subgoals.
\cite{liu1999constructive} is an extension of this system
with the support of the constructive negation.
Works~\cite{bartak1998constructive, alvez2004constructive} 
implement a constraint logic programming systems
with the support of constructive negation.
Yet, as with our implementation, the
constructive negation in these systems
supports only equality and disequality
constraints over first-order terms.
We are not aware of any practical implementation 
that is parametric over arbitrary admissible closed constraint structures.

Many tools were developed to 
compute stable models of logic programs,
among them are~\cite{gebser2007clasp, giunchiglia2006answer}.
These systems usually require
to perform grounding of logic program.
The problem of finding stable models of 
ground logic program then is encoded 
as propositional formula and solved 
by some SAT solver.
Unfortunately, some logic programs
do not have finite grounding, 
but even if a program has it,
grounding may cause an exponential blow-up.
Recently, a goal-directed system
for computing stable models was developed~ 
\cite{marple2012goal, marple2017s, arias2018constraint}.
To the best of our knowledge, it is the only ASP system,
that does not require grounding.
The key components of this system are the usage of tabling, 
constructive negation, coinductive logic programming, 
and non-monotonic reasoning check.
It is an interesting and challenging task
to extend \textsc{MiniKanren} with the support of 
stable model semantics in the spirit of this line of work.
