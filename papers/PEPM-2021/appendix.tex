\appendix

\section{The Proof of Theorem~\ref{thm:main}}
\label{sec:app}

First, we make a number of observations which will simplify following reasoning.

We need to relate two semantics~--- the angelic one presented in Section~\ref{sec:angelic-semantics} and
the fair semantics by well quasi-ordering presented in Section~\ref{sec:fair-semantics}~--- and prove that
arbitrary goal simultaneously converges or diverges in them. Both semantics are represented
as labeled transition systems with the same alphabet of labels and tree-shaped states in which interior
nodes correspond to disjunctions and leaves contain ordered conjunctions of relation applications (plus a
substitution, a number of first fresh semantic variable, etc.) Moreover, both semantics share the same
set of rules dealing with disjunction; both semantics are sound and complete w.r.t. the denotational
semantics of \mk. The completeness means that in a state no leaf can be excluded from consideration~---
each leaf of a state sooner of later (meaning: in a finite number of steps) will be considered and some
its relation application will be unfolded. This allows us to abstract from concrete shapes of states and
instead consider them as sets of leaves. The following lemma justifies this transition.

\begin{lemma}[Convergence of leaves]
  \label{lem:converge_of_leaves}
  Let $s$ be a state with leaves $\{\omega_i\}$. Then $s$ converges (in angelic or fair semantics)
  iff each $\omega_i$ converges.
\end{lemma}
\begin{proof}
  The lemma follows from a more general fact: let

  \[
  s\to s_1\to s_2 \to \dots
  \]

  be a (finite or infinite) derivation sequence in either semantics of interest (we may consider
  only angelic semantics since the semantics by well quasi-ordering constitutes its special case).
  Then:

  \begin{itemize}
    \item each state $s_i$ can be represented as
      
      \[
      s_i\,[w_1,\,\dots,\,w_k]
      \]
      
      where $\{w_j\}$ is a collection of disjoint subtrees of $s_i$, containing all its leaves;
      
    \item each step can be represented either as
      
      \[
      s_i\,[\dots w_j\dots]\to s_i\,[\dots w^\prime_j\dots]
      \]

      where there is a derivation step

      \[
      w_j\to w^\prime_j
      \]

      or as

      \[
      s_i\,[\dots w_j\dots]\to s^\prime_i\,[\dots w_{j-1},\,w_{j+1},\,\dots]
      \]

      where $s^\prime_i$ is a reordering of all $\{w_l\mid l\ne j\}$;
     
    \item for each $w_j$ there is a derivation sequence

      \[
      \omega_k\to\dots\to w_j
      \]

      for some $k$.
  \end{itemize}

  Indeed, in the base case all leaves of $s$ are precisly $\{\omega_i\}$, which satisfy all requirements. Inductively, let
  us aleady have an intermediate state

  \[
  s_i\,[w_1,\,\dots,\,w_k]
  \]

  with all $\{w_j\}$ satisfying the claim. The application of rules for disjunction will traverse the state from the root to
  a certain leaf (if the state does not contain disjunctions then the whole case degenerates to $s_i=w_1$ and the claim
  holds trivially) and, hence, will arrive at a certain $w_j$. Then the next state can be represented as

  \[
  s^\prime_i\,[\dots w^\prime_j\dots]
  \]
  

  if $w_j\to w^\prime_j$ and $w^\prime\ne\times$ or as
  
  \[
  s_i\,[\dots w_{j-1},\,w_{j+1},\,\dots]
  \]

  if $w_j\to\times$ and $s^\prime_i[\dots,\,w_{j-1},\,w_{j+1},\,\dots]$ is a reordering of remaining $\{w_l\mid l\ne j\}$.
  
  In other words, any derivation for a state can be decomposed into the derivations for its leaves and vice-versa.  
\end{proof}

The next observation concerns the ``commutativity'' of sequential unifications. Let $p_{1,2},\,q_{1,2}$ be some terms. Then

\[
\begin{array}{c}
  mgu\,(p_1,\,p_2)\cdot mgu\,(q_1,\,q_2)=\\
  \phantom{XXXXXXX}mgu\,(q_1,\,q_2)\cdot mgu\,(p_1,\,p_2)
\end{array}
\]

Indeed, both

\[
mgu\,(p_1,\,p_2)\cdot mgu\,(q_1,\,q_2)
\]

and

\[
mgu\,(q_1,\,q_2)\cdot mgu\,(p_1,\,p_2)
\]

are most general unifiers for pairs $(p_1,\,p_2)$ and $(q_1,\,q_2)$ simultaneously. If there were different, then terms $C\,(p_1,\,q_1)$ and
$C\,(p_2,\,q_2)$, where $C$ is some binary constructor, would have \emph{different} most general unifiers, which is impossible\footnote{Strictly
  speaking, the validity of these claims depends on the representation of substitution; for example, in concrete representation
  the substitutions $[x\mapsto y]$ and $[y\mapsto x]$ may not be equal, but equivalent.}. Taking this into account we arrive at the
following property. Let

\[
\theta\xrightarrow{u_1;\,u_2;\dots u_k}{\theta^\prime}
\]

be a sequence of unifications which transforms a substitution $\theta$ into a substitution $\theta^\prime$. Then

\[
\theta\xrightarrow{u_{\pi(1)};\,u_{\pi(2)};\dots u_{\pi(k)}}{\theta^\prime}
\]

for any permutation $\pi$.


Another subtlety concerns the order of fresh semantic variables allocation. Let us have a leaf state $\inbr{\theta,\,i,\,ab}$, where
$\theta$ is a substitution, $a$ and $b$ are relation invocations and $i$ is the next free semantic variable. In angelic semantics we
may perform the unfoldings of $a$ and $b$ in different orders. Since the bodies for both $a$ and $b$ may contain \lstinline|fresh| constructs,
these constructs will be evaluated in different orders delivering different concrete semantic variables and different substitutions. However
these substitutions will be $\alpha$-equivalent; this observation reflects the intuitively trivial fact that the order of semantic variable
allocation is not essential as long as no variable is allocated more than once in the evalutaion within the same branch. Thus, we discard
the \lstinline|fresh| construct from considerations and will treat all substitutions up to $\alpha$-equivalence. This, in particular,
will allow us to omit the next fresh variable component from states.

\begin{lemma}[Effect of One-step Unfolding]
\label{lem:one-step-unfolding}

Let $r$ be an application and $\theta$~--- a substitution. One-step unfolding transforms a state $\inbr{\theta,\,r}$ into
a set of states $\{\inbr{\theta_i,\,\rho_i}\}$, where $\rho_i$ is a conjunction of applications, $\theta_i$~--- some substitution. Then
for each $i$ there is a sequence of unifications $u^i_1\dots u^i_{k_i}$ such that

\[
\theta\xrightarrow{u^i_1\dots u^i_{k_i}}{\theta_i}
\]
\end{lemma}
\begin{proof}
By induction on the derivation for one-step unfolding.
\end{proof}

Thus, we may introduce a denotation $[r\to\rho_i]$ for the result of the sequence of unifications performed during a one-step unfolding of $r$ which
delivers a leaf state $\rho_i$. 

\begin{lemma}[Weakening in Angelic Semantics]
  \label{lem:weakening}

  Let $\theta$ be a substitution, $\omega$ be a list of applications, and let $\inbr{\theta,\,\omega}$ converge to $\{\}$ (an empty state) in angelic semantics.
  Then for arbitrary substitution $\sigma$ and arbitrary list of applications $\psi$

  \[
  \inbr{\theta\cdot\sigma,\,\omega\psi}
  \]

  converges to $\{\}$.
\end{lemma}
\begin{proof}
  By induction on the derivation; this lemma shows rather a trivial fact that putting additional constraints (by considering a more specific substitution and adding
  more conjuncts) can not make a converging program diverge.
\end{proof}

\begin{lemma}[Convergence Preservation]
  \label{lem:convergence_preservation}

  Let $\phi a \phi^\prime$ be a list of applications with one distinguished element ``$a$'', $\theta$ be a substitution. Let

  \[
  \{\inbr{\theta\cdot[a\to\alpha_i],\,\alpha_i}\}
  \]
  
  be the result of one-step unfolding for $\inbr{\theta,\,a}$. Then

  \[
  \inbr{\theta,\,\phi a \phi^\prime}
  \]

  converges iff all

  \[
  \{\inbr{\theta\cdot[a\to\alpha_i],\,\phi\alpha_i\phi^\prime}\}
  \]

  converge.
\end{lemma}
\begin{proof}

  \mbox{}
  
  \begin{itemize}
  \item[$\Leftarrow$] Let all $\{\inbr{\theta\cdot[a\to\alpha_i],\,\phi\alpha_i\phi^\prime}\}$ converge. Then unfolding of $a$ delivers a converging derivation for $\inbr{\theta,\,\phi a \phi^\prime}$.
  \item[$\Rightarrow$] Let $\inbr{\theta,\,\phi a \phi^\prime}$ converges. Two cases are possible.
    \begin{enumerate}
    \item There is no unfolding of $a$ in the converging derivation. Then
      
      \[
      \inbr{\theta,\,\phi a \phi^\prime}
      \]
      
      converges to $\{\}$ (each state contains $a$). Moreover
      
      \[
      \inbr{\theta,\,\phi \phi^\prime}\eqno{(*)}
      \]
      
      converges to $\{\}$ (we can repeat the converging derivation of $\inbr{\theta,\,\phi a \phi^\prime}$).      
      By weakening all
      
      \[
      \inbr{\theta\cdot[a\to\alpha_i],\,\phi\alpha_i\phi^\prime}
      \]
      
      converge to $\{\}$ taking into account $(*)$.
      
    \item There is a sequence of derivation
      
      \[
      \inbr{\theta,\,\phi a \phi^\prime}\to \inbr{\theta\cdot\sigma,\,\pi a \rho}\eqno{(**)}
      \]
      
      where $\sigma$ is some sequence of unifications and the next step is unfolding of $a$. Then its result is
      
      \[
      \{\inbr{\theta\cdot\sigma\cdot[a\to\alpha_i],\,\pi \alpha_i \rho}\}
      \]
      
      Taking the state
      
      \[
      \inbr{\theta\cdot[a\to\alpha_i],\,\phi\alpha_i \phi^\prime}
      \]
      
      and repeating the unfoldings of $(**)$ we can (at most) arrive at the state
      
      \[
      \inbr{\theta\cdot[a\to\alpha_i]\cdot\sigma,\,\pi \alpha_i \rho}
      \]
      
      which converges since
      
      \[
      \theta\cdot[a\to\alpha_i]\cdot\sigma=\theta\cdot\sigma\cdot[a\to\alpha_i]
      \]
    \end{enumerate}
  \end{itemize}
\end{proof}

\begin{corollary}
  \label{corollary:corollary}
  Let $\inbr{\theta,\,\omega}$~--- some state. Let $\{\inbr{\theta^\prime,\,\omega^\prime}\}$ be a set of states, obtained from $\inbr{\theta,\,\omega}$
  in a finite number of unfoldings. Then $\inbr{\theta,\,\omega}$ converges iff all $\{\inbr{\theta^\prime,\,\omega^\prime}\}$~--- converge.
\end{corollary}

\begin{proof}[The Proof of Theorem~\ref{thm:main}]
  Let $g$ converges in angelic semantics. Take the first unfolded application, say, $r$, in the converging derivation. Two cases are possible:

  \begin{enumerate}
    \item The fair semantics performs exactly the same unfolding; nothing more to consider.
    \item The fair semantics chooses some other application to unfold. Due to the property of well quasi-ordering there will be only a finite
      number of unfoldings until $r$ will be unfolded as well. Repeating these finite number unfoldings in angelic semantics before unfolding $r$
      by Corollary~\ref{corollary:corollary} results in a converging set of pairs.
  \end{enumerate}
  
\end{proof}
