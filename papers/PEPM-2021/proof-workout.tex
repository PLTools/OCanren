%%
%% This is file `sample-sigplan.tex',
%% generated with the docstrip utility.
%%
%% The original source files were:
%%
%% samples.dtx  (with options: `sigplan')
%% 
%% IMPORTANT NOTICE:
%% 
%% For the copyright see the source file.
%% 
%% Any modified versions of this file must be renamed
%% with new filenames distinct from sample-sigplan.tex.
%% 
%% For distribution of the original source see the terms
%% for copying and modification in the file samples.dtx.
%% 
%% This generated file may be distributed as long as the
%% original source files, as listed above, are part of the
%% same distribution. (The sources need not necessarily be
%% in the same archive or directory.)
%%
%% The first command in your LaTeX source must be the \documentclass command.
\documentclass{article}[12pt]

\usepackage[T2A]{fontenc}
\usepackage[utf8]{inputenc}

\usepackage[english]{babel}
\usepackage{listings}
\usepackage[section]{placeins}
\usepackage{multirow}
\usepackage{pgfplots}
\usepackage{yfonts}
\usepackage{subcaption}
\usepackage{xspace}
\usepackage{amsmath}
\usepackage{mathtools}
\usepackage{comment}
\usepackage{url}
%\usepackage{tikz}
%\usetikzlibrary{trees}
\usepackage{amsthm}

\sloppy

\newcommand{\ruleno}[1]{\mbox{[\textsc{#1}]}}
\newcommand{\rulen}[1]{[\textsc{#1}]}
\newcommand{\mk}{\textsc{miniKanren}\xspace}
\newcommand{\inbr}[1]{\langle #1 \rangle}
\renewcommand{\emptyset}{\varnothing}
\newcommand{\primi}[1]{\mbox{\bf #1}}

\newtheorem{theorem}{Theorem}[section]
\newtheorem{corollary}{Corollary}[theorem]
\newtheorem{lemma}[theorem]{Lemma}


\begin{document}

\section{``Commutativity'' of Unification}

Let $p_{1,2},\,q_{1,2}$~--- some terms. Then

\[
MGU\,(p_1,\,p_2)\cdot MGU\,(q_1,\,q_2)=MGU\,(q_1,\,q_2)\cdot MGU\,(p_1,\,p_2)
\]

Indeed, both $MGU\,(p_1,\,p_2)\cdot MGU\,(q_1,\,q_2)$ and $MGU\,(q_1,\,q_2)\cdot MGU\,(p_1,\,p_2)$ are most general unifiers
for pairs $(p_1,\,p_2)$ and $(q_1,\,q_2)$ simultaneously. If there were different, then terms $C\,(p_1,\,q_1)$ and
$C\,(p_2,\,q_2)$, where $C$~--- some binary constructor, would have \emph{different} most general unifiers, which is impossible.

Let

\[
\theta\xrightarrow{u_1;\,u_2;\dots u_k}{\theta^\prime}
\]

be sequence of unifications which transforms a substitution $\theta$ into substitution $\theta^\prime$. Then

\[
\theta\xrightarrow{u_{\pi(1)};\,u_{\pi(2)};\dots u_{\pi(k)}}{\theta^\prime}
\]

for any permutation $\pi$.

\section{Properties of One-step Unfolding}

Let $r$ be an application (atomic formula), $\theta$~--- a substitution; one-step unfolding transforms a pair $\inbr{\theta,\,r}$ into
a set $\{\inbr{\theta_i,\,\rho_i}\}$, where $\rho_i$~--- a list (conjunction) of applications, $\theta_i$~--- some substitution. Then
for each $i$ there is a sequence of unifications $u^i_1\dots u^i_{k_i}$ such that

\[
\theta\xrightarrow{u^i_1\dots u^i_{k_i}}{\theta_i}
\]

Proof by derivation of one-step unfolding. Thus, we may introduce a denotation

\[
\theta_i=\theta\cdot[p\to\pi_i]
\]

\section{Weakenings in Angelic Semantics}

Let $\theta$ be a substitution, $\omega$~--- a list of applications, and let $\inbr{\theta,\,\omega}$ converges. Then for arbitrary substitution $\sigma$ and
arbitrary list of applications $\psi$

\[
\inbr{\theta\cdot\sigma,\,\omega\psi}
\]

converges.

\section{Properties of Unfolding in Angelic Semantics}

\begin{lemma}
  Let $\{\omega_i\}$ be a state (a set of lists (conjunctions) of applications). Then $\{\omega_i\}$ converges iff each $\omega_i$ converges.
\end{lemma}

\begin{lemma}
Let $\phi a \phi^\prime$ be a list of applications with one distinguished element $a$, $\theta$~--- a substitution. Let

\[
\{\inbr{\theta\cdot[a\to\alpha_i],\,\alpha_i}\}
\]

be the result of one-step unfolding for $\inbr{\theta,\,a}$. Then

\[
\inbr{\theta,\,\phi a \phi^\prime}
\]

converges iff all

\[
\{\inbr{\theta\cdot[a\to\alpha_i],\,\phi\alpha_i\phi^\prime}\}
\]

converge.
\end{lemma}

%\begin{proof}
Proof sketch.

\paragraph{$\Leftarrow$} Let all $\{\inbr{\theta\cdot[a\to\alpha_i],\,\phi\alpha_i\phi^\prime}\}$ converge. Then unfolding of $a$ delivers a converging derivation for $\inbr{\theta,\,\phi a \phi^\prime}$.
\paragraph{$\Rightarrow$} Let $\inbr{\theta,\,\phi a \phi^\prime}$ converges. Take the first step of its converging derivation. Two cases are possible.
\begin{enumerate}
\item The first step is unfolding of $a$. Then we have precisely

  \[
  \{\inbr{\theta\cdot[a\to\alpha_i],\,\phi\alpha_i\phi^\prime}\}
  \]
  
  as the next state, which means, that all of them converge.
\item The first step is an unfolding of some other application $b$. Then we can rewrite $\phi a \phi^\prime$ as $\phi a \psi b \omega$, and denote

  \[
  \{\theta\cdot[b\to\beta_j],\,\phi a \psi \beta_j \omega\}
  \]

  the result of one-step unfolding of $\inbr{\theta,\,b}$. Since

  \[
  \{\theta\cdot[b\to\beta_j],\,\phi a \psi \beta_j \omega\}
  \]

  is a state in a converging derivation, each

  \[
  \inbr{\theta\cdot[b\to\beta_j],\,\phi a \psi \beta_j \omega}
  \]

  converges. Take an arbitrary one. Two cases are possible.

  \begin{itemize}
  \item There is no unfolding of $a$ in the converging derivation. Then
    
    \[
    \inbr{\theta\cdot[b\to\beta_j],\,\phi\psi\beta_j\omega}\eqno{(*)}
    \]
    
    converges as well with the same results. Take the pair

    \[
    \inbr{\theta\cdot[a\to\alpha_i],\,\phi\alpha_i\psi b\omega}
    \]

    and perform a one-step unfolding of $b$; this will give us a state

    \[
    \inbr{\theta\cdot[a\to\alpha_i]\cdot[b\to\beta_j],\,\phi\alpha_i\psi\beta_j\omega}
    \]

    which converges by weakening taking into account $(*)$.
    
  \item There is a sequence of derivation 
    
    \[
    \inbr{\theta\cdot[b\to\beta_j],\,\phi a \psi\beta_j\omega}\to \inbr{\theta\cdot[b\to\beta_j]\cdot\sigma,\,\pi a \rho}\eqno{(**)}
    \]

    where $\sigma$ is some sequence of unifications and the next step is unfolding of $a$. Then its result is

    \[
    \{\inbr{\theta\cdot[b\to\beta_j]\cdot\sigma\cdot[a\to\alpha_i],\,\pi \alpha_i \rho}\}
    \]

    Take the pair

    \[
    \inbr{\theta\cdot[a\to\alpha_i],\,\phi\alpha_i\psi b\omega}
    \]

    and perform a one-step unfolding of $b$; this will give us a state

    \[
    \inbr{\theta\cdot[a\to\alpha_i]\cdot[b\to\beta_j],\,\phi\alpha_i\psi\beta_j\omega}
    \]

    Repeating the unfoldings of $(**)$ we can (at most) arrive at state

    \[
     \inbr{\theta\cdot[a\to\alpha_i]\cdot[b\to\beta_j]\cdot\sigma,\,\pi \alpha_i \rho}
     \]

     which converges since

     \[
     \theta\cdot[a\to\alpha_i]\cdot[b\to\beta_j]\cdot\sigma=\theta\cdot[b\to\beta_j]\cdot\sigma\cdot[a\to\alpha_i]
     \]
    
  \end{itemize}
%\end{proof}

\end{enumerate}

\begin{corollary}
  Let $\inbr{\theta,\,\omega}$~--- some pair of a substitution and a list of applications. Let $\inbr{\theta^\prime,\,\omega^\prime}$~---
  some other pair, obtained from $\inbr{\theta,\,\omega}$ in a finite number of unfoldings. Then if $\inbr{\theta,\,\omega}$ converges, then
  $\inbr{\theta^\prime,\,\omega^\prime}$~--- converges.
\end{corollary}

\end{document}

