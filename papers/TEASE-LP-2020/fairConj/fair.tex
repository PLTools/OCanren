\documentclass[submission,copyright,creativecommons]{eptcs}
\providecommand{\event}{TEASE-LP 2020} % Name of the event you are submitting to
\usepackage{breakurl}             % Not needed if you use pdflatex only.
\usepackage{underscore}           % Only needed if you use pdflatex.

\usepackage[utf8]{inputenc}
\usepackage{amsmath,amssymb}
\usepackage[english,russian]{babel}
\usepackage{amssymb}
\usepackage{mathtools}
\usepackage{listings}
\usepackage{comment}
\usepackage{indentfirst}
\usepackage{hyperref}
\usepackage{amsthm}
\usepackage{xcolor}
\usepackage{stmaryrd}
\usepackage{eufrak}
\usepackage{placeins}
\newtheorem{theorem}{Theorem}
\newtheorem{hyp}{Hypothesis}

\lstdefinelanguage{ocanren}{
keywords={run, conde, fresh, let, in, match, with, when, class, type,
object, method, of, rec, repeat, until, while, not, do, done, as, val, inherit,
new, module, sig, deriving, datatype, struct, if, then, else, open, private, virtual, include, success, failure,
true, false},
sensitive=true,
commentstyle=\small\itshape\ttfamily,
keywordstyle=\ttfamily\underbar,
identifierstyle=\ttfamily,
basewidth={0.5em,0.5em},
columns=fixed,
fontadjust=true,
literate={fun}{{$\lambda$}}1 {->}{{$\to$}}3 {===}{{$\equiv$}}1 {=/=}{{$\not\equiv$}}1 {|>}{{$\triangleright$}}3 {\\/}{{$\vee$}}2 {/\\}{{$\wedge$}}2 {^}{{$\uparrow$}}1,
morecomment=[s]{(*}{*)}
}

\lstset{
mathescape=true,
%basicstyle=\small,
identifierstyle=\ttfamily,
keywordstyle=\bfseries,
commentstyle=\scriptsize\rmfamily,
basewidth={0.5em,0.5em},
fontadjust=true,
language=ocanren
}

\usepackage{letltxmacro}
\newcommand*{\SavedLstInline}{}
\LetLtxMacro\SavedLstInline\lstinline
\DeclareRobustCommand*{\lstinline}{%
  \ifmmode
    \let\SavedBGroup\bgroup
    \def\bgroup{%
      \let\bgroup\SavedBGroup
      \hbox\bgroup
    }%
  \fi
  \SavedLstInline
}

\def\transarrow{\xrightarrow}
\newcommand{\setarrow}[1]{\def\transarrow{#1}}

\def\padding{\phantom{X}}
\newcommand{\setpadding}[1]{\def\padding{#1}}

\def\subarrow{}
\newcommand{\setsubarrow}[1]{\def\subarrow{#1}}

\newcommand{\trule}[2]{\frac{#1}{#2}}
\newcommand{\crule}[3]{\frac{#1}{#2},\;{#3}}
\newcommand{\withenv}[2]{{#1}\vdash{#2}}
\newcommand{\trans}[3]{{#1}\transarrow{\padding{\textstyle #2}\padding}\subarrow{#3}}
\newcommand{\ctrans}[4]{{#1}\transarrow{\padding#2\padding}\subarrow{#3},\;{#4}}
\newcommand{\llang}[1]{\mbox{\lstinline[mathescape]|#1|}}
\newcommand{\pair}[2]{\inbr{{#1}\mid{#2}}}
\newcommand{\inbr}[1]{\left<{#1}\right>}
\newcommand{\highlight}[1]{\color{red}{#1}}
%\newcommand{\ruleno}[1]{\eqno[\scriptsize\textsc{#1}]}
\newcommand{\ruleno}[1]{\mbox{[\textsc{#1}]}}
\newcommand{\rulename}[1]{\textsc{#1}}
\newcommand{\inmath}[1]{\mbox{$#1$}}
\newcommand{\lfp}[1]{fix_{#1}}
\newcommand{\gfp}[1]{Fix_{#1}}
\newcommand{\vsep}{\vspace{-2mm}}
\newcommand{\supp}[1]{\scriptsize{#1}}
\newcommand{\sembr}[1]{\llbracket{#1}\rrbracket}
\newcommand{\cd}[1]{\texttt{#1}}
\newcommand{\free}[1]{\boxed{#1}}
\newcommand{\binds}{\;\mapsto\;}
\newcommand{\dbi}[1]{\mbox{\bf{#1}}}
\newcommand{\sv}[1]{\mbox{\textbf{#1}}}
\newcommand{\bnd}[2]{{#1}\mkern-9mu\binds\mkern-9mu{#2}}
\newcommand{\meta}[1]{{\mathcal{#1}}}
\newcommand{\dom}[1]{\mathtt{dom}\;{#1}}
\newcommand{\primi}[2]{\mathbf{#1}\;{#2}}
\renewcommand{\dom}[1]{\mathcal{D}om\,({#1})}
\newcommand{\ran}[1]{\mathcal{VR}an\,({#1})}
\newcommand{\fv}[1]{\mathcal{FV}\,({#1})}
\newcommand{\tr}[1]{\mathcal{T}r_{#1}}
\newcommand{\diseq}{\not\equiv}
\newcommand{\reprfunset}{\mathcal{D}}
\newcommand{\reprfun}{\mathfrak{f}}
\newcommand{\cstore}{\Omega}
\newcommand{\cstoreinit}{\cstore_\epsilon^{init}}
\newcommand{\csadd}[3]{\mathbf{add}\,(#1, #2 \diseq #3)}  %{#1 + [#2 \diseq #3]}
\newcommand{\csupdate}[2]{\mathbf{update}\,(#1, #2)}  %{#1 \cdot #2}

\let\emptyset\varnothing
\let\eps\varepsilon

\title{A Step Towards a Fair Conjunction for \textsc{miniKanren}}
\author{Petr Lozov
\institute{Saint Petersburg State University and \\ JetBrains Research, Russia}
\email{lozov.peter@gmail.com}
\and
Dmitry Boulytchev
\institute{Saint Petersburg State University and \\ JetBrains Research, Russia}
\email{dboulytchev@math.spbu.ru}
}
\def\titlerunning{A Step Towards a Fair Conjunction for \textsc{miniKanren}}
\def\authorrunning{P. Lozov \& D. Boulytchev}
\begin{document}
\maketitle

\textsc{miniKanren}~\cite{TRS,WillThesis} is known for its complete interleaving search: any answer for a query will be discovered
in a finite number of steps. In other terms this property can be reformulated as commutativity of disjunction: the 


Одной из особенностей реляционного программирования~\cite{TRS,MicroKanren} является полный поиск, достигаемый благодаря
\emph{interleaving search}. Помимо полноты такой поиск делает реляционную дизъюнкцию справедливой: завершаемость программы
не зависит от порядка дизъюктов. Однако конъюнкция таким свойстом не обладает. Например, реализации отношения обращения
списка \lstinline{revers$^o$}, представленные на рисунке~\ref{fig:reverso}, отличаются только порядком конъюнктов
(строки \ref{line:reverso} и \ref{line:appendo}), однако в первом случае при обращении списка будет получен ответ, после
чего запрос разойдется. Во втором случае --- запрос завершится после обнаружения ответа.
Также для более длинных списков время обнаружения первого ответа сильно отличается: первая реализация оказывается в несколько раз быстрее.

\begin{comment}
\begin{figure}[h]
    \centering
    \begin{tabular}{c@{\hskip2cm}c}

\begin{lstlisting}[numbers=left,numberstyle=\small,escapechar=|]
let rec revers$^o_1$ x y =
  x === [] /\ y === [] \/
  fresh (e xs ys) (
    x === e : xs /\
    |\label{line:reverso}|revers$^o_1$ xs ys /\
    |\label{line:appendo}|appendo ys [e] y)
\end{lstlisting}

&

\begin{lstlisting}[numbers=left,numberstyle=\small]
let rec revers$^o_2$ x y =
  x === [] /\ y === [] \/
  fresh (e xs ys) (
    x === e : xs /\
    appendo ys [e] y /\
    revers$^o_2$ xs ys)
\end{lstlisting}

\end{tabular}
\caption{Two implementations of \lstinline{reverso}}
\label{fig:reverso}
\end{figure}
\end{comment}

На текущий момент существует техника, управляющая порядком конъюнктов: Relational Search via Divergence Test~\cite{DivTest}. Данная техника, однако,
консервативно подходит задаче перестановки конъюнктов. Также этот критерий не делает конъюнкцию справедливой.

Мы предлагаем подход, позволяющий сохранить информацию при перестановке конъюнктов за счёт преобразования частично вычисленного конъюнкта в дизъюнктивную
нормальную форму с явно представленными подстановками $\sigma_i$.

$$S \; \wedge \; X \Rightarrow (S_1 \sigma_1 \; \vee \ldots \vee \; S_n \sigma_n) \wedge \; X$$
В такой форме можно раскрыть скобки по правилу дистрибутивности, после чего поменять конъюнкты местами, сохранив содержимое подстановок.

$$(S_1 \sigma_1 \; \vee \ldots \vee \; S_n \sigma_n) \wedge \; X \Rightarrow (X \sigma_1 \; \wedge S_1) \; \vee \ldots \vee \; (X \sigma_n \; \wedge S_n)$$

Однако, использование такой перестановки после каждого шага редукции конъюнкта оказывается неэффективным, во-первых, из-за накладных расходов, связанных с
частым перестраиванием структуры каждого конъюнкта, а во-вторых, из-за потери приоритета вычислению левого конъюкта.  Таким образом, нам необходим критерий,
обнаруживающий необходимость перестановки конъюнктов.

Классический критерий, используемый в суперкомпиляции~\cite{turchin1986concept}, который называется \emph{embedding}~\cite{Nash-Williams1987} оказался неэффективен.
Хоть при данном критерии конъюнкция становится справедливой, ведь при счетном количестве шагов, перестановка конъюнктов обязательно произойдёт, но вычисление
данного критерия крайне трудоемко.

Более эффективным оказался критерий, основанный на анализе опровергающей способности текущего конъюнкта. Для этого перед исполнением программы для каждого
отношения строится аппроксимирующие отношение, из которого исключены все вызовы (пример для \lstinline{revers$^o$} на рисунке~\ref{fig:reverso_approx}).
Количество ответов такого отношения, вычисленное со свободными аргументами в пустой подстановке, всегда конечно и соответствует количеству содержательных
ветвей в исходном отношении. И если количество ответов этой аппроксимации, вычисленное на конкретных аргументах, меньше количества содержательных ветвей,
значит хотя бы одна из ветвей будет опровергнута. В этом случае мы продолжаем вычисление конъюнкта, в противном --- делаем перестановку. Отметим, что
такая конъюнкция в общем случае справедливой не является, но на практике часто оказывается справедливой.

\begin{figure}[h]
\centering
\begin{tabular}{c}
\begin{lstlisting}
let rec revers$^o$ x y =
  x === [] /\ y === [] \/
  fresh (e xs) (x === e : xs)
\end{lstlisting}
\end{tabular}
\caption{Approximation of \lstinline{reverso}}
\label{fig:reverso_approx}
\end{figure}

Для апробации данного критерия был разработан интерпретатор, основанный на операционной семантике~\cite{CertifiedSemantics} языка miniKanren, которая
была расширена возможностью перестановки конъюнктов. В качестве benchmarks были использованы два простых примера: \lstinline{revers$^o$} и \lstinline{sorto$^o$},
а также две более содержательные программы: \lstinline{hanoi$^o$} --- поиск решения задачи Ханойских башен\footnote{https://en.wikipedia.org/wiki/Tower_of_Hanoi},
и \lstinline{bridge$^o$} - поиск решения Bridge and torch problem\footnote{https://en.wikipedia.org/wiki/Bridge_and_torch_problem}. Каждая из этих программ была
взята в двух версиях: optimistic --- с хорошим порядком конъюнктов, и pessimistic --- с намеренно плохим порядком конъюнктов.

\begin{figure}[h]
  \small
  \centering
  \begin{tabular}{ c | c | c | c | c | c }
    relation & size &  \multicolumn{2}{c}{directed conjunction} & \multicolumn{2}{c}{rearranging conjunction} \\
    \cline{3-6}
    & & optimistic & pessimistic & optimistic & pessimistic  \\
    \hline
    revers$^o$   & 30         & 0.124 & 0.318 & 0.118 & 0.292 \\
                 & 60         & 0.255 & 1.791 & 0.287 & 0.636 \\
                 & 90         & 0.692 & 5.728 & 0.796 & 1.027 \\
    \hline
    sort$^o$     & 3          & 0.088 & 0.106 & 0.088  & 0.103  \\
                 & 4          & 0.087 & 0.975 & 0.099  & 0.117  \\
                 & 5          & 0.092 & >300  & 0.108  & 0.124  \\
                 & 30         & 8.601 & >300  & 11.813 & 42.853 \\
    \hline
    hanoi$^o$    & -          & 2.265 & >300  & 2.284 & 2.557  \\
    \hline
    bridge$^o$   & -          & 24.825 & >300 & 20.977 & 25.367

  \end{tabular}
  \caption{The results of a evaluation: running times of benchmarks in seconds}
  \label{evaluation_results}
\end{figure}

Как видно из таблицы~\ref{evaluation_results}, в оптимистичном случае обычная конъюнкция, как правило, незначительно быстрее. Это связано с тем, переставляющая конъюнкция ведет себя аналогично обычной, но присутствуют накладные расходы из-за вычисления критерия. Однако в пессимистичном случае переставляющая конъюнкция в разы быстрее обычной.

\bibliographystyle{eptcs}
\bibliography{fair}

\end{document}
