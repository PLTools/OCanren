\begin{figure}[t]
\[
\begin{array}{rcl}
  x\,[t/x] &=& t \\
  y\,[t/x] &=& y,\;\; y\ne x\\
  C_i^{k_i}\,(t_1,\dots,t_{k_i})\,[t/x]&=&C_i^{k_i}\,(t_1\,[t/x],\dots,t_{k_i}\,[t/x])\\[2mm]
  (t_1 \equiv t_2)\,[t/x]&=&t_1\,[t/x] \equiv t_2\,[t/x]\\
  (g_1 \wedge g_2)\,[t/x]&=&g_1\,[t/x] \wedge g_2\,[t/x]\\
  (g_1 \vee g_2)\,[t/x]&=&g_1\,[t/x] \vee g_2\,[t/x]\\
  (\mbox{\lstinline|fresh|}\;x\,.\,g)\,[t/x]&=&\mbox{\lstinline|fresh|}\;x\,.\,g\\
  (\mbox{\lstinline|fresh|}\;y\,.\,g)\,[t/x]&=&\mbox{\lstinline|fresh|}\;y\,.\,(g\,[t/x]),\;\;y\ne x\\
  (R_i^{k_i}\,(t_1,\dots,t_{k_i})\,[t/x]&=&R_i^{k_i}\,(t_1\,[t/x],\dots,t_{k_i}\,[t/x])
\end{array}
\]
  \caption{Substitutions for terms and goals}
  \label{substitution}
\end{figure}

\section{Denotational Semantics}
\label{denotational}

In this section we present a denotational semantics for the language we defined above. We use a simple set-theoretic
approach which can be considered as an analogy to the least Herbrand model for definite logic programs~\cite{LHM}.
Strictly speaking, instead of developing it from scratch we could have just described the conversion of specifications
into definite logic form and took their least Herbrand model. However, in that case we would still need to define
the least Herbrand model semantics for definite logic programs in a certified way. In addition, while for
this concrete language the conversion to definite logic form is trivial, it may become less trivial for
its extensions (with, for examples, nominal constructs~\cite{AlphaKanren}) which we plan to do in future.

We also must make the following observation. Usually, building inductive denotational semantics amounts to
constructing a complete lattice and a monotone function and taking its least fixed point~\cite{TarskiKnaster}.
As we deal with a first-order language with only monotonic constructs (conjunction/disjunction) these steps
are trivial. Moreover, we express the semantics in \textsc{Coq}, where all well-formed inductive definitions already
have proper semantics, which removed the neccesity to justify the validity of steps we perform.

To motivate further development, we first consider the following example. Let us have the following goal:

\begin{lstlisting}
   x === Cons (y, z)
\end{lstlisting}

There are three free variables, and solving the goal delivers us the following single answer:

\begin{lstlisting}
   $\alpha\mapsto\;$ Cons ($\beta$, $\gamma$)
\end{lstlisting}

where semantic variables $\alpha$, $\beta$ and $\gamma$ correspond to the syntactic ones ``\lstinline|x|'', ``\lstinline|y|'', ``\lstinline|z|''. The
goal does not put any constraints on ``\lstinline|y|'' and ``\lstinline|z|'', so there are no bindings for ``$\beta$'' and ``$\gamma$'' in the answer.
This answer can be seen as the following ternary relation over the set of all ground terms:

\[
\{(\mbox{\lstinline|Cons ($\beta$, $\,\gamma$)|}, \beta, \gamma) \mid \beta\in\mathcal{D},\,\gamma\in\mathcal{D}\}\subset\mathcal{D}^3
\]

The order of ``dimensions'' is important, since each dimension corresponds to a certain free variable. Our main idea is to represent this relation as a set of total
functions 

\[
\mathfrak{f}:\mathcal{A}\mapsto\mathcal{D}
\]

from semantic variables to ground terms. We call these functions \emph{representing functions}. Thus, we may reformulate the same relation as

\[
\{(\mathfrak{f}\,(\alpha),\mathfrak{f}\,(\beta),\mathfrak{f}\,(\gamma))\mid\mathfrak{f}\in\sembr{\mbox{\lstinline|$\alpha$ === Cons ($\beta$, $\,\gamma$)|}}\}
\]

where we use conventional semantic brackets ``$\sembr{\bullet}$'' to denote the semantics. For the top-level goal we need to substitute its free syntactic
variables with distinct semantic ones, calculate the semantics, and build the explicit representation for the relation as shown above. The relation, obviously,
does not depend on concrete choice of semantic variables, but depends on the order in which the values of representing functions are tupled. This order can be
conventionalized, which gives us a completely deterministic semantics.

Now we implement this idea. First, for a representing function

\[
\mathfrak{f} : \mathcal{A}\to\mathcal{D}
\]

we introduce its homomorphic extension 

\[
  \overline{\mathfrak{f}}:\mathcal{T_A}\to\mathcal{D}
\]

which maps terms to terms:

\[
\begin{array}{rcl}

  \overline{\mathfrak f}\,(\alpha) & = & \mathfrak f\,(\alpha)\\
  \overline{\mathfrak f}\,(C_i^{k_i}\,(t_1,\dots.t_{k_i})) & = & C_i^{k_i}\,(\overline{\mathfrak f}\,(t_1),\dots \overline{\mathfrak f}\,(t_{k_i}))
\end{array}
\]

Let us have two terms $t_1, t_2\in\mathcal{T_A}$. If there is a unifier for $t_1$ and $t_2$ then, clearly, there is a substitution $\theta$ which
turns both $t_1$ and $t_2$ into the same \emph{ground} term (we do not require $\theta$ to be the most general). Thus, $\theta$ maps
(some) ground variables into ground terms, and its application to $t_{1(2)}$ is exactly $\overline{\theta}(t_{1(2)})$. This reasoning can be
performed in the opposite direction: a unification $t_1\equiv t_2$ defines the set of all representing functions $\mathfrak{f}$ for which
$\overline{\mathfrak{f}}(t_1)=\overline{\mathfrak{f}}(t_2)$. 

We remind the conventional notions of pointwise modification of a function

\[
f\,[x\gets v]\,(z)=\left\{
\begin{array}{rcl}
  f\,(z) &,& z \ne x \\
  v      &,& z = x
\end{array}
\right.
\]

and substitution of a free variable with a term in terms and goals (see Figure~\ref{substitution}).

For a representing function $\mathfrak{f}:\mathcal{A}\to\mathcal{D}$ and a semantic variable $\alpha$ we define
the following \emph{generalization} operation:

\[
\mathfrak{f}\uparrow\alpha = \{ \mathfrak{f}\,[\alpha\gets d] \mid d\in\mathcal D\}
\]

Informally, this operation generalizes a representing function into a set of representing functions in such a way that the
values of these functions for a given variable cover the whole $\mathcal{D}$. We extend the generalization operation for sets of
representing functions $\mathfrak{F}\subseteq\mathcal{A}\to\mathcal{D}$:

\[
  \mathfrak{F}\uparrow\alpha = \bigcup_{\mathfrak{f}\in\mathfrak{F}}(\mathfrak{f}\uparrow\alpha)
\]

Now we are ready to specify the semantics for goals (see Figure~\ref{denotational_semantics_of_goals}).
 We've already given the motivation for
  the semantics of unification: the condition $\overline{\mathfrak{f}}(t_1)=\overline{\mathfrak{f}}(t_2)$ gives us the set of all (otherwise
  unrestricted) representing functions which ``equate'' terms $t_1$ and $t_2$.
  Set union and intersection provide a conventional interpretations
for disjunction and conjunction of goals. Now for the case of ``\lstinline|fresh $x$ . $g$|''. First, we take an arbitrary semantic variable $\alpha$,
not free in $g$, and substitute $x$ with $\alpha$. Then we calculate the semantics of $g\,[\alpha/x]$. The interesting part is the next step:
as $x$ can not be free in ``\lstinline|fresh $x$ . $g$|'', we need to generalize the result over $\alpha$ since in our model the semantics of a
goal specifies a relation over its free variables. We introduce some nondeterminism by choosing arbitrary $\alpha$, but we have proven by
structural induction that with different choices of free variable the semantics of a goal does not change (and this proof turned out to
be the most cumbersome among all others). Consider the following example:

\renewcommand{\overset}[2]{#2}
\[
\begin{array}{lc}
  \sembr{\mbox{\lstinline|fresh y . ($\alpha$ === y) $\,\wedge\,$ (y === Zero)|}}&\overset{\mbox{(by \textsc{Fresh$_D$})}}{=}\\[1mm]
  \multicolumn{2}{r}{\mbox{\tiny (by \textsc{Fresh$_D$})}}\\
  (\sembr{\mbox{\lstinline|($\alpha$ === $\beta$) $\,\wedge\,$ ($\beta$ === Zero)|}})\uparrow\beta&\overset{\mbox{(by \textsc{Conj$_D$})}}{=}\\[1mm]
  \multicolumn{2}{r}{\mbox{\tiny (by \textsc{Conj$_D$})}}\\
  (\sembr{\mbox{\lstinline|$\alpha$ === $\beta$|}} \,\cap\, \sembr{\mbox{\lstinline|$\beta$ === Zero)|}})\uparrow\beta&\overset{\mbox{(by \textsc{Unify$_D$})}}{=}\\[1mm]
  \multicolumn{2}{r}{\mbox{\tiny (by \textsc{Unify$_D$})}}\\
  (\{\mathfrak{f}\mid \overline{\mathfrak{f}}\,(\alpha)=\overline{\mathfrak{f}}\,(\beta)\} \,\cap\, \{\mathfrak{f}\mid \overline{\mathfrak{f}}\,(\beta)=\overline{\mathfrak{f}}\,(\mbox{\lstinline|Zero|})\})\uparrow\beta&\overset{\mbox{(by the definition of ``$\overline{\mathfrak{f}}$'')}}{=}\\[1mm]
  \multicolumn{2}{r}{\mbox{\tiny (by the definition of ``$\overline{\mathfrak{f}}$'')}}\\
  (\{\mathfrak{f}\mid \mathfrak{f}\,(\alpha)=\mathfrak{f}\,(\beta)\} \,\cap\, \{\mathfrak{f}\mid \mathfrak{f}\,(\beta)=\mbox{\lstinline|Zero|}\})\uparrow\beta&\overset{\mbox{(by the definition of ``$\cap$'')}}{=}\\[1mm]
  \multicolumn{2}{r}{\mbox{\tiny (by the definition of ``$\cap$'')}}\\
  (\{\mathfrak{f}\mid \mathfrak{f}\,(\alpha)=\mathfrak{f}\,(\beta)=\mbox{\lstinline|Zero|}\})\uparrow\beta&\overset{\mbox{(by the definition of ``$\uparrow$'')}}{=}\\[1mm]
  \multicolumn{2}{r}{\mbox{\tiny (by the definition of ``$\uparrow$'')}}\\
  \{\mathfrak{f}\mid \mathfrak{f}\,(\alpha)=\mbox{\lstinline|Zero|}, \mathfrak{f}\,(\beta)=d, d\in\mathcal{D}\}&\overset{\mbox{(by the totality of representing functions)}}{=}\\[1mm]
  \multicolumn{2}{r}{\mbox{\tiny (by the totality of representing functions)}}\\
  \{\mathfrak{f}\mid \mathfrak{f}\,(\alpha)=\mbox{\lstinline|Zero|}\}&
\end{array}
\]

In the end we've got the set of representing functions, each of which restricts only the value of free variable $\alpha$. 

The final case is relational invocation, in which we unfold the definition of corresponding relational symbol and substitute its formal parameters with
actual ones.

\begin{figure}[t]
  \[
  \begin{array}{cclr}
    \sembr{t_1\equiv t_2}&=&\{\mathfrak f : \mathcal{A}\to\mathcal{D}\mid \overline{\mathfrak{f}}\,(t_1)=\overline{\mathfrak{f}}\,(t_2)\}& \ruleno{Unify$_D$}\\
    \sembr{g_1\wedge g_2}&=&\sembr{g_1}\cap\sembr{g_1}&\ruleno{Conj$_D$}\\
    \sembr{g_1\vee g_2}&=&\sembr{g_1}\cup\sembr{g_1}&\ruleno{Disj$_D$}\\
    \sembr{\mbox{\lstinline|fresh|}\,x\,.\,g}&=&(\sembr{g\,[\alpha/x]})\uparrow\alpha,\;\alpha\not\in FV(g)& \ruleno{Fresh$_D$}\\
    \sembr{R\,(t_1,\dots,t_k)}&=&\sembr{g\,[t_1/x_1,\dots,t_k/x_k]} & \ruleno{Invoke$_D$}\\
    & & \multicolumn{2}{c}{\mbox{where}\;R=\lambda\,x_1\dots x_k\,.\,g}
  \end{array}
  \]
  \caption{Denotational semantics of goals}
  \label{denotational_semantics_of_goals}
\end{figure}

To formalize denotational semantics in \textsc{Coq} we can define representing functions simply as \textsc{Coq} functions:

\begin{lstlisting}[language=Coq,basicstyle=\footnotesize]
   Definition repr_fun : Set := var -> ground_term.
\end{lstlisting}

We define the semantics via inductive proposition ``\lstinline|in_denotational_sem_goal|'' (with notation ``\lstinline[mathescape=true]{[| $\bullet$ , $\bullet$ |]}'')
such that

\[
\forall g,\mathfrak{f}\;:\;\mbox{\lstinline|in_denotational_sem_goal|}\;g\;\mathfrak{f}\Longleftrightarrow\mathfrak{f}\in\sembr{g}
\]

The head of the definition is as follows

\begin{lstlisting}[language=Coq,basicstyle=\footnotesize,morekeywords={where,at,level}]
  Reserved Notation "[| g , f |]" (at level 0).
  Inductive in_denotational_sem_goal :
    goal -> repr_fun -> Prop :=
    ...
  where "[| g , f |]" := (in_denotational_sem_goal g f).
\end{lstlisting}

and the body just eumerates the cases shown on Fig.~\ref{denotational_semantics_of_goals}.
  
We formulate the following important \emph{completeness condition} for the semantics of a goal $g$:

\[
\forall {\mathfrak f}, {\mathfrak f'}:  \left.{\mathfrak f}\right|_{FV(g)} = \left.{\mathfrak f'}\right|_{FV(g)}, \quad {\mathfrak f} \in \sembr{g} \Leftrightarrow {\mathfrak f'} \in \sembr{g}
\]

In other words, representing functions for a goal $g$ restrict only the values of free variables of $g$ and do not introduce any ``hidden'' correlations.
This condition guarantees that our semantics is complete in the sense that it does not introduce artificial restrictions for the relation it defines.
The following lemma proves that the semantics of goals always satisfies this condition:

\begin{lstlisting}[language=Coq,basicstyle=\footnotesize]
  Lemma completeness_condition (f f' : repr_fun) (g : goal),
    (forall x, is_fv_of_goal x g -> gt_eq (f x) (f' x)) ->
    [| g , f |] ->
    [| g , f' |].
\end{lstlisting}

Here ``\lstinline[language=Coq]{gt_eq}'' stands for the equality of ground terms.
