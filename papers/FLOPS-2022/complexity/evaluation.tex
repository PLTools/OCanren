\section{Evaluation}
\label{sec:evaluation}

The theory we've built was so far applied to only one relation~--- \lstinline|append$^o$|~--- which we used as a motivation example. With our framework, it turned out
to be possible to explain the difference in performance between two nearly identical implementations, and the difference~--- linear vs. quadratic asymptotic
complexity~--- was just as expected from the experimental performance evaluation. In this section, we present some other results of complexity estimations and
discuss the adequacy of these estimations w.r.t. the real \mK implementations.

Derived complexity estimations for a few other relations are shown in \figureword~\ref{fig:examples_all_complexities}. Besides concatenation, we deal with
naive list reversing and Peano numbers addition and multiplication. We show both $d-$ and $t-$ factors since the difference between the two indicates the
cases when scheduling strikes in. We expect that for simple relations like those presented the procedure of deriving estimations should be easy; however,
for more complex ones the dealing with extracted inequalities may involve a non-trivial metatheory reasoning.

The justification of the adequacy of our complexity estimations w.r.t. the existing \mK implementations faces the following problem: it is
not an easy task to separate the contribution of scheduling from other components of the search procedure~--- unification and occurs check. However, it is common
knowledge among \textsc{Prolog} users that in practice unification takes a constant time almost always; some theoretical basis for this is given in~\cite{UnificationAverageCost}.
There are some specifics of unification implementation in \mK. First, for the simplicity of backtracking in a non-mutable fashion triangular
substitution~\cite{UnificationTheory} is used instead of the idempotent one. It brings in an additional overhead which is analyzed
in some detail in~\cite{WillsThesis}, but the experience shows that in the majority of practical cases this overhead is insignificant. Second, \mK by default
performs the ``occurs check'', which contributes a significant overhead and often subsumes the complexity of all other search components. Meanwhile, it is known, that occurs checks are rarely
violated~\cite{OccursCheckNotAProblem}. Having said this, we expect that in the majority of the cases the performance of \mK programs with the occurs check disabled are
described by scheduling complexity alone. In particular, this is true for all cases in \figureword~\ref{fig:examples_all_complexities}. To confirm the adequacy of
our model we evaluated the running time of these and some other goals (under the conditions we've mentioned) and found that it confirms the estimations derived
using our framework. The details of implementation, evaluation, and results can be found in an accompanying repository.~\footnote{https://www.dropbox.com/sh/ciceovnogkeeibz/AAAoclpTSDeY3OMagOBJHNiSa?dl=0}

\begin{figure}[t]
  \setlength{\belowcaptionskip}{-20pt plus 3pt minus 2pt}
  \[ \begin{tabular}{|l|c|c||l|c|c|}
    \hline
          & $d$ & $t$  & & $d$ & $t$ \\
    \hline
    \lstinline|append$^o$ $\mathbf{a}$ $\mathbf{b}$ ${ab}$|      & $len\,(\mathbf{a})$ & $len^2\,(\mathbf{a})$   & \lstinline|plus$^o$ $\mathbf{n}$ $\mathbf{m}$ ${r}$| & $|\mathbf{n}|$ & $|\mathbf{n}|$ \\
    \lstinline|append$_{opt}^o$ $\mathbf{a}$ $\mathbf{b}$ ${ab}$| & $len\,(\mathbf{a})$ & $len\,(\mathbf{a})$     & \lstinline|plus$^o$ $\mathbf{n}$ ${m}$ $\mathbf{r}$| & $\min\,\{|\mathbf{n}|, |\mathbf{r}|\}$ & $\min\,\{|\mathbf{n}|, |\mathbf{r}|\}$ \\
    \lstinline|append$_{opt}^o$ ${a}$ ${b}$ $\mathbf{ab}$|        & $len\,(\mathbf{ab})$ & $len\,(\mathbf{ab})$   & \lstinline|plus$^o$ ${n}$ ${m}$ $\mathbf{r}$| & $|\mathbf{r}|$ & $|\mathbf{r}|$ \\
    \lstinline|revers$^o$ $\mathbf{a}$ ${r}$|                    & $len^2\,(\mathbf{a})$ & $len^3\,(\mathbf{a})$ & \lstinline|mult$^o$ $\mathbf{n}$ $\mathbf{m}$ ${r}$| & $|\mathbf{n}|\cdot|\mathbf{m}|$ & $|\mathbf{n}|^2\cdot|\mathbf{m}|$ \\
    \lstinline|revers$^o$ ${a}$ $\mathbf{r}$|                    & $len^2\,(\mathbf{r})$ & $len^2\,(\mathbf{r})$ & \lstinline|mult$^o$ (S ${n}$) (S ${m}$) $\mathbf{r}$| & $|\mathbf{r}|^2$ & $|\mathbf{r}|^2$ \\
    \hline
  \end{tabular} \]
  \caption{Derived $d$- and $t$-factors for some goals; $len\,(\bullet)$ stands for the length of a ground list, $|\bullet|$~--- for the value of a Peano number, represented as ground term.}
  \label{fig:examples_all_complexities}
\end{figure}

