\section{Scheduling Complexity}
\label{sec:scheduling}

We may notice that operational semantics described in the previous section can be used to calculate the number of elementary scheduling steps.
In this section, we define a specific value that estimates the scheduling time and give some equations to calculate this value for a given \emph{semantic
state}. However, our ultimate goal is to provide a way to deduce the complexity factor recursively for a given query. This problem will be addressed in
Section~\ref{sec:symbolic}, which will make use of recurrent equations presented here.

We also restrict our considerations only by the case when the evaluation of a goal in question terminates. Indeed,
if the search diverges, no reasonable complexity estimation for the time of the whole search can be given. A much more interesting question would be
the complexity estimation for coming up with some \emph{specific} answer. For now we do not have a complete solution for this case; we discuss the issue in details in Section~\ref{sec:discussion}.

Our first idea is to take the number of states $d\,(s)$ in the \emph{finite} trace for a given state $s$:

\[ d\,(s) \; \eqdef \; | \trs{s} |  \]

However, it turns out, that this value alone is not enough to provide an accurate scheduling complexity estimation. The reason is that some
elementary steps in the semantics are not elementary in existing implementations. Namely, a careful analysis discovers that
each of semantic steps involves a navigation to the leftmost leaf of the state which in implementations corresponds to multiple elementary actions,
which number is proportional to the length of the leftmost branch of the state in question. Here we provide an \emph{ad-hoc} definition for this value, $t\,(s)$, which we call the \emph{scheduling factor}:

\[
t\,(s) \eqdef \sum\limits_{s_i \in \trs{s}} lh\,(s_i) 
\]

where

\[
\begin{array}{rcl}
 lh\,(\taskst{g}{e})  &\eqdef& 1 \\
 lh\,(s_1 \oplus s_2) &\eqdef& lh\,(s_1) + 1 \\
 lh\,(s \otimes g)    &\eqdef& lh\,(s) + 1 \\
\end{array}
\]


The following lemma provides a fundamental relation between these two estimations of the scheduling complexity:

\begin{lemma}
  For any state $s$

  \[
  d\,(s) \le t\,(s) \le d^2\,(s)
  \]
  
\end{lemma}

Our next goal is to come up with the equations which relate the scheduling complexity of states with the scheduling complexity of their
(immediate) substates. It turns out that to come up with such equations both $t$ and $d$ values need to be estimated.  


% We take scheduling factor $t\,(s)$ as a value that determines the scheduling complexity $T_s$, but we will also need to calculate $d\,(s)$ as it will be used in the equations for $t\,(s)$. 

The following trivial lemma handles the basic (leaf state) case:

\begin{lemma}
  If

  \[\taskst{g}{e} \xrightarrow{l} s^\prime\]

  then

  \[d\,(\taskst{g}{e}) = d\,(s^\prime) + 1\]

  and

  \[t\,(\taskst{g}{e}) = t\,(s^\prime) + 1\]
\end{lemma}

%In $\oplus$-states the substates are evaluated separately, one step at a time for each substate,
%so the total number of semantic steps is just a sum.
%However, for the scheduling factor, there is an extra summand since the heights of the states in
%the trace becomes bigger (by one additional $\oplus$-node on the top).
%This additional node exists in the trace until one of the substates is evaluated completely, so the
%scheduling factor is increased by the number of steps before such an event.
%So we have the following lemma.

The next lemma provides the equations for $\oplus$-states:
\begin{lemma}
\label{lem:sum_estimation}
For any two states $s_1$ and $s_2$

\[
\begin{array}{rcl}
  d\,(s_1 \oplus s_2) &=& d\,(s_1) + d\,(s_2) \\

%  t\,(s_1 \oplus s_2) &=& t\,(s_1) + t\,(s_2) + \min\,\{2\cdot d\,(s_1) - 1, 2\cdot d\,(s_2)\}
    t\,(s_1 \oplus s_2) &=& t\,(s_1) + t\,(s_2) + \costdisj{s_1}{s_2}
\end{array}
\]

where

\[ \costdisj{s_1}{s_2} = \min\,\{2\cdot d\,(s_1) - 1, 2\cdot d\,(s_2)\} \] 

\end{lemma}

Informally, for a state in the form $s_1 \oplus s_2$ the substates are evaluated separately, one step at a time for
each substate, so the total number of semantic steps is the sum of those for the substates. However, for the scheduling factor, 
there is an extra summand $\costdisj{s_1}{s_2}$ since the ``leftmost heights'' of the states in the trace are one node greater then those for the
original substates due to the introduction of one additional $\oplus$-node on the top. This additional node persists in the trace until the evaluation of one of the substates comes to the end, so the scheduling factor is increased by the number of steps until that.

The next lemma provides the equations for $\otimes$-states:
\begin{lemma}
For any state $s$ and any goal $g$

\[
\begin{array}{rcl}
d\,(s \otimes g)  &=&  d\,(s) + \smashoperator[lr]{\sum\limits_{a_i \in \tra{s}}} d\,(\taskst{g}{a_i}) \\

 t\,(s \otimes g)  &=&  t\,(s) + \costconj{s}{g} + \smashoperator[lr]{\sum\limits_{a_i \in \tra{s}}} (t\,(\taskst{g}{a_i}) + \costdisj{\taskst{g}{a_i}}{(s'_i \otimes g)})
\end{array}
\]

where 

\[
\begin{array}{rcl}
\costdisj{s_1}{s_2} & = & \min\,\{2\cdot d\,(s_1) - 1, 2\cdot d\,(s_2)\} \\
\costconj{s}{g} & = & d\,(s) \\
s'_i & = & \mbox{the next state in the trace for s after} \\
 & & \mbox{the transition labeled with the answer $a_i$} \\
\end{array}
\]
\end{lemma}

For the states of the form $s \otimes g$ the reasoning is the same, but the result is more complicated.
In $\otimes$-state the left substate is evaluated until an answer is found, which is then taken as
\emph{an environment} for the evaluation of the right subgoal.
Thus, in the equations for $\otimes$-states we sum the evaluation time of the second goal for all
the answers generated for the first substate.
The tasks of evaluating the right subgoal in different environments are added to the
evaluation of the left substate by the creation of an $\oplus$-state, so for scheduling factor there is
an additional summand $\costdisj{\taskst{g}{a_i}}{s'_i}$ for each answer with $s'_i$ being the state
after discovering the answer.
There is also an extra summand $\costconj{s}{g}$ to the scheduling factor because of the
$\otimes$-node that increases the height in the trace, analogous to the one caused by
$\oplus$-nodes.
We can notice that $\otimes$-node is always placed immediately over the left substate so this
addition is exactly the number of steps for the left substate.

{ \color{red}

After unfolding the auxilary costs definitions the last equation becomes quite cumbersome and moreover
it is hard to identify all intermidiate states that occur in the equation.
Luckily, we can nicely approximate this additional costs up to a multiplicative factor.
Such approximation will be sufficient to find compexity.
First, we rewrite the equation for $t(s \otimes g)$ in the following form:
\[ t\,(s \otimes g)  =  t\,(s) + \smashoperator[lr]{\sum\limits_{a_i \in \tra{s}}} t\,(\taskst{g}{a_i}) + C\,(s \otimes g) \]
where
\[ C\,(s \otimes g) = \costconj{s}{g} + \smashoperator[lr]{\sum\limits_{a_i \in \tra{s}}} \costdisj{\taskst{g}{a_i}}{(s'_i \otimes g)} \]

Next, we will need the following definition.

\begin{definition}
Let $E$ be a set of environments, $g$~--- a goal. Then we call the value

\[
\alpha\,(g, E) = \argmax{e \in E} d\,(\taskst{g}{e})
\]

a \emph{senior environment}.
\end{definition}

In other words, $\alpha\,(g, E)$ is an element of $E$ which delivers the longest trace of $g$. 
With senior environment defined we can prove the following lemma\footnote{We assume the following definition for 
$h\,(x) = \Theta\,(f\,(x))$: \[\exists C_1, C_2 \in \mathcal{R^{+}}, \, \forall x : C_1 \cdot f\,(x) \le h\,(x) \le C_2 \cdot f\,(x) \]} (see Appendix~\ref{A}):

\begin{lemma}

\[
C\,(s \otimes g) =
\Theta\,(d\,(s) + \smashoperator[lr]{\sum\limits_{\begin{array}{c}
                                                                                   \scalebox{0.7}{$a_i \in \tra{s}$} \\
                                                                                   \scalebox{0.7}{$a_i \ne \alpha\,(g, \tra{s}) $} \\
                                                                              \end{array}}} d\,(\taskst{g}{a_i}))	
\]
\end{lemma}

We can see that this approximation is very similar to the expression for $d$ value for $\otimes$-state except that we exclude the senior environment. This correction is essential for complexity and, as we will see later, it is in fact responsible for the difference in complexities for our motivational example.

As a result we have the following approximation for the scheduling factor of $\otimes$-states.

\begin{corollary}
\[
 t\,(s \otimes g)  =  t\,(s) + \left({\sum\limits_{a_i \in \tra{s}}} t\,(\taskst{g}{a_i})\right) +
 \Theta\,(d\,(s) + \smashoperator[lr]{\sum\limits_{\begin{array}{c}
                                                                                   \scalebox{0.7}{$a_i \in \tra{s}$} \\
                                                                                   \scalebox{0.7}{$a_i \ne \alpha\,(g, \tra{s}) $} \\
                                                                              \end{array}}} d\,(\taskst{g}{a_i}))	
\]
\end{corollary}

}









\begin{comment}

\textcolor{red}{\bf OR}

\begin{definition}
For two functions $f \colon X_1 \times \dots \times X_n \to \mathcal{R}$ and $g \colon X_1 \times \dots \times X_n \colon \mathcal{R} \to \mathcal{R}$ we say that $f(x_1,\, \dots,\, x_n) \sim_C g(x_1,\, \dots,\, x_n, C)$ if there exist two positive real constants $c^{low}$ and $c^{up}$ such that for all $(x_1,\, \dots,\, x_n) \in X_1 \times \dots \times X_n$, $g(x_1,\, \dots,\, x_n,\, c^{low}) \le f(x_1,\, \dots,\, x_n) \le g(x_1,\, \dots,\, x_n,\, c^{up})$.
\end{definition}

\begin{lemma}
\label{lem:prod_approximation}
For any state $s$ and any goal $g$
\[ 
\begin{array}{lcrl}

t\,(s \otimes g) & \lowupbound{1}{2} & & t\,(s) + (\displaystyle\sum\limits_{a_i \in \tra{s}} t\,(\taskst{g}{a_i})) + \\ 
& & C \cdot ( & d\,(s) + (\displaystyle\sum\limits_{a_i \in \tra{s}} d\,(\taskst{g}{a_i})) {\color{blue} - d\,(\taskst{g}{a_m})} ) 

\end{array} \]

where $a_m = \argmax{a_i \in \tra{s}} d(\taskst{g}{a_i})$
\end{lemma}

\textcolor{red}{\bf OR}

\begin{lemma}
\label{lem:prod_approximation}
\[ 
\begin{array}{lcrl}

t\,(s \otimes g) & \sim_C & & t\,(s) + (\displaystyle\sum\limits_{a_i \in \tra{s}} t\,(\taskst{g}{a_i})) + \\ 
& & C \cdot ( & d\,(s) + (\displaystyle\sum\limits_{a_i \in \tra{s}} d\,(\taskst{g}{a_i})) {\color{blue} - d\,(\taskst{g}{a_m})} ) 

\end{array} \]

where $a_m = \argmax{a_i \in \tra{s}} d(\taskst{g}{a_i})$
\end{lemma}

The blue addend is called ...

\end{comment}

\begin{comment}

One option is to go with the first argument of ``$\min$'' in $\costdisj{\taskst{g}{a_i}}{s'_i}$.
It should be a good approximation in the case when there are several answers passed to the second
goal and for none of them the number of steps surpasses the \emph{overall} number of steps for all
other answers (the second argument of ``$\min$'' will include the sum for the rest of the answers).

\begin{corollary}
\label{lem:prod_estimation_multiple_answers}
For any state $s$ and any goal $g$
\[ t\,(s \otimes g) \le t\,(s) + d\,(s) + \displaystyle\sum\limits_{a_i \in \tra{s}} (t\,(\taskst{g}{a_i}) + 2\cdot d\,(\taskst{g}{a_i}) \]
\end{corollary}

In the case when there is only one answer, however, we should rather go with the second argument of ``$\min$''. 

In this case the number of steps $d\,(s'_1 \otimes g)$ is equal to the number of steps $d\,(s'_1)$
since no more answers are produced, and we can approximate it by the length $d\,(s)$ of the whole
trace for $s$. 

\begin{corollary}
\label{lem:prod_estimation_single_answer}
  For any state $s$ and any goal $g$, if $\tra{s} = \{a\}$, then
  
\[ t\,(s \otimes g) \le t\,(s) + 3\cdot d\,(s) + t\,(\taskst{g}{a}) \]
\end{corollary}

Finally, since we estimate only up to a multiplicative constant (in particular, it does not matter by what constants we multiply the values of $d\,(\cdot)$ when calculating
the scheduling factor) we can derive from these results two compact scheduling time approximations for goals in the form of sequences of disjuncts/conjuncts.
These two approximations work regardless of the associativity/grouping of subformulas; thus a certain constant $c_k$, depending only on $k$, comes in.

For conjunctions, we have the following one.

\begin{lemma}
\label{lem:conjunction_metrics_calc}

Let $g = g_1 \land \dots \land g_k$ and let $A_i$ be a set of all answers that are passed to $g_i$ at some stage starting from some initial environment $e_0$

\[
\begin{array}{rcl}
A_1 &=& \{ e_0 \} \\
A_{i + 1} & = & \bigcup\limits_{a \in A_i} \tra{\taskst{g_i}{a}} 
\end{array}
\]

Then

\[
\begin{array}{rcl}
d\,(\taskst{g}{e}) &=& \displaystyle\sum\limits_{1 \le i \le k} \;\; \displaystyle\sum\limits_{a \in A_i} d\,(\taskst{g_i}{a}) \\
t\,(\taskst{g}{e}) &\le& \displaystyle\sum\limits_{1 \le i \le k} \;\; \displaystyle\sum\limits_{a \in A_i} t\,(\taskst{g_i}{a}) + c_k \cdot \displaystyle\sum\limits_{1 \le i \le k} \;\; \displaystyle\sum\limits_{a \in A_i} d\,(\taskst{g_i}{a}), \\
\end{array}
\]

In the case when all $A_i$ contain only one answer

\[
t\,(\taskst{g}{e}) \le \displaystyle\sum\limits_{1 \le i \le k} \;\; \displaystyle\sum\limits_{a \in A_i} t\,(\taskst{g_i}{a}) + c_k \cdot \displaystyle \sum\limits_{1 \le i \le k - 1} \;\; \displaystyle\sum\limits_{a \in A_i} d\,(\taskst{g_i}{a})
\]

\end{lemma}

When applying the estimation from corollary~\ref{lem:prod_estimation_multiple_answers} we have an extra summand in the form of the number of steps (multiplied by some constant) for all conjuncts.
The only exception is the case when every conjunct produces no more than one answer, then we can use the corollary~\ref{lem:prod_estimation_single_answer} everywhere instead and drop out the
additional number of steps for the last conjunct. Besides that, a constant number of steps is required to turn each conjunction into a $\otimes$-state, but we may integrate this extra constant into $c_k$.

For disjunctions, the lemma is the following one.

\begin{lemma}
\label{lem:disjunction_metrics_calc}

Let $g = g_1 \lor \dots \lor g_k$ and $1 \le l \le k$; then

\[
\renewcommand{\arraystretch}{1.5}
\begin{array}{rcl}
  d\,(\taskst{g}{e}) &\le& \displaystyle\sum\limits_{1 \le i \le k} d\,(\taskst{g_i}{e}) \\
  t\,(\taskst{g}{e}) &\le& \displaystyle\sum\limits_{1 \le i \le k} t\,(\taskst{g_i}{e}) + c_k\cdot \displaystyle\sum\limits_{\renewcommand{\arraystretch}{1}\begin{array}{c}1 \le i \le k \\ i \ne l\end{array}} d\,(\taskst{g_i}{e}).
\end{array}
\]

\end{lemma}

Roughly speaking, if we have a disjunct $g_m$ with a number of steps larger than all the steps for other disjuncts combined, then when applying lemma~\ref{lem:sum_estimation} we again will have an
extra summand in the form of the number of steps for all disjuncts except for the $g_m$ (it will always be the largest argument of ``$\min$''). But if we can drop out the \emph{largest}
the number of steps among disjuncts, we can also drop out any other instead, that's where arbitrary $l$ comes from. The case when there is no such $g_m$ has to be considered separately; it is simpler
since then all the numbers of steps are the same up to a multiplicative constant.

\end{comment}
