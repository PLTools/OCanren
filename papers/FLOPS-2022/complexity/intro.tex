\section{Introduction}
\label{sec:intro}

\begin{figure}[t]
\begin{tabular}{p{6cm}p{6cm}}
\begin{lstlisting}[basicstyle=\small]
   append$^o_{opt}$ = fun a b ab ->
     ((a === Nil) /\ (ab === b)) \/
     (fresh (h t tb)
        (a === Cons(h, t)) /\
        (ab === Cons(h, tb) /\
        (append$^o_{opt}$ t b tb)) 
     )
\end{lstlisting} &
\begin{lstlisting}[basicstyle=\small]
   append$^o$ = fun a b ab ->
     ((a === Nil) /\ (ab === b)) \/
     (fresh (h t tb)
        (a === Cons(h, t)) /\
        (append$^o$ t b tb) /\
        (ab === Cons(h, tb)) 
     )
\end{lstlisting}
\end{tabular}
\caption{Two implementations of list concatenation relation}
\label{fig:length_implementations}
\end{figure}

A family of embedded languages for logic and, more specifically, relational programming \mK~\cite{TRS} has demonstrated an interesting potential is various fields of 
program synthesis~\cite{SevenProblems,Twines,Matching} or declarative pogramming~\cite{?}. A distinctive feature of \mK is \emph{interleaving search}~\cite{Transformers} which,
in particular, delivers such an important feature as completeness.

However, being a different search strategy than conventional BFS/DFS/incremental deepening, etc., interleaving search makes the conventional ways of reasoning about the complexity
of logical programs irrelevant. Moreover, some intrinsic properties of interleaving search can manifest themselves in a number of astounding and, at the first glance, unexplanable
performance effects. 

As an example, let's consider two implementations of list concatenation relation (\figureword~\ref{fig:length_implementations}). The only difference between the two is
the position of the recursive call. The common wizdom of \mK programming prescribes putting a recursive call \emph{last} in a conjunction (if possible), since otherwise multiple
negative effects (for example, divergence) can be encountered when running a relation in a backward direction. What is unexpected is that the right implementation
works significantly slower being run is a \emph{forward} direction as well, although it performs exactly the same number of unifications. A careful analysis discovers that the difference
is caused not by unifications, but by the process of \emph{scheduling} goals during the search. In \mK a lazy structure is build that decomposes the
goals into unifications, performs these unifications in a certain order, and threads the results appropriately. For the \lstinline|append$^o$| this structure becomes linear in
size, increasing the time of scheduling.

In the paper we present a model to estimate scheduling complexity of interleaving search in \mK. We use a formal semantics which, in turn, reflects the
behaviour of actual implementations~\cite{CertifiedSemantics}. The roadmap of our approach is as follows: we identify two complexity measures (\emph{scheduling factors}) and
give exact and approximate recursive formulas to calculate the scheduling factors (\sectionword~\ref{sec:scheduling}); then we present an approach to
automaticaly extract inequalities for the scheduling factors for a whole goal using symbolic execution (\sectionword~\ref{src:symbolic}). These inequalities have
to be reformulated and solved in terms of a certain \emph{metatheory}, which, on success, provides an asymptotic bounds for the complexity of
the goal evaluation. To be applicable, our approach puts a number of limitation on the goal being estimated as well as on the relational program as whole. We discuss
these limitations in \sectionword~\ref{sec:discussion} as well as the contribution of scheduling in overall interleaving search complexity in \mK.

