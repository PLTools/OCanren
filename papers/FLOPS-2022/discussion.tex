\section{Discussion and Future Work}
\label{sec:discussion}

\textcolor{blue}{
The formal framework presented here analyzes the basic aspects of scheduling cost for interleaving search strategy from the theoretical viewpoint.
As we have shown, it is sufficiently powerful to explain some surprising asymptotic behaviour for simple standard programs in \mK,
but the applicability of this framework in practice for real implementations of \mK requires further investigation.
Two key aspects that determine practical applicability are the admissibility of the imposed reqiurements and
correspodance of specific \mK implementations to the reference operational semantics, which should be studied individually for each application.
We see our work as the ground for the future development of methods for analysing the cost of interleaving search.
}

Our approach imposes three requirements on the analyzed programs: disjunctive normal form, uniqueness of answers, and grounding of relational calls.
The first two are rather non-restrictive: DNF is equivalent to the description of relation as a set of Horn clauses in \textsc{Prolog},
and the majority of well-known examples in \mK are written in this or very similar form. Repetition of answers is usually an indication
of a mistake in a program~\cite{WillsThesis}. The groundness condition is more serious: it prohibits program execution from presenting infinitely
many individual ground solutions in one answer using free variables, which is a useful pattern. At the same time, this requirement is
not unique for our work (the framework for \textsc{CASLOG} system mentioned above imposes exactly the same condition) and the experience
shows that many important kinds of programs satisfy it (although it is hard to characterize the class of such programs precisely).
Relaxing any of these restrictions will likely mess up the current relatively compact description of symbolic execution (for
the conditions on relational calls) or the form of the extracted inequalities (for the DNF condition).

Also, for now, we confine ourselves to the problem of estimating the time of the full search for a given goal. Estimating the time before
the first (or some specific) answer is believed to be an important and probably more practical task. Unfortunately, the technique we describe
can not be easily adjusted for this case. The reason for this is that the reasoning about time (scheduling time in particular) in our
terms becomes non-compositional for the case of interrupted search: if an answer is found in some branch, the search is cut short in
other branches, too. Dealing with such a non-compositionality is a subject of future research.

%And we still need to analyze every branch, since the height of states can be different in different branches, so
%the equal number of semantic steps in different branches can take different amounts of time to evaluate. This picture requires more
%complicated notions with non-trivial dependencies between them and the model becomes impractical. We still can use our technique
%to establish some rough lower and upper bounds (for example via relation between two complexity factors from \lemmaword~\ref{lem:d_t_relation}),
%but in general this problem requires a separate thorough analysis,


