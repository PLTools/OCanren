\section{Background: Syntax and Semantics of \mK}
\label{sec:background}

In this section, we recollect some known formal descriptions for \mK language that will be used as a basis for our development.
The descriptions here are taken from~\cite{CertifiedSemantics} (with a few non-essential adjustments for presentation purposes) to make
the paper self-contained, more details and explanations can be found there.

The syntax of core \mK is shown in Fig.~\ref{fig:syntax}. 
All data is presented using terms $\mathcal{T}_X$ built from a fixed set of constructors $\mathcal{C}$ with known arities and variables
from a given set $X$.
We parameterize the terms with an alphabet of variables since in the semantic description we will need \emph{two} kinds of variables:
\emph{syntactic} variables $\mathcal{X}$, used for bindings in the definitions, and \emph{logic} variables $\mathcal{A}$, which are
introduced and unified during the evaluation. We assume the set $\mathcal{A}$ ordered and use the notation $\alpha_i$ 
to specify a position of a logical variable w.r.t. this order.

There are five types of goals: unification of two terms, conjunction, and disjunction of goals,
%(the ``\lstinline|conde|'' operator from the canonical versions of \mK, split in the two for simplicity),
fresh logic variable introduction, and invocation of some relational definition. For the sake of brevity, in code snippets, we abbreviate
immediately nested ``\lstinline|fresh|'' constructs into the one, writing ``\lstinline|fresh x y $\dots$ . $g$|'' instead of
``\lstinline|fresh x . fresh y . $\dots$ $g$|''. The \emph{specification} $\mathcal{S}$ consists of a set of relational definitions and a top-level goal.
A top-level goal represents a search procedure that returns a stream of substitutions for the free variables of the goal.
%The language we consider is first-order, as goals can not be passed as parameters, returned, or constructed at runtime.


\begin{figure}[t]
\centering
\[
\begin{array}{ccll}
  \mathcal{C} & = & \{C_i^{k_i}\} & \mbox{constructors} \\
  \mathcal{T}_X & = & X \cup \{C_i^{k_i} (t_1, \dots, t_{k_i}) \mid t_j\in\mathcal{T}_X\} & \mbox{terms over the set of variables $X$} \\
  \mathcal{D} & = & \mathcal{T}_\emptyset & \mbox{ground terms}\\
  \mathcal{X} & = & \{ \ttt{x}, \ttt{y}, \ttt{z}, \dots \} & \mbox{syntactic variables} \\
  \mathcal{A} & = & \{ x, y, z \dots \} & \mbox{logic variables} \\
  \mathcal{R} & = & \{ R_i^{k_i}\} &\mbox{relational symbols with arities} \\[2mm]
  \mathcal{G} & = & \mathcal{T_X}\equiv\mathcal{T_X}   &  \mbox{equality} \\
              &   & \mathcal{G}\wedge\mathcal{G}     & \mbox{conjunction} \\
              &   & \mathcal{G}\vee\mathcal{G}       &\mbox{disjunction} \\
              &   & \mbox{\lstinline|fresh|}\;\mathcal{X}\;.\;\mathcal{G} & \mbox{fresh variable introduction} \\
              &   & R_i^{k_i} (t_1,\dots,t_{k_i}),\;t_j\in\mathcal{T_X} & \mbox{relational symbol invocation} \\[2mm]
  \mathcal{S} & = & \{R_i^{k_i} = \lambda\;\ttt{x}_1^i\dots \ttt{x}_{k_i}^i\,.\, g_i;\}\; g, \; \textcolor{blue}{g_i, g\in\mathcal{G}} & \mbox{specification}
\end{array}
\]
\caption{The syntax of \mK}
\label{fig:syntax}
\end{figure}

During the evaluation of \mK program an environment, consisting of a substitution for logic variables and a counter of allocated logic
variables, is threaded through the computation and updated in every unification and fresh variable introduction.
The substitution in the environment at a given point and given branch of evaluation contains all the information about relations between
the logical variables at this point.
%Hereafter we refer to the substitution in the environment at a given point as ``current substitution''
%in informal explanations.
Different branches are combined via \emph{interleaving search} procedure~\cite{Transformers}.
The answers for a given goal are extracted from the final environments.

% \begin{wrapfigure}{r}{0.5\textwidth}
% \setlength{\abovecaptionskip}{0pt plus 3pt minus 2pt}
\begin{figure}[t]
\centering
\[
\begin{array}{ccllcccll}
  \Sigma & = & \mathcal{A} \to \mathcal{T}_\mathcal{A} & \mbox{substitutions} &\qquad\qquad&        S & = & \taskst{\mathcal{G}}{E} & \mbox{task} \\
       E & = & \Sigma \times \mathbb{N}              & \mbox{environments}  &\qquad\qquad&          &   & S \oplus S              & \mbox{sum} \\
         &   &                                       &                      &\qquad\qquad&          &   & S \otimes \mathcal{G}   & \mbox{product} \\
%         &   &                                       &                      &\qquad\qquad&  \hat{S} & = & \diamond \; \mid \; S   & \mbox{states} \\
       L & = & \circ \; \mid \; E      & \mbox{labels}        &\qquad\qquad&  \hat{S} & = & \diamond \; \mid \; S   & \mbox{states} \\
%         &   &                                       &                      &\qquad\qquad&        L & = & \circ \; \mid \; E      & \mbox{labels} 
\end{array}
\]
\caption{States and labels in the LTS for \mK}
\label{fig:operanional_semantics_states_labels}
%\end{wrapfigure}
\end{figure}

This search procedure is formally described by operational semantics in the form of a labeled transition system.
This semantics corresponds to the canonical implementation of interleaving search. 

The form of states and labels in the transition system is defined in \figureword~\ref{fig:operanional_semantics_states_labels}.
Non-terminal states $S$ have a tree-like structure with intermediate nodes corresponding to partially evaluated conjunctions
(``$\otimes$'') or disjunctions (``$\oplus$'').
A leaf in the form $\taskst{g}{e}$ determines a task to evaluate a goal $g$ in an environment $e$. For a conjunction node, its right child
is always a goal since it cannot be evaluated unless some result is provided by the left conjunct.
We also need a terminal state $\diamond$ to represent the end of the evaluation.
The label ``$\circ$'' is used to mark those steps which do not provide an answer; otherwise, a transition is labeled by an updated
environment.

\begin{figure*}[h]
  \renewcommand{\arraystretch}{1.6}
  \[
  \begin{array}{crcr}
    \multicolumn{3}{c}{\taskst{t_1 \equiv t_2}{(\sigma, n)} \xrightarrow{\circ} \Diamond , \, \, \nexists\; mgu\,(t_1 \sigma, t_2 \sigma)} &\ruleno{UnifyFail} \\
    \multicolumn{3}{c}{\taskst{t_1 \equiv t_2}{(\sigma, n)} \xrightarrow{(mgu\,(t_1 \sigma, t_2 \sigma) \circ \sigma),\, n)} \Diamond} & \ruleno{UnifySuccess} \\
    \multicolumn{3}{c}{\taskst{\mbox{\lstinline|fresh|}\, \ttt{x}\, .\, g}{(\sigma, n)} \xrightarrow{\circ} \taskst{g\,[\bigslant{\alpha_{n + 1}}{\ttt{x}}]}{( \sigma, n + 1)}} & \ruleno{Fresh} \\
    \multicolumn{3}{c}{\dfrac{R_i^{k_i}=\lambda\,\ttt{x}_1\dots \ttt{x}_{k_i}\,.\,g}{\taskst{R_i^{k_i}\,(t_1,\dots,t_{k_i})}{e} \xrightarrow{\circ} \taskst{g\,[\bigslant{t_1}{\ttt{x}_1}\dots\bigslant{t_{k_i}}{\ttt{x}_{k_i}}]}{e}}} & \ruleno{Invoke}\\[3mm]
    \taskst{g_1 \lor g_2}{e} \xrightarrow{\circ} \taskst{g_1}{e} \oplus \taskst{g_2}{e} & \ruleno{Disj} &
    \taskst{g_1 \land g_2}{e} \xrightarrow{\circ} \taskst{g_1}{e} \otimes g_2 & \ruleno{Conj} \\    
    \dfrac{s_1 \xrightarrow{l} \Diamond}{(s_1 \oplus s_2) \xrightarrow{l} s_2} & \ruleno{DisjStop} &
    \dfrac{s_1 \xrightarrow{l} s'_1}{(s_1 \oplus s_2) \xrightarrow{l} (s_2 \oplus s'_1)} &\ruleno{DisjStep} \\
    \dfrac{s \xrightarrow{\circ} \Diamond}{(s \otimes g) \xrightarrow{\circ} \Diamond} &\ruleno{ConjStop} &
    \dfrac{s \xrightarrow{e} \Diamond}{(s \otimes g) \xrightarrow{\circ} \taskst{g}{e}}  & \ruleno{ConjStopAns}\\
    \dfrac{s \xrightarrow{\circ} s'}{(s \otimes g) \xrightarrow{\circ} (s' \otimes g)} &\ruleno{ConjStep} &
    \dfrac{s \xrightarrow{e} s'}{(s \otimes g) \xrightarrow{\circ} (\taskst{g}{e} \oplus (s' \otimes g))} & \ruleno{ConjStepAns} 
  \end{array}
  \]
  \caption{Operational semantics of interleaving search}
  \label{fig:operanional_semantics_rules}
\end{figure*}

The transition rules are shown in Fig.~\ref{fig:operanional_semantics_rules}.
\textcolor{blue}{
The first six rules define evaluation of leaf states.
For the disjunction and conjunction the corresponding node states are constructed.
For other types of goals the environment and the evaluated goal are updated in accordance with the task given by the goal: for an equality the most general unifier of the terms is incorporated into the substitution (or execution halts if the terms are non-unifiable); for a fresh construction a new variable is introduced and the counter of allocated variables is incremented; for a relational call the body of the relation is taken as the next goal.
The rest of the rules define composition of evaluation of substates for partial disjunctions and conjunctions.
For a partial disjunction the first constituent is evaluated for one step, then the constituents are swapped (which constutes the \emph{interleaving}), and answer is propagated.
When the evalution of the first constituent of partial disjunction halts, the evaluation proceedes with the second constituent.
For a partial conjunction the first consituent is evaluated until the answer is obtained, then the evaluation of the second constituent with this answer as the environment is scheduled for evaluation together with the remaining partial conjunction (via partial disjunction node).
When the evalution of the first constituent of partial conjunction halts, the evaluation of the conjunction halts too.
}

The introduced transition system is completely deterministic,
therefore a derivation sequence for a state $s$ determines a certain \emph{trace}~--- a sequence of states and labeled transitions between
them. It may be either finite (ending with the terminal state $\Diamond$) or infinite. We will denote by $\trs{s}$ the sequence of states in
the trace for initial state $s$ and by $\tra{s}$ the sequence of answers in the trace for initial state $s$. The sequence $\tra{s}$ corresponds
to the stream of answers in the reference \mK implementations.

In the following we rely on the following property of leaf states:

\begin{definition}
  A leaf state $\taskst{g}{\mkenv{\sigma}{n}}$ is well-formed iff $\fv{g}\cup\Dom\,(\sigma)\cup\VRan\,(\sigma)\subseteq\{\alpha_1,\dots,\alpha_n\}$, where
  $\fv{g}$ denotes the set of free variables in a goal $g$, $\Dom\,(\sigma)$ and $\VRan\,(\sigma)$~--- the domain of a substitution $\sigma$ and
  a set of all free variables in its image respectively.
\end{definition}

Informally, in a well-formed leaf state all free variables in goals and substitution respect the counter of free logical variables.
This definition is in fact an instance of a more general definition of well-formedness for all states, introduced in~\cite{CertifiedSemantics}, where it is
proven that the initial state is well-formed and any transition from a well-formed state results in a well-formed one.

Besides operational semantics, we will make use of a denotational one analogous to the least Herbrand model. For a relation $R^k$, its denotational semantics $\sembr{R^k}$ is
treated as a $k$-ary relation on the set of all ground terms, where each ``dimension'' corresponds to a certain argument of $R^k$. For example,
$\sembr{\lstinline|append$^o$|}$ is a set of all triplets of ground lists, in which the third component is a
concatenation of the first two. The concrete description of the denotational semantics is given in~\cite{CertifiedSemantics} as well as the proof of
the soundness and completeness of the operational semantics w.r.t. to the denotational one.

\begin{figure}[t]
\centering
\[
\begin{array}{ccll}
B_{nf} & = &  \unigoal{\mathcal{T}_\mathcal{X}}{\mathcal{T}_\mathcal{X}} \; \mid \;
                     \invokegoal{R^k}{\mathcal{T}_\mathcal{X}}{\mathcal{T}_\mathcal{X}} \\
C_{nf} & = & B_{nf} \; \mid \; \conjgoal{C_{nf}}{B_{nf}} \\
F_{nf} & = & C_{nf} \; \mid \; \freshgoal{X}{F_{nf} } \\
D_{nf} & = & F_{nf} \; \mid \; \disjgoal{D_{nf}}{F_{nf}}
\end{array}
\]
\caption{Disjunctive Normal Form for goals}
\label{fig:dnf}
\end{figure}

Finally, we explicitly enumerate all the restrictions required by our method to work:

\begin{itemize}
\item All relations have to be in DNF \textcolor{blue}{(set $D_{nf}$ in \figureword\ref{fig:dnf})}. 
\item We only consider goals which converge with a finite number of answers.
\item All answers have to be ground (\emph{groundness} condition) for all relation invocations encountered
  during the evaluation.
\item All answers have to be unique (\emph{answer uniqueness} condition) for all relation invocations encountered
  during the evaluation.
\end{itemize}

