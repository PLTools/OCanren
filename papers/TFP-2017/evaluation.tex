\section{Implementation and evaluation}
\label{evaluation}

We implemented the relational conversion for the subset of Objective Caml language, using the infrastructure of the 
original compiler\footnote{\url{http://github.com/ocaml/ocaml}}. We rewrite the original abstract syntax tree with
the types, inferred by the compiler, into relational form, using the set of combinators from \ocanren. 

Our preliminary evaluation has shown, that our convertor generates a lot of abstractions, most of which can be inlined.
So we additionally implemented an optimization pass, which performs inlining where possible. This optimization
greatly improves the performance of the converted program.

As the first example of the conversion we consider a plain implementation of the \lstinline|map| function for lists:

\begin{lstlisting}
   let rec map = fun f.fun l.
     match l with
     | Nil          -> Nil
     | Cons (x, xs) -> Cons (f x, map f xs)
\end{lstlisting}

It's relational counterpart without optimizations is shown below:

\begin{lstlisting}
   let rec map = funf.funl.funq.
     fresh (q$_1$) ((l q$_1$) /\
       (((q$_1$ ===  ^Nil) /\ ((fun q$_2$. q$_2$ ===   ^Nil) q)) \/
       (fresh (q$_3$ q$_4$)
         (q$_1$ ===  ^Cons (q$_3$, q$_4$)) /\
         (fun x.fun xs. (fun q$_5$.
           (fresh (q$_6$ q$_7$) 
             (f x q$_6$) /\
             (map f xs q$_7$) /\
             (q$_5$ ===  ^Cons (q$_6$, q$_7$)))) q
         ) ((===) q$_3$) ((===) q$_4$))))
\end{lstlisting}

After inlining we get the following final relational form:

\begin{lstlisting}
   let rec map = fun f.fun l.fun q.
     fresh (q$_1$) ((l q$_1$) /\
       (((q$_1$ ===  ^Nil) /\ (q ===  ^Nil)) \/
       (fresh (q$_3$ q$_4$)
         (q$_1$ ===  ^Cons (q$_3$, q$_4$))
           (fresh (q$_6$ q$_7$)
             (f ((===) q$_3$) q$_6$) \/
             (map f ((===) q$_4$) q$_7$) \/
             (q === ^Cons (q$_6$, q$_7$))))))
\end{lstlisting}

In the next subsections we consider more elaborated and interesting examples.

\subsection{High-order Lambda Interpreter}

The construction of this interpreter was inspired by Felleisen-style semantic description~\cite{Felleisen}, when the rule for finding a redex
inside a lambda term is expressed by mean of a context. Our interpreter takes as a first argument a function, which decomposes the term, passed
as a second argument, into the redex and the context. Thus, various reduction orders can be expressed by changing only decomposition function.

We represent terms in the De Bruijn notation, thus the types for terms, contexts and the result of decomposition look like as follows:

\begin{lstlisting}
   type ('nat, 'lambda) lambda = 
   | Index of 'nat
   | Abst  of 'lambda
   | App   of 'lambda * 'lambda
  
   type ('lambda, 'context) context = 
   | Hole
   | AppL  of 'context * 'lambda
   | AppR  of 'lambda * 'context
   | AbstH of 'context

   type ('lambda, 'context) split = 
   | Nothing 
   | Beta of 'context * 'lambda * 'lambda
\end{lstlisting}

We also implemented three reduction strategies~--- call-by-name, call-by-value and normal order and converted 
the interpreter into the relational form. Our relational interpreter now allows, first, to reduce lambda terms

\begin{lstlisting}
   eval normal_order (=== (^App^(Abst^(Index ^Z), ^Index ^(S ^Z)))) q ==> [q=Index (S Z)]
\end{lstlisting}

and, second, to generate from a normal form a set of terms, which can be reduced to this normal form, using specified
reduction order:

\begin{lstlisting}
   eval normal_order (=== q) ^(Index ^Z) ==>              
   [
    q=Index Z; 
    q=App (Abst (Index Z), Index Z); 
    q=App (Abst (Index (S Z)), _.18); 
    q=App (Abst (Index Z), App (Abst (Index Z), Index Z)); 
    q=App (Abst (Index Z), App (Abst (Index (S Z), _.45)); 
    ...
   ]
\end{lstlisting}

\subsection{Syntactic Unification}

Syntactic unification~\cite{Unification} for two terms finds a substitution, which transforms these terms into equal. We 
suppose, that a term can be either a variable, or an application of a constructor to some terms, hence the following definition:

\begin{lstlisting}
   type ('var, 'const, 'terms) term = 
   | Var  of 'var
   | Cnst of 'const * 'terms
\end{lstlisting}

In our implementation we perform a unification in the context of some initial substitution; as the unification can fail, it
returns optional substitution. We represents a substitution with a type \lstinline|subst|, which is simply an associative
list. 

Relational unification allows, first, to perform regular unification:

\begin{lstlisting}
   unify (=== Var "x") (=== ^(Cnst (^"A", (=== [])))) [] q ==> 
   [q=Some [("x", Cnst ("A", []))]]
\end{lstlisting}

then, to generate one or both terms, which unification results in a given substitution:

\begin{lstlisting}
   unify (=== q) (=== ^(Cnst (^"A", []))) (=== []) ^(Some [(^"z", ^(Cnst (^"A", [])))])) ==> 
   [q=Var "z"]
\end{lstlisting}

and, finally, to calculate a pair of terms, which transform one substitution to another:

\begin{lstlisting}
   unify (=== q) (=== r) (=== []) ^(Some ^[(^"z", ^(Cnst (^"A", ^[])))]) ==> 
   [
    (q=Var "x"               , r=Cnst ("A", []));
    (q=Cnst ("A", [])        , r=Var "x");
    (q=Cnst (_.32, [Var "x"]), r=Cnst (_.32, [Cnst ("A", [])]))
   ]
\end{lstlisting}

\begin{comment}
\subsection{Вывод типов Хиндли-Милнера}

Данный алгоритм по лямбда-терму вычисляет наиболее общий тип этого терма.

Для описания используемой системы типов были написаны алгебраические типы $lambda\_type$ и $gen\_type$. Первый тип определяет две типовые контстанты $TInt$ и $TBool$, типовые именовынные переменные, а также объединение пары типов $a$ и $b$ в функциональный тип $a \rightarrow b$. Тип $gen\_type$ является полиморфным замыканием типа по списку имен типовых переменных.

\begin{lstlisting}[language=ocaml,mathescape=true,numberstyle=\small,stepnumber=1,numbersep=-5pt]
type ('var, 'lambda_type) lambda_type =
 | TInt
 | TBool
 | TVar of 'var
 | TFun of 'lambda_type * 'lambda_type

type ('list, 'lambda_type) gen_type =
 | Gen of 'list * 'lambda_type
\end{lstlisting}

Синтаксис языка (типы $literal$ и $lambda$) состоит из констат, переменных, применения, абстакции и конструкции $let$. 

\begin{lstlisting}[language=ocaml,mathescape=true,numberstyle=\small,stepnumber=1,numbersep=-5pt]
type literal =
 | LInt of int
 | LBool of bool

type ('var, 'literal, 'lambda) lambda =
 | Var  of 'var
 | Lit  of 'literal
 | App  of 'lambda * 'lambda
 | Abst of 'var * 'lambda
 | Let  of 'var * 'lambda * 'lambda
\end{lstlisting}

Итоговая функция $type\_inference$ имеет следующий тип:
\begin{lstlisting}[language=ocaml,mathescape=true,numberstyle=\small,stepnumber=1,numbersep=-5pt]
type lambda_type$_0$ = (num, lambda_type$_0$) lambda_type
type lambda$_0$ = (list, literal, lambda$_0$) lambda

type_inference ::
  'a ->('a ->'a) ->'b lambda$_0$ ->'a lambda_type$_0$ option
\end{lstlisting}
Данная функция абстрагируется от типа имен типовых переменных с помощью первого и второго аргумента, а именно имени первой свободной типовой переменной, а также функции-генеретора следующей свободной переменной по предыдущей.

Например, если в качестве имен типовых переменных использовать числа Пеано, получим:
\begin{lstlisting}[language=ocaml,mathescape=true,numberstyle=\small,stepnumber=1,numbersep=-5pt]
let nat_type_inference term = type_inference Z (fun n -> S n) term
\end{lstlisting}

Полученное с помощью реляционного преобразования отношение можо использовать для
\begin{itemize}
\item вычисления типа терма;
\begin{lstlisting}[language=ocaml,mathescape=true,numberstyle=\small,stepnumber=1,numbersep=-5pt]
run (q) 1 (nat_type_inference (=== Abst ("x", Var "x")) q)

==> {
q=Just (TFun (TVar Z, TVar Z)); 
}
\end{lstlisting}
\item проверки населенности типа с помощью генерации термов, которые имеют данный тип.
\begin{lstlisting}[language=ocaml,mathescape=true,numberstyle=\small,stepnumber=1,numbersep=-5pt]
run (q) 5 (nat_type_inference ((===) q) (Just TBool))

==> {}
q=Lit (LBool _.24); 
q=Let (_.18, Lit (LInt _.32), Lit (LBool _.80)); 
q=Let (_.18, Lit (LBool _.32), Lit (LBool _.80)); 
q=Let (_.89, Lit (LBool _.32), Var _.89); 
q=Let (_.18, Abst (_.26, Lit LInt (_.48)), Lit (LBool _.149)); 
}
\end{lstlisting}
\end{itemize}

Полученное отношение также абстрагировано от типа имен типовых переменных.

\subsection{Интерпретатор пожмножества языка Scheme}
Входным языком данного call-by-value интерпретатора является нетипизированное лямда-исчисление, дополненное конструктором списков и оператором цитирования. Оператор цитирования возвращает входной терм, не изменяя его.

\begin{lstlisting}[language=ocaml,mathescape=true,numberstyle=\small,stepnumber=1,numbersep=-5pt]
type 'var ident = 
  | Lambda 
  | Quote 
  | List 
  | Var of 'var

type ('terms, 'ident) term = 
  | Ident of 'ident 
  | Seq of 'terms
\end{lstlisting}

Результатом выполнения входной программы является либо вычисленное значение, либо замыкание функции.

\begin{lstlisting}[language=ocaml,mathescape=true,numberstyle=\small,stepnumber=1,numbersep=-5pt]
type ('term, 'ident, 'env) result = 
  | Val of 'term 
  | Closure of 'ident * 'term * 'env
\end{lstlisting}

Итоговый функциональный интерпретатор $eval$ имеет представленную ниже сингатуру.

\begin{lstlisting}[language=ocaml,mathescape=true,numberstyle=\small,stepnumber=1,numbersep=-5pt]
type term$_0$ = (term$_0$ list, string ident) term
type env = (string * term$_0$) list
type result$_0$ = (term$_0$, string ident, env) result

eval :: term$_0$ ->env ->result$_0$
\end{lstlisting}

Реляционная форма данного интерпретатора помимо интерпретирования входной программы позволяет находидить квайны произвольного порядка. Например, для вычисления квайна первого порядка достаточно запросить запросить программу, результатом вычисления которой в пустом окружении является она сама.

\begin{lstlisting}[language=ocaml,mathescape=true,numberstyle=\small,stepnumber=1,numbersep=-5pt]
run (q) 1 (eval (=== q) (===[]) (Val q))

==> {
q=Seq [
  Seq [Ident Lambda; Seq [Ident (_.253 [=/= List; =/= Quote])]; 
    Seq [Ident List; Ident (_.253 [=/= List; =/= Quote]); 
      Seq [Ident List; Seq [Ident Quote; Ident Quote];
        Ident (_.253 [=/= List; =/= Quote])]]];
    Seq [Ident Quote; 
      Seq [Ident Lambda; Seq [Ident (_.253 [=/= List; =/= Quote])]; 
        Seq [Ident List; Ident (_.253 [=/= List; =/= Quote]); 
          Seq [Ident List; Seq [Ident Quote; Ident Quote]; 
          Ident (_.253 [=/= List; =/= Quote])]]]]];
}
\end{lstlisting}
Если полученный результат представить в синтаксисе Scheme и подставить вместо семантической переменной $(\_.253)$ переменную $x$, получим
\begin{lstlisting}[language=ocaml,mathescape=true,numberstyle=\small,stepnumber=1,numbersep=-5pt]
((lambda(x) (list x (list (quote quote) x)))
 (lambda(x) (list x (list (quote quote) x))))
\end{lstlisting}

Полученная программа является квайном, что было проверено с помощью функциональной программы.

\section{Обозримый пример 1}

\begin{lstlisting}[language=ocaml,mathescape=true,numbers=left,numberstyle=\small,stepnumber=1,numbersep=-5pt,caption={Функция map},captionpos=b]
  let rec map = funf.funl.
    match l with
    | Nil          -> Nil
    | Cons (x, xs) -> Cons (f x, map f xs)
\end{lstlisting}

\begin{lstlisting}[language=ocaml,mathescape=true,numbers=left,numberstyle=\small,stepnumber=1,numbersep=-5pt,caption={Отношение map},captionpos=b]
  let rec map = funf.funl.funq.
    fresh (q$_1$) ((l q$_1$) &&&
      (((q$_1$ === Nil) &&& ((funq$_2$.q$_2$ === Nil) q)) |||
      (fresh (q$_3$ q$_4$)
        (q$_1$ === Cons (q$_3$, q$_4$)) &&&
        (funx.funxs.(funq$_5$.
          (fresh(q$_6$ q$_7$) 
            (f x q$_6$) &&&
            (map f xs q$_7$) &&&
            (q$_5$ === Cons(q$_6$, q$_7$)))) q
        ) ((===) q$_3$) ((===) q$_4$))))
\end{lstlisting}

\begin{lstlisting}[language=ocaml,mathescape=true,numbers=left,numberstyle=\small,stepnumber=1,numbersep=-5pt,caption={Отношение map после бета-редукций},captionpos=b]
  let rec map = funf.funl.funq.z
    fresh (q$_1$) ((l q$_1$) &&&
      (((q$_1$ === Nil) &&& (q === Nil)) |||
      (fresh (q$_3$ q$_4$)
        (q$_1$ === Cons (q$_3$, q$_4$))
          (fresh(q$_6$ q$_7$)
            (f ((===) q$_3$) q$_6$) &&&
            (map f ((===) q$_4$) q$_7$) &&&
            (q === Cons(q$_6$, q$_7$))))))
\end{lstlisting}

\section{Обозримый пример 2}
Здесь не описаны функции $index\_subst$ и $hole\_subst$.

\begin{lstlisting}[language=ocaml,mathescape=true,numbers=left,numberstyle=\small,stepnumber=1,numbersep=-5pt,caption={Функция eval},captionpos=b]
  let rec eval =funbeta_finder.funterm.
    match beta_finder term with
    | Beta (hole, body, arg) -> 
      let new_body = index_subst body Z arg in
      let new_term = hole_subst hole new_body in
      eval beta_finder new_term
    | Nothing -> term  
\end{lstlisting}

\begin{lstlisting}[language=ocaml,mathescape=true,numbers=left,numberstyle=\small,stepnumber=1,numbersep=-5pt,caption={Отношение eval после бета-редукций},captionpos=b]
  let rec eval=funbeta_finder.funterm.funq.
    fresh (q$_1$) ((term q$_1$) &&&
      (fresh (q$_2$ q$_3$ q$_4$) 
        (q$_1$ === Beta (q$_2$, q$_3$, q$_4$)) &&&
        (let new_body = index_subst ((===)q$_3$) ((===)Z) ((===)q$_4$) in
         let new_term = hole_subst ((===)q$_2$) new_body in
         eval funbeta_finder new_term)) |||
      (q$_1$ === Nothing) &&& (term q))
  
\end{lstlisting}
\end{comment}
