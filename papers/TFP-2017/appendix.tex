\section{Appendix}
\label{appendix}

In this appendix we present a proof of partial semantic correctness of relational conversion, or, to be precise, 
a number of observations, definitions, and claims, which, we believe, are sufficient to reconstruct
the complete proof. 

We remind, that our goal is to prove the following statement:

\begin{theorem} 
\normalfont For arbitrary functional program $p$ of a ground type $t$, arbitrary value $v$, and
arbitrary variable $x$

$$
\begin{array}{c}
p\leadsto^f v \Rightarrow \lstinline|fresh ($x$) ($\sembr{p}^c x$)| \leadsto^r (\theta, \emptyset)\\
\mbox{and}\\
\theta(\mathfrak{s})=v
\end{array}
$$

\noindent where $\mathfrak{s}$ is a semantic variable, associated with
$x$ on the first step of the relational evaluation.
\end{theorem}
  
We first comment on the empty set as the set of negative substitutions. A disequality constraint can
come only from a polymorphic equality, which is applied when both its operands are reduced to
values. In the relational counterpart, being run in a forward direction, this corresponds to the evaluation of disequality constraints for
closed terms only, which, in turn, means, that they will immediately succeed or fail. Both cases
add nothing to the set of negative substitutions, which is initially empty. 

Next, we cannot prove the theorem, using an induction by a derivation length, since in the case of
application, for example, the type of the term in the head position is not ground. This 
obstacle could be lifted, if we could prove the following generalization:

$$
p\leadsto^f f \Rightarrow \sembr{p}^c\leadsto^r\sembr{f}^c
$$ 

\noindent for arbitrary $p$ of any type. This claim, however, turned out to be false~--- a term
\lstinline|C ((fun x.x) A)| can be taken as an example.  

The origin of the problem is that we \emph{functionalize} the constructors, \lstinline|match|, and
equality expressions, and, hence, change the order of reductions in the relational counterpart in 
comparison with the original functional program. Thus, we need to take this change into account.

First, we develop a modified functional semantics, which corresponds better to the reduction
order in the relational case. We call this semantics \emph{deferred}, as it defers the evaluation
of constructors, \lstinline|match|, and equality expressions. This semantics can be acquired in
two steps: first, we consider a reduced version of the original functional semantics, in which
we treat arbitrary constructor, \lstinline|match|, and equality expressions as values. Then, the
deferred semantics is just an iterative application of the reduced version to the arguments 
of these new values (arguments of constructors or equality operator, or scrutinees of \lstinline|match| 
expressions).

Next, we claim, that if a term of some ground type is reduced to some value by the original semantics,
then it as well is reduced to the same value by the deferred one. This claim is based on the following
observations:

\begin{itemize}
\item progress and type preservation properties for both semantics (which can be proven in a standard
way);
\item Church-Rosser property for lambda-calculus;
\item the fact, that the reduced semantics applies a proper subset of rules of the original one.
\end{itemize}

Now, we are going to prove the theorem by a simulation between the deferred semantics for the original program
and the relational one for the relationally converted. Before that, we formulate the number of lemmas and 
definitions.

\begin{lemma}
\label{stack_split}
\normalfont Let us separate all the contexts into two disjoint kinds: 

\begin{itemize}
\item functional

$$
C_f = \Box\;e\mid v\;\Box\mid\lstinline|let $x$ = $\Box$ in $e$|
$$

\item ground

$$
C_g = \lstinline|match $\;\Box\;$ with $\{p_i$->$e_i\}$|\mid C^n(\bar{v},\Box,\bar{e})\mid\Box\lstinline|=e|\mid\lstinline|v=|\Box
$$

Let $\left<{\mathcal S},\,e\right>$ be an arbitrary state in a derivation sequence w.r.t. the deferred
semantics. Then $\mathcal S=C_f^*C_g^*$.
\end{itemize}

In other words, during the evaluation w.r.t. the deferred semantics, the stack of contexts is separated into the two
(possibly empty) segments: all ground contexts reside below all functional. The proof is by the induction on the
length of derivation sequence.
\end{lemma}

\begin{definition}
\normalfont
We as well separate all terms of the source language into the two disjoint kinds:

\begin{itemize}
\item functional

$$
e_1\,e_2\mid \lambda x.e \mid \mu f.\lambda x.e \mid \lstinline|let $x$ = $e_1$ in $e_2$| \mid \lstinline|let rec $f$ = $\lambda x.e_1$ in $e_2$|
$$

\item ground

$$
e_1 = e_2 \mid \lstinline|match $e$ with {$p_i$ -> $e_i$ }| \mid \lstinline|C$^k$ ($e_1\dots e_k$)|
$$

\end{itemize}

\end{definition}

\begin{definition}
\normalfont Augmented conversion of a term w.r.t. to a substitution $\sembr{\bullet}_\theta$ is defined as follows: 

$$
\begin{array}{rcl}
\sembr{p}_\theta&=&\sembr{p}^c\\
\sembr{v}_\theta&=&(\lambda x.x\equiv\mathfrak{s}),\,\mbox{if}\;\;\theta(\mathfrak s)=v
\end{array}
$$

Here $\theta$ is a substitution, $p$~--- arbitrary functional term, $v$~--- arbitrary value of a
ground type in the sense of the original semantics (i.e. the composition of constructors). Note, the
cases in this definition are not disjoint, and in the second case there can be more, than one
variable with the requested property, so augmented conversion defines a set of relational terms.
\end{definition}

\begin{lemma}
\label{substitution}
\normalfont Let $f$, $e$ be two arbitrary terms of the source language, $\theta$~--- arbitrary
substitution. Then

$$
\sembr{f[x\gets e]}_\theta=\sembr{f}_\theta[x\gets\sembr{e}_\theta]
$$

The equality here is understood in a set-theoretic sense. The proof is by structural 
induction.
\end{lemma}

\begin{definition}
\normalfont For arbitrary substitution $\theta$ define a conversion of a functional context  
$\sembr{\bullet}_\theta$ as follows:

$$
\begin{array}{rcl}
\sembr{\Box\,e}_\theta&=&\Box\,\sembr{e}_\theta\\
\sembr{v\,\Box}_\theta&=&\sembr{v}_\theta\,\Box\\
\sembr{\lstinline|let $\;x\; = \;\Box\;$ in $\;e$|}_\theta&=&\lstinline|let $\;x\; = \;\Box\;$ in $\;\sembr{e}_\theta$|
\end{array}
$$

Here $e$ is an arbitrary functional term, $v$~--- abstraction. This conversion is an extension of augmented
conversion for functional contexts, hence the same denotation.
\end{definition}

\begin{definition}
\normalfont For arbitrary semantic variables ${\mathfrak s}_1$, ${\mathfrak s}_2$ and arbitrary substitution $\theta$ 
define a conversion of ground context $\sembr{\bullet}^{{\mathfrak s}_1{\mathfrak s}_2}_\theta$ as follows:

$$ 
\begin{array}{rcl}
\sembr{C^k(v_1, \ldots, v_{i-1}, \Box, e_{i+1}, \ldots, e_k)}^{{\mathfrak s}_1{\mathfrak s}_2}_\theta&=&\Box \; \wedge \\
       & & (\sembr{e_{i+1}}_\theta \; {\mathfrak s}^\prime_{i+1}) \; \wedge \\
       & & \ldots  \\
       & & (\sembr{e_k}_\theta \; {\mathfrak s}^\prime_k) \; \wedge \\
       & & ({\mathfrak s}_2 \equiv\; \uparrow C^k({\mathfrak s}^\prime_1, \ldots, {\mathfrak s}^\prime_{i-1}, {\mathfrak s}_1, {\mathfrak s}^\prime_{i+1}, \ldots, {\mathfrak s}_k)),\,\mbox{if}\;\theta({\mathfrak s}^\prime_j)=v_j,\,j<i
\end{array}
$$

$$
\begin{array}{rcl}
\sembr{\Box = e}^{{\mathfrak s}_1{\mathfrak s}_2}_\theta&=&\Box\, \wedge \\
 & & (\sembr{e}_\theta\; {\mathfrak s}^\prime) \wedge \\
 & & ((({\mathfrak s}_1 \equiv {\mathfrak s}^\prime) \wedge ({\mathfrak s}_2 \equiv \lstinline|^true|))\, \vee \\ 
 & & (({\mathfrak s}_1 \not \equiv {\mathfrak s}^\prime) \wedge ({\mathfrak s}_2 \equiv \lstinline|^false|))) 
\end{array}
$$

$$
\begin{array}{rcl}
\sembr{v = \Box}^{{\mathfrak s}_1{\mathfrak s}_2}_\theta&=&\Box\,\wedge \\
 & & ((({\mathfrak s}^\prime \equiv {\mathfrak s}_1) \wedge ({\mathfrak s}_2 \equiv \lstinline|^true|))\, \vee \\ 
 & & (({\mathfrak s}^\prime \not \equiv {\mathfrak s}_1) \wedge ({\mathfrak s}_2 \equiv \lstinline|^false|))),\,\mbox{if}\;\theta({\mathfrak s})=v 
\end{array}
$$

$$
\begin{array}{rcl}
\sembr{\lstinline|match $\;\Box\;$ with \{$C^{n_i}_i$($y^i_1$, ..., $y^i_{n_i}$) -> $\;e_i$\}|}^{{\mathfrak s}_1{\mathfrak s}_2}_\theta&=&\Box \; \wedge \bigvee_i\\
& &(\lstinline|fresh ($s^i_1 \ldots s^i_{n_i}$)| \\
& &\qquad({\mathfrak s}_1 \equiv \;\uparrow C_i^{n_i}(s^i_1, \ldots, s^i_{n_i})) \\
& &\qquad(\lambda y^i_1. \ldots \lambda  y^i_{n_i}. \sembr{e_i}_\theta) \; (\equiv s^i_1) \ldots (\equiv s^i_{n_i})\;{\mathfrak s}_2)
\end{array}
$$

Here we assume ${\mathfrak s}^\prime$ and ${\mathfrak s}^\prime_i$ to be arbitrary semantic variables, $v_i$~--- arbitrary values w.r.t. the original 
functional semantics, $e_i$~--- arbitrary terms of the source language. We also claim, that $\theta$ is
undefined for all mentioned semantic variables, unless the opposite is specified explicitly.

\end{definition}

\begin{definition}
\normalfont For arbitrary substitution $\theta$, arbitrary semantic variable ${\mathfrak s}_m$ and a functional 
term $e$ define a conversion of a stack $\sembr{\bullet}^{e,{\mathfrak s}_m}_\theta$ as follows:

$$
\def\arraystretch{1.5}
\sembr{f_n\dots f_1g_m\dots g_1}^{e,{\mathfrak s}_m}_\theta=\left\{
\begin{array}{lcl}
\sembr{g_m}^{{\mathfrak s}_m{\mathfrak s}_{m-1}}_\theta\dots\sembr{g_1}^{{\mathfrak s}_1{\mathfrak s}_0}_\theta&,&n=0\;\;\mbox{and $e$~--- ground}\\
\sembr{f_n}_\theta\dots\sembr{f_1}_\theta(\Box\,{\mathfrak s}_m)\sembr{g_m}^{{\mathfrak s}_m{\mathfrak s}_{m-1}}_\theta\dots\sembr{g_1}^{{\mathfrak s}_1{\mathfrak s}_0}_\theta&,&\mbox{otherwise}
\end{array}
\right.
$$

Here ${\mathfrak s}_0\dots {\mathfrak s}_{m-1}$ designate arbitrary distinct semantic variables.
\end{definition}

\begin{definition}
\normalfont For arbitrary substitution $\theta$ and arbitrary semantic variable ${\mathfrak s}_m$ define a simulation
conversion $\sembr{\bullet}^{{\mathfrak s}_m}_\theta$ of the source language term as follows:

$$
\begin{array}{rcl}
\sembr{e_1 = e_2}^{{\mathfrak s}_m}_\theta&=& (\sembr{e_1}_\theta\; {\mathfrak s}^\prime_1) \wedge \\
                           & & (\sembr{e_2}_\theta\; {\mathfrak s}^\prime_2) \wedge \\
                           & & ((({\mathfrak s}^\prime_1 \equiv {\mathfrak s}^\prime_2) \wedge ({\mathfrak s}_m \equiv \lstinline|^true|))\, \vee \\ 
                           & & (({\mathfrak s}^\prime_1 \not \equiv {\mathfrak s}^\prime_2) \wedge ({\mathfrak s}_m \equiv \lstinline|^false|)))
\end{array}
$$

$$
\begin{array}{rcl}
\sembr{v = e}^{{\mathfrak s}_m}_\theta&=& (\sembr{e}_\theta\; {\mathfrak s}^\prime_2) \wedge \\
                        & & ((({\mathfrak s}^\prime_1 \equiv {\mathfrak s}^\prime_2) \wedge ({\mathfrak s}_m \equiv \lstinline|^true|))\, \vee \\ 
                        & & (({\mathfrak s}^\prime_1 \not \equiv {\mathfrak s}^\prime_2) \wedge ({\mathfrak s}_m \equiv \lstinline|^false|))),\,\mbox{if}\;\theta({\mathfrak s}^\prime_1)=v
\end{array}
$$

$$
\begin{array}{rcl}
\sembr{v_1 = v_2}^{{\mathfrak s}_m}_\theta&=& ((({\mathfrak s}^\prime_1 \equiv {\mathfrak s}^\prime_2) \wedge ({\mathfrak s}_m \equiv \lstinline|^true|))\, \vee \\ 
                           & & (({\mathfrak s}^\prime_1 \not \equiv {\mathfrak s}^\prime_2) \wedge ({\mathfrak s}_m \equiv \lstinline|^false|))),\,\mbox{if}\;\theta({\mathfrak s}^\prime_j)=v_j
\end{array}
$$

$$ 
\begin{array}{rcl}
\sembr{C^k(v_1, \ldots, v_{i-1}, e_i, \ldots, e_k)}^{{\mathfrak s}_m}_\theta&=&(\sembr{e_i}_\theta \; {\mathfrak s}^\prime_i) \; \wedge \\
       & & \ldots  \\
       & & (\sembr{e_k}_\theta \; {\mathfrak s}^\prime_k) \; \wedge \\
       & & ({\mathfrak s}_m \equiv\; \uparrow C^k({\mathfrak s}^\prime_1, \ldots, {\mathfrak s}^\prime_k)),\,\mbox{if}\;\theta({\mathfrak s}^\prime_j)=v_j,\,j<i
\end{array}
$$

$$ 
\sembr{C^k(v_1, \ldots, v_k)}^{{\mathfrak s}_m}_\theta = ({\mathfrak s}_m \equiv\; \uparrow C^k({\mathfrak s}^\prime_1, \ldots, {\mathfrak s}^\prime_k)),\,\mbox{if}\;\theta({\mathfrak s}^\prime_j)=v_j
$$

$$ 
\sembr{C^k(v_1, \ldots, v_k)}^{{\mathfrak s}_m}_\theta = ({\mathfrak s}_m \equiv\; {\mathfrak s}^\prime),\;\mbox{if}\;\theta({\mathfrak s}^\prime)=C^k(v_1, \ldots, v_k)
$$

$$
\begin{array}{rcl}
\sembr{\lstinline|match $\;e\;$ with \{$C^{n_i}_i$($y^i_1$, ..., $y^i_{n_i}$) -> $\;e_i$\}|}^{{\mathfrak s}_m}_\theta&=&\sembr{e}_\theta\;{\mathfrak s}^\prime\;\wedge\;\bigvee_i\\
& &(\lstinline|fresh ($s^i_1 \ldots s^i_{n_i}$)| \\
& &\qquad({\mathfrak s}^\prime \equiv \;\uparrow C_i^{n_i}(s^i_1, \ldots, s^i_{n_i})) \\
& &\qquad(\lambda y^i_1. \ldots \lambda  y^i_{n_i}. \sembr{e_i}_\theta) \; (\equiv s^i_1) \ldots (\equiv s^i_{n_i})\;{\mathfrak s}_m)
\end{array}
$$

$$
\begin{array}{rcl}
\sembr{\lstinline|match $\;v\;$ with \{$C^{n_i}_i$($y^i_1$, ..., $y^i_{n_i}$) -> $\;e_i$\}|}^{{\mathfrak s}_m}_\theta&=&\bigvee_i\\
& &(\lstinline|fresh ($s^i_1 \ldots s^i_{n_i}$)| \\
& &\qquad({\mathfrak s}^\prime \equiv \;\uparrow C_i^{n_i}(s^i_1, \ldots, s^i_{n_i})) \\
& &\qquad(\lambda y^i_1. \ldots \lambda  y^i_{n_i}. \sembr{e_i}_\theta) \; (\equiv s^i_1) \ldots (\equiv s^i_{n_i})\;{\mathfrak s}_m),\,\mbox{if}\;\theta({\mathfrak s}^\prime)=v
\end{array}
$$

Here all ${\mathfrak s}^\prime$ and ${\mathfrak s}^\prime_i$ designate arbitrary semantic variables, $e$~--- arbitrary term, $v$~--- arbitrary value w.r.t. the
original semantics. We also claim, that $\theta$ is undefined for all mentioned semantic variables, unless the opposite is specified explicitly.
\end{definition}

\begin{definition}
\normalfont Let 
\begin{itemize}
\item \mbox{$\left<\mathcal S,\,e\right>$}~--- a state w.r.t. the deferred semantics;
\item \mbox{$\left<\Sigma, \hat{\mathcal S}, \hat{e}, (\theta, \emptyset)\right>$}~--- a state w.r.t. the
relational semantics.
\end{itemize} 

We say, that these states are connected, if there exists a semantic variable $q_m$, such, that:\vspace{1mm}

\begin{enumerate}
\item \mbox{$\hat{\mathcal S}\in\sembr{\mathcal S}^{e,{\mathfrak s}_m}_\theta$}\vspace{1mm}
\item \mbox{$\hat{e}\in\left\{
                          \begin{array}{lcl}
                            \sembr{e}^{{\mathfrak s}_m}_\theta&,&e\mbox{~--- ground and }\mathcal S\mbox{ does not contain functional contexts}\\[1mm]
                            \sembr{e}_\theta&,&\mbox{otherwise}
                          \end{array}
                       \right.
            $} 
\item $\Sigma$ contains all semantic variables from $\hat{e}$, $\hat{\mathcal S}$, and $\theta$.
\end{enumerate}

\end{definition}

\begin{lemma}
\label{constructor}
\normalfont Let $v=\lstinline|C$^k$($v_1$,...,$v_k$)|$ be a value. Then
for arbitrary $\Sigma$, $\mathcal S$, $\theta$, $\hat{v}\in \sembr{v}_\theta$, and 
semantic variable ${\mathfrak s}$, such, that ${\mathfrak s}\not\in dom(\theta)$ either

$$
\left<\Sigma,\,\mathcal S, (\hat{v}\,{\mathfrak s}),\, (\theta,\,\emptyset)\right>\leadsto^*\left<\Sigma^\prime,\,\mathcal S,\,{\mathfrak s}\equiv\lstinline|C$^k$(${\mathfrak s}^\prime_1$,...,${\mathfrak s}^\prime_k$)|,\,(\theta^\prime,\,\emptyset)\right>\;\mbox{and}\;\theta^\prime({\mathfrak s}^\prime_i)=v_i
$$

or

$$
\left<\Sigma,\,\mathcal S, (\hat{v}\,{\mathfrak s}),\, (\theta,\,\emptyset)\right>\leadsto\left<\Sigma,\,\mathcal S,\,{\mathfrak s}\equiv {\mathfrak s}^\prime,\,(\theta,\,\emptyset)\right>\;\mbox{and}\;\theta({\mathfrak s}^\prime)=v
$$
 
The proof is by induction on the height of $v$.
\end{lemma}

\begin{lemma}
\label{evaluation_lemma}
\normalfont Let $s=\left<\mathcal S=g_m\dots g_1,\,e\right>$ be a state w.r.t. the deferred semantics, 
$g_i$~--- ground contexts, $e$~--- expression of a ground type, $\theta$~--- some substitution,
${\mathfrak s}_m$~--- some semantic variable, \mbox{$\hat{\mathcal{S}}\in\sembr{\mathcal S}^{e,\,{\mathfrak s}_m}_\theta$}, 
\mbox{$\hat{e} \in \sembr{e}_\theta$}. Then there is a sequence of steps w.r.t. the relational
semantics, such, that

$$
\left<\Sigma, \hat{\mathcal S}, (\hat{e} \, {\mathfrak s}_m), (\theta,\,\emptyset) \right>\leadsto^*\hat{s}
$$

\noindent and $s$ and $\hat{s}$ are connected. Here we assume $\Sigma$ to contain all semantic variables from
$\hat{\mathcal S}$ and $\theta$. The proof is by case analysis on $e$, using Lemma~\ref{constructor}.
\end{lemma}

\begin{lemma} 
\label{connection}
\normalfont Let \mbox{$s_1 \to s_2$}~--- a single evaluation step w.r.t. the deferred semantics,
$\hat{s_1}$~--- a state of the relational semantics, such, that $s_1$ and $\hat{s_1}$ are connected. Then
there exists a sequence of steps in the relational semantics \mbox{$\hat{s_1}\leadsto^*\hat{s_2}$}, such, 
that $s_2$ and $\hat{s_2}$ are connected. The proof is by case analysis and definition of connection
relation, using Lemmas~\ref{substitution},~\ref{constructor},~\ref{evaluation_lemma}. 
\end{lemma}

\begin{lemma}
\label{prefix}
\normalfont Let $s_0=\left<\emptyset,\,\epsilon,\,\lstinline|fresh ($x$) $(\sembr{e}^c\;x)$|,\,\iota\right>$ be an
initial state of evaluation w.r.t. the relational semantics. Then there is a sequence of steps
\mbox{$s_0\leadsto^*\hat{s}$}, such, that \mbox{$\left<\epsilon,\,e\right>$} (an initial state of
evaluation of $e$ w.r.t. the deferred semantics) and $\hat{s}$ are connected. Immediately follows from
Lemma~\ref{evaluation_lemma}.
\end{lemma}

Now we can prove the partial correctness theorem. Let us have a term $e$ of a ground type in the source language, which
reduces to a value $v=\lstinline|C$^k$($v_1$,...,$v_k$)|$ w.r.t. the original call-by-value semantics. Then it reduces to the same value w.r.t. the
deferred semantics: 

$$
\left<\epsilon,\,e\right>\to^*\left<\epsilon,\,v\right>
$$

By Lemma~\ref{prefix} 

$$
\left<\emptyset,\,\epsilon,\lstinline|fresh ($x$) $(\sembr{e}^c\;x)$|,\iota\right>\leadsto^*\hat{s}
$$

\noindent where \mbox{$\left<\epsilon,\,e\right>$} and $\hat{s}$ are connected. By Lemma~\ref{connection}, there is
a state $\hat{s^\prime}$ w.r.t. the relational semantics, such, that

$$
\hat{s}\leadsto^*\hat{s^\prime}
$$

\noindent and \mbox{$\left<\epsilon,\,v\right>$} and $\hat{s^\prime}$ are connected. By the definition of
the connection relation, $\hat{s^\prime}$ has one of the following forms:

$$
\left<\Sigma,\,\epsilon,\,{\mathfrak s}_0\equiv\lstinline|C$^k$(${\mathfrak s}^\prime_1$,...,${\mathfrak s}^\prime_k$)|,\,(\theta,\,\emptyset)\right>,\,\theta({\mathfrak s}^\prime_i)=v_i
$$

\noindent or

$$
\left<\Sigma,\,\epsilon,\,{\mathfrak s}_0\equiv {\mathfrak s}^\prime,\,(\theta,\,\emptyset)\right>,\,\theta({\mathfrak s}^\prime)=v
$$

\noindent where ${\mathfrak s}_0$ is the first semantic variable, added to $\Sigma$, and \mbox{${\mathfrak s}_0\not\in dom(\theta)$}. In
both cases, we can make the one last step in the relational semantics, which completes the proof. 
