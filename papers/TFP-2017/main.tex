\documentclass{llncs}

\usepackage{makeidx}
\usepackage{amssymb}
\usepackage{listings}
\usepackage{indentfirst}
\usepackage{verbatim}
\usepackage{amsmath, amssymb}
\usepackage{graphicx}
\usepackage{xcolor}
\usepackage{url}
\usepackage{stmaryrd}
\usepackage{xspace}
\usepackage{comment}
\usepackage{wrapfig}
\usepackage{placeins}
\usepackage{tabularx}
\usepackage{ragged2e}


\usepackage [sorting = none] {biblatex}

\addbibresource {main.bib}

\def\transarrow{\xrightarrow}
\newcommand{\setarrow}[1]{\def\transarrow{#1}}

\newcommand{\trule}[2]{\frac{#1}{#2}}
\newcommand{\crule}[3]{\frac{#1}{#2},\;{#3}}
\newcommand{\withenv}[2]{{#1}\vdash{#2}}
\newcommand{\trans}[3]{{#1}\transarrow{#2}{#3}}
\newcommand{\ctrans}[4]{{#1}\transarrow{#2}{#3},\;{#4}}
\newcommand{\llang}[1]{\mbox{\lstinline[mathescape]|#1|}}
\newcommand{\pair}[2]{\inbr{{#1}\mid{#2}}}
\newcommand{\inbr}[1]{\left<{#1}\right>}
\newcommand{\highlight}[1]{\color{red}{#1}}
\newcommand{\ruleno}[1]{\eqno[\scriptsize\textsc{#1}]}
\newcommand{\inmath}[1]{\mbox{$#1$}}
\newcommand{\lfp}[1]{fix_{#1}}
\newcommand{\gfp}[1]{Fix_{#1}}
\newcommand{\vsep}{\vspace{-2mm}}
\newcommand{\supp}[1]{\scriptsize{#1}}
\newcommand{\G}{\mathfrak G}
\newcommand{\sembr}[1]{\llbracket{#1}\rrbracket}
\newcommand{\cd}[1]{\texttt{#1}}
\newcommand{\miniKanren}{miniKanren\xspace}
\newcommand{\ocanren}{OCanren\xspace}
\newcommand{\free}[1]{\boxed{#1}}
\newcommand{\binds}{\;\mapsto\;}
\newcommand{\dbi}[1]{\mbox{\bf{#1}}}

\let\emptyset\varnothing

\lstdefinelanguage{ocanren}{
keywords={fresh, let, in, match, with, when, class, type,
object, method, of, rec, repeat, until, while, not, do, done, as, val, inherit,
new, module, sig, deriving, datatype, struct, if, then, else, open, private, virtual, include, success, failure,
true, false},
sensitive=true,
commentstyle=\small\itshape\ttfamily,
keywordstyle=\ttfamily\underbar,
identifierstyle=\ttfamily,
basewidth={0.5em,0.5em},
columns=fixed,
fontadjust=true,
literate={fun}{{$\lambda$}}1 {->}{{$\to$}}3 {===}{{$\equiv$}}1 {=/=}{{$\not\equiv$}}1 {|>}{{$\triangleright$}}3 {|||}{{$\vee$}}2 {/\\}{{$\wedge$}}2 {^}{{$\uparrow$}}1,
morecomment=[s]{(*}{*)}
}

\lstset{
mathescape=true,
%basicstyle=\small,
identifierstyle=\ttfamily,
keywordstyle=\bfseries,
commentstyle=\scriptsize\rmfamily,
basewidth={0.5em,0.5em},
fontadjust=true,
language=ocanren
}

\usepackage{letltxmacro}
\newcommand*{\SavedLstInline}{}
\LetLtxMacro\SavedLstInline\lstinline
\DeclareRobustCommand*{\lstinline}{%
  \ifmmode
    \let\SavedBGroup\bgroup
    \def\bgroup{%
      \let\bgroup\SavedBGroup
      \hbox\bgroup
    }%
  \fi
  \SavedLstInline
}
\addtolength{\parskip}{-2pt}

\begin{document}

\mainmatter

\title{Typed Relational Conversion}

\author{
  Petr Lozov \and Andrei Vyatkin \and Dmitry Boulytchev
}

\institute{
St.Petersburg State University\\
Universitetski pr., 28, 198504, St.Petersburg, Russia\\
JetBrains Research\\
Universitetskaya emb., 7-9-11, bldg. 5A, 199034, St.Petersburg, Russia}

%\email{lozov.peter@gmail.com}
%\and
%St.Petersburg State University\\
%Universitetski pr., 28, 198504, St.Petersburg, Russia\\
%\email{dewshick@gmail.com}
%\and
%St.Petersburg State University\\
%Universitetski pr., 28, 198504, St.Petersburg, Russia\\
%JetBrains Research\\
%\email{dboulytchev@math.spbu.ru}
%}

\maketitle

\begin{abstract}
We address the problem of transforming typed functional programs into relational form. 
In this form a program can be run in various ``directions'' with some arguments left free, 
making it possible to acquire different behaviors from a single specification. We specify the 
syntax, typing rules and semantics for the source language as well as its relational extension, 
describe the conversion and prove its correctness both in terms of typing and dynamic semantics. 
We also discuss the limitations of our approach, present the implementation of the conversion for 
the subset of OCaml and evaluate it on a number of realistic examples.
\end{abstract}

\section{Introduction}
\label{intro}

Relational programming is an attractive technique, based on the idea of constructing programs as relations.
While in general some relational effects can be reproduced with a number of languages for logic programming, such as
Prolog, Mercury\footnote{\url{https://mercurylang.org}}, or Curry\footnote{\url{http://www-ps.informatik.uni-kiel.de/currywiki}}, in
a narrow sense relational programming amounts to writing relational specifications in \miniKanren~\cite{TRS}. \miniKanren\footnote{\url{http://minikanren.org}},
initially designed as a small relational DSL, embedded in Scheme/Racket, was later implemented for a number of general-purpose host languages,
including Scala, Haskell, Standard ML and OCaml.

With the relational approach, it becomes possible to give simple and elegant solutions for the problems, otherwise
considered as tricky, tough, tedious, or boring~\cite{unified}. For example, relational interpreters can be used to derive
\emph{quines}~--- programs, which reduce to itself, as well as \emph{twines} or \emph{thrines} (pairs or triples of
programs, reducing to each other)~\cite{Untagged}; a straightforward relational description of
simply typed lambda calculus~\cite{Lambda} inference rules works both as type inferencer and inhabitation problem solver~\cite{WillThesis};
relational list sorting can be used to generate all permutations~\cite{ocanren}, etc. 

On the other hand, writing relational specifications can sometimes be a tricky and error-prone task. Fortunately, many 
specifications can be written systematically by ``generalizing'' a certain functional program. From the very beginning, 
the conversion from functional to relational form was considered as an element of relational programming thesaurus~\cite{TRS}. However,
the traditional approach~--- \emph{unnesting}~--- was formulated for an untyped case, worked only for specifically written
programs and was never implemented.

We present a generalized form of relational conversion, which can be applied to typed terms in general form. We study the relational conversion 
for a small ML-like language (essentially, a certain subset of OCaml), equipped with Hindley-Milner type system with let-polymorphism~\cite{Types}. 
We start from retelling the syntax, typing rules, and operational semantics, and then extend the source language with a conventional set of 
relational constructs. This set corresponds to existing typed embedding of \miniKanren into OCaml~\cite{ocanren}. We then present typing rules and 
develop operational semantics for this relational extension; to our knowledge, this is the first attempt to specify formal semantics for
\miniKanren. Next, we develop formal rules for relational conversion and prove, that these rules respect both typing and
semantics. Finally, we describe the implementation of a relational converter and demonstrate its application for a number of problems, for some
of which we present a relational solution for the first time.

\section{Relational Programming in \miniKanren}
\label{ocanren}

In the context of this paper, we will use a certain concrete implementation of \miniKanren~--- a shallow DSL for 
OCaml\footnote{https://github.com/dboulytchev/ocanren}, called \ocanren~\cite{ocanren}. \ocanren corresponds to \miniKanren with
disequality constraints~\cite{CKanren}, and (modulo typing) follows the original implementation~\cite{MicroKanren,SmallEmbedding}. Here we describe the external view 
on \ocanren, giving the only intuitive meaning of its constructs; the formal description will be presented in Section~\ref{relational_extension}.
We also use a simplified syntax, which is a little bit different from the concrete syntax in actual implementation, but assumed to
be easier to read.

The central notion of \miniKanren is \emph{goal}; in \ocanren a goal can be an arbitrary expression of reserved goal type, which we denote $\G$.
There are only five syntactic forms of goals (denoted below as $g, g_1, g_2$, etc.):

\begin{itemize}
  \item conjunction $g_1\wedge g_2$;
  \item disjunction $g_1\vee g_2$;
  \item fresh variable introduction $\lstinline|fresh ($x$) $\;g$|$;
  \item unification $t_1\;\equiv\;t_2$;
  \item disequality constraint $t_1\;\not\equiv\;t_2$.
\end{itemize}

Two last forms of goals constitute a basis for goal construction; here $t_1$ and $t_2$ are \emph{terms}. In \ocanren a term is an arbitrary expression of polymorphic logic type $\alpha^o$. The postfix notation $\Box^o$ is a traditional way to denote relational entities, and we will use it for types as well\footnote{In the real implementation the terms have a more complex two-parametric type, which encodes a tagging, needed to be performed when the results of the relational program are returned into the functional word; these details, however, are irrelevant to the objectives of the paper, and we stick with the simplified version.}.

The simplest expression of logic type is a variable, bound in \lstinline|fresh|. Another example is a primitive value, \emph{injected} into the logic domain with a built-in primitive ``$\uparrow$'', such as $\uparrow\!3$ (of type \lstinline|int$^o$|) or \lstinline|$\uparrow$true| (of type \lstinline|bool$^o$|). With these simplest forms some relational programs can already be written:

\begin{lstlisting}
   val is_zero$^o$ : int$^o$ -> bool$^o$ -> $\G$
   let is_zero$^o$ n b = 
     (n === ^0 /\ b === ^true) |||
     (n =/= ^0 /\ b === ^false)
\end{lstlisting}

The function \lstinline|is_zero$^o$| implements a binary relation between integers and booleans; when called with specific arguments, it returns a goal, which can be executed or combined with another goal. In the former case, a stream of \emph{answers} is returned. An element of the stream contains the description of certain constraints for logical variables, which have to be respected in order for the relation to hold. We denote running primitive ``$\leadsto$'', so

\begin{lstlisting}
   iz_zero$^o$ ^0 ^true $\leadsto$ [()]
\end{lstlisting}

\noindent returns a stream with one ``empty'' element since no additional constraints are discovered. 
Running 

\begin{lstlisting}
   iz_zero$^o$ q ^true $\leadsto$ [q$\binds$0]
\end{lstlisting}

\noindent (where \lstinline|q| is a fresh logical variable) returns a stream, which represents the single expected answer. The goal

\begin{lstlisting}
   iz_zero$^o$ q ^false $\leadsto$ [q$\binds\free{0}$ (=/= 0)]
\end{lstlisting}

\noindent provides a stream, which can be interpreted as ``\lstinline|q| can be an arbitrary integer as long as it is different from zero'' ($\free{0}$, $\free{1}$, etc. denote free logical variables). Finally, the goal

\begin{lstlisting}
   iz_zero$^o$ q p $\leadsto$ [(q$\binds$0, p$\binds$true); (q$\binds\free{0}$ (=/= 0), p$\binds$false)]
\end{lstlisting}

\noindent returns a stream with two elements, which represents all possible pairs in the relation.

Other types (pairs, lists, user-defined algebraic datatypes, etc.) can be used in relational specifications as well, being injected by the same primitive. For example, expression \lstinline{^(1, "abc")} has type \lstinline{(int * string)$^o$}, \lstinline{^[1; 2; 3]}~--- type \lstinline{(int list)$^o$}, etc. The subtle part is that (since the unification only works for logical types) the placement of ``$^o$'' determines the granularity of unification. Indeed, a logical variable can only be placed where logical type is expected. Thus, in unification one can use a value of type \lstinline{(int * int)$^o$} as \emph{a whole}, but in order to control the \emph{contents} of the pair relationally the type \lstinline{(int$^o$, int$^o$)$^o$} is required. This makes it impossible to reuse some built-in or standard types in relational code~--- for example, predefined list type is not flexible enough, since it does not allow the tail of the list to be logical. Instead, logical list type has to be introduced:

\begin{lstlisting}
   type $\alpha$ llist = Nil | Cons of $\alpha^o$ * ($\alpha$ llist)$^o$
\end{lstlisting}

With logical list type, we can implement some relations for lists:

\begin{lstlisting}
   val append : ($\alpha$ llist)$^o$ -> ($\alpha$ llist)$^o$ -> ($\alpha$ llist)$^o$ -> $\G$
   let rec append$^o$ x y xy =
     (x === ^Nil /\ xy === y) |||
     (fresh (h t ty)
        x  === ^(Cons (h, t)) /\
        xy === ^(Cons (h, ty)) /\
        append$^o$ t y ty
     ) 
\end{lstlisting}

Here we defined relational list concatenation \lstinline{append$^o$}, a canonical example in the field. This definition takes three logical lists \lstinline{x}, \lstinline{y} and \lstinline{xy} as arguments, and constructs a goal, which, being run, finds all bindings for free variables in the lists (if any), trying to satisfy the desired relation. The search is described in terms of case analysis and recursion:

\begin{enumerate}
\item If the first list is empty, then the second and the third lists must be equal.
\item Otherwise, the first list can be split into a head and a tail, and two fresh variables \lstinline{h} and \lstinline{t} are needed to denote them. We also need a fresh variable \lstinline{ty} to denote the list, such that appending \lstinline{y} to \lstinline{t} equals \lstinline{ty}. To ensure this property we use a recursive call to \lstinline{append$^o$}. Finally, we acquire the final result by consing \lstinline{h} and \lstinline{ty}. 
\end{enumerate}

As it can be seen from the type, relational concatenation is polymorphic, like its functional counterpart. However, the query

\begin{lstlisting}
   append$^o$ ^(Cons (^$\lambda$x.x, ^Nil)) q ^(Cons (^$\lambda$y.y, ^Nil))  
\end{lstlisting}

\noindent ends with a run-time error due to inability to unify closures. This is a fundamental limitation in original \miniKanren as well, as it deals only with first-order syntactic unification~\cite{Unification}. This example demonstrates, that, unlike pure OCaml, the typing in OCanren is somewhat weak. In order to restore the strong typing, some of the type variables have to be bounded to range over only non-functional types. The lack of direct support for bounded polymorphism~\cite{cardelli} in OCaml makes this step problematic. Our experience, however, shows, that in practice this deficiency rarely gets in the way. In the following development, we assume, that in polymorphic types some type variables may be implicitly bounded to only non-function types, and these boundings are respected in all instantiations of those type variables.

Finally, we describe the unnesting technique~\cite{TRS}, which was introduced as a method for manual transformation
of functional programs into relational form. As an example, we take list concatenation:

\begin{lstlisting}
   let rec append x y =
     match x with
     | Nil -> y
     | Cons (h, t) -> Cons (h, append t y)
\end{lstlisting}

Unnesting introduces a new name for each nested function call; in our case, there is only one nested call, namely,
the recursive call to \lstinline|append| itself, so the unnested function looks like

\begin{lstlisting}
   let rec append x y =
     match x with 
     | Nil -> y
     | Cons (h, t) -> 
        let ty = append t y in
        Cons (h, ty)
\end{lstlisting}

Now the conversion is straightforward: the pattern-matching construct is transformed into a disjunction, and new names,
introduced in pattern bindings and unnestings, are transformed into \lstinline|fresh| variables:

\begin{lstlisting}
   let rec append$^o$ x y xy =
     (t === ^Nil /\ xy === y) |||
     (fresh (h t ty)
        (x  === ^Cons (h, t)) /\
        (xy === ^Cons (h, ty)) /\
        (append$^o$ t y ty)
     )
\end{lstlisting}

However, not every definition can be converted to a relational form by unnesting. Consider, for example, the following definition:

\begin{lstlisting}
   let bar y =
     let f x = x in
     let g a = f in
     g A y
\end{lstlisting}

Unnesting would transform this program into the following form

\begin{lstlisting}
   let bar$^o$ y r =
     let f x r = x === r in
     let g a r = f === r in
     g ^A y r
\end{lstlisting}

\noindent which is obviously invalid since it unifies a function $f$ with a logical variable $r$. In order to apply unnesting, one
needs to $\eta$-expand the definition of $g$, making the functional nature of its return type syntactically visible.

We stress, that relational conversion, described in Section~\ref{conversion}, is essentially different from unnesting. In particular, 
we use $\eta$-expansion in a very limited manner (only in one case) and for the aforementioned example the result of relational
conversion looks as follows:

\begin{lstlisting}
   let bar$^o$ y =
     let f x = x in
     let g a = f in
     g (fun q. q === ^A) y
\end{lstlisting}

Note, the majority of definitions are left intact; the only difference with the functional version comes from the use of the constructor 
\lstinline|A|, which was transformed into a goal-returning function.




\section{The Syntax and Semantics of Relational Language}
\label{language}

In this section we describe the syntax and semantics of the language, which is used in the rest of the paper. To some extent this description serves as
a short introduction to miniKanren. The main distinction between ``the real'' miniKanren and our version is that we give a proper semantics only to converging programs, 
which deliver a finite set of answers, while in the reality of relational programming the result is represented as an infinite stream, from which any number of answers can be requested, and the request of a non-existing answer can
lead to a divergence. Our semantics, thus, corresponds to the scenario, when \emph{all} answers are requested from the stream. On the other hand, we do not distinguish programs, calculating the infinite number 
of answers, from those diverging with no results at all.   
However, we consider the finite version of the semantics as an important case, which is justified by the evaluation, presented in Section~\ref{evaluation}.


\begin{figure}[t]
$$
\begin{array}{rcll}
\meta{C}    & = & \{C^k,\dots\}                                        & \;\;\mbox{\emph{(constructors)}}\\
\meta{T}(X) & = & x\in X \mid C^k\,(t_1,\dots,t_k),\,t_i\in \meta{T}(X) & \;\;\mbox{\emph{(terms)}}\\
\meta{V}    & = & \{x, y, z, \dots\}                                   & \;\;\mbox{\emph{(syntactic variables)}}\\
\meta{T_V}  & = & \meta{T}(\meta{V})                                   & \;\;\mbox{\emph{(syntactic terms)}}\\
\meta{R}    & = & \{r^k,\dots\}                                        & \;\;\mbox{\emph{(relational symbols)}}\\
\meta{G}    & = & t_1\equiv t_2,\,t_i\in\meta{T_V}                      & \;\;\mbox{\emph{(unification)}}\\ 
            &   & g_1\wedge g_2                                        & \;\;\mbox{\emph{(conjunction)}}\\
            &   & g_1\vee g_2                                          & \;\;\mbox{\emph{(disjunction)}}\\
            &   &\lstinline|fresh|\,(x)\;g                             & \;\;\mbox{\emph{(fresh variable introduction)}}\\
            &   &r^k\;t_1\dots t_k,\,t_i\in\meta{T_V}                   & \;\;\mbox{\emph{(relational reference)}}\\
\meta{D}    & = & r^k\binds\lambda x_1\dots x_k\,.\,g,\,x_i\in\meta{V}  & \;\;\mbox{\emph{(relational definition)}}\\
\meta{S}    & = & d_1,\dots,\,d_k; g                                    & \;\;\mbox{\emph{(specification)}}
\end{array}
$$
\caption{The syntax of source language}
\label{syntax}
\end{figure}

The syntax of our relational language is shown on Fig.~\ref{syntax}. First, we introduce the alphabet of constructors $\meta{C}$, each of which is equipped with a 
non-negative arity. Then we in a conventional fashion inductively define the set of all terms $\meta{T}(X)$, parameterized by the set of variables $X$. We need this parameterization 
since later we will be dealing with two sorts of variables~--- \emph{syntactic} and \emph{semantic}, and, therefore, two sorts of terms. Next, we choose the set of syntactic variables
$\meta{V}$ and the set of \emph{relational symbols} $\meta{R}$ with arities, which will be used as names for relational definitions. We also introduce a shortcut
$\meta{T_V}$ for the set of all terms over syntactic variables since it will be used in all other syntactic definitions.

The core syntax category in the language is a \emph{goal}. There are five types of goals: unification of two terms, conjunction and disjunction of two
goals, fresh variable introduction and a call of some relational definition. We stipulate, that the calls of relational definitions respect their arities; we 
will also use a shortcut form \lstinline|fresh ($x$ $y$ $z$ ...) ...| instead of \lstinline|fresh($x$) (fresh ($y$) (fresh ($z$) ...)| where needed.

Next, \emph{relational definition} $\meta{D}$ binds some relational symbol to a parameterized goal; the number of parameters corresponds to the arity of the symbol, and
we assume, that all parameter variables are pairwise distinct. Finally, the top-level syntax category is \emph{specification} $\meta{S}$~--- a goal in the context 
of some relational definitions. 

Note, we define here a language with first-order relations; in particular, we do not allow partial application. As we see later, our approach critically depends
on recursive call identification, which is a trivial task in the first-order case. Some existing frameworks for relational programming~\cite{OCanren,RelConversion}
do not impose such a limitation; extending our approach for the higher-order case is a subject of future research.

\setarrow{\xRightarrow}
\newcommand{\otrans}[4]{\withenv{#1}{\trans{#2}{\mbox{$#3$}}{#4}}}
\newcommand{\cotrans}[5]{\withenv{#1}{\ctrans{#2}{\mbox{$#3$}}{#4}{#5}}}
 
\begin{figure}[t]
$$
\cotrans{\Gamma,\,\iota}{(\sigma,\,\delta)}{t_1\equiv t_2}{\emptyset}{mgu\,(t_1\iota\sigma,\,t_2\iota\sigma) = \bot}\ruleno{UnifyFail}
$$

$$
\cotrans{\Gamma,\,\iota}{(\sigma,\,\delta)}{t_1\equiv t_2}{(\sigma\circ\Delta,\,\delta)}{mgu\,(t_1\iota\sigma,\,t_2\iota\sigma) = \Delta\ne\bot}\ruleno{UnifySuccess}
$$

$$
\trule{\otrans{\Gamma,\,\iota}{(\sigma,\,\delta)}{g_1}{S_1},\;\;\;\;
       \otrans{\Gamma,\,\iota}{(\sigma,\,\delta)}{g_2}{S_2}
      }
      {\otrans{\Gamma,\,\iota}{(\sigma,\,\delta)}{g_1\vee g_2}{S_1\cup S_2}}\ruleno{Disj}
$$

$$
\trule{\otrans{\Gamma,\,\iota}{(\sigma,\,\delta)}{g_1}{\{(\sigma_i,\,\delta_i)\}},\;\;\;\;
       \otrans{\Gamma,\,\iota}{(\sigma_i,\,\delta_i)}{g_2}{S_i}
      }
      {\otrans{\Gamma,\,\iota}{(\sigma,\,\delta)}{g_1\wedge g_2}{\bigcup_i S_i}}\ruleno{Conj}
$$

$$
\crule{\otrans{\Gamma,\,\iota[x\gets\alpha]}{(\sigma,\,\delta\cup\{\alpha\})}{g}{S}}
      {\otrans{\Gamma,\,\iota}{(\sigma,\,\delta)}{\lstinline|fresh($x$) $\;g$|}{S}}
      {\alpha\in\meta{W}\setminus\delta}\ruleno{Fresh}
$$

$$
\crule{\otrans{\Gamma,\,[x_i\gets v_i]}{(\epsilon,\,\delta)}{g}{\{(\sigma_i,\,\delta_i)\}}}
      {\otrans{\Gamma,\,\iota}{(\sigma,\,\delta)}{r^k\,t_1\dots t_k}{\{(\sigma\circ\sigma_i,\,\delta_i)\}}}
      {\Gamma(r^k)=\lambda x_1\dots x_k.g,\,v_i=t_i\iota\sigma}\ruleno{Invoke}
$$
\caption{Big-step operational semantics}
\label{semantics}
\end{figure}

\begin{figure}[t]
\arraycolsep=5pt
\def\arraystretch{2.2}
\subfloat[\label{appendo_eval_a}]{
$
\begin{array}{c|c}
  \multicolumn{2}{c}{\otrans{\bot}{(\epsilon,\,\emptyset)}{\lstinline|fresh ($q$)|\;...}{...}}\\
  \hline
  \multicolumn{2}{c}{\otrans{[\bnd{q}{\sv{0}}]}{(\epsilon,\,\{\sv{0}\})}{\lstinline|append$^o$ (Cons (A, Nil)) Nil $\;q$|}{...}}\\
  \hline
  \multicolumn{2}{c}{\otrans{[\bnd{x}{\lstinline|Cons(A, Nil)|},\,\bnd{y}{\lstinline|Nil|},\,\bnd{xy}{\sv{0}}]}{(\dots)}{\lstinline|(...) $\vee$ (...)|}{...}}\\
  \hline
  \otrans{\dots}{(\dots)}{\lstinline|($x\;$ === $\;$Nil) $\wedge$ (...)|}{\emptyset} & \multirow{2}*{Fig.~\ref{appendo_eval_b}}\\
  \cline{1-1}
  \otrans{\dots}{(\dots)}{\lstinline|$x\;$ === $\;$Nil|}{\emptyset} & 
\end{array}
$
}\\
\subfloat[\label{appendo_eval_b}]{
$
\begin{array}{c|c}
\multicolumn{2}{c}{\otrans{[\bnd{x}{\lstinline|Cons(A, Nil)|},\,\bnd{y}{\lstinline|Nil|},\,\bnd{xy}{\sv{0}}]}{(\epsilon,\,\{\sv{0}\})}{\lstinline|fresh($h\;t\;ty$) ...|}{...}}\\
\hline
\multicolumn{2}{c}{\otrans{[\dots,\,\bnd{h}{\sv{1}},\,\bnd{t}{\sv{2}},\,\bnd{ty}{\sv{3}}]}{(\epsilon,\,\{\sv{0}..\sv{3}\})}{\lstinline|($x\;$ === $\;$Cons ($h$, $\;t$)) $\wedge$ (...)|}{...}}\\
\hline
\otrans{\dots}{(\epsilon,\,\{\sv{0}..\sv{3}\})}{\lstinline|$x\;$ === $\;$Cons ($h$, $\;t$)|}{\{([\bnd{\sv{1}}{\lstinline|A|},\,\bnd{\sv{2}}{\lstinline|Nil|}],\,\{\sv{0}..\sv{3}\})\}} & \mbox{Fig.~\ref{appendo_eval_c}}
\end{array}
$
}\\
\subfloat[\label{appendo_eval_c}]{
$
\begin{array}{c|c|c}
\multicolumn{3}{c}{\otrans{\dots}{\{([\bnd{\sv{1}}{\lstinline|A|},\,\bnd{\sv{2}}{\lstinline|Nil|}],\,\{\sv{0}..\sv{3}\})\}}{\lstinline|(append$^o$ $\;t\;$ $y\;$ $ty$) $\wedge$ (...)|}{...}}
\\
\hline
\multicolumn{2}{c|}{\otrans{\dots}{(\dots)}{\lstinline|append$^o$ $\;t\;$ $y\;$ $ty$|}{...}} & 
\multirow{4}*{Fig.~\ref{appendo_eval_d}} \\
\cline{1-2}
\multicolumn{2}{c|}{\otrans{[\bnd{x}{\lstinline|Nil|},\,\bnd{y}{\lstinline|Nil|},\,\bnd{xy}{\sv{3}}]}{(\epsilon,\,\{\sv{0}..\sv{3}\})}{\lstinline|(...) $\vee$ (...)|}{...}} & \\
\cline{1-2}
\multicolumn{2}{c|}{\otrans{\dots}{(\dots)}{\lstinline|($x\;$ === $\;$Nil) $\wedge$ ($xy\;$ === $\;y$)|}{...}} &  \\
\cline{1-2}
\otrans{\dots}{(\dots)}{\lstinline|$x\;$ === $\;$Nil|}{(\dots)} & 
\otrans{\dots}{(\dots)}{\lstinline|$xy\;$ === $\;y$|}{\{([\bnd{\sv{3}}{\lstinline|Nil|}],\,\{\sv{0}..\sv{3}\})\}} &
\end{array}
$}\\
\subfloat[\label{appendo_eval_d}]{
$
\begin{array}{c}
\otrans{\dots}{([\bnd{\sv{1}}{\lstinline|A|},\,\bnd{\sv{2}}{\lstinline|Nil|},\,\bnd{\sv{3}}{\lstinline|Nil|}],\,\{\sv{0}..\sv{3}\})}{\lstinline|$xy\;$ === $\;$Cons ($h$, $\;ty$)|}{\{([\dots,\,\bnd{\sv{0}}{\lstinline|Cons (A, Nil)|}],\,\{\sv{0}..\sv{3}\})\}}
\end{array}
$}
\caption{An example of relational evaluation}
\label{appendo_eval}
\end{figure}

We describe the semantics of our language using a conventional big-step style inference system. First, we choose an infinite
set of \emph{semantic variables} $\meta{W}$. As we will see shortly, the \lstinline|fresh($x$)...| construct allocates a fresh variable, not being
used before, and associates it with the syntactic variable $x$. Thus, in the semantics we will need an infinite supply of fresh variables.

Next, we introduce the \emph{interpretation} of syntactic variables $\iota$ as a (partial) mapping

$$
\iota : \meta{V} \to \meta{T}(\meta{W})
$$

The role of the interpretation is twofold: first, it binds syntactic variables, used in the \lstinline|fresh| construct, to their semantic counterparts, and second, 
it binds relational parameters to their values~--- terms over semantic variables. For a syntactic term $t$ and an interpretation $\iota$ we denote
$t\iota$ the result of substitution of all syntactic variables in $t$ by their interpretations according to $\iota$; we assume $t\iota$ to be defined 
only when $\iota$ is defined for all variables in $t$. Thus, $t\iota$, if defined, is always an element of $\meta{T}(\meta{W})$.

Then, we borrow some conventional machinery from unification theory~\cite{Unification,UnificationRevisited}. Namely, we define a substitution $\sigma$ to
be a partial mapping between semantic variables and semantic terms:

$$
\sigma : \meta{W} \to \meta{T}(\meta{W})
$$

For any substitution $\sigma$ we assume, that the set of all free variables of all terms in the range of $\sigma$ has an empty intersection with the
domain of $\sigma$, and we denote by $\sigma\circ\theta$ the composition of substitutions, defined in a usual way. For arbitrary $t\in\meta{T}(\meta{W})$ and
a substitution $\sigma$ we denote the result of application of $\sigma$ to $t$ as $t\sigma$.

The basic inference relation for our semantics has the form

$$
\otrans{\Gamma,\iota}{(\sigma,\,\delta)}{g}{S}
$$

\noindent where $\Gamma$ is an environment, which binds relational symbols to their definitions, $\iota$~--- an interpretation, $\sigma$~--- a substitution, 
$\delta$~--- a set of allocated semantic variables, $g$~--- a goal, and $S$~--- a set of pairs $(\sigma^\prime,\,\delta^\prime)$, where $\sigma^\prime$ and
$\delta^\prime$~--- a substitution and a set of allocated semantic variables respectively. Informally speaking, we interpret a goal $g$ in the context of
relational definitions $\Gamma$, current interpretation $\iota$, current substitution $\sigma$ and current set of allocated semantic variables $\sigma$. As a 
result, we obtain a (possibly empty) set of answers. Each answer consists of a new substitution, accumulated through the execution of $g$, and a new set of
allocated semantic variables (note, in original miniKanren a goal can produce the same answer multiple number of times, but this property is not important
in our case).

The inference rules themselves are shown on Fig.~\ref{semantics}. The first two rules handle two possible outcomes of the unification. Note, we use here the most 
general unifier ($mgu$) of two semantic terms; we assume ``occurs check'' to be incorporated in the unification algorithm. Since the unification goal is built of 
syntactic terms, we have to interpret them first (by applying $\iota$), and take into account current substitution $\sigma$.

The rule for the disjunction first interprets the constituents of the disjunction in the same state and then combines the outcomes.
% using the multiset union operation ``$\biguplus$''. 

The rule for the conjunction threads the execution of its subgoals in a left-to-right successive manner: first the
left conjunct is evaluated, providing a set $\{(\sigma_i,\,\delta_i)\}$. Then the second conjunct is evaluated for each element of the set, and the
results are eventually combined. Note, the evaluation of both conjuncts is performed under the \emph{same} interpretation $\iota$ since both of them occur in the 
\emph{same} bounding context. The substitution and the set of allocated semantic variables, on the other hand, are inherited from left to right since the evaluation 
of the right conjunct has to be performed in the context of the results, provided by the left one. 

The rule for the \lstinline|fresh| construct allocates arbitrary semantic variable, not taken before, and evaluates the unique subgoal in the updated interpretation, which
associates the syntactic variable, bound in this \lstinline|fresh|, with taken semantic one.

Finally, the rule for relational definition invocation describes its evaluation in a few steps. First, the body of the definition is found, using the environment $\Gamma$. 
Then, the terms $t_i$, specified as the arguments of the invocation, are converted into their semantic forms $v_i$ using current interpretation $\iota$ and current 
substitution $\sigma$. Next, the body of the definition is evaluated in the context of \emph{empty} substitution $\epsilon$ and an interpretation, containing nothing
else, than the bindings for the formal parameters of the definition. This way of handling interpretation models the behavior of call stack in conventional languages
with no nested functions. Finally, the result substitutions are composed with the original one\footnote{We could use the original
substitution instead of the empty one without the need to use composition; however we found the approach we took more proof-friendly since each relational definition is evaluated
in initially empty substitution.}. 

Given this big-step evaluation relation for goals, we can describe the evaluation for the top-level specification $s=d_1,\dots,d_k;g$. First, we construct the associated environment
$\Gamma_s$, which properly binds all relational symbols in $s$ to their bodies. Then, we evaluate the top-level goal

$$
\otrans{\Gamma_s,\,\bot}{(\epsilon,\,\emptyset)}{g}{S_s}
$$

\noindent obtaining the set of results $S_s$; here we use empty (everywhere undefined) interpretation $\bot$ and empty substitution $\epsilon$ as a starting point. 
Finally, we choose all substitutions from $S_s$. 

Our semantics is almost deterministic~--- the only source of ambiguity is the rule for the \lstinline|fresh| construct, where we choose a new semantic variable
arbitrarily. If we fix the order, in which semantic variables are allocated, the semantics becomes completely deterministic. It is also easy to see, that if each
relational symbol is unambiguously defined in the specification and called with a proper number of parameters, and all goals in all relational definitions and the 
top-level goal are closed (i.e. each variable occurrence is bound either in some \lstinline|fresh| constructor in a parameter list of enclosing definition), 
then during the evaluation all syntactic variables are properly interpreted~--- in other words, the execution cannot break down halfway through and either diverges or 
finishes with some results.

We illustrate the evaluation, determined by this semantics, by the canonical example for relational programming~--- list concatenation relation \lstinline|append$^o$| 
(we respect here the convention to add the ``$^o$'' suffix to all names of relational entities):

\begin{lstlisting}  
   append$^o$ $\binds$ $\lambda\;x\;y\;xy$ . 
     (($x$ === $\;$Nil) /\ ($xy$ === $\;y$)) \/
     (fresh ($h$ $t$ $ty$)
        ($x$  === $\;$Cons ($h$, $t$)) /\
        (append$^o$ $t$ $y$ $ty$) /\
        ($xy$ === $\;$Cons ($h$, $ty$)) /\
     );
   fresh ($q$) (append$^o$ (Cons (A, Nil)) Nil $q$)
\end{lstlisting}

For the simplicity we omitted the arities of constructors \lstinline|A|, \lstinline|Cons|, \lstinline|Nil| and relational symbol \lstinline|append$^o$|. Since we require the 
top-level goal to be closed, from now on we conventionalize the use of the top-level \lstinline|fresh| construct as a binder for the variables whose values we are 
interested in (in this particular example $q$).

The evaluation is illustrated on Fig.~\ref{appendo_eval}; here we use numbers in bold font to denote semantic variables. For the sake of brevity and in order to
make the illustration observable we do not show the binding environment for relational definitions and as a rule denote by ellipses the inherited components
of derivation tree (the components in the left side of ``$\Rightarrow$'' are inherited top-down, in the right side~--- bottom-up).

\FloatBarrier
We start from the top-level goal and first apply the rule $\rulename{Fresh}$ (see Fig.~\ref{appendo_eval_a}). Since 
we did not use any semantic variables yet, we allocate the first one ($\sv{0}$), update the interpretation and the set of used semantic variables and continue. The next construct
is the call for \lstinline|append$^o$|, so we unfold its definition, replace the interpretation of syntactic variables by the bindings for the formal parameters, and 
evaluate the body w.r.t. the empty substitution (which has no difference from the current one yet). The body of \lstinline|append$^o$| definition is a disjunction, so we
take its left constituent, which is a conjunction, so we in turn take its left constituent, which is a unification \lstinline|$x\;$ === $\;$Nil|. This unification clearly fails, as
current interpretation binds $x$ to \lstinline|Cons (A, Nil)|. This completes the whole branch for the first disjunct of \lstinline|append$^o$| with empty result.

The evaluation of the second disjunct is shown on Fig.~\ref{appendo_eval_b}. Its top-level construct is \lstinline|fresh ($h\;t\;ty$)|, so we allocate three successive 
semantic variables $\sv{1}$, $\sv{2}$ and $\sv{3}$ and save the bindings in the interpretation. The next construct is a conjunction of three goals (assuming
``$\wedge$'' is right-associative in the concrete syntax), and we proceed with the first one, which is a unification \mbox{\lstinline|$x\;$ === $\;$Cons($h$, $\;t$)|}. 
Since $x$, $h$ and $t$ are free in current substitution and $x$ is bound to \lstinline|Cons(A, Nil)| by current interpretation, the unification succeeds with the 
substitution \mbox{$[\bnd{\sv{1}}{\lstinline{A}},\,\bnd{\sv{2}}{\lstinline|Nil|}]$}. The evaluation of remaining conjuncts is shown on Fig.~\ref{appendo_eval_c}.

\FloatBarrier
The first one is a recursive call to \lstinline|append$^o$|. We evaluate the actual parameters~--- $t$, $y$ and $ty$~--- in current interpretation and substitution, 
obtaining \lstinline|Nil|, \lstinline|Nil| and $\sv{3}$ respectively, update the interpretation to bind formal parameters to these values, and recurse to the body with
empty current substitution. Again, we have the disjunction, and the first disjunct is a conjunction \mbox{\lstinline|($x\;$ === $\;$Nil) $\wedge$ ($xy\;$ === $\;y$)|}.
Now \mbox{\lstinline|$x\;$ === $\;$Nil|} succeeds, as $x$ is already bound to \lstinline|Nil| by the interpretation, and \mbox{\lstinline|$xy\;$ === $\;y$|} succeeds
as well, providing a new substitution \mbox{$[\bnd{\sv{3}}{\lstinline|Nil|}]$}. We omit the detailed evaluation of the second top-level disjunct of \lstinline|append$^o$| since
it contains a unification \lstinline|$x\;$ === $\;$Cons (_, _)| which, clearly, does not contribute anything.

Finally, we return from the recursive call to \lstinline|append$^o$| and take the composition of substitutions~--- one before the call, and another after~--- which
gives us \mbox{$[\bnd{\sv{1}}{\lstinline|A|},\,\bnd{\sv{2}}{\lstinline|Nil|},\,\bnd{\sv{3}}{\lstinline|Nil|}]$} (see Fig.~\ref{appendo_eval_d}). We only need now to 
interpret the last conjunct of the second disjunct of \lstinline|append$^o$|~--- \mbox{\lstinline|$xy\;$ === $\;$Cons ($h$, $\;ty$)|}~--- which gives us the
final substitution \mbox{$[\bnd{\sv{1}}{\lstinline|A|},\,\bnd{\sv{2}}{\lstinline|Nil|},\,\bnd{\sv{3}}{\lstinline|Nil|},\,\bnd{\sv{0}}{\lstinline|Cons (A, Nil)|}]$}. Now, we
have to remember, that the topmost bound variable of the top-level goal is $q$, and corresponding semantic variable is $\sv{0}$. Thus, the answer is 
\mbox{$q\;=\;\lstinline|Cons (A, Nil)|$}, which is rather expected.
\section{Relational Conversion}
\label{conversion}
\def\arraystretch{1}

Before we describe the relational conversion itself, we formulate some limitations for the source
programs. Functional programs tend to operate with higher-order values, while miniKanren is
limited by a first-order unification. Therefore, it would be unreasonable to expect, that arbitrary
functional program can be converted into a relational form (at least using reasonably simple 
transformations). 

First, we introduce the set of ground types $\mathcal G$:

$$
\mathcal G=\alpha \mid T^k(g_1,\dots,g_k)
$$

Informally, a value of a ground type cannot contain closures. Then we formulate the following limitations for
the programs to be converted into a relational form:

\begin{itemize}
  \item all constructor parameter types must be type variables;
  \item constructors and polymorphic equality can only be applied to the values of ground types;
  \item all \lstinline|match|-expressions must by of ground types.
\end{itemize}

The first condition means, that all algebraic datatypes (which we consider defined implicitly, see Section~\ref{source_language}) 
have to be fully-polymorphic. The first two limitations then allow us to specify the polymorphism restriction for 
relational programs, which we mentioned informally in Section~\ref{ocanren}: all type variables are bounded to
range only over ground types (this condition, of course, is sufficient, but not necessary).

The third limitation is not essential and introduced only to simplify the presentation. If a \lstinline|match|-expression does not
have a ground type, it can always be transformed to have one by applying $\eta$-expansion:

\begin{lstlisting}
   match $e$ with {$p_i$ -> $e_i$} $\leadsto$ fun $\bar{x}$.match $e$ with {$p_i$ -> $e_i\,\bar{x}$}
\end{lstlisting}

\noindent where $\bar{x}$ is a vector of new variables, different from those in $e$, $e_i$, and $p_i$. In fact, our implementation,
described in Section~\ref{evaluation}, performs this expansion as long as a non-ground type \lstinline|match|-expression is encountered. 
This is the single case when we actually use types and perform $\eta$-expansion.

The general idea behind the conversion can be illustrated on a type level: an expression of type $t$ in the source
language is transformed into the expression of type $\sembr{t}^t$ in relational extension, where
the transformation $\sembr{\bullet}^t$ is defined as follows:

$$
\begin{array}{rcl}
\sembr{g}^t                     & = & g \to \G \\
\sembr{t_1 \to t_2}^t           & = & \sembr{t_1}^t \to \sembr{t_2}^t \\
%\sembr{\forall \alpha. \: t} & = & \forall \alpha. \: \sembr{t}
\end{array}
$$

In other words, an expression of a ground type is converted into a goal-returning function. The informal semantics
of this function is to make its argument respect a certain contract. As the argument can have some free variable occurrences, 
the goal tries to substitute these variables with some values in order to respect the contract this goal represents. 
For example, a constant \lstinline|Nil| is converted into a function \lstinline|fun $q$ . $q\,$=== ^Nil|.

The conversion itself is described in terms of transformation $\sembr{\bullet}^c$, see Figure~7.%\ref{relational_conversion}. 
The first five rules
simply propagate the conversion through the expression; the last three actually do the work. These rules themselves may look complicated,
but the idea is rather simple.

\begin{figure}[t]
  \centering
  \begin{tabular}{rcp{6cm}}
     $\sembr{x}^c$                &=&$x$\\
     $\sembr{\lambda x.e}^c$      &=&$\lambda x.\sembr{e}^c$\\
     $\sembr{f\;e}^c$             &=&$\sembr{f}^c\;\sembr{e}^c$\\
     $\sembr{\lstinline|let $\;x\;$ = $\;e_1\;$ in $\;e_2$|}^c$&=&\lstinline|let $x$ = $\sembr{e_1}^c$ in $\sembr{e_2}^c$|\\
     $\sembr{\lstinline|let rec $\;f\;$ = $\lambda x.e_1\;$ in $\;e_2$|}^c$&=&\lstinline|let rec $f$ = $\sembr{\lambda x.e_1}^c$ in $\sembr{e_2}^c$|\\[2mm]
     $\sembr{C^k (e_1,\dots,e_k)}^c$&=&\lstinline|fun $q$.fresh ($q_1 \dots q_k$)|
\begin{lstlisting}
  ($\sembr{e_1}^c\; q_1$) /\
  ...
  ($\sembr{e_k}^c\; q_k$) /\
  ($q$ === $\;\uparrow(C^n (q_1, \dots, q_k)$))
\end{lstlisting}\\[-2mm]
     $\sembr{\lstinline|match $\;e\;$ with \{$C^{n_i}_i(x^i_1,\dots,x^i_{n_i})\;$ -> $\;e_i$\}|}^c$&=&\lstinline|fun $q$.fresh ($q_e$)|
\begin{lstlisting}
    ($\sembr{e}^c\;q_e$) /\
    $\bigvee_i$ ((fresh ($q^i_1\dots q^i_{n_i}$)
           ($q_e$ === $\;\uparrow C^{n_i}_1(q^i_1,\dots,q^i_{n_i})$) /\
           (fun $x^i_1\dots x^i_{n_i}$.$\sembr{e_i}^c$) ($\equiv q^i_1$) ... ($\equiv q^i_{n_i}$) $q$
     ) 
    )
\end{lstlisting}\\[-2mm]
     $\sembr{\lstinline|$e_1\,$=$\,e_2$|}^c$&=&\lstinline|fun $q$.fresh ($q_1\,q_2$)|
\begin{lstlisting}
  $\sembr{e_1}^c\,q_1$ /\
  $\sembr{e_2}^c\,q_2$ /\
  (($q_1$ === $\;q_2$ /\ $q$ === $\;$^true) |||
   ($q_1$ =/= $\;q_2$ /\ $q$ === $\;$^false)
  )
\end{lstlisting}
  \end{tabular}
\label{relational_conversion}
\caption{Relational conversion}
\end{figure}

In the case of constructor we know, that all expressions $e_i$ have ground types. Thus, their relational images are goal-returning
functions. We create a set of fresh variables (one for each expression) and pass them as arguments to these functions to associate
them with the values of the expressions. The result of conversion for the constructor application itself has to be a 
goal-returning function as well. We surround expression constructed so far with abstraction and unify its argument $q$ with the
constructor, applied to corresponding logical variables. We also apply logical constructor $\uparrow$ to respect the typing rule
for unification.

The rule for pattern-matching conversion operates similarly. First, the scrutinee must have a ground type (since it is matched against
constructors). We create a fresh variable $q_e$ and associate it with the value of the scrutinee exactly as in the previous
case. Then, for each branch we create a number of fresh variables (one for each variable in the pattern for the branch) and
express pattern-matching in terms of unification, using these variables and corresponding constructor. Finally, the body $e_i$ of the branch
is an expression with free variables, corresponding to those in the pattern. We, therefore, convert $e_i$ and surround the result with
lambdas, closing all these variables. To pass the bindings $q^i_j$ for pattern variables to the body we apply this function to
 goal-returning functions $(\equiv q^i_j)$. This, again, gives us a goal-returning function, which we apply to the topmost result variable $q$.

The last rule follows the same pattern: both arguments of polymorphic equality are transformed into goal-returning functions, and we know, that
the arguments of these functions are of some ground type. We apply these functions to fresh variables and perform case analysis. Note, this is
the only case when we actually use disequality constraints.

An interesting property of relational conversion is that it does not change terms, which do not use constructors, equality, and pattern-matching. Thus,
a lot of useful higher-order functions~--- application, composition, fixed point, etc.~--- are already relational and can be used in
relational specifications.

Another observation is that our transformation is compositional (a relational image of application is an application of relational
images). This means, that relational conversion is compatible with separate compilation~--- multiple source files can be
converted independently without losing the possibility to work properly when combined.

Then, it is interesting, that the result of relational conversion runs in a forward direction
deterministically. Thus, relational conversion imposes only a constant-time slowdown in a forward
direction.

Finally, we formulate the following properties for relational conversion:

\begin{itemize}
\item Static correctness: if an expression $e$ has a type $t$ in the source language, then $\sembr{e}^c$ has a 
type $\sembr{t}^t$ in relational extension. In other words, relational conversion transforms properly typed
programs into properly typed. Proof is by structural induction (and trivial).
\item Partial semantic correctness: if an expression $e$ has a ground type $t$ and \mbox{$e \leadsto^f v$} for some
  value $v$, then \mbox{$\lstinline|fresh($x$)($\sembr{e}^c\;x$)| \leadsto^r (\theta,\emptyset)$}, and 
\mbox{$\theta(\mathfrak{s})=v$}, where $\mathfrak{s}$ is a semantic variable, associated with $x$ on the
first step of the relational evaluation.
  The essential part of the proof is given in the Appendix~\ref{appendix}.
%Proof
%is by induction on the length of derivation sequence (a number of lemmas have to be justified on the way).
\end{itemize}

In order to prove the complete correctness, we need some means to interpret the results of relational 
derivation with free variables in functional case. This is a subject of future research.

%\begin{figure}[t]
%\centering
%\includegraphics{graph2.pdf}
%\caption{The Second Set of Benchmarks}
%\label{eval:second}
%\end{figure}

\section{Performance Evaluation}
\label{sec:evaluation}

One of our initial goals was to evaluate, what performance impact would choosing OCaml as a host language make. In addition we spent some 
efforts in order to implement \miniKanren in an efficient, tagless manner, and, of course, the outcome of this decision also has to be 
measured. For comparison we took faster-miniKanren\footnote{\url{https://github.com/webyrd/faster-miniKanren}}~--- a full-fledged 
\miniKanren implementation for Scheme/Racket. It turned out that faster-miniKanren implements a number of optimizations~\cite{WillThesis, Optimizations} 
to speedup the search; moreover, the search order in our implementation initially was a little bit different. In order to make the comparison
fair, we additionally implemented all these optimizations and adjusted the search order to exactly coincide with 
what faster-miniKanren does.

\begin{figure}[t]
\centering
\includegraphics[scale=0.4]{graph.png}
\caption{The Results of the Performance Evaluation}
\label{eval}
\end{figure}

\FloatBarrier 

For the set of benchmarks we took the following problems:

\begin{itemize}
%\item \textbf{sorto, permo}~--- sorting and permutation for lists of Peano numbers (shown as example in Section~\ref{sec:examples}).
%The concrete tests are the sorting of the list \lstinline{[3; 2; 1; 0]} and taking all permutations of the list \lstinline{[0; 1; 2; 3; 4; 5; 6; 7]}.
\item \textbf{pow, logo}~--- exponentiation and logarithm for integers in binary form. The concrete tests relationally computed
$3^5$ (which in 243) and $log_3 243$ (which is, conversely, 5).
\item \textbf{quines, twines, trines}~--- self/co-evaluating program synthesis problems from~\cite{Untagged}. The
concrete tests took the first 100, 15 and 3 answers for these problems respectively.
\end{itemize}

%Since the last bundle of benchmarks uses disequality constraints (and, hence, $\mu$Kanren is ruled out) we
%split all benchmarks into two sets.

The evaluation was performed on a desktop computer with Intel Core i7-4790K CPU @ 4.00GHz processor and 16GB of memory.
For OCanren \mbox{ocaml-4.04.0+frame_pointer+flambda} was used, for faster-miniKanren~--- Chez~Scheme~9.4.1.
All benchmarks were ran in the natively compiled mode ten times, then average user time was taken. The results of the evaluation
are shown on Figure~\ref{eval}. The whole evaluation repository with all scripts and detailed description is accessible 
from \lstinline{GitHub}\footnote{\url{https://github.com/Kakadu/ocanren-perf}}.

The first conclusion, which is rather easy to derive from the results, is that the tagless approach indeed matters. Our initial
implementation did not show essential speedup in comparison even with $\mu$Kanren (and was even \emph{slower} on the logarithm
and permutations benchmarks). The situation was improved drastically, however, when we switched to the tagless version.

Yet, in comparison with faster-miniKanren our implementation is still lagging behind. We can conclude, that the optimizations, 
used in Scheme/Racket version, have a different impact in the OCaml case; we save this problem for future research.


\section{Conclusion}
\label{conclusion}

We presented an improvement of a search strategy for relational programming, which is aimed at
improving refutationally completeness. We've proven, that in the case of a finite number of 
answers our modification is a strict improvement over the original strategy. Our evaluation 
shows, that w.r.t. the improved search many practically important refutationally incomplete 
queries became refutationally complete; in addition in a number of cases the performance was greatly 
improved since our modification, as a side effect, causes the search to choose more 
``optimistic'' branches. 

We can identify the following directions for future work. 

First, we believe, that our result on refutational improvement for a finite number of answers 
can be extended to the general case as well (note, in our current development we did not make 
any use of the \emph{completeness} property of miniKanren search). For this, we would also need 
another, more general, semantics. 

Another direction is extending the language with disequality constraints. Our evaluation has 
shown, that disequality constraints do not compromise our improvement in all user benchmarks, 
but we do not have a proof, that they are indeed harmless.

Next, we are working on a certified proof of the main theorem in Coq.

Finally, our practical evaluation is performed only for a prototype. 
We consider the embedding of our improvement in a full-fledged implementation to be
an important task.

\begin{comment}
\section{The Source Language}

\begin{figure}
\centering
{\bf Supplementary syntax categories:}
$$
\begin{array}{rcll}
  \mathcal C &=&\lstinline|True|,\,\lstinline|False|,\,C^n,\dots                &\mbox{\supp(constructors with arity)}\\
  \mathcal X &=&x,\,y,\,z,\,\dots                                               &\mbox{\supp(variables)}\\
  \mathcal P &=&C^n\,(x_1,\,\dots,\,x_n)                                         &\mbox{\supp(shallow patterns)}
\end{array}
$$
{\bf Expressions:}
$$
\begin{array}{rcll}
  \mathcal E &=&x                                                               &\mbox{\supp(variable occurrence)}\\
             & &\lambda x.e                                                     &\mbox{\supp(abstraction)}\\
             & &e_1\;e_2                                                        &\mbox{\supp(application)}\\ 
             & &C^n(e_1,\dots, e_n)                                             &\mbox{\supp(constructor application)}\\
             & &\lstinline|let $x$ = $e_1$ in $e_2$|                            &\mbox{\supp(let-binding)}\\
             & &\lstinline|let rec $f$ = $\lambda x.e_1$ in $e_2$|              &\mbox{\supp(recursive let-binding)}\\
             & &e_1\,=\,e_2                                                     &\mbox{\supp(equality test)}\\
             & &\lstinline|match $e$ with $\{p_i$ -> $e_i\}$| &\mbox{\supp(pattern matching)}
\end{array}
$$
\caption{Syntax of the source language}
\label{functional_syntax}
\end{figure}

\setarrow{:}
\newcommand{\typed}[3]{\withenv{#1}{\trans{#2}{}{#3}}}

\begin{figure}
\centering
{\bf Types:}
$$
\begin{array}{rcll}
  \mathcal X &=&\alpha, \beta, \dots                                            &\mbox{\supp{(type variables)}}\\
  \mathcal D &=&T^n,...                                                         &\mbox{\supp{(datatype constructors)}}\\
  \mathcal T &=&\lstinline|bool|\mid\alpha\mid T^n(t_1,\dots,t_n)\mid t_1\to t_2 &\mbox{\supp{(types)}}\\
  \mathcal S &=&\forall\bar{\alpha}.t                                           &\mbox{\supp{(type schemas)}}
\end{array}
$$
{\bf Typing rules:}
\begin{tabular}{p{7cm}p{7cm}}
$$
\typed{\Gamma}{\lstinline|True|,\;\lstinline|False|}{\lstinline|bool|}
\ruleno{Bool$_T$}
$$ 
&
$$
\trule{\typed{\Gamma}{e_1}{t}\;\;\;\;\typed{\Gamma}{e_2}{t}}
      {\typed{\Gamma}{e_1=e_2}{\lstinline|bool|}}
\ruleno{Eq$_T$}
$$
\\
$$
\trule{\typed{\Gamma}{e_i}{t^C_i}}
      {\typed{\Gamma}{C^n(e_1,\dots,e_n)}{t^C}}
\ruleno{Constr$_T$}
$$
&
$$
\typed{\Gamma,x:\forall\bar{\alpha}.t}{x}{t[\bar{\alpha}\gets\bar{t^\prime}]}
\ruleno{Var$_T$}
$$
\\
$$
\trule{\typed{\Gamma}{f}{t_1\to t_2}\;\;\;\;\typed{\Gamma}{e}{t_1}}
      {\typed{\Gamma}{f\;e}{t_2}}
\ruleno{App$_T$}
$$
&
$$
\trule{\typed{\Gamma,\,x:t_1}{f}{t_2}}
      {\typed{\Gamma}{\lambda x.f}{t_1\to t_2}}
\ruleno{Abs$_T$}
$$
\\
\multicolumn{2}{p{14cm}}{
$$
\trule{\typed{\Gamma}{e_1}{t_1}\;\;\;\;\typed{\Gamma,x:\forall\bar{\alpha}.t_1}{e_2}{t}}
      {\typed{\Gamma}{\lstinline|let $\;x\;$ = $\;e_1\;$ in $\;e_2$|}{t}},\;\bar{\alpha}=FV(t_1)\setminus FV(\Gamma)
\ruleno{Let$_T$}
$$}\\
\multicolumn{2}{p{14cm}}{
$$
\trule{\typed{\Gamma,f:t_1}{\lambda x.e_1}{t_1}\;\;\;\;\typed{\Gamma,f:\forall\bar{\alpha}.t_1}{e_2}{t}}
      {\typed{\Gamma}{\lstinline|let rec $\;f\;$ = $\;\lambda x.e_1\;$ in $\;e_2$|}{t}},\;\bar{\alpha}=FV(t_1)\setminus FV(\Gamma)
\ruleno{LetRec$_T$}
$$}\\
\multicolumn{2}{p{14cm}}{
$$
\trule{\typed{\Gamma}{e}{t^C}\;\;\;\;\typed{\Gamma,x^i_1:t^{C_i}_1,\dots,x^i_{k_i}:t^{C_i}_{k_i}}{e_i}{t}}
      {\typed{\Gamma}{\lstinline|match $\;e\;$ with $\;\{C_i^{k_i}(x^i_1,\dots,x^i_{k_i})$ -> $e_i\}$|}{t}}
\ruleno{Match$_T$}
$$}
\end{tabular}
\caption{Typing rules for the source language}
\label{functional_typing}
\end{figure}

\setarrow{\to}
\newcommand{\step}[2]{\trans{\inbr{#1}}{}{\inbr{#2}}}

\begin{figure}
\centering
{\bf Values:}
$$
\mathcal V = \lstinline|True|\mid\lstinline|False|\mid C^n(v_1,\dots,v_n)\mid\lambda x.e\mid\mu f\lambda x.e
$$
{\bf Contexts:}
$$
\mathcal C = \Box\;e\mid v\;\Box\mid\lstinline|let $x$ = $\Box$ in $e$|\mid\lstinline|match $\;\Box\;$ with $\{p_i$->$e_i\}$|\mid C^n(\bar{v},\Box,\bar{e})\mid\Box=e\mid v=\Box 
$$
$$
C[e]\mbox{\supp{~--- a context $C$ with an expression $e$ plugged into a hole}}
$$
{\bf Stack of contexts:}
$$
\mathcal S=\epsilon\mid\mathcal C : \mathcal S
$$
{\bf States:}
$$
\inbr{\mathcal S, e}\mbox{\supp{(stack of contexts, expression)}};\;\inbr{\epsilon,e}\mbox{\supp{(initial state)}};\;\inbr{\epsilon,v}\mbox{\supp{(final state)}}
$$
{\bf Transitions:}
\vskip2mm
\bgroup
\def\arraystretch{0}
\begin{tabular}{p{7cm}p{7cm}}
\multicolumn{2}{p{14cm}}{
$$
\step{C:\mathcal S,\, v}{\mathcal S,\, C[v]}\ruleno{Value}
$$}\\
$$
\step{\mathcal S,\, f\;e}{\Box\;e:\mathcal S,\, f}\ruleno{AppL}
$$&
$$
\step{\mathcal S,\, v\;e_2}{v\;\Box:\mathcal S,\, e_2}\ruleno{AppR}
$$\\
$$
\step{\mathcal S,\,e_1=e_2}{\Box=e_2:\mathcal S,\,e_1}\ruleno{EqL}
$$&
$$
\step{\mathcal S,\,v=e}{v=\Box:\mathcal S,\,e}\ruleno{EqR}
$$\\
\multicolumn{2}{p{14cm}}{
$$
\step{C:\mathcal S,\,v=v}{\mathcal S,\,C[\lstinline|True|]}\ruleno{EqTrue}
$$}\\
\multicolumn{2}{p{14cm}}{
$$
\step{C:\mathcal S,\,v_1=v_2}{\mathcal S,\,C[\lstinline|False|]},\;v_1\ne v_2\ruleno{EqFalse}
$$}\\
\multicolumn{2}{p{14cm}}{
$$
\step{\mathcal S,\, (\lambda x.e)\;v}{\mathcal S,\, e[x\gets v]}\ruleno{Beta}
$$}\\
\multicolumn{2}{p{14cm}}{
$$
\step{\mathcal S,\, (\mu f\lambda x.e)\;v}{\mathcal S,\, e[f\gets\mu f\lambda x.e,\, x\gets v]}\ruleno{Mu}
$$}\\
\multicolumn{2}{p{14cm}}{
$$
\step{\mathcal S,\, C^n(v_1,\dots,v_{k-1},e_k,\dots,e_n)}{C^n(v_1,\dots,v_{k-1},\Box,\dots,e_n):\mathcal S,\, e_k}\ruleno{Constr}
$$}\\
\multicolumn{2}{p{14cm}}{
$$
\step{\mathcal S,\, \lstinline|let $\;x\;$ = $\;e_1\;$ in $\;e_2$|}{\lstinline|let $\;x\;$ = $\;\Box\;$ in $\;e_2$|:\mathcal S,\, e_1}\ruleno{Let}
$$}\\
\multicolumn{2}{p{14cm}}{
$$
\step{\mathcal S,\, \lstinline|let $\;x\;$ = $\;v\;$ in $\;e$|}{\mathcal S,\,e[x\gets v]}\ruleno{LetVal}
$$}\\
\multicolumn{2}{p{14cm}}{
$$
\step{\mathcal S,\, \lstinline|let rec $\;f\;$ = $\;\lambda x.e_1\;$ in $\;e_2$|}{\mathcal S,\, e_2[f\gets\mu f\lambda x.e_1]}\ruleno{LetRec}
$$}\\
\multicolumn{2}{p{14cm}}{
$$
\step{\mathcal S,\,\lstinline|match $\;e\;$ with $\;\{p_i$->$e_i\}$|}{\lstinline|match $\;\Box\;$ with $\;\{p_i$->$e_i\}$|:\mathcal S,\, e}\ruleno{Match}
$$}\\
\multicolumn{2}{p{14cm}}{
$$
\step{\mathcal S,\,\lstinline|match $\;C_k^{n_k}(v_1,\dots,v_{n_k})\;$ with $\;\{C_i^{n_i}(x^i_1,\dots,x^i_{n_i})\to e_i\}$|}{\mathcal S,\,e_k[x^k_j\gets v_j]}\ruleno{MatchVal}
$$}
\end{tabular}
\egroup
\caption{Semantics for the source language}
\label{functional_semantics}
\end{figure}

\begin{figure}
\centering
$$
\begin{array}{rcll}
  \mathcal E &\mathrel{{+}{=}}&\lstinline|fresh ($x$) $\;e$| &\mbox{\supp{(fresh logical variable binder)}}\\
             &                &e_1\equiv e_2                 &\mbox{\supp{(unification)}}                   \\
             &                &e_1\not\equiv e_2             &\mbox{\supp{(disequality constraint)}}        \\
             &                &e_1\vee e_2                   &\mbox{\supp{(disjunction)}}                   \\
             &                &e_1\wedge e_2                 &\mbox{\supp{(conjunction)}}
\end{array}
$$
\caption{Syntax of the relational extension}
\label{relational_syntax}
\end{figure}

\setarrow{:}
\begin{figure}
\centering
{\bf Types:}
$$
\begin{array}{rcll}
 \mathcal L &=               &\;\uparrow\!\alpha \mid\;\uparrow\!\lstinline|bool|\mid\;\uparrow\!T^n(l_1,\dots,l_n)&\mbox{\supp{(type of logical terms)}}\\
 \mathcal T &\mathrel{{+}{=}}& \G                                                                            &\mbox{\supp{(type of logical goals)}}
\end{array}
$$
{\bf Typing rules:}
\begin{tabular}{p{7cm}p{7cm}}
\multicolumn{2}{p{14cm}}{
$$
\trule{\typed{\Gamma,x:l}{e}{\G}}
      {\typed{\Gamma}{\lstinline|fresh ($x$) $\;e$|}{\G}}
\ruleno{Fresh$_T$}
$$}\\
$$
\trule{\typed{\Gamma}{e_1}{l}\;\;\;\;\typed{\Gamma}{e_2}{l}}
      {\typed{\Gamma}{e_1\equiv e_2}{\G}}
\ruleno{Unify$_T$}
$$&
$$
\trule{\typed{\Gamma}{e_1}{l}\;\;\;\;\typed{\Gamma}{e_2}{l}}
      {\typed{\Gamma}{e_1\not\equiv e_2}{\G}}
\ruleno{Disequality$_T$}
$$\\
$$
\trule{\typed{\Gamma}{e_1}{\G}\;\;\;\;\typed{\Gamma}{e_2}{\G}}
      {\typed{\Gamma}{e_1\wedge e_2}{\G}}
\ruleno{Conjunction$_T$}
$$&
$$
\trule{\typed{\Gamma}{e_1}{\G}\;\;\;\;\typed{\Gamma}{e_2}{\G}}
      {\typed{\Gamma}{e_1\vee e_2}{\G}}
\ruleno{Disjunction$_T$}
$$
\end{tabular}
\caption{Typing rules for the relational extension}
\label{relational_typing}
\end{figure}

\setarrow{\to}
\begin{figure}
\centering
{\bf Semantic variables:}
\begin{gather*}
\mathfrak S = \mathfrak s_1, \mathfrak s_2, \dots\\
\Sigma, \Sigma^\prime\dots \subset 2^{\mathcal S}\;\mbox{\supp{(sets of allocated semantics variables)}}\\
\inbr{\Sigma^\prime, \mathfrak s}\gets\lstinline|new|\;\Sigma,\;\Sigma^\prime=\Sigma\cup\{\mathfrak s\}\;\mbox{\supp{(allocation of a new semantic variable)}}
\end{gather*}
{\bf Values:}
$$
\mathcal V \mathrel{{+}{=}} \lstinline|success|\mid\mathfrak s
$$
{\bf Contexts:}
$$
\mathcal C \mathrel{{+}{=}}\Box\equiv e\mid v\equiv\Box\mid\Box\not\equiv e\mid v\not\equiv\Box\mid\Box\wedge e\mid e\wedge\Box
$$
{\bf States:}
\begin{gather*}
\inbr{\Sigma,\mathcal S,e,\sigma}\mbox{\supp{(set of allocated semantic variables, stack of contexts, expression, logical state)}}\\
\inbr{\emptyset,\epsilon,e,\iota}\mbox{\supp{(initial state)}}\\
\inbr{\_,\epsilon,\lstinline|success|,\sigma}\mbox{\supp{(final state)}}
\end{gather*}
{\bf Transitions:}
\vskip2mm
\bgroup
\def\arraystretch{0}
\begin{tabular}{p{14cm}}
$$
\step{\Sigma,\,\mathcal S,\,\lstinline|fresh($x$) $\;e$|,\,\sigma}{\Sigma^\prime,\,\mathcal S,\,e[x\gets\mathfrak s],\,\sigma},\,\inbr{\Sigma^\prime,\mathfrak s}\gets\lstinline|new|\;\Sigma\ruleno{Fresh}
$$\\
$$
\step{\Sigma,\,\mathcal S,\,e_1\equiv e_2,\,\sigma}{\Sigma,\,\Box\equiv e_2:\mathcal S,\,e_1,\,\sigma}\ruleno{UnifyL}
$$\\
$$
\step{\Sigma,\,\mathcal S,\,v\equiv e,\,\sigma}{\Sigma,\,v\equiv\Box:\mathcal S,\,e,\,\sigma}\ruleno{UnifyR}
$$\\
$$
\step{\Sigma,\,\mathcal S,\,v_1\equiv v_2,\,\sigma}{\Sigma,\,\mathcal S,\,\lstinline|success|,\,\sigma^\prime},\,{\bf unify}\,(\sigma,\,v_1,\,v_2)=\sigma^\prime\ruleno{Unify}
$$\\
$$
\step{\Sigma,\,\mathcal S,\,e_1\not\equiv e_2,\,\sigma}{\Sigma,\,\Box\not\equiv e_2:\mathcal S,\,e_1,\,\sigma}\ruleno{DisEqL}
$$\\
$$
\step{\Sigma,\,\mathcal S,\,v\not\equiv e,\,\sigma}{\Sigma,\,v\not\equiv\Box:\mathcal S,\,e,\,\sigma}\ruleno{DisEqR}
$$\\
$$
\step{\Sigma,\,\mathcal S,\,v_1\not\equiv v_2,\,\sigma}{\Sigma,\,\mathcal S,\,\lstinline|success|,\,\sigma^\prime},\,{\bf diseq}\,(\sigma,\,v_1,\,v_2)=\sigma^\prime\ruleno{DisEq}
$$\\
$$
\step{\Sigma,\,\mathcal S,\,e_1\vee e_2,\,\sigma}{\Sigma,\,\mathcal S,\,e_1,\,\sigma}\ruleno{DisjL}
$$\\
$$
\step{\Sigma,\,\mathcal S,\,e_1\vee e_2,\,\sigma}{\Sigma,\,\mathcal S,\,e_2,\,\sigma}\ruleno{DisjR}
$$\\
$$
\step{\Sigma,\,\mathcal S,\,e_1\wedge e_2,\,\sigma}{\Sigma,\,\Box\wedge e_2:\mathcal S,\,e_1,\,\sigma}\ruleno{ConjStartL}
$$\\
$$
\step{\Sigma,\,\mathcal S,\,e_1\wedge e_2,\,\sigma}{\Sigma,\,e_1\wedge\Box:\mathcal S,\,e_2,\,\sigma}\ruleno{ConjStartR}
$$\\
$$
\step{\Sigma,\,\mathcal S,\,\lstinline|success|\wedge e,\,\sigma}{\Sigma,\,\mathcal S,\,e,\,\sigma}\ruleno{ConjL}
$$\\
$$
\step{\Sigma,\,\mathcal S,\,e\wedge\lstinline|success|,\,\sigma}{\Sigma,\,\mathcal S,\,e,\,\sigma}\ruleno{ConjR}
$$
\end{tabular}
\egroup
\caption{Semantics for the relational extension}
\label{relational_semantics}
\end{figure}

\begin{figure}
\centering
{\bf Terms:}
$$
\mathfrak T = \mathfrak s\mid\;\uparrow\!C^n(\mathfrak t_1,\dots,\mathfrak t_n)
$$
{\bf Substitution:}
\begin{gather*}
\theta:\mathfrak S\to\mathfrak T\;\mbox{\supp{(a partial mapping from semantic variables to terms)}}\\
\forall\theta\;\forall\mathfrak s,\mathfrak s^\prime\in dom(\theta)\;:\;\theta(\mathfrak s)\not\ni\mathfrak s^\prime\\
\mbox{\supp{Substitution application:}}\;\mathfrak t\theta=\left\{\begin{array}{rcl}
                           \theta\;\mathfrak s&,&\mathfrak s\in dom(\theta)\\
                           \uparrow\!C^n(\mathfrak t_1\theta,\dots,\mathfrak t_n\theta)&,&\mathfrak t=\uparrow\!C^n(\mathfrak t_1,\dots,\mathfrak t_n)\\
                           \mathfrak t&,&\mbox{\supp{otherwise}}
                         \end{array}
                  \right.\\
\phi\;\theta=\lambda\mathfrak s\,.\,(\theta\;\mathfrak s)\phi\;\mbox{\supp{(substitution composition)}}
\end{gather*}
{\bf Logical state:}
$$
\sigma=\inbr{\theta, \{\zeta_1,\dots,\zeta_k\}}\;\mbox{\supp{(a pair of a substitution and a set of substitutions)}}
$$
{\bf Unification:}
$$
{\bf unify}\,(\inbr{\theta, \{\zeta_1,\dots,\zeta_k\}}, \mathfrak t_1,\mathfrak t_2)
$$
{\bf Disequality constraint:}
$$
{\bf diseq}\,(\inbr{\theta, \{\zeta_1,\dots,\zeta_k\}}, \mathfrak t_1,\mathfrak t_2)
$$
\caption{Logical states and transitions}
\end{figure}
%\end{comment}

\begin{figure}
\centering
{\bf Ground types:}
$$
\mathcal G=\alpha\mid\lstinline|bool|\mid T^n(g_1,\dots,g_n)
$$
{\bf Type conversion:}
$$
\begin{array}{rcl}
\left[g\right]              &=&g\to\G\\
\left[t_1\to t_2\right]     &=&\left[t_1\right]\to\left[t_2\right]\\
\left[\forall\alpha.t\right]&=&\forall\alpha.\left[t\right]
\end{array}
$$
{\bf Term conversion:}
\vskip2mm
\bgroup
\def\arraystretch{0.2}
\begin{tabular}{p{7cm}p{7cm}}
\multicolumn{2}{p{14cm}}{$$
\sembr{x} = x\ruleno{Var$_{RC}$}
$$}\\
$$
\sembr{\lambda x.e}=\lambda x.\sembr{e}\ruleno{Abs$_{RC}$}
$$&
$$
\trule{e : \_\to\_}
      {\sembr{f\;e}=\sembr{f}\;\sembr{e}}\ruleno{App$_{RC}$}
$$\\
$$
\sembr{\lstinline|True|}=\lambda q\,.\,q\;\equiv\;\uparrow\!\lstinline|True|\ruleno{True$_{RC}$}
$$&
$$
\sembr{\lstinline|False|}=\lambda q\,.\,q\;\equiv\;\uparrow\!\lstinline|False|\ruleno{False$_{RC}$}
$$\\
\multicolumn{2}{p{14cm}}{$$
\sembr{\lstinline|let $\;x\;$ = $\;e_1\;$ in $\;e_2$|}=\lstinline|let $\;x\;$ = $\;\sembr{e_1}\;$ in $\;\sembr{e_2}$|\ruleno{Let$_{RC}$}
$$}\\
\multicolumn{2}{p{14cm}}{$$
\sembr{\lstinline|let rec $\;f\;$ = $\;e_1\;$ in $\;e_2$|}=\lstinline|let rec $\;f\;$ = $\;\sembr{e_1}\;$ in $\;\sembr{e_2}$|\ruleno{LetRec$_{RC}$}
$$}\\
\multicolumn{2}{p{14cm}}{$$
\trule{e : g\;\;\;\;f : g\to t_1\to\dots\to t_n\to g_0}
      {\sembr{f\;e}=\lambda y_1\dots y_nq\,.\,\lstinline|fresh ($q^\prime$) $\;(\sembr{e}\;q^\prime)\wedge(\sembr{f}\,(\equiv q^\prime)\,y_1\dots y_n\,q)$|}\ruleno{AppG$_{RC}$}
$$}\\
\multicolumn{2}{p{14cm}}{$$
\sembr{C^n(x_1,\dots,x_n)}=\lambda q\,.\,\lstinline|fresh ($q_1\dots q_n$)|\;(\bigwedge\sembr{x_i}\;q_i)\wedge(q\;\equiv\;\uparrow\!C^n(q_1,\dots,q_n))\ruleno{Constr$_{RC}$}
$$}\\
\multicolumn{2}{p{14cm}}{$$
\trule{\lstinline|match $\;e\;$ with $\;\{C^{n_i}_i(x^i_1,\dots,x^i_{n_i})\;$->$e_i\}$|:t_1\to\dots\to t_r\to g}
      {
       \begin{array}{ccc}
                        \multicolumn{3}{l}{\sembr{\lstinline|match $\;e\;$ with $\;\{C^{n_i}_i(x^i_1,\dots,x^i_{n_i})\;$->$e_i\}$|}=}\\ 
            \phantom{X}&\multicolumn{2}{l}{
                           \begin{array}{l}
                              \lambda q_1\dots q_rq\,.\,\lstinline|fresh ($s$)|\\
                              \phantom{XX}(\sembr{e}\;s)\wedge\\
                              \phantom{XX}\bigvee\lstinline|fresh ($s^i_1\dots s^i_{n_i}$)|\\
                              \phantom{XXXX}(s\;\equiv\;\uparrow\!C^{n_i}_i(s^i_1,\dots,s^i_{n_i}))\wedge\\
                              \phantom{XXXX}(\lambda x^i_1,\dots,x^i_{n_i}\,.\,\sembr{e_i}\;q_1\dots q_r\;q)(\equiv\;s^i_1)\dots(\equiv\;s^i_{n_i})
                           \end{array}}                        
       \end{array}
      }\ruleno{Match$_{RC}$}
$$}\\
\multicolumn{2}{p{14cm}}{$$
\begin{array}{ccc}
                \multicolumn{3}{l}{\sembr{e_1=e_2}=}\\
    \phantom{X}&\multicolumn{2}{l}{
       \begin{array}{l}
        \lambda q\,.\,\lstinline|fresh ($q_1\;q_2$)|\\
        \phantom{XX}(\sembr{e_1}\;q_1)\wedge\\
        \phantom{XX}(\sembr{e_2}\;q_2)\wedge\\
        \phantom{XX}((q_1\;\equiv\;q_2\wedge q\;\equiv\;\uparrow\!\lstinline|True|)\vee(q_1\;\not\equiv\;q_2\wedge q\;\equiv\;\uparrow\!\lstinline|False|))
       \end{array}
    }
\end{array}\ruleno{Eq$_{RC}$}
$$}
\end{tabular}
\egroup
\caption{Relational conversion rules}
\end{figure}
\end{comment}

\printbibliography

\begin{comment}
\begin{thebibliography}{99}
\bibitem{TRS}
Daniel P. Friedman, William E.Byrd, Oleg Kiselyov. The Reasoned Schemer. The MIT
Press, 2005.

\bibitem{MicroKanren}
Jason Hemann, Daniel P. Friedman. $\mu$Kanren: A Minimal Core for Relational Programming //
Proceedings of the 2013 Workshop on Scheme and Functional Programming (Scheme '13).

\bibitem{alphaKanren}
William E. Byrd, Daniel P. Friedman. alphaKanren: A Fresh Name in Nominal Logic Programming //
Proceedings of the 2007 Workshop on Scheme and Functional Programming (Scheme '07).

\bibitem{CKanren}
Claire E. Alvis, Jeremiah J. Willcock, Kyle M. Carter, William E. Byrd, Daniel P. Friedman.
cKanren: miniKanren with Constraints //
Proceedings of the 2011 Workshop on Scheme and Functional Programming (Scheme '11).

\bibitem{Untagged}
William E. Byrd, Eric Holk, Daniel P. Friedman.
miniKanren, Live and Untagged: Quine Generation via Relational Interpreters (Programming Pearl) //
Proceedings of the 2012 Workshop on Scheme and Functional Programming (Scheme '12).

%\bibitem{Implicits}
%Leo White, Fr\'ed\'eric Bour, Jeremy Yallop.
%Modular Implicits // Workshop on ML, 2014, arXiv:1512.01438.

%\bibitem{Unparsing}
%Olivier Danvy.
%Functional Unparsing // Journal of Functional Programming, Vol.~8, Issue~6, November 1998.

%\bibitem{DoWeNeed}
%Daniel Fridlender, Mia Indrika.
%Do we need dependent types? // Journal of Functional Programming, Vol.~10, Issue~4, July 2000.

%\bibitem{DGP}
%Jeremy Gibbons. Datatype-generic Programming //
%Proceedings of the 2006 International Conference on Datatype-generic Programming.

%\bibitem{Deriving}
%Jeremy Yallop.
%Practical Generic Programming in OCaml // Proceedings of 2007 Workshop on ML.

%\bibitem{InstantGenerics}
%Manuel M. T. Chakravarty, Gabriel C. Ditu, Roman Leshchinskiy.
%Instant Generics: Fast and Easy. \url{http://www.cse.unsw.edu.au/~chak/papers/CDL09.html}, 2009.

%\bibitem{ALaCarte}
%Wouter Swierstra. Data Types \'a la Carte  // Journal of Functional Programming, Vol.~18, Issue~4, 2008.

%\bibitem{Kumar}
%Ramana Kumar. Mechanising Aspects of miniKanren in HOL. Bachelor Thesis, The Australian National University, 2010.

\bibitem{Unification}
Franz Baader, Wayne Snyder. Unification theory. In John Alan Robinsonand Andrei Voronkov, editors,
Handbook of Automated Reasoning. Elsevier and MIT Press, 2001.

%\bibitem{triangular}
%David C Bender, Lindsey Kuper, William E Byrd, Daniel P Friedman.
%Efficient Representations for Triangular Substitutions: a Comparison in miniKanren. Indiana University, 2009.

%\bibitem{HKinded}
%Jeremy Yallop, Leo White. Lightweight Higher-Kinded Polymorphism. FLOPS 2014.

\bibitem{Lambda}
Henk Barendregt. Lambda Calculi with Types, Handbook of Logic in Computer Science (Vol.~2), 1992.

\bibitem{WillThesis}
William E. Byrd. Relational Programming in miniKanren: Techniques, Applications, and Implementations. PhD Thesis,
Indiana University, Bloomington, IN, September 30, 2009.

\bibitem{ocanren}
Dmitry Kosarev, Dmitry Boulytchev. Typed Embedding of a Relational Language in OCaml // International Workshop on ML, 2016.

\bibitem{Types}
Benjamin Pierce. Types and Programming Languages. MIT Press, 2002.

\bibitem{Felleisen}
Andrew Wright, Matthias Felleisen. A Syntactic Approach to Type Soundness // Information and Computation, Vol.~115, No.~1, 1994.

\bibitem{cardelli}
Luca Cardelli, Peter Wegner. On Understanding Types, Data Abstraction, and Polymorphism // ACM Computing Surveys, Vol.~17, No.~4, 1985.

\bibitem{unified}
William E. Byrd, Michael Ballantyne, Gregory Rosenblatt, Matthew Might. A Unified Approach to Solving Seven Programming Problems // 
Proceedings of the International Conference on Functional Programming, 2017.

\bibitem{WillOnHM}
William E. Byrd. Personal communications.

\end{thebibliography}
\end{comment}

\end{document}

