\section{Performance Evaluation}
\label{sec:evaluation}

One of our initial goals was to evaluate, what performance impact would choosing
OCaml as a host language make. In addition we spent some efforts in order to implement \miniKanren in
efficient, tagless manner, and, of course, the outcome of this decision also has to be evaluated. 
Since our library generally follows $\mu$Kanren\footnote{https://github.com/jasonhemann/microKanren}, we've chosen it as a reference implementation.
In addition, we took \texttt{faster-miniKanren}\footnote{https://github.com/webyrd/faster-miniKanren}~--- more elaborated 
implementation with a little different search~--- since it implements disequality constraints. 

For the set of benchmarks we took the following problems:

\begin{itemize}
\item \textbf{expo}~--- exponentation $3^5(=243)$ for integers in binary form is calculated relationally;
\item \textbf{logo}~--- the inverse problem $log_3243(=5)$;
\item \textbf{sorto}~--- sorting a list of Peano numbers (shown as example in Section~\ref{sec:examples});
\item \textbf{quines, twines, trines}~--- self/co-evaluating program construction problems from~\cite{Untagged}.
\end{itemize}

Since the last bundle of benchmarks uses disequality constraints (and, hence, $\mu$Kanren is ruled out) we
split all benchmarks into two sets. 

The evaluation was performed on a desktop computer with Intel Core i7-4790K CPU @ 4.00GHz processor and 32GB of memory.
For OCanren \mbox{\texttt{ocaml-4.04.0+frame_pointer+flambda}} was used, for other implementations~--- Chez~Scheme~9.4.1. 
All benchmarks were run in natively compiled mode ten times, then average user time was taken. The results of evaluation
are shown on figures~\ref{eval:first} and~\ref{eval:second}.

The first conclusion, which is rather easy to derive from the results, is that ``taglessless'' indeed matters. Our initial
implementation did not show essential speedup in comparison with $\mu$Kanren (and was even \emph{slower} on the logarithm 
benchmark). The situation was improved drastically, however, when we switched to a tagless version.

Yet, in comparison with \texttt{faster-miniKanren} our implementation is still lagging behind. We did not discover yet the
reasons, and saved this problem for future research.

\begin{figure}[t]
\centering
\includegraphics{graph1.pdf}
\caption{The First Set of Benchmarks}
\label{eval:first}
\end{figure}

\begin{figure}[h]
\centering
\includegraphics{graph2.pdf}
\caption{The Second Set of Benchmarks}
\label{eval:second}
\end{figure}
