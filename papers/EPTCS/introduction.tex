\section{Introduction}
\label{intro}

Relational programming~\cite{TRS} is an attractive technique, based on the idea
of constructing programs as relations.  As a result, relational programs can be
``queried'' in various ``directions'', making it possible, for example, to simulate
reversed execution. Apart from being interesting from a purely theoretical standpoint,
this approach may have a practical value: some problems look much simpler
when considered as queries to some relational specification~\cite{WillThesis}. There are a
number of appealing examples confirming this observation: a type checker
for simply typed lambda calculus (and, at the same time, a type inferencer and solver
for the inhabitation problem), an interpreter (capable of generating ``quines''~---
programs producing themselves as a result)~\cite{Untagged}, list sorting (capable of
producing all permutations), etc.

Many logic programming languages, such as Prolog, Mercury~\cite{MercuryFirstPaper},
or Curry~\cite{CurryFirstPaper} to some extent
can be considered relational. We have chosen \miniKanren\footnote{\url{http://minikanren.org}}
as a model language, because it was specifically designed as a relational DSL, embedded in Scheme/Racket.
Being rather a minimalistic language, which can be implemented with just a few data structures and
combinators~\cite{MicroKanren, MuKanrenNew}, \miniKanren found its way into dozens of host languages, including Scala, Haskell and Standard ML.
The paradigm behind \miniKanren can be described as ``lightweight logic programming''\footnote{An in-depth comparison of \miniKanren
and Prolog can be found here: \url{http://minikanren.org/minikanren-and-prolog.html}}.

This paper addresses the problem of embedding \miniKanren into OCaml\footnote{\url{http://ocaml.org}}~--- a statically-typed functional language with
a rich type system. A statically-typed implementation would bring us a number of benefits. First, as always,
we expect typing to provide a certain level of correctness guarantees, ruling out some pathological programs, which
otherwise would provide pathological results. In the context of relational programming, however, typing would additionally
help us to interpret the results of queries. Often an answer to a relational query contains a number of
free variables, which are supposed to take arbitrary values. In the typed case these variables become typed,
facilitating the understanding of the answers, especially those with multiple free variables. Next, a number of \miniKanren
applications require additional constraints to be implemented. In the untyped setting, when everything can be anything,
some symbols or data structures tend to percolate into undesirable contexts~\cite{Untagged}. In order to prevent this from happening, some
auxiliary constraints (``\lstinline{absent$^o$}'', ``\lstinline{symbol$^o$}'', etc.) were introduced. These constraints play a role
of a weak dynamic type system, cutting undesirable answers out at runtime. Conversely, in a typed language this work can be
entrusted to the type checker (at the price of enforcing an end user to write properly typed specifications), not only improving the
performance of the system but also reducing the number of constraints which have to be implemented. Finally, it is rather natural
to expect better performance of a statically-typed implementation.

We present an implementation of a set of relational combinators and syntax extensions for
OCaml\footnote{The source code of our implementation is accessible from \url{https://github.com/dboulytchev/OCanren}.},
which, technically speaking, corresponds to $\mu$Kanren~\cite{MicroKanren} with disequality
constraints~\cite{CKanren}. The contribution of our work is as follows:

\begin{enumerate}
\item Our embedding allows an end user to enjoy strong static typing and type inference in relational
specifications; in particular, all type errors are reported at compile time and the types for
all logical variables are inferred from the context.

\item Our implementation is based on the \emph{polymorphic unification}, which, like the polymorphic comparison,
can be used for arbitrary types. The implementation of polymorphic unification uses unsafe features and
relies on the intrinsic knowledge of the runtime representation of values; we show, however, that this does not
compromise type safety.

\item We describe a uniform and scalable pattern for using types for relational programming, which
helps in converting typed data to and from the relational domain. With this pattern, only one
generic feature (``\lstinline{Functor}'') is needed, and thus virtually any generic
framework for OCaml can be used. Although being rather a pragmatic observation, this pattern, we
believe, would lead to a more regular and easy way to maintain relational specifications.

\item We provide a simplified way to integrate relational and functional code. Our approach utilizes
a well-known pattern~\cite{Unparsing, DoWeNeed} for variadic function implementation and makes it
possible to hide the reification of the answers phase from an end user.
\end{enumerate}

The rest of the paper is organized as follows: in the next section we provide a short overview of the related
works. Then we briefly introduce \miniKanren in
its original form to establish some notions; we do not intend to describe the language in its full bloom (interested readers can
refer to~\cite{TRS}). In Section~\ref{sec:goals} we describe some basic constructs behind a \miniKanren implementation, this time
in OCaml. In Section~\ref{sec:unification} we discuss polymorphic unification, and show that unification with
triangular substitution respects typing. Then we present our approach to handle user-defined types by injecting them
into the logic domain, and describe a convenient generic programming pattern, which can be used to implement the conversions from/to logic
domain. We also consider a simple approach and a more elaborate and efficient tagless variant (see Section~\ref{sec:injection}).
Section~\ref{sec:reification} describes top-level primitives and addresses the problem of relational and functional code integration.
Then, in Section~\ref{sec:examples} we present a set of relational examples, written with the aid of our
library. Section~\ref{sec:evaluation} contains the results of a performance evaluation and a comparison of our implementations
with existing implementation for Scheme. The final section concludes.

The authors would like to express a deep gratitude to the anonymous rewievers for their numerous constructive comments, Michael Ballantyne, Greg Rosenblatt, 
and the other attendees of the miniKanren Google Hangouts, who helped the authors in understanding the subtleties of the original miniKanren
implementation, Daniel Friedman for his remarks on the preliminary version of the paper, and William Byrd for all his help and support, which cannot be
overestimated.

