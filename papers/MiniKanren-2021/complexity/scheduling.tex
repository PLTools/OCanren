\section{Scheduling Complexity}
\label{sec:scheduling}

We may notice that operational semantics described in the previous section can be used to calculate the number of elementary scheduling steps.
In this section we define a specific value which estimates the scheduling time and give some equations to calculate this value for a given \emph{semantic
state}. However, our ultimate goal is to provide a way to deduce the complexity factor for a \emph{syntactic goal}. This problem will be addressed in
Section~\ref{which}, which will make use of recurrent equations presented here.

We also restrict our considerations only by the case when the evaluation of a goal in question delivers a final number of answers. Indeed,
if the number of answers is infinite, no reasonable complexity estimation can be given. A much more interesting question would be
the complexity estimation for coming up with some \emph{specific} answer; however for now this problem seems to be too hard to
tackle.

Our first idea is to take the number of states $d\,(s)$ in the \emph{finite} trace for a given state $s$:

\[ d\,(s) \; \eqdef \; | \trs{s} |  \]

However, it turns out, that this value alone in not enough to provide an accurate scheduling complexity estimation. The reason is that some
elementary steps in the semantics are not elementary in existing implementations. Namely, a careful analysis of the semantics discovers that
it involves a navigation to the leftmost leaf of the state, which in implementation corresponds to a number of
steps, proportional to the length of the leftmost branch of the state in question. Here we provide an \emph{ad-hoc} definition for this value, $t\,(s)$, and
assess its adequacy in Section~\ref{adequacy}:

\[
t\,(s) \eqdef \sum\limits_{s_i \in \trs{s}} lh\,(s_i) 
\]

where

\[
\begin{array}{rcl}
 lh\,(\taskst{g}{e})  &\eqdef& 1 \\
 lh\,(s_1 \oplus s_2) &\eqdef& lh\,(s_1) + 1 \\
 lh\,(s \otimes g)    &\eqdef& lh\,(s) + 1 \\
\end{array}
\]


The following lemma provides a fundamental relation between these two scheduling complexity factors:

\begin{lemma}
  For any state $s$

  \[
  d\,(s) \le t\,(s) \le d^2\,(s)
  \]
  
\end{lemma}

Our next goal is to come up with the equations which relate the complexity of states with the complexity of their
(immediate) substates. The following lemma deals with a basic (leaf state) case:

\begin{lemma}
  If

  \[\taskst{g}{e} \rightarrow s^\prime\]

  or

  \[\taskst{g}{e} \xrightarrow{a} s^\prime\]

  then

  \[d\,(\taskst{g}{e}) = d\,(s^\prime) + 1\]

  and

  \[t\,(\taskst{g}{e}) = t\,(s^\prime) + 1\]
\end{lemma}

Two remaining cases are those for $\oplus$- and $\otimes$- states. The first case is dealt with by the following lemma:

\begin{lemma}
For any two states $s_1$ and $s_2$

\[
\begin{array}{rcl}
  d\,(s_1 \oplus s_2) &=& d\,(s_1) + d\,(s_2) \\

  t\,(s_1 \oplus s_2) &=& t\,(s_1) + t\,(s_2) + \min\,\{2\times d\,(s_1) - 1, 2\times d\,(s_2)\}
\end{array}
\]

\end{lemma}

The last one is the most cumbersome. Remember, in ``$\otimes$'' the left substate is evaluated until the first answer is found, which
then is taken as \emph{an environment} for evaluation the left subgoal. Thus, in the equations for $\otimes$-states answers
has to be taken into account. The following lemma give us an exact equation:

\begin{lemma}
For any state $s$ and any goal $g$

\[
d\,(s \otimes g) = d\,(s) + \displaystyle\sum\limits_{a_i \in \tra{s}} d\,(\taskst{g}{a_i})
\]

\[
\begin{array}{rclc}
  t\,(s \otimes g) &=& t\,(s) & + \\
                   & & d\,(s) & + \\
  & & \displaystyle\sum\limits_{a_i \in \tra{s}} (t\,(\taskst{g}{a_i}) + \min\,\{2\times d\,(\taskst{g}{a_i}),\,2\times (d\,(s) - d_{a_i}\,(s) + \displaystyle\sum\limits_{j > i} d\,(\taskst{g}{a_j}))\} - 1)
\end{array}
\]

where $d_{a_i}\,(s)$ is the number of steps in the trace for state $s$ until the answer $a_i$ is produced.
\end{lemma}

The last equation is too cumbersome to use directly, so as a rule we will use some its approximation instead. One option is to go with the fist argument of ``$\min$''. It can be a good approximation
in the case when there are several answers passed to the second goal and for none of their scheduling times surpass the \emph{overall} scheduling time for all other answers.

\begin{corollary}
For any state $s$ and any goal $g$
\[ t\,(s \otimes g) \le t\,(s) + d\,(s) +  \displaystyle\sum\limits_{a_i \in \tra{s}} (t\,(\taskst{g}{a_i}) + 2\times d\,(\taskst{g}{a_i}) \]
\end{corollary}

In the case when there is only one answer, however, we should rather go with the second argument of ``$\min$''.

\begin{corollary}
  For any state $s$ and any goal $g$, if $\tra{s} = \{a\}$, then
  
\[ t\,(s \otimes g) \le t\,(s) + 3\times d\,(s) + t\,(\taskst{g}{a}) \]
\end{corollary}

Finally, we provide two approximations for goals in the form of sequences of disjuncts/conjuncts. These two approximations work regardless the associativity/grouping of subformulas; thus a
certain constant $c_k$, depending only on $k$, comes in.

\begin{lemma}

Let $g = g_1 \lor \dots \lor g_k$ and $1 \le l \le k$; then

\[
\renewcommand{\arraystretch}{1.5}
\begin{array}{rcl}
  d\,(\taskst{g}{e}) &\le& \displaystyle\sum\limits_{1 \le i \le k} d\,(\taskst{g_i}{e}) \\
  t\,(\taskst{g}{e}) &\le& \displaystyle\sum\limits_{1 \le i \le k} t\,(\taskst{g_i}{e}) + k\times \displaystyle\sum\limits_{\renewcommand{\arraystretch}{1}\begin{array}{c}1 \le i \le k \\ i \ne l\end{array}} d\,(\taskst{g_i}{e}).
\end{array}
\]

\end{lemma}

\begin{lemma}

Let $g = g_1 \land \dots \land g_k$ and let $A_i$ be a set of all answers that are passed to $g_i$ at some stage starting from some initial environment $e_0$

\[
\begin{array}{rcl}
A_1 &=& \{ e_0 \} \\
A_{i + 1} & = & \bigcup\limits_{a \in A_i} \tra{\taskst{g_i}{a}} 
\end{array}
\]

Then

\[
\begin{array}{rcl}
d\,(\taskst{g}{e}) &=& \displaystyle\sum\limits_{1 \le i \le k} \;\; \displaystyle\sum\limits_{a \in A_i} d\,(\taskst{g_i}{a}) \\
t\,(\taskst{g}{e}) &\le& \displaystyle\sum\limits_{1 \le i \le k} \;\; \displaystyle\sum\limits_{a \in A_i} t\,(\taskst{g_i}{a}) + c_k \times \displaystyle\sum\limits_{1 \le i \le k} \;\; \displaystyle\sum\limits_{a \in A_i} d\,(\taskst{g_i}{a}), \\
\end{array}
\]

In the case when all $A_i$ contain only one answer

\[
t\,(\taskst{g}{e}) \le \displaystyle\sum\limits_{1 \le i \le k} \;\; \displaystyle\sum\limits_{a \in A_i} t\,(\taskst{g_i}{a}) + c_k \times \displaystyle \sum\limits_{1 \le i \le k - 1} \;\; \displaystyle\sum\limits_{a \in A_i} d\,(\taskst{g_i}{a})
\]

\end{lemma}
