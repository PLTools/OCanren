\section{Conclusion}

We presented a first attempt to build a theory which would allow one to calculate an adequate worst-case time
complexity estimations for relational programs in \mK. While our research is not completed the current results still
allow to explain some observable phenomena in some relational programs behavior.

For now we confine ourselves to the problem of estimating the time of full search for a give query. Estimating the time before the first answer is found is an important and probably more practical task. Unfortunatelly, the described model can not be adjusted naturally to this case. The reason for this is that the reasoning about time (scheduling time in particular) in our terms becomes non-compositional for the case of interrupted search: if the answer is found in some branch, the search is cut short in other branches too. We can still calculate the number of semantic steps for this case, just focusing at one branch in such cases. But for scheduling factor this will not work, because the size of states can be different in different branches, so the equal number of semantic steps can take different time in different branches. This picture requires more complicated notions with non-trivial dependencies between them and the model becomes impractical. We are currently looking into the ways to tame it.

Among other directions for future research we can mention relaxing the requirements which we put on
terms and substitutions in order to provide an accurate unification time estimations. It would be also
interesting to come up with an automated (or semi-automated) procedure to deduce the complexity
estimations in a symbolic forms by making use of symbolic execution model presented in Section~\ref{sec:symbolic}.
