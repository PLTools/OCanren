%\documentclass[a4paper,UKenglish,cleveref,autoref]{lipics-v2019}

\documentclass[runningheads]{llncs}

\usepackage{amsmath,amssymb}
\usepackage[utf8]{inputenc}
\usepackage[english]{babel}
\usepackage{amssymb}
\usepackage{mathtools}
\usepackage{listings}
\usepackage{comment}
\usepackage{indentfirst}
\usepackage{hyperref}
%\usepackage{amsthm}
\usepackage{stmaryrd}
\usepackage{eufrak}
\usepackage{lstcoq}
\usepackage{placeins}
\usepackage[dvipsnames]{xcolor}
\usepackage{tcolorbox}
\usepackage{soul}
\usepackage{enumitem}
\setlist{nosep}
%\usepackage{biblatex}

\let\llncssubparagraph\subparagraph
%% Provide a definition to \subparagraph to keep titlesec happy
\let\subparagraph\paragraph
%% Load titlesec
%% Revert \subparagraph to the llncs definition
\usepackage{titlesec}
\let\subparagraph\llncssubparagraph

\titlespacing*{\section}{0pt}{1ex plus 1ex minus .2ex}{1ex plus .2ex}
\titlespacing*{\subsection}{0pt}{1ex plus 1ex minus .2ex}{1ex plus .2ex}
\titlespacing*{\paragraph}{0pt}{0ex}{1em}

%\newtheorem{theorem}{Theorem}
%\newtheorem{lemma}{Lemma}
%\newtheorem{corollary}{Corollary}
\newtheorem{hyp}{Hypothesis}
%\newtheorem{definition}{Definition}


\lstdefinelanguage{minikanren}{
basicstyle=\normalsize,
keywords={fresh},
sensitive=true,
commentstyle=\itshape\ttfamily, % \footnotesize\itshape\ttfamily,
keywordstyle=\textbf,
identifierstyle=\ttfamily,
basewidth={0.5em,0.5em},
columns=fixed,
fontadjust=true,
literate={fun}{{$\lambda\;\;$}}1 {->}{{$\to$}}3 {===}{{$\,\equiv\,$}}1 {=/=}{{$\not\equiv$}}1 {|>}{{$\triangleright$}}3 {/\\}{{$\wedge$}}2 {\\/}{{$\vee$}}2,
morecomment=[s]{(*}{*)}
}

\lstset{
mathescape=true,
language=minikanren
}

\usepackage{letltxmacro}
\newcommand*{\SavedLstInline}{}
\LetLtxMacro\SavedLstInline\lstinline
\DeclareRobustCommand*{\lstinline}{%
  \ifmmode
    \let\SavedBGroup\bgroup
    \def\bgroup{%
      \let\bgroup\SavedBGroup
      \hbox\bgroup
    }%
  \fi
  \SavedLstInline
}

\usepackage{todonotes}
\newcommand{\rednote}[1]{\todo[inline, color=red!20]{#1}}
\newcommand{\orangenote}[1]{\todo[inline, color=orange!20]{#1}}
\newcommand{\forestnote}[1]{\todo[inline, color=ForestGreen!20]{#1}}
\newcommand{\myhl}[2]{\colorbox{#1!50}{\parbox{\textwidth}{#2}}}

\def\transarrow{\xrightarrow}
\newcommand{\setarrow}[1]{\def\transarrow{#1}}

\def\padding{\phantom{X}}
\newcommand{\setpadding}[1]{\def\padding{#1}}

\def\subarrow{}
\newcommand{\setsubarrow}[1]{\def\subarrow{#1}}

\newcommand{\trule}[2]{\frac{#1}{#2}}
\newcommand{\crule}[3]{\frac{#1}{#2},\;{#3}}
\newcommand{\withenv}[2]{{#1}\vdash{#2}}
\newcommand{\trans}[3]{{#1}\transarrow{\padding{\textstyle #2}\padding}\subarrow{#3}}
\newcommand{\ctrans}[4]{{#1}\transarrow{\padding#2\padding}\subarrow{#3},\;{#4}}
\newcommand{\llang}[1]{\mbox{\lstinline[mathescape]|#1|}}
\newcommand{\pair}[2]{\inbr{{#1}\mid{#2}}}
\newcommand{\inbr}[1]{\left<{#1}\right>}
\newcommand{\highlight}[1]{\color{red}{#1}}
%\newcommand{\ruleno}[1]{\eqno[\scriptsize\textsc{#1}]}
\newcommand{\ruleno}[1]{\mbox{[\textsc{#1}]}}
\newcommand{\rulename}[1]{\textsc{#1}}
\newcommand{\inmath}[1]{\mbox{$#1$}}
\newcommand{\lfp}[1]{fix_{#1}}
\newcommand{\gfp}[1]{Fix_{#1}}
\newcommand{\vsep}{\vspace{-2mm}}
\newcommand{\supp}[1]{\scriptsize{#1}}
\newcommand{\sembr}[1]{\llbracket{#1}\rrbracket}
\newcommand{\cd}[1]{\texttt{#1}}
\newcommand{\free}[1]{\boxed{#1}}
\newcommand{\binds}{\;\mapsto\;}
\newcommand{\dbi}[1]{\mbox{\bf{#1}}}
\newcommand{\sv}[1]{\mbox{\textbf{#1}}}
\newcommand{\bnd}[2]{{#1}\mkern-9mu\binds\mkern-9mu{#2}}
\newcommand{\meta}[1]{{\mathcal{#1}}}
\newcommand{\dom}[1]{\mathtt{dom}\;{#1}}
\newcommand{\primi}[2]{\mathbf{#1}\;{#2}}
\renewcommand{\dom}[1]{\mathcal{D}om\,({#1})}
\newcommand{\ran}[1]{\mathcal{VR}an\,({#1})}
\newcommand{\fv}[1]{\mathcal{FV}\,({#1})}
\newcommand{\tr}[1]{\mathcal{T}r_{#1}}
\newcommand{\step}{\circ}

\newcommand{\searchRule}[6] {
  #1, #2 \vdash (#3, #4) \xRightarrow{#5} #6}
\newcommand{\extSearchRule}[8] {
  #1, #2, #3, #4 \vdash (#5, #6) \xRightarrow{#7}_{e} #8}
\newcommand{\q}{\hspace{0.5em}}
\newcommand{\bigcdot}{\boldsymbol{\cdot}}
\newcommand{\bigslant}[2]{{\raisebox{.2em}{$#1$}\left/\raisebox{-.2em}{$#2$}\right.}}

\let\emptyset\varnothing
\let\eps\varepsilon

\DeclareUnicodeCharacter{2212}{-}

\sloppy

\title{Certified Semantics for Relational Programming\thanks{The reported study was funded by RFBR, project number 18-01-00380.}}

\author{Dmitry Rozplokhas\inst{1,3}\orcidID{0000-0001-7882-4497} \and
Andrey Vyatkin\inst{2} \and \\
Dmitry Boulytchev\inst{2,3}\orcidID{0000−0001−8363−7143}}
%
\authorrunning{D. Rozplokhas et al.}
% First names are abbreviated in the running head.
% If there are more than two authors, 'et al.' is used.
%
\institute{Higher School of Economics, Russia \and
Saint Petersburg State University, Russia \and
JetBrains Research, Russia}

\begin{document}

\setlength{\belowcaptionskip}{-5pt}
\setlength{\abovecaptionskip}{0pt}

\setlength{\abovedisplayskip}{-3pt}
\setlength{\belowdisplayskip}{-2pt}
\setlength{\abovedisplayshortskip}{0pt}
\setlength{\belowdisplayshortskip}{2pt}

\maketitle
\setcounter{footnote}{0}

%TODO mandatory: add short abstract of the document
\begin{abstract}
  We present a formal study of semantics for the relational programming language \textsc{miniKanren}. First,
  we formulate a denotational semantics which corresponds to the minimal Herbrand model for definite logic
  programs. Second, we present operational semantics which models interleaving, the distinctive feature of \textsc{miniKanren}
  implementation, and prove its soundness and completeness w.r.t. the denotational semantics.
  Our development is supported by a \textsc{Coq} specification, from which a reference interpreter can be
  extracted. We also derive from our main result a certified semantics (and a reference interpreter) for SLD resolution
  with cut and prove its soundness.
\end{abstract}

\section{Introduction}
\label{sec:intro}

\colorbox{red!20}{\textbf{Motivational example:}}
\begin{lstlisting}[basicstyle=\small]
   append$^o_{naive}$ = fun a b ab ->
     ((a === Nil) /\ (ab === b)) \/
     (fresh (h t tb)
        (a === Cons(h, t)) /\
        (append$^o_{naive}$ t b tb) /\
        (ab === Cons(h, tb)) 
     )
\end{lstlisting}

\begin{lstlisting}[basicstyle=\small]
   append$^o_{opt}$ = fun a b ab ->
     ((a === Nil) /\ (ab === b)) \/
     (fresh (h t tb)
        (a === Cons(h, t)) /\
        (ab === Cons(h, tb) /\
        (append$^o_{opt}$ t b tb)) 
     )
\end{lstlisting}

\colorbox{red!20}{\parbox{\textwidth}{\textbf{State requirements for our method (informally)}}}

We prove the theorem only for goals in DNF.

\[
\begin{array}{lcl}
B_{nf} & = &  \unigoal{\mathcal{T}_\mathcal{X}}{\mathcal{T}_\mathcal{X}} \; \mid \;
                     \invokegoal{R^k}{\mathcal{T}_\mathcal{X}}{\mathcal{T}_\mathcal{X}} \\
C_{nf} & = & B_{nf} \; \mid \; \conjgoal{C_{nf}}{B_{nf}} \\
F_{nf} & = & C_{nf} \; \mid \; \freshgoal{X}{F_{nf} } \\
D_{nf} & = & F_{nf} \; \mid \; \disjgoal{D_{nf}}{F_{nf}}
\end{array}
\]

We also have a requirement that all recursive calls performed during unification are \emph{grounding} and \emph{non-repetitive}.

\begin{definition}
  We call relational invocation
  
  \[\taskst{\invokegoal{R^k}{t_1}{t_k}}{e}\]

  \emph{grounding} and \emph{non-repetitive} if 

  \[ \forall (\sigma^{a}, n^{a}) \in \tra{\taskst{\invokegoal{R^k}{t_1}{t_k}}{e}} \,:\, FV(t_i \sigma^{a}) = \emptyset \]

  and
  
  \[ \forall (\sigma_1^{a},\, n_1^{a}),\, (\sigma_2^{a},\, n_2^{a}) \in \tra{\taskst{\invokegoal{R^k}{t_1}{t_k}}{e}} \,:\, (t_1 \sigma_1^{a}, \dots, t_k \sigma_1^{a}) \ne (t_1 \sigma_2^{a},\, \dots, t_k \sigma_2^{a}) \]
\end{definition}


\section{The Source Language and Relational Extension}

In this section we present a formal description of a small functional language, taken as a source
for relational conversion. We describe its syntax, typing rules and semantics, and then extend it
with relational constructs. Thus, relational conversion maps a program in the source language into 
the program in extended. We specify the typing rules and semantics for the extension as well.

\subsection{The Source Language}
\label{source_language}

The syntax of our source functional language is shown on Figure~\ref{functional_syntax}. It consists of a lambda calculus, 
enriched with constructors with fixed arities $C^n$, patterns $p$ and pattern-matching constructs. Among constructors
we distinguish two nullary interpreted constructors \lstinline|true| and \lstinline|false|, and we add a boolean equality
operator ``$=$'' and expressions for recursive/non recursive let-bindings. In a pattern matching we only allow shallow
patterns (which is not an essential limitation) and do not allow wildcards (which is important~--- converting 
wildcard pattern matching into relational form would require essentially different projections). This choice of language may 
look quite a restrictive. However, in terms of relational programming the language contains virtually everything we would need. Indeed, from
relational conversion standpoint the standard built-in integer arithmetics, for example, is of no use~--- 
there is simply no way to convert integer expression into relational form, using integer expressions. In order to use relational 
arithmetics one needs to reimplement everything from scratch, using, for example, Peano encoding; but Peano arithmetics can be
easily expressed in the language we present.

\begin{wrapfigure}{r}{0.5\textwidth}
\centering
%\scalebox{0.9}{
$$
\begin{array}{rcl}
   e &=&x\\
     & &\lambda x.e\\
     & &e_1\;e_2\\
     & &C^n(e_1,\dots, e_n)\\
     & &\lstinline|true|\\
     & &\lstinline|false|\\
     & &\lstinline|let $x$ = $e_1$ in $e_2$|\\
     & &\lstinline|let rec $f$ = $\lambda x.e_1$ in $e_2$|\\
     & &e_1\,=\,e_2\\
     & &\lstinline|match $e$ with $\{p_i$ -> $e_i\}$|\\
     & &\\
   p &=&C^n(x_1,\dots,x_n)\\
\end{array}
$$
%}
\caption{The syntax of the source language}
\label{functional_syntax}
\end{wrapfigure}

Our language is equipped with Hidley-Milner type system, and we present the typing rules in conventional syntax-directed form 
on Figure~\ref{functional_typing}. Besides primitive boolean type, type variables and function types our system contains
a number of implicitly defined algebraic datatypes $T^k$, and we stipulate, that each constructor $C^n$ belongs to exactly one
datatype. In rule \textsc{Constr$_T$} we assume, that type $t^C$ has the form $T^k(t_1,\dots,t_k)$, where each of the types
$t_i$ is recovered from the types $t_i^C$ of arguments of constructor $C^n$ and, moreover, these types are agree in the sense of
constructor application. Similarly, in rule \textsc{Match$_T$} the types of all $C_i^{k_i}(x^i_1,\dots,x^i_{k_i})$ are expected
to be equal $t^C$, and $t^{C_i}_j$ is a type of $j$-th argument of constructor $C_i$, used in a pattern.

\setarrow{:}
\newcommand{\typed}[3]{\withenv{#1}{\trans{#2}{}{#3}}}

\begin{figure}
\centering
{\bf Types:}
$$
\begin{array}{rcll}
  \mathcal X &=&\alpha, \beta, \dots                                            &\mbox{\supp{(type variables)}}\\
  \mathcal D &=&T^n,...                                                         &\mbox{\supp{(datatype constructors)}}\\
  \mathcal T &=&\lstinline|bool|\mid\alpha\mid T^k(t_1,\dots,t_k)\mid t_1\to t_2 &\mbox{\supp{(types)}}\\
  \mathcal S &=&\forall\bar{\alpha}.t                                           &\mbox{\supp{(type schemas)}}
\end{array}
$$
{\bf Typing rules:}
\def\arraystretch{0}
\begin{tabular}{p{7cm}p{7cm}}
$$
\typed{\Gamma}{\lstinline|true|,\;\lstinline|false|}{\lstinline|bool|}
\ruleno{Bool$_T$}
$$ 
&
$$
\trule{\typed{\Gamma}{e_1}{t}\;\;\;\;\typed{\Gamma}{e_2}{t}}
      {\typed{\Gamma}{e_1=e_2}{\lstinline|bool|}}
\ruleno{Eq$_T$}
$$
\\
$$
\trule{\typed{\Gamma}{e_i}{t^C_i}}
      {\typed{\Gamma}{C^n(e_1,\dots,e_n)}{t^C}}
\ruleno{Constr$_T$}
$$
&
$$
\typed{\Gamma,x:\forall\bar{\alpha}.t}{x}{t[\bar{\alpha}\gets\bar{t^\prime}]}
\ruleno{Var$_T$}
$$
\\
$$
\trule{\typed{\Gamma}{f}{t_1\to t_2}\;\;\;\;\typed{\Gamma}{e}{t_1}}
      {\typed{\Gamma}{f\;e}{t_2}}
\ruleno{App$_T$}
$$
&
$$
\trule{\typed{\Gamma,\,x:t_1}{f}{t_2}}
      {\typed{\Gamma}{\lambda x.f}{t_1\to t_2}}
\ruleno{Abs$_T$}
$$
\\
\multicolumn{2}{p{14cm}}{
$$
\trule{\typed{\Gamma}{e_1}{t_1}\;\;\;\;\typed{\Gamma,x:\forall\bar{\alpha}.t_1}{e_2}{t}}
      {\typed{\Gamma}{\lstinline|let $\;x\;$ = $\;e_1\;$ in $\;e_2$|}{t}},\;\bar{\alpha}=FV(t_1)\setminus FV(\Gamma)
\ruleno{Let$_T$}
$$}\\
\multicolumn{2}{p{14cm}}{
$$
\trule{\typed{\Gamma,f:t_1}{\lambda x.e_1}{t_1}\;\;\;\;\typed{\Gamma,f:\forall\bar{\alpha}.t_1}{e_2}{t}}
      {\typed{\Gamma}{\lstinline|let rec $\;f\;$ = $\;\lambda x.e_1\;$ in $\;e_2$|}{t}},\;\bar{\alpha}=FV(t_1)\setminus FV(\Gamma)
\ruleno{LetRec$_T$}
$$}\\
\multicolumn{2}{p{14cm}}{
$$
\trule{\typed{\Gamma}{e}{t^C}\;\;\;\;\typed{\Gamma,x^i_1:t^{C_i}_1,\dots,x^i_{k_i}:t^{C_i}_{k_i}}{e_i}{t}}
      {\typed{\Gamma}{\lstinline|match $\;e\;$ with $\;\{C_i^{k_i}(x^i_1,\dots,x^i_{k_i})$ -> $e_i\}$|}{t}}
\ruleno{Match$_T$}
$$}
\end{tabular}
\caption{Typing rules for the source language}
\label{functional_typing}
\end{figure}

We describe the semantics of our language in the form of transition system. The transition relation

\setarrow{\to}
\newcommand{\step}[2]{\trans{\inbr{#1}}{}{\inbr{#2}}}

$$
\step{\mathcal S,\,e}{\mathcal S^\prime,\,e^\prime}
$$

\noindent describes one step of evaluation of expression $e$ with a stack of contexts $\mathcal S$, which results in
a new stack $\mathcal S^\prime$ and a new expression $e^\prime$. A context is an expression with a unique hole; informally speaking, 
a stack of contexts describes a path in the expression being evaluated from the topmost construct to the point, where the evaluation 
currently is taking place. For a context $C$ and an expression $e$ we denote by $C[e]$ a complete expression with no holes, which is 
obtained by plugging $e$ into the unique hole of $C$. From each state $\inbr{C_1:C_2:\dots:C_k,e}$ we can build an 
expression $C_k[\dots[C_2[C_1[e]]\dots]$, which represents an intermediate result of evaluation according to a small-step semantics. 
This form of semantic description originates from Felleisen-style~\cite{Felleisen} approach for small-step semantics, and we've
chosen it since it can be naturally extended for relational case.

Our semantics describes call-by-value left-to-right evaluation; in the rules $\textsc{Beta}$, $\textsc{Mu}$, $\textsc{LetVal}$,
$\textsc{LetRec}$ and $\textsc{MatchVal}$ we perform capture-avoiding substitutions, which respect the names in abstractions and let-bindings,
and in the rules $\textsc{EqTrue}$ and  $\textsc{EqFalse}$ we assume, that values being compared do not contain subvalues of the 
form $\lambda x.e$ or $\mu f\lambda x.e$. Finally, in the rule $\textsc{MatchVal}$ we assume, that at most one pattern matches the 
scrutinee~--- this is an important distinction with usual semantics of pattern matching, when the patterns are examined in a top-down 
manner until the match succeeds.

Finally, for a closed expression $e$ and a value $v$ we write $\sembr{e}=v$, iff 

$$\inbr{\epsilon,e}\to^*\inbr{\epsilon,v}$$

\noindent where $\epsilon$~--- an empty stack, and ``$\to^*$'' is a reflexive-transitive closure of ``$\to$''.

\begin{figure}[t]
\centering
{\bf Values:}
$$
\mathcal V = \lstinline|true|\mid\lstinline|false|\mid C^n(v_1,\dots,v_n)\mid\lambda x.e\mid\mu f\lambda x.e
$$
{\bf Contexts:}
$$
\mathcal C = \Box\;e\mid v\;\Box\mid\lstinline|let $x$ = $\Box$ in $e$|\mid\lstinline|match $\;\Box\;$ with $\{p_i$->$e_i\}$|\mid C^n(\bar{v},\Box,\bar{e})\mid\Box=e\mid v=\Box 
$$
$$
C[e]\mbox{\supp{~--- a context $C$ with an expression $e$ plugged into a hole}}
$$
{\bf Stack of contexts:}
$$
\mathcal S=\epsilon\mid\mathcal C : \mathcal S
$$
{\bf States:}
$$
\inbr{\mathcal S, e}\mbox{\supp{(stack of contexts, expression)}};\;\inbr{\epsilon,e}\mbox{\supp{(initial state)}};\;\inbr{\epsilon,v}\mbox{\supp{(final state)}}
$$
{\bf Transitions:}
\vskip2mm
\bgroup
\def\arraystretch{0}
\begin{tabular}{p{7cm}p{7cm}}
\multicolumn{2}{p{14cm}}{
$$
\step{C:\mathcal S,\, v}{\mathcal S,\, C[v]}\ruleno{Value}
$$}\\
$$
\step{\mathcal S,\, f\;e}{\Box\;e:\mathcal S,\, f}\ruleno{AppL}
$$&
$$
\step{\mathcal S,\, v\;e_2}{v\;\Box:\mathcal S,\, e_2}\ruleno{AppR}
$$\\
$$
\step{\mathcal S,\,e_1=e_2}{\Box=e_2:\mathcal S,\,e_1}\ruleno{EqL}
$$&
$$
\step{\mathcal S,\,v=e}{v=\Box:\mathcal S,\,e}\ruleno{EqR}
$$\\
\multicolumn{2}{p{14cm}}{
$$
\step{C:\mathcal S,\,v=v}{\mathcal S,\,C[\lstinline|true|]}\ruleno{EqTrue}
$$}\\
\multicolumn{2}{p{14cm}}{
$$
\step{C:\mathcal S,\,v_1=v_2}{\mathcal S,\,C[\lstinline|false|]},\;v_1\ne v_2\ruleno{EqFalse}
$$}\\
\multicolumn{2}{p{14cm}}{
$$
\step{\mathcal S,\, (\lambda x.e)\;v}{\mathcal S,\, e[x\gets v]}\ruleno{Beta}
$$}\\
\multicolumn{2}{p{14cm}}{
$$
\step{\mathcal S,\, (\mu f\lambda x.e)\;v}{\mathcal S,\, e[f\gets\mu f\lambda x.e,\, x\gets v]}\ruleno{Mu}
$$}\\
\multicolumn{2}{p{14cm}}{
$$
\step{\mathcal S,\, C^n(v_1,\dots,v_{k-1},e_k,\dots,e_n)}{C^n(v_1,\dots,v_{k-1},\Box,\dots,e_n):\mathcal S,\, e_k}\ruleno{Constr}
$$}\\
\multicolumn{2}{p{14cm}}{
$$
\step{\mathcal S,\, \lstinline|let $\;x\;$ = $\;e_1\;$ in $\;e_2$|}{\lstinline|let $\;x\;$ = $\;\Box\;$ in $\;e_2$|:\mathcal S,\, e_1}\ruleno{Let}
$$}\\
\multicolumn{2}{p{14cm}}{
$$
\step{\mathcal S,\, \lstinline|let $\;x\;$ = $\;v\;$ in $\;e$|}{\mathcal S,\,e[x\gets v]}\ruleno{LetVal}
$$}\\
\multicolumn{2}{p{14cm}}{
$$
\step{\mathcal S,\, \lstinline|let rec $\;f\;$ = $\;\lambda x.e_1\;$ in $\;e_2$|}{\mathcal S,\, e_2[f\gets\mu f\lambda x.e_1]}\ruleno{LetRec}
$$}\\
\multicolumn{2}{p{14cm}}{
$$
\step{\mathcal S,\,\lstinline|match $\;e\;$ with $\;\{p_i$->$e_i\}$|}{\lstinline|match $\;\Box\;$ with $\;\{p_i$->$e_i\}$|:\mathcal S,\, e}\ruleno{Match}
$$}\\
\multicolumn{2}{p{14cm}}{
$$
\step{\mathcal S,\,\lstinline|match $\;C_k^{n_k}(v_1,\dots,v_{n_k})\;$ with $\;\{C_i^{n_i}(x^i_1,\dots,x^i_{n_i})\to e_i\}$|}{\mathcal S,\,e_k[x^k_j\gets v_j]}\ruleno{MatchVal}
$$}
\end{tabular}
\egroup
\caption{Semantics for the source language}
\label{functional_semantics}
\end{figure}

\FloatBarrier

\subsection{Relational Extension}
\label{relational_extension}

The relational extension adds five conventional miniKanren expressions for constructing goals; the syntax is shown on Figure~\ref{relational_syntax}.
Since relational constructs are added to regular functional ones, it becomes possible to construct expressions like \lstinline|fun x.x = (true === fun y.y)| etc.
In order to rule such pathological expressions out we devised an extension for the type system of source language. In fact, this approach follows the
actual implementation for OCaml, where a careful choice of types for representing terms and goals made it possible to reject the majority of non-well-formed
programs at compile-time.

Our extension for the type system introduces one interpreted datatype constructor $\Box^o$ with one data constructor $\uparrow$~--- a polymorphic type and
a constructor for logical terms. In addition we introduce interpreted type of goals $\G$, which is distinct from all other types. The typing rules for the relational 
extension are shown on Figure~\ref{relational_typing}. These rules describe rather expected typing: in unification and disequality constraints only
terms of the same logical type can be used, and conjuction and disjunction can only be taken for goals. Note, in our extension a term can be calculated as
a result of arbitrary expression in initial functional language (as long as this expression has expected logical type), but such ``higher-order'' terms will
never appear as a result of relational conversion, so, in fact, relational extension we describe here defines a richer language, than we actually need.

\begin{wrapfigure}{r}{0.5\textwidth}
\centering
$$
\begin{array}{rl}
   e\mathrel{{+}{=}}&\lstinline|fresh ($x$) $\;e$|\\
                    &e_1\equiv e_2\\
                    &e_1\not\equiv e_2\\
                    &e_1\vee e_2\\
                    &e_1\wedge e_2
\end{array}
$$
\caption{Syntax of the relational extension}
\label{relational_syntax}
\end{wrapfigure}

\setarrow{:}
\begin{figure}[t]
\centering
{\bf Types:}
$$
\begin{array}{rcl}
 \mathcal L &=               &\alpha^o \mid\lstinline|bool|^o\mid (T^n(l_1,\dots,l_n))^o\\
 \mathcal T &\mathrel{{+}{=}}&\G
\end{array}
$$
{\bf Typing rules:}
\def\arraystretch{0}
\begin{tabular}{p{7cm}p{7cm}}
\multicolumn{2}{p{14cm}}{
$$
\trule{\typed{\Gamma,x:l}{e}{\G}}
      {\typed{\Gamma}{\lstinline|fresh ($x$) $\;e$|}{\G}}
\ruleno{Fresh$_T$}
$$}\\
$$
\trule{\typed{\Gamma}{e_1}{l}\;\;\;\;\typed{\Gamma}{e_2}{l}}
      {\typed{\Gamma}{e_1\equiv e_2}{\G}}
\ruleno{Unify$_T$}
$$&
$$
\trule{\typed{\Gamma}{e_1}{l}\;\;\;\;\typed{\Gamma}{e_2}{l}}
      {\typed{\Gamma}{e_1\not\equiv e_2}{\G}}
\ruleno{Disequality$_T$}
$$\\
$$
\trule{\typed{\Gamma}{e_1}{\G}\;\;\;\;\typed{\Gamma}{e_2}{\G}}
      {\typed{\Gamma}{e_1\wedge e_2}{\G}}
\ruleno{Conjunction$_T$}
$$&
$$
\trule{\typed{\Gamma}{e_1}{\G}\;\;\;\;\typed{\Gamma}{e_2}{\G}}
      {\typed{\Gamma}{e_1\vee e_2}{\G}}
\ruleno{Disjunction$_T$}
$$
\end{tabular}
\caption{Typing rules for the relational extension}
\label{relational_typing}
\end{figure}

\setarrow{\leadsto}
\def\arraystretch{0}
\begin{figure}[t]
\centering
{\bf Semantic variables:}
\begin{gather*}
\mathfrak S = \mathfrak s_1, \mathfrak s_2, \dots\\[-2mm]
\Sigma, \Sigma^\prime\dots \subset 2^{\mathcal S}\;\mbox{\supp{(sets of allocated semantics variables)}}\\[-1mm]
\inbr{\Sigma^\prime, \mathfrak s}\gets\lstinline|new|\;\Sigma,\;\Sigma^\prime=\Sigma\cup\{\mathfrak s\},\;{\mathfrak s}\notin\Sigma\;\mbox{\supp{(allocation of a new semantic variable)}}
\end{gather*}
{\bf Values:}
$$
\mathcal V \mathrel{{+}{=}} \lstinline|success|\mid\mathfrak s
$$
{\bf Contexts:}
$$
\mathcal C \mathrel{{+}{=}}\Box\equiv e\mid v\equiv\Box\mid\Box\not\equiv e\mid v\not\equiv\Box\mid\Box\wedge e\mid e\wedge\Box
$$
{\bf States:}
\begin{gather*}
\inbr{\Sigma,\mathcal S,e,\sigma}\mbox{\supp{(set of allocated semantic variables, stack of contexts, expression, logical state)}}\\[-1mm]
\inbr{\emptyset,\epsilon,e,\iota}\mbox{\supp{(initial state)}}
\end{gather*}
{\bf Transitions:}
{\def\arraystretch{0}
\begin{tabular}{p{14cm}}
$$
\step{\Sigma,\,\mathcal S,\,\lstinline|fresh($x$) $\;e$|,\,\sigma}{\Sigma^\prime,\,\mathcal S,\,e[x\gets\mathfrak s],\,\sigma},\,\inbr{\Sigma^\prime,\mathfrak s}\gets\lstinline|new|\;\Sigma\ruleno{Fresh}
$$\\
$$
\step{\Sigma,\,\mathcal S,\,e_1\equiv e_2,\,\sigma}{\Sigma,\,\Box\equiv e_2:\mathcal S,\,e_1,\,\sigma}\ruleno{UnifyL}
$$\\
$$
\step{\Sigma,\,\mathcal S,\,v\equiv e,\,\sigma}{\Sigma,\,v\equiv\Box:\mathcal S,\,e,\,\sigma}\ruleno{UnifyR}
$$\\
$$
\step{\Sigma,\,\mathcal S,\,v_1\equiv v_2,\,\sigma}{\Sigma,\,\mathcal S,\,\lstinline|success|,\,\sigma^\prime},\,{\bf unify}\,(\sigma,\,v_1,\,v_2)=\sigma^\prime\ruleno{Unify}
$$\\
$$
\step{\Sigma,\,\mathcal S,\,e_1\not\equiv e_2,\,\sigma}{\Sigma,\,\Box\not\equiv e_2:\mathcal S,\,e_1,\,\sigma}\ruleno{DisEqL}
$$\\
$$
\step{\Sigma,\,\mathcal S,\,v\not\equiv e,\,\sigma}{\Sigma,\,v\not\equiv\Box:\mathcal S,\,e,\,\sigma}\ruleno{DisEqR}
$$\\
$$
\step{\Sigma,\,\mathcal S,\,v_1\not\equiv v_2,\,\sigma}{\Sigma,\,\mathcal S,\,\lstinline|success|,\,\sigma^\prime},\,{\bf diseq}\,(\sigma,\,v_1,\,v_2)=\sigma^\prime\ruleno{DisEq}
$$\\
$$
\step{\Sigma,\,\mathcal S,\,e_1\vee e_2,\,\sigma}{\Sigma,\,\mathcal S,\,e_1,\,\sigma}\ruleno{DisjL}
$$\\
$$
\step{\Sigma,\,\mathcal S,\,e_1\vee e_2,\,\sigma}{\Sigma,\,\mathcal S,\,e_2,\,\sigma}\ruleno{DisjR}
$$\\
$$
\step{\Sigma,\,\mathcal S,\,e_1\wedge e_2,\,\sigma}{\Sigma,\,\Box\wedge e_2:\mathcal S,\,e_1,\,\sigma}\ruleno{ConjStartL}
$$\\
$$
\step{\Sigma,\,\mathcal S,\,e_1\wedge e_2,\,\sigma}{\Sigma,\,e_1\wedge\Box:\mathcal S,\,e_2,\,\sigma}\ruleno{ConjStartR}
$$\\
$$
\step{\Sigma,\,\mathcal S,\,\lstinline|success|\wedge e,\,\sigma}{\Sigma,\,\mathcal S,\,e,\,\sigma}\ruleno{ConjL}
$$\\
$$
\step{\Sigma,\,\mathcal S,\,e\wedge\lstinline|success|,\,\sigma}{\Sigma,\,\mathcal S,\,e,\,\sigma}\ruleno{ConjR}
$$
\end{tabular}}
\caption{Semantics for the relational extension}
\label{relational_semantics}
\end{figure}

The semantics of extended language is shown on Figure~\ref{relational_semantics}. First, the state is extended: besides the stack of contexts and
current expression it now contains a set of used \emph{semantic variables} $\Sigma$ and a \emph{logical state} $\sigma$. 
Semantic variables are allocated and substituted for syntactic logic variable occurrences when \lstinline|fresh| expression is evaluated 
(see rule \textsc{Fresh}). Logical states are affected when unification or disequality constraint is evaluated; we explain them
in details below. All existing rules for the initial language are considered rewritten to propagate newly added components of states unchanged.
Then, we modify the substitution to respect names, bounded in \lstinline|fresh| as well. 
Next, we consider two new kinds of values: a semantic variable and a special value \lstinline|success|. The former is a result of evaluation for
a free logic variable, the latter~--- the result of evaluation for a succeeded goal.

We also extend the definition of context to handle new kinds of expressions. In unification and disequality constraint the terms are evaluated left-to right.
Conjunction and disjunction, however, evaluate nondeterministically: in disjunction only one subgoal is chosen (see rules \textsc{DisjL} and \textsc{DisjR}),
a conjunction can evaluate either left, or right subgoal first (see rules \textsc{ConjStartL} and \textsc{ConjStartR}). When chosen subgoal is evaluated
to a value \lstinline|success|, the other subgoal starts its evaluation (rules \textsc{ConjL} and \textsc{ConjR}).
We have chosen a nondeterministic variant for the semantics since different existing miniKanren implementations use (a little bit) different search, and we do 
not want to depend on implementation details. An opposite side of this solution is that for a concrete program and a concrete miniKanren implementation 
the result of evaluation might not coincide with that, prescribed by the semantics: in concrete implementation a program can diverge, while
nondeterministic semantics may still define a certain scenario to complete with a result. We argue, that in this case it will always be possible to
rewrite a program or/and interpreter to converge according to that scenario.

Finally, we describe the structure of a logical state and the implementation of unification and disequality constraint. The development is mainly based on the existing implementation~\cite{CKanren} and standard approaches for implementing unification~\cite{Unification}. We therefore assume the familiarity of the reader with the following notions:

\begin{itemize}
  \item substitution ($\theta$);
  \item application of substitution $\theta$ to a term $t$ ($t\,\theta$);
  \item composition of substitutions ($\theta\theta^\prime$);
  \item most general unifier of two terms ($mgu\,(t_1, t_2)$).
\end{itemize} 

As it can be seen from the semantics and typing rules, a unification (or disequality constraint) can only
be applied to equally-typed logical values, and we consider substitutions to be partial functions from
semantic variables ($\mathfrak S$) to logical values.

A logical state contains two components

$$
\sigma=(\theta,\Theta^-)
$$

\noindent where $\theta$ is a substitution, $\Theta^-$~--- a set of substitutions describing disequality constraints, 
which can potentially be violated. The initial state contains undefined substitution and empty set:

$$
\iota=(\bot,\emptyset)
$$

The effect of unification is described by the following primitive:

$$
{\bf unify}\,(\sigma,\,t_1,\,t_2)={\bf unify}\,((\theta,\Theta^-),\,t_1,\,t_2)
$$

First, it calculates the most general unifier for the terms under consideration w.r.t. current substitution:

$$
\rho=mgu\,(t_1\,\theta,t_2\,\theta)
$$

If there is no such $\rho$ the unification fails, and the evaluation terminates unsuccessfully. Otherwise,
$\rho$ has to be checked against disequality constraints, represented by $\Theta^-$ (if $\Theta^-$ is empty, the
check succeeds immediately).

Being a substitution, $\rho$ at the same time can be considered as the following unification problem: we can try to unify a pair of terms 

$$
\begin{array}{rcl}
t_l&=&(\mathfrak s_1,\dots,\mathfrak s_k)\\
t_r&=&(\rho(\mathfrak s_1),\dots,\rho(\mathfrak s_k))
\end{array}
$$

\noindent where $\{\mathfrak s_i\}=dom\,(\rho)$. We pick every substitution $\theta^-\in\Theta^-$ and calculate 
$mgu\,(t_l\,\theta^-,t_r\,\theta^-)$. There are three possible outcomes:

\begin{enumerate}
\item The unification fails. This means, that disequality constraint, represented by $\theta^-$, can no
longer be violated. We remove $\theta^-$ from $\Theta^-$ and continue with the next disequality constraint.
\item The unification succeeds with an empty substitution. This means, that the
disequality constraint, represented by $\theta^-$, is violated. The check stops, and the whole top-level 
unification fails.
\item The unification succeeds with a non-empty substitution $\theta^{\prime-}$. This means, that in order not to 
voilate the disequality constraint, represented by $\theta^-$, $\theta^{\prime-}$ has to be respected. We replace
$\theta^-$ with $\theta^{\prime-}$ in $\Theta^-$ and continue with the next disequality constraint.
\end{enumerate}

If the disequality check succeeds, by the end we have a modified set $\Theta^{\prime-}$, and we assume

$$
{\bf unify}\,((\theta,\Theta^-),\,t_1,\,t_2)=(\theta\rho,\Theta^{\prime-})
$$

The evaluation of disequality constraint is performed in a similar manner using the primitive

$$
{\bf diseq}\,(\sigma,\,t_1,\,t_2)={\bf diseq}\,((\theta,\Theta^-),\,t_1,\,t_2)
$$

First, an $mgu\,(t_1\,\theta,t_2\,\theta)$ is calculated. Again, there are three
possible cases:
\FloatBarrier

\begin{enumerate}
\item The unification fails. This means, that disequality constraint succeeds.
\item The unification succeeds with an empty substitution. This means, that disequality
constraint fails.
\item The unification succeeds with a non-empty substitution $\theta^{\prime-}$. This means, that 
this substitution describes the disequality constraint, which have to be respected in
the future, so we add it to $\Theta^-$. 
\end{enumerate}

If disequality constraint succeeds, we obtain (potentially) modified set $\Theta^{\prime-}$, and we
assume

$$
{\bf diseq}\,((\theta,\Theta^-),\,t_1,\,t_2)=(\theta,\Theta^{\prime-})
$$

Finally, for a closed goal $g$ and a logical state $\sigma$ we define $\sembr{g}^r=\sigma$, iff

$$
\inbr{\emptyset,\epsilon,g,\iota}\leadsto^*\inbr{\Sigma,\epsilon,\lstinline|success|,\sigma}\mbox{ for some $\Sigma$}
$$
 
\noindent where ``$\leadsto^*$'' is a reflexive-transitive closure of ``$\leadsto$''. 

%\subsection{Adequacy of Relational Semantics}





\section{Denotational Semantics}
\label{denotational}

To motivate further development we first consider the following example. Let us have the following goal:

\begin{lstlisting}
   x === Cons (y, z)
\end{lstlisting}

There are three free variables, and solving the goal delivers us the following single answer:

\begin{lstlisting}
   $\alpha\mapsto\;$ Cons ($\beta$, $\gamma$)
\end{lstlisting}

where semantic variables $\alpha$, $\beta$ and $\gamma$ correspond to the syntactic ones ``\lstinline|x|'', ``\lstinline|y|'', ``\lstinline|z|''. The
goal does not put any constraints on ``\lstinline|y|'' and ``\lstinline|z|'', so there are no bindings for ``$\beta$'' and ``$\gamma$'' in the answer.
This answer can be seen as the following ternary relation over the set of all ground terms:

\[
\{(\mbox{\lstinline|Cons ($\beta$, $\,\gamma$)|}, \beta, \gamma) \mid \beta\in\mathcal{D},\,\gamma\in\mathcal{D}\}\subset\mathcal{D}^3
\]

The order of ``dimensions'' is important since each dimension corresponds to a certain free variable. Our main idea is to represent this relation as a set of functions 

\[
\mathfrak{f}:\mathcal{A}\to\mathcal{D}
\]

from semantic variables to ground terms. We call these functions \emph{representing functions}. Thus, we may reformulate the same relation as

\[
\{(\mathfrak{f}\,(\alpha),\mathfrak{f}\,(\beta),\mathfrak{f}\,(\gamma))\mid\mathfrak{f}\in\sembr{\mbox{\lstinline|$\alpha$ === Cons ($\beta$, $\,\gamma$)|}}\}
\]

where we use conventional semantic brackets ``$\sembr{\bullet}$'' to denote the semantics. Now we implement this idea.

First, for a representing function

\[
\mathfrak{f} : \mathcal{A}\to\mathcal{D}
\]

we introduce its homomorphic extension 

\[
  \overline{\mathfrak{f}}:\mathcal{T_A}\to\mathcal{D}
\]

which maps terms to terms:

\[
\begin{array}{rcl}

  \overline{\mathfrak f}\,(\alpha) & = & \mathfrak f\,(\alpha)\\
  \overline{\mathfrak f}\,(C_i^{k_i}\,(t_1,\dots.t_{k_i})) & = & C_i^{k_i}\,(\overline{\mathfrak f}\,(t_1),\dots \overline{\mathfrak f}\,(t_{k_i}))
\end{array}
\]

Then, the semantic function for goals is parameterized over environments which prescribe semantic functions to relational symbols:

\[
  \Gamma : \mathcal{R} \to (\mathcal{T_A}^*\to 2^{\mathcal{A}\to\mathcal{D}})
\]

An environment associates with a relational symbol a function which takes a string of terms (the arguments of the relation) and returns a set of
representing functions. The signature for semantic brackets for goals is as follows:

\[
\sembr{\bullet}_{\Gamma} : \mathcal{G}\to 2^{\mathcal{A}\to\mathcal{D}}
\]

It maps a goal into the set of representing functions w.r.t. an environment $\Gamma$.

We formulate the following important \emph{coverage condition} for the semantics of a goal $g$:

\[
\forall\alpha\not\in FV\,(g)\; :\; \{\mathfrak{f}\,(\alpha)\mid\mathfrak{f}\in\sembr{g}_\Gamma\}=\mathcal{D}\eqno{(\star)}
\]

Informally, it means that the values of all representing functions on each non-free variable of $g$ covers $\mathcal{D}$. In other words, representing
functions restrict only the values of free variables. This condition guarantees that our semantics is complete in the sense that it does not
introduce artificial restrictions for the relation it defines.

We remind conventional notions of pointwise modification of a function

\[
f\,[x\gets v]\,(z)=\left\{
\begin{array}{rcl}
  f\,(z) &,& z \ne x \\
  v      &,& z = x
\end{array}
\right.
\]

and substitution of a free variable with a term in terms and goals (see Figure~\ref{substitution}).

\begin{figure}[t]
\[
\begin{array}{rcll}
  x\,[t/x] &=& t &\\
  y\,[t/x] &=& y & y\ne x\\
  C_i^{k_i}\,(t_1,\dots,t_{k_i})\,[t/x]&=&C_i^{k_i}\,(t_1\,[t/x],\dots,t_{k_i}\,[t/x])&\\[2mm]
  (t_1 \equiv t_2)\,[t/x]&=&t_1\,[t/x] \equiv t_2\,[t/x]&\\
  (g_1 \wedge g_2)\,[t/x]&=&g_1\,[t/x] \wedge g_2\,[t/x]&\\
  (g_1 \vee g_2)\,[t/x]&=&g_1\,[t/x] \vee g_2\,[t/x]&\\
  (\mbox{\lstinline|fresh|}\;x\,.\,g)\,[t/x]&=&\mbox{\lstinline|fresh|}\;x\,.\,g&\\
  (\mbox{\lstinline|fresh|}\;y\,.\,g)\,[t/x]&=&\mbox{\lstinline|fresh|}\;y\,.\,(g\,[t/x])&y\ne x\\
  (R_i^{k_i}\,(t_1,\dots,t_{k_i})\,[t/x]&=&R_i^{k_i}\,(t_1\,[t/x],\dots,t_{k_i}\,[t/x])&
\end{array}
\]
  \caption{Substitutions for terms and goals}
  \label{substitution}
\end{figure}

For a representing function $\mathfrak{f}:\mathcal{A}\to\mathcal{D}$ and a semantic variable $\alpha$ we define
the following \emph{generalization} operation:

\[
\mathfrak{f}\uparrow\alpha = \{ \mathfrak{f}\,[\alpha\gets d] \mid d\in\mathcal D\}
\]

Informally, this operation generalizes a representing function into a set of representing functions in such a way that the
values of these functions for the given variable cover the whole $\mathcal{D}$. We extend the generalization operation for the sets of
representing functions $\mathfrak{F}\subseteq\mathcal{A}\to\mathcal{D}$:

\[
  \mathfrak{F}\uparrow\alpha = \bigcup_{\mathfrak{f}\in\mathfrak{F}}(\mathfrak{f}\uparrow\alpha)
\]

Now we are ready to specify the sematics for goals (see Figure~\ref{denotational_semantics_of_goals}).

\begin{figure}[t]
  \[
  \begin{array}{cclr}
    \sembr{t_1\equiv t_2}_\Gamma&=&\{\mathfrak f : \mathcal{A}\to\mathcal{D}\mid \overline{\mathfrak{f}}\,(t_1)=\overline{\mathfrak{f}}\,(t_2)\}& \mbox{[\textsc{Unify$_D$}]}\\
    \sembr{g_1\wedge g_2}_\Gamma&=&\sembr{g_1}_\Gamma\cap\sembr{g_1}_\Gamma&\mbox{[\textsc{Conj$_D$}]}\\
    \sembr{g_1\vee g_2}_\Gamma&=&\sembr{g_1}_\Gamma\cup\sembr{g_1}_\Gamma&\mbox{[\textsc{Disj$_D$}]}\\
    \sembr{\mbox{\lstinline|fresh|}\,x\,.\,g}_\Gamma&=&(\sembr{g\,[\alpha/x]}_\Gamma)\uparrow\alpha,\;\alpha\not\in FV(g)& \mbox{[\textsc{Fresh$_D$}]}\\
    \sembr{R\,(t_1,\dots,t_k)}_\Gamma&=&(\Gamma\,R)\,t_1\dots t_k & \mbox{[\textsc{Invoke$_D$}]}
  \end{array}
  \]
  \caption{Denotational semantics of goals}
  \label{denotational_semantics_of_goals}
\end{figure}

\begin{figure}[t]
  \begin{gather*}
    \sembr{\{R_i=\lambda\,x_1^i\dots x_{k_i}^i\,.\,g_i\}_{i=1}^n\;g}=\sembr{g}_{\Gamma_{fix}}\\[2mm]
    \Gamma_{fix}\,R_i=(fix\;\mathcal{F})\,[i]\\[2mm]
    \mathcal{F} : (\mathcal{T_A}^*\to 2^{\mathcal{A}\to\mathcal{D}})^n\to (\mathcal{T_A}^*\to 2^{\mathcal{A}\to\mathcal{D}})^n\\[2mm]
    \begin{array}{rcl}
      \mathcal{F}\,(p_1,\dots,p_n)& = &(t^1_1\dots t^1_{k_1}\mapsto\sembr{g^1\,[t^1_1/x^1_1,\dots,t^1_{k_1}/x^1_{k_1}]}_\Gamma,\\
                                  &  &\phantom{(}\dots\\
                                  &  &\phantom{(}t^n_1\dots t^n_{k_n}\mapsto\sembr{g^n\,[t^n_1/x^n_1,\dots,t^n_{k_n}/x^n_{k_n}]}_\Gamma)
    \end{array}\\
    \mbox{where}\;\Gamma\, R_i=p_i
  \end{gather*}
  \caption{Denotational semantics of specifications}
  \label{denotational_semantics_of_relations}
\end{figure}

\subsection{Semantics for goals}

\[ V = \{ \alpha_1, \alpha_2, \dots \} \]

\[ \llbracket g \rrbracket_{\Gamma} \in 2^{V \to D} \]

Where $(f \in \llbracket g \rrbracket_{\Gamma}) \land (\alpha \not\in FV(g)) \Rightarrow \forall d \in D, \; f[\alpha \to d] \in \llbracket g \rrbracket_{\Gamma}$

\[\]

\[ \llbracket t_1 \equiv t_2 \rrbracket_{\Gamma} = \{ f \mid \overline{f}(t_1) = \overline{f}(t_2) \} \]

\[ \llbracket g_1 \lor g_2 \rrbracket_{\Gamma} = \llbracket g_1 \rrbracket_{\Gamma} \cup \llbracket g_2 \rrbracket_{\Gamma} \]

\[ \llbracket g_1 \land g_2 \rrbracket_{\Gamma} = \llbracket g_1 \rrbracket_{\Gamma} \cap \llbracket g_2 \rrbracket_{\Gamma} \]

\[ \llbracket fresh \, (x, g) \rrbracket_{\Gamma} = \{ f[\alpha \to d] \, \mid \, f \in \llbracket g[\bigslant{\alpha}{x}] \rrbracket_{\Gamma}, \; d \in D \}, \quad \alpha \not\in FV(g) \]

\[ \llbracket r^n \, t_1 \, \dots \, t_n \rrbracket_{\Gamma} = \Gamma(r^n)(t_1, \dots, t_n) \]

\[\]

\subsection{Semantics for relations}

\[ \begin{array}{l}
r_1^{k_1} = \lambda x_1 \dots x_{k_1}. \, g_1 \\
\dots \\
r_n^{k_n} = \lambda x_1 \dots x_{k_n}. \, g_n
\end{array} \]

\[ F : (T^{k_1} \to 2^{V \to D}) \times \dots \times (T^{k_1} \to 2^{V \to D}) \to (T^{k_1} \to 2^{V \to D}) \times \dots \times (T^{k_1} \to 2^{V \to D}) \]
\[ F(p_1, \dots, p_n) = (\lambda t_1 \dots t_{k_1}. \, \llbracket g_1[\bigslant{t_j}{x_j}] \rrbracket_{\Gamma}, \dots, \lambda t_1 \dots t_{k_n}. \, \llbracket g_n[\bigslant{t_j}{x_j}] \rrbracket_{\Gamma}), \quad \texttt{where} \quad \Gamma = r_i^{n_i} \mapsto p_i  \]

\[ \]

$2^{V \to D}$ --- complete lattice $\Rightarrow$

$(T^{k_1} \to 2^{V \to D}) \times \dots \times (T^{k_1} \to 2^{V \to D})$ --- complete lattice

\[ \Gamma_{fix} \stackrel{def}{=}  r_i^{k_i} \mapsto (fix \; F)_i \]

\[ \Gamma_{fix}(r_i^{k_i}) = \lambda t_1 \dots t_{k_i}. \, \llbracket g_i[\bigslant{t_j}{x_j}] \rrbracket_{\Gamma_{fix}}\]

\[ \llbracket g \rrbracket \stackrel{def}{=} \llbracket g \rrbracket_{\Gamma_{fix}} \]

\[\]


\section{Operational Semantics}
\label{operational}

In this section we describe operational semantics of \textsc{miniKanren}, which corresponds to the known
implementations with interleaving search. The semantics will be given in the form of labeled transition system (LTS). From now on we
assume the set of semantic variables to be linearly ordered ($\mathcal{A}=\{\alpha_1,\alpha_2,\dots\}$).

We introduce the notion of substitution

\[
  \sigma : \mathcal{A}\to\mathcal{T_A}
\]

as a (partial) mapping from semantic variables to terms over the set of semantic variables. We denote $\Sigma$ the
set of all substitutions, $\dom{\sigma}$~--- the domain for a substitution $\sigma$,
$\ran{\sigma}=\bigcup_{\alpha\in\mathcal{D}om\,(\sigma)}\fv{\sigma\,(\alpha)}$~--- its range (the set of all free variables in the image).

The states in the transition system have the following shape

\[
S = \mathcal{G}\times\Sigma\times\mathbb{N}\mid S\oplus S \mid S \otimes \mathcal{G}
\]

As we will see later, an evaluation of a goal is separated into elementary steps, and these steps are performed interchangeably for different subgoals. 
Thus, a state has a tree-like structure with intermediate nodes corresponding to partially-evaluated conjunctions (``$\otimes$'') or
disjunctions (``$\oplus$''). A leaf in the form $\inbr{g, \sigma, n}$ determines a goal in a context, where $g$~--- a goal, $\sigma$~--- a substitution accumulated so far,
and $n$~--- a natural number, which corresponds to a number of semantic variables used to this point. For a conjunction node its right child is always a goal since
it cannot be evaluated unless some result is provided by the left conjunct.

We also need extended states

\[
\overline{S} = \diamond \mid S
\]

where $\diamond$ symbolizes the end of evaluation, and the following well-formedness condition:

\begin{definition}
  Well-formedness condition for extended states:
  
  \begin{itemize}
  \item $\diamond$ is well-formed;
  \item $\inbr{g, \sigma, n}$ is well-formed iff $\fv{g}\cup\dom{\sigma}\cup\ran{\sigma}\subset\{\alpha_1,\dots,\alpha_n\}$;
  \item $s_1\oplus s_2$ is well-formed iff $s_1$ and $s_2$ well-formed;
  \item $s\otimes g$ is well-formed iff $s$ is well-formed and for all leaf triplets $\inbr{\_,\_,n}$ in $s$ $\fv{g}\subseteq\{\alpha_1,\dots,\alpha_n\}$.
  \end{itemize}
  
\end{definition}

Informally the well-formedness restricts the set of states to those in which all goals use only allocated variables.

Finally, we define the set of labels:

\[
L = \circ \mid \Sigma\times \mathbb{N}
\]

The label ``$\circ$'' is used to mark those steps which do not provide an answer; otherwise a transition is labeled by a pair of a substitution and a number of allocated
variables. The substitution is one of the answers, and the number is threaded through the derivation to keep track of allocated variables; we ignore it in further explanations.

\begin{figure}
  \[
  \begin{array}{cr}
    \inbr{t_1 \equiv t_2, \sigma, n} \xrightarrow{\circ} \Diamond , \, \, \nexists\; mgu\,(t_1, t_2, \sigma) &\ruleno{UnifyFail} \\[2mm]
    \inbr{t_1 \equiv t_2, \sigma, n} \xrightarrow{(mgu\,(t_1, t_2, \sigma),\, n)} \Diamond & \ruleno{UnifySuccess} \\[2mm]
    \inbr{g_1 \lor g_2, \sigma, n} \xrightarrow{\circ} \inbr{g_1, \sigma, n} \oplus \inbr{g_2, \sigma, n} & \ruleno{Disj} \\[2mm]
    \inbr{g_1 \land g_2, \sigma, n} \xrightarrow{\circ} \inbr{ g_1, \sigma, n} \otimes g_2 & \ruleno{Conj} \\[2mm]
    \inbr{\mbox{\lstinline|fresh|}\, x\, .\, g, \sigma, n} \xrightarrow{\circ} \inbr{g\,[\bigslant{\alpha_{n + 1}}{x}], \sigma, n + 1} & \ruleno{Fresh}\\[2mm]
    \dfrac{R_i^{k_i}=\lambda\,x_1\dots x_{k_i}\,.\,g}{\inbr{R_i^{k_i}\,(t_1,\dots,t_{k_i}),\sigma,n} \xrightarrow{\circ} \inbr{g\,[\bigslant{t_1}{x_1}\dots\bigslant{t_{k_i}}{x_{k_i}}], \sigma, n}} & \ruleno{Invoke}\\[5mm]
    \dfrac{s_1 \xrightarrow{\circ} \Diamond}{(s_1 \oplus s_2) \xrightarrow{\circ} s_2} & \ruleno{DisjStop}\\[5mm]
    \dfrac{s_1 \xrightarrow{r} \Diamond}{(s_1 \oplus s_2) \xrightarrow{r} s_2} & \ruleno{DisjStopAns}\\[5mm]
    \dfrac{s \xrightarrow{\circ} \Diamond}{(s \otimes g) \xrightarrow{\circ} \Diamond} &\ruleno{ConjStop}\\[5mm]
    \dfrac{s \xrightarrow{(\sigma, n)} \Diamond}{(s \otimes g) \xrightarrow{\circ} \inbr{g, \sigma, n}}  & \ruleno{ConjStopAns}\\[5mm]
    \dfrac{s_1 \xrightarrow{\circ} s'_1}{(s_1 \oplus s_2) \xrightarrow{\circ} (s_2 \oplus s'_1)} &\ruleno{DisjStep}\\[5mm]
    \dfrac{s_1 \xrightarrow{r} s'_1}{(s_1 \oplus s_2) \xrightarrow{r} (s_2 \oplus s'_1)} &\ruleno{DisjStepAns}\\[5mm]
    \dfrac{s \xrightarrow{\circ} s'}{(s \otimes g) \xrightarrow{\circ} (s' \otimes g)} &\ruleno{ConjStep}\\[5mm]
    \dfrac{s \xrightarrow{(\sigma, n)} s'}{(s \otimes g) \xrightarrow{\circ} (\inbr{g, \sigma, n} \oplus (s' \otimes g))} & \ruleno{ConjStepAns} 
  \end{array}
  \]
  \caption{Operational semantics of interleaving search}
  \label{lts}
\end{figure}

The transition rules are shown on Figure~\ref{lts}. The first two rules specify the semantics of unification. If two terms are not unifiable under the current substitution
$\sigma$ then the evaluation stops with no answer; otherwise it stops with the answer equal to the most general unifier.

The next two rules describe the steps performed when disjunction (conjunction) is encountered on the top level of the current goal. For disjunction it schedules both goals (using ``$\oplus$'') for
evaluating in the same context as the parent state, for conjunction~--- schedules the left goal and postpones the right one (using ``$\otimes$'').

The rule for ``\lstinline|fresh|'' substitutes bound syntactic variable with a newly allocated semantic one and proceeds with the goal; no answer provided at this step.

The rule for relation invocation finds a corresponding definition, substitutes its formal parameters with the actual ones, and proceeds with the body.

The rest of the rules specify the steps performed during the evaluation of two remaining types of the states~--- conjunction and disjunction. In all cases the left state
is evaluated first. If its evaluation stops with a result then the right state (or goal) is scheduled for evaluation, and the label is propagated. If there is no result then
the conjunction evaluation stops with no result (\textsc{ConjStop}) as well while the disjunction evaluation proceeds with the right state (\textsc{DisjStop}).

The last four rules describe \emph{interleaving}, which occurs when the evaluation of the left state suspends with some residual state (with or without an answer). In the case of disjunction
the answer (if any) is propagated, and the constituents of the disjunction are swapped (\textsc{DisjStep}, \textsc{DisjStepAns}). In case of conjunction, if the evaluation step in
the left conjunct did not provide any answer, the evaluation is continued in the same order since there is still no information to proceed with the evaluation of the right
conjunct (\textsc{ConjStep}); if there is some answer, then the disjunction of the right conjunct in the context of the answer and the remaining conjunction is
scheduled for evaluation (\textsc{ConjStepAns}).

The introduced transition system is completely deterministic. There was, however, some freedom in choosing the order of evaluation for conjunction and
disjunction states. For example, instead of evaluating the left substate first we could choose to evaluate the right one, etc. In each concrete case we would
end up with a different (but still deterministic) system which would prescribe different semantics to a concrete goal. This choice reflects the inherent
non-deterministic nature of search in relational (and, more generally, logical) programming. However, as long as deterministic search procedures
are sound and complete, we can consider them ``equivalent''\footnote{There still can be differences in observable behavior of concrete goals under different
sound and complete search strategies: a goal can be refutationally complete~\cite{WillThesis} under one strategy and non-complete under another.}.

A derivation sequence for a certain state determines a \emph{trace}~--- a finite or infinite sequence of answers. We may define a set of finite or infinite
sequences $X^\omega$ over an alphabet $X$ as a set of functions from natural numbers into a lifted set $X_\bot=X\cup\{\bot\}$:

\[
X^\omega=\{\omega : \mathbb{N}\to X_\bot\ \mid \forall n\in\mathbb{N},\, \omega\,(n)=\bot\Rightarrow \omega\,(n+1)=\bot\}
\]

Informally speaking, we represent a sequence as a function which maps positions (treated as natural numbers) into the elements of the sequence. We use ``$\bot$''
to specify that there is no element at given position, and we stipulate, that there are no ``holes'' in this representation: if there is no element at given
position then there are no elements at greater positions as well. 

For this representation we may define the empty sequence $\epsilon$ and operations of prepending a sequence $\omega$ with an element $a$ and taking a suffix of
a sequence $\omega$ from a position $n$ as follows:

\begin{gather*}
  \epsilon = i \mapsto \bot\\[2mm]
  a\omega = i \mapsto \left\{
  \begin{array}{rcl}
    a &,& i = 0\\
    \omega\,(i-1)&,&\mbox{otherwise}
  \end{array}
  \right.\\[2mm]
  \omega\,[n:]=i\mapsto\omega\,(n+i)
\end{gather*}

For a given state $s$ a trace $\tr{s}\in L^\omega$ is a sequence of labels, defined as follows simultaneously with the sequence of states $\{s_i\}$:

\[
\begin{array}{ccccl}
  \multicolumn{5}{c}{s_o=s}\\
  \tr{s}\,(n)=a &,& s_{n+1}=s'&\mbox{ if }& s_n\ne\diamond,\, s_n\xrightarrow{a} x'\\
  \tr{s}\,(n)=\bot&,&s_{n+1}=\diamond&\mbox{ if }& s_n=\diamond
\end{array}
\]

The trace corresponds to the stream of answers in the reference \textsc{miniKanren} implementations.

To formalize the operational part in \textsc{Coq} we first need to define all preliminary notions from unification theory~\cite{Unification} which our semantics uses.

In particular, we need to implement the notion of the most general unifier (MGU). As is it well-known~\cite{UnificationMcBride} all standard recursive algorithms for calculating
MGU are not decreasing on argument terms, so we can't define it as a simple recursive function in \textsc{Coq} due to the termination check. There is no such obstacle when we define
MGU as a proposition:

\begin{lstlisting}[language=Coq]
  Inductive MGU : term -> term -> option subst -> Set := ...
\end{lstlisting}

However, we still need to use a well-founded induction to prove the existence of the most general unifier and its defining properties:

\begin{lstlisting}[language=Coq]
  Lemma MGU_ex : forall t1 t2, {r & MGU t1 t2 r}.
  
  Definition unifier (s : subst) (t1 t2 : term) : Prop := apply_subst s t1 = apply_subst s t2.

  Lemma MGU_unifies:
    forall t1 t2 s, MGU t1 t2 (Some s) -> unifier s t1 t2.
  
  Definition more_general (m s : subst) : Prop :=
    exists (s' : subst), forall (t : term), apply_subst s t = apply_subst s' (apply_subst m t).

  Lemma MGU_most_general :
    forall (t1 t2 : term) (m : subst),
      MGU t1 t2 (Some m) ->
      forall (s : subst), unifier s t1 t2 -> more_general m s.

  Lemma MGU_non_unifiable :
    forall (t1 t2 : term),
      MGU t1 t2 None -> forall s,  ~ (unifier s t1 t2).
\end{lstlisting}

For this well-founded induction we use the number of free variables in argument terms as a well-founded order on pairs of terms:

\begin{lstlisting}[language=Coq]
  Definition terms := term * term.

  Definition fvOrder (t : terms) := length (union (fv_term (fst t)) (fv_term (snd t))).

  Definition fvOrderRel (t p : terms) := fvOrder t < fvOrder p.

  Lemma fvOrder_wf : well_founded fvOrderRel.
\end{lstlisting}

After this preliminary work, the described transition relation can be encoded naturally as an inductively defined proposition (here ``\lstinline|state'|''
stands for an extended state):

\begin{lstlisting}[language=Coq]
  Inductive eval_step : state -> label -> state' -> Set := ...
\end{lstlisting}

We state the fact that our system is deterministic through existence and uniqueness of a transition for every state:

\begin{lstlisting}[language=Coq]
  Lemma eval_step_ex : forall (st : state), {l : label & {st' : state' & eval_step st l st'}}.

  Lemma eval_step_unique :
    forall (st : state) (l1 l2 : label) (st'1 st'2 : state'),
      eval_step st l1 st'1 -> eval_step st l2 st'2 -> l1 = l2 /\ st'1 = st'2.
\end{lstlisting}

To work with (possibly) infinite sequences we use the standard approach in \textsc{Coq}~--- coinductively defined streams:

\begin{lstlisting}[language=Coq]
  Context {A : Set}.

  CoInductive stream : Set :=
  | Nil : stream
  | Cons : A -> stream -> stream.
\end{lstlisting}

Although the definition of the datatype is coinductive some of its properties we are working with make sense only when defined inductively:

\begin{lstlisting}[language=Coq]
  Inductive in_stream : A -> stream -> Prop :=
  | inHead : forall x t, in_stream x (Cons x t)
  | inTail : forall x h t, in_stream x t -> in_stream x (Cons h t).

  Inductive finite : stream -> Prop :=
  | fNil : finite Nil
  | fCons : forall h t, finite t -> finite (Cons h t).
\end{lstlisting}

Then we define a trace coinductively as a stream of labels in transition steps and prove that there exists a unique trace from any extended state:

\begin{lstlisting}[language=Coq]
  Definition trace : Set := $@$stream label.

  CoInductive op_sem : state' -> trace -> Set :=
  | osStop : op_sem Stop Nil
  | osState : forall st l st' t, eval_step st l st' ->
                            op_sem st' t ->
                            op_sem (State st) (Cons l t).

  Lemma op_sem_ex (st' : state') : {t : trace & op_sem st' t}.

  Lemma op_sem_unique :
    forall st' t1 t2, op_sem st' t1 -> op_sem st' t2 -> equal_streams t1 t2.
\end{lstlisting}

Note, for the equality of streams we need to define a new coinductive proposition instead of using the standard syntactic equality in order for coinductive proofs to work~\cite{CPDT}.

One thing we can prove using operational semantics is the \emph{interleaving} properties of disjunction. Specifically, we can prove that a trace for a disjunction is
a one-by-one interleaving of streams for its disjuncts:

\begin{lstlisting}[language=Coq]
  CoInductive interleave : stream -> stream -> stream -> Prop :=
  | interNil : forall s s', equal_streams s s' -> interleave Nil s s'
  | interCons : forall h t s rs, interleave s t rs -> interleave (Cons h t) s (Cons h rs).

  Lemma sum_op_sem : forall st1 st2 t1 t2 t, op_sem (State st1) t1 ->
                                        op_sem (State st2) t2 ->
                                        op_sem (State (Sum st1 st2)) t ->
                                        interleave t1 t2 t.
\end{lstlisting}

This allows us to prove the expected properties of interleaving in a more general setting of arbitrary streams:

\begin{itemize}
\item  the elements of the interleaved stream are exactly those of two interleaved streams;
\item  the interleaved stream is finite iff both interleaving streams are finite.
\end{itemize}

The corresponding \textsc{Coq} lemmas are as follows:

\begin{lstlisting}[language=Coq]
  Lemma interleave_in : forall s1 s2 s, interleave s1 s2 s ->
                   forall x, in_stream x s <-> in_stream x s1 \/ in_stream x s2.

  Lemma interleave_finite : forall s1 s2 s, interleave s1 s2 s ->
                   (finite s <-> finite s1 /\ finite s2).
\end{lstlisting}

\section{Equivalence of Semantics}
\label{equivalence}

Now when we defined two different kinds of semantics for \textsc{miniKanren} we can relate them and show that the results given by these two semantics are the same for any specification.
This will actually say something important about the search in the language: since operational semantics describes precisely the behavior of the search and denotational semantics
ignores the search and describes what we \emph{should} get from a mathematical point of view, by proving their equivalence we establish the \emph{completeness} of the search, which
means that the search will get all answers satisfying the described specification and only those.

But first, we need to relate the answers produced by these two semantics as they have different forms: a trace of substitutions (along with the numbers of allocated variables)
for the operational one and a set of representing functions for the denotational one. We can notice that the notion of representing function is close to substitution, with only two differences:

\begin{itemize}
\item representing function is total;
\item terms in the domain of representing function are ground.
\end{itemize}

Therefore we can easily extend (perhaps ambiguously) any substitution to a representing function by composing it with an arbitrary representing function preserving
all variable dependencies in the substitution. So we can define a set of representing functions that correspond to a substitution as follows:

\[
\sembr{\sigma} = \{\overline{\mathfrak f} \circ \sigma \mid \mathfrak{f}:\mathcal{A}\mapsto\mathcal{D}\}
\]

\begin{comment}
In \textsc{Coq} this notion boils down to the following definition:

\begin{lstlisting}[language=Coq]
   Definition in_denotational_sem_subst
     (s : subst) (f : repr_fun) : Prop :=
       exists (f' : repr_fun),
         repr_fun_eq (subst_repr_fun_compose s f') f.
\end{lstlisting}

where ``\lstinline[language=Coq]|repr_fun_eq|'' stands for representing functions extensional equality, ``\lstinline[language=Coq]|subst_repr_fun_compose|''~---
for a composition of a substitution and a representing function.
\end{comment}

And the \emph{denotational analog} of operational semantics (a set of representing functions corresponding to the answers in the trace) for a given state $\hat{s}$ is
then defined as the union of sets for all substitution in the trace:

\[
\sembr{\hat{s}}_{op} = \cup_{(\sigma, n) \in \tr{\hat{s}}} \sembr{\sigma}
\]

\begin{comment}
In \textsc{Coq} we again use a proposition instead:

\begin{lstlisting}[language=Coq]
   Definition in_denotational_analog
      (t : trace) (f : repr_fun) : Prop :=
      exists s n, in_stream (Answer s n) t /\
             in_denotational_sem_subst s f.
   Notation "{| t , f |}" := (in_denotational_analog t f).
\end{lstlisting}
\end{comment}

This allows us to state theorems relating the two semantics.

\begin{theorem}[Operational semantics soundness]
\label{lem:soundness}
If indices of all free variables in a goal $g$ are limited by some number $n$, then $\sembr{\inbr{g, \epsilon, n}}_{op} \subset \sembr{g}$.
\end{theorem}

It can be proven by nested induction, but first, we need to generalize the statement so that the inductive hypothesis would be strong enough for the inductive step.
To do so, we define denotational semantics not only for goals but for arbitrarily states. Note that this definition does not need to have any intuitive
interpretation, it is introduced only for the proof to go smoothly. The definition of the denotational semantics for extended states is shown on Fig.~\ref{denotational_semantics_of_states}.
The generalized version of the theorem uses it.

\begin{figure}[t]
  \[
  \begin{array}{ccl}
    \sembr{\Diamond}&=&\emptyset\\
    \sembr{\inbr{g, \sigma, n}}&=&\sembr{g}\cap\sembr{\sigma}\\
    \sembr{s_1 \oplus s_2}&=&\sembr{s_1}\cup\sembr{s_2}\\
    \sembr{s \otimes g}&=&\sembr{s}\cap\sembr{g}\\
  \end{array}
  \]
  \caption{Denotational semantics of states}
  \label{denotational_semantics_of_states}
\end{figure}

\begin{lemma}[Generalized soundness]
\label{lem:gen_soundness}
For any well-formed state $\hat{s}$

\[
\sembr{\hat{s}}_{op} \subset \sembr{\hat{s}}.
\]
\end{lemma}

It can be proven by the induction on the number of steps in which a given answer (more accurately, the substitution that contains it) occurs in the trace.
We break the proof in two parts and separately prove by induction on evidence that for every transition in our system the semantics of both the label (if there is one)
and the next state are subsets of the denotational semantics for the initial state.

\begin{lemma}[Soundness of the answer]
\label{lem:answer_soundness}
For any transition $s \xrightarrow{(\sigma, n)} \hat{s}$, \mbox{$\sembr{\sigma} \subset \sembr{s}$}.
\end{lemma}

\begin{lemma}[Soundness of the next state]
\label{lem:next_state_soundness}
For any transition $s \xrightarrow{l} \hat{s}$, \mbox{$\sembr{\hat{s}} \subset \sembr{s}$}.
\end{lemma}

It would be tempting to formulate the completeness of operational semantics as soundness with the inverted inclusion, but it does not hold in such generality.
The reason for this is that the denotational semantics encodes only the dependencies between free variables of a goal, which is reflected by the closedness condition,
while the operational semantics may also contain dependencies between semantic variables allocated in \lstinline|fresh| constructs. Therefore we formulate completeness
with representing functions restricted on the semantic variables allocated in the beginning (which includes all free variables of a goal). This does not
compromise our promise to prove the completeness of the search as \textsc{miniKanren} provides the result as substitutions only for queried variables,
which are allocated in the beginning.

\begin{theorem}[Operational semantics completeness]
%\label{lem:gen_completeness}
If the indices of all free variables in a goal $g$ are limited by some number $n$, then

\[
\{\mathfrak{f}|_{\{\alpha_1,\dots,\alpha_n\}} \mid \mathfrak{f} \in \sembr{g}\} \subset \{\mathfrak{f}|_{\{\alpha_1,\dots,\alpha_n\}} \mid \mathfrak{f} \in \sembr{\inbr{g, \epsilon, n}}_{op}\}.
\]
\end{theorem}

Similarly to the soundness, this can be proven by nested induction, but the generalization is required. This time it is enough to generalize it from goals
to states of the shape $\inbr{g, \sigma, n}$. We also need to introduce one more auxiliary semantics~--- \emph{a bounded denotational semantics}:

\[
\sembr{\bullet}^l : \mathcal{G} \to 2^{\mathcal{A}\to\mathcal{D}}
\]

Instead of always unfolding the definition of relation for invocation goal, it does so only given number of times. So for a given set of relational
definitions $\{R_i^{k_i} = \lambda\;x_1^i\dots x_{k_i}^i\,.\, g_i\}$ the definition of bounded denotational semantics is exactly the same as for the conventional denotational semantics, except that for the invocation case we have

\[
\sembr{R_i^{k_i}\,(t_1,\dots,t_{k_i})}^{l+1} = \sembr{g_i[t_1/x_1^i, \dots, t_{k_i}/x_{k_i}^i]}^{l}
\]

It is convenient to define bounded semantics for level zero as the empty set:

\[
\sembr{g}^{0} = \emptyset
\]

The bounded denotational semantics is an approximation of the conventional denotational semantics; it is clear that any answer in conventional denotational semantics will also be in bounded denotational semantics for some level.

\begin{lemma}
$\sembr{g} \subset \cup_l \sembr{g}^l$
\end{lemma}

Now the generalized version of the completeness theorem is as follows.

\begin{lemma}[Generalized completeness]
\label{lem:gen_completeness}
For any set of relational definitions, for any level $l$, for any well-formed state $\inbr{g, \sigma, n}$,

\[
\{\mathfrak{f}|_{\{\alpha_1,\dots,\alpha_n\}} \mid \mathfrak{f} \in \sembr{g}^l \cap \sembr{\sigma}\} \subset \{\mathfrak{f}|_{\{\alpha_1,\dots,\alpha_n\}} \mid \mathfrak{f} \in \sembr{\inbr{g, \sigma, n}}_{op}\}.
\]
\end{lemma}

The proof is by the induction on level $l$. The induction step is proven by structural induction on goal $g$. We use lemmas~\ref{lem:sum_answers} and~\ref{lem:prod_answers} for evaluation of a disjunction and a conjunction respectively, and lemma~\ref{lem:den_sem_change_var} in the case of fresh variable introduction to move from an arbitrary semantic variable in denotational semantics to the next allocated fresh variable. The details of this proof may be found in the Appendix~\ref{appendix_gen_completeness_proof}, the full proof script is in the specification in Coq.

\section{Specification in \textsc{Coq}}
\label{specification}

We certified all the definitions and propositions from the previous sections using the \textsc{Coq} proof assistant.\footnote{The specification is available at \url{https://github.com/dboulytchev/miniKanren-coq}} The \textsc{Coq} specification for the most part closely follows the formal descriptions we gave by means of inductive definitions (and inductively defined propositions in particular) and structural induction in proofs. The detailed description of the specification, including code snippets, is provided in the extended version of the paper, and in this section we address only some non-trivial parts of it and some design choices.

The language formalized in \textsc{Coq} has a few non-essential simplifications for the sake of convenience. Specifically, we restrict the arities of all constructors to be either zero or two and require all relations to have exactly one argument. These restrictions do not make the language less expressive in any way since we can always represent a sequence of terms as a list using constructors \lstinline|Nil$^0$| and \lstinline|Cons$^2$|. 

In our formalization of the language we use higher-order abstract syntax~\cite{HOAS} for variable binding, therefore we work explicitly only with semantic variables. We preferred it to the first-order syntax because it gives us the ability to use substitution and the induction principle provided by \textsc{Coq}. On the other hand, we need to explicitly specify a requirement on the syntax representation, which is trivially fulfilled in the first-order case: all bindings have to be ``consistent'', i.e. if we instantiate a higher-order \lstinline|fresh| construct with different semantic variables the results will be the same up to some renaming (provided that both those variables are not free in the body of the binder). Another requirement we have to specify explicitly (independent of HOAS/FOAS dichotomy) is a requirement that the definitions of relations do not contain unbound semantic variables.

To formalize the operational semantics in \textsc{Coq} we first need to define all preliminary notions from unification theory~\cite{Unification} which our semantics uses. In particular, we need to implement the notion of the most general unifier (MGU). As it is well-known~\cite{StructuralMGU} all standard recursive algorithms for calculating MGU are not decreasing on argument terms, so we can't define them as simple recursive functions in \textsc{Coq} due to the termination check failure. The standard approach to tackle this problem is to define the function through well-founded recursion. We use a distinctive version of this approach, which is more convenient for our purposes: we define MGU as a proposition (for which there is no termination requirement in \textsc{Coq}) with a dedicated structurally-recursive function for one step of unification, and then we use a well-founded induction to prove the existence of a corresponding result for any arguments and defining properties of MGU. For this well-founded induction, we use the number of distinct free variables in argument terms as a well-founded order on pairs of terms.

In the operational semantics, to define traces as (possibly) infinite sequences of transitions we use the standard approach in \textsc{Coq}~--- coinductively defined streams. Operating with them requires a number of well-known tricks, described by Chlipala~\cite{CPDT}, to be applied, such as the use of a separate coinductive definition of equality on streams.

The final proofs of soundness and completeness of operational semantics are relatively small, but the large amount of work is hidden in the proofs of auxiliary facts that they use (including lemmas from the previous sections and some technical machinery for handling representing functions).

\section{Application}

In this section we consider some applications of the framework and results, described in the previous sections.

\subsection{Correctness of Transformations}

One important immediate corollary of theorems we've proven is the justification of correctness for certain program transformations.
The completeness of interleaving search guarantees the correctness of any transformation which preserves the denotational semantics,
for example:

\begin{itemize}
\item changing the order of constituents in conjunctions and disjunctions;
\item distributing conjunctions over disjunctions and vice versa, for example, normalising goals info CNF or DNF;
\item moving fresh variable introduction upwards/downwards, for example, transforming any relation into a top-level fresh
  construct with a freshless body.
\end{itemize}

Note that this way we can guarantee only the preservation of results as \emph{sets of ground terms}; the other aspects of program behavior,
such as termination, may be affected by some of these transformations. 

As an example of these transformations application we consider the transformation from \textsc{Prolog} to conventional \textsc{miniKanren} and,
more interesting, in the opposite direction.

Since in \textsc{Prolog} a rather limited form of goals (an implicit conjunction of atoms) is used the conversion to \textsc{miniKanren} is easy:

\begin{itemize}
  \item built-in constructors in \textsc{Prolog} terms (for examples, for lists) are converted into \textsc{miniKanren} representation;
  \item each atom is converted into a corresponding relational invocation with the same parameters (modulo the conversion of terms);
  \item the list of atoms in the body of \textsc{Prolog} clause is converted into explicit conjunction;
  \item a number of fresh variables (one for each argument) is created and these variables are unified with terms in corresponding
    argument position in the head of corresponding clause;
  \item different clauses for the same predicate are combined using disjunction.
\end{itemize}

For example, consider the result of conversion from \textsc{Prolog} definition for list appending relation into \textsc{miniKanren}.
Like in \textsc{miniKanren}, the definition consists of two clauses. The first one is

\begin{lstlisting}
  append ([], X, X).
\end{lstlisting}

and the result of conversion is

\begin{lstlisting}
  append$^o$ = fun x y z .
    fresh X . x == Nil /\
              y == X   /\
              z == X
\end{lstlisting}

The second one is

\begin{lstlisting}
  append ([H$|$T], Y, [H$|$TY]) :- append (T, Y, TY).
\end{lstlisting}

which is convered into

\begin{lstlisting}
  append$^o$ = fun x y z .
    fresh H T Y TY . x == Cons (H, T)  /\
                     y == Y            /\
                     z == Cons (H, TY) /\
                     append$^o$ (T, Y, TY)
\end{lstlisting}

The overall result is not literally the same as what we've shown in Section~\ref{language}, but denotationally equivalent.

The conversion in the opposite direction involves the following steps:

\begin{itemize}
  \item converting between term representation;
  \item moving all ``\lstinline|fresh|'' constructs into the top-level;
  \item transforming the freshless body into DNF;
  \item replacing all unifications with calls for a specific predicate ``\lstinline|unify/2|'', defined as

    \begin{lstlisting}
      unify (X, X).
    \end{lstlisting}    

  \item splitting top-level disjunctions into separate clauses with the same head.
\end{itemize}

The correctness of these, again, can be justified denotationally. For the appendo relation in Section~\ref{language} the result
can be as follows:

\begin{lstlisting}
  append (X, Y, Z) :- unify (X, []), unify (Z, Y).
  append (X, Y, Z) :-
    unify (X, [H$|$T]),
    unify (Z, [H$|$TY]),
    append (T, Y, TY).
\end{lstlisting}

This transformations show that we can, for example, interpret \textsc{Prolog} specifications in interleaving semantics; moreover, we can,
using the certified framework we developed, describe conventional \textsc{Prolog} search strategies.

\begin{figure*}
\[
\begin{array}{cr|cr|cr|cr}
  \dfrac{s_1 \xrightarrow{\circ}_c \Diamond}{(s_1 \oplus s_2) \xrightarrow{\circ} \Diamond} & \ruleno{PlusStopC}&
  \dfrac{s_1 \xrightarrow{r}_c \Diamond}{(s_1 \oplus s_2) \xrightarrow{r} \Diamond} & \ruleno{PlusStopAnsC}&
  \dfrac{s_1 \xrightarrow{\circ}_c s'_1}{(s_1 \oplus s_2) \xrightarrow{\circ} s'_1} &\ruleno{PlusStepC}&
  \dfrac{s_1 \xrightarrow{r}_c s'_1}{(s_1 \oplus s_2) \xrightarrow{r} s'_1} &\ruleno{PlusStepAnsC}\\
  \dfrac{s_1 \xrightarrow{\circ}_c \Diamond}{(s_1 \circledast s_2) \xrightarrow{\circ}_c \Diamond} & \ruleno{AstStopC}&
  \dfrac{s_1 \xrightarrow{r}_c \Diamond}{(s_1 \circledast s_2) \xrightarrow{r}_c \Diamond} & \ruleno{AstStopAnsC}&
  \dfrac{s_1 \xrightarrow{\circ}_c s'_1}{(s_1 \circledast s_2) \xrightarrow{\circ}_c s'_1} &\ruleno{AstStepC}&
  \dfrac{s_1 \xrightarrow{r}_c s'_1}{(s_1 \circledast s_2) \xrightarrow{r}_c s'_1} &\ruleno{AstStepAnsC}
\end{array}
\]
\caption{Cut signal propagation rules}
\label{cut-signal-propagation}
\end{figure*}
%\FloatBarrier

\subsection{SLD Semantics}

The conventional for \textsc{Prolog} SLD search differs from the interleaving one in just one aspect~--- it does not perform interleaving.
Thus, changing just two rules in the operational semantics converts interleaving search into the depth-first one:

\[
  \begin{array}{cr}
    \dfrac{s_1 \xrightarrow{\circ} s'_1}{(s_1 \oplus s_2) \xrightarrow{\circ} (s'_1 \oplus s_2)} &\ruleno{DisjStep}\\
    \dfrac{s_1 \xrightarrow{r} s'_1}{(s_1 \oplus s_2) \xrightarrow{r} (s'_1 \oplus s_2)} &\ruleno{DisjStepAns}
  \end{array}
\]

With this definition we can almost completely reuse the mechanized proof of soundness (with just cosmetic changes); the completeness, however,
can no longer be proven (as it does not hold anymore).

\subsection{Cut}

Dealing with the ``cut'' construct is known to be a cornerstone feature in study of operational semantics for \textsc{Prolog}. It turned out that
in our case the semantics of ``cut'' can be expressed naturally (but a bit verbose). Unlike SLD-resolution, it does not amount to an incremental
change in semantics description. It also would work only for programs, directly converted from \textsc{Prolog} specifications.

The key observation in dealing with the ``cut'' in our settings is that a state in our semantics in fact encodes the whole current
search tree (including all backtracking possiblities). This opens the opportunity to organize proper ``navigation'' through the tree
to reflect the effect of ``cut''.

First, we add ``cut'' as a new sort of goals:

\begin{lstlisting}[language=Coq,basicstyle=\footnotesize]
  Inductive goal : Set := ... | Cut : goal.
\end{lstlisting}

In denotational semantics we interpret ``cut'' as success (thus, denotationally we treat all cuts as ``green''). Operationally, we
modify SLD semantics is such a way that a ``cut'' cuts all other branches of all enclosing nodes, marked with ``$\oplus$'', up to
the moment when the evaluation of the disjunct, containing the ``cut'', was started. It is easy to see that this node will always
be the nearest ``$\oplus$'', derived from the disjunction. Unfortunately, in the tree other ``$\oplus$'' nodes can
appear due to the evaluation of ``$\otimes$'' nodes, thus we need a way to distinguish these two sorts of ``$\oplus$''. We
denote the new sort of nodes as ``$\circledast$'', and modify the definition of states.

In the semantics the rule \textsc{[ConjStepAns]} is replaced with

\[
\begin{array}{cr}
  \dfrac{s \xrightarrow{(\sigma, n)} s'}{(s \otimes g) \xrightarrow{\circ} (\inbr{g, \sigma, n} \circledast (s' \otimes g))} & \ruleno{ConjStepAns} 
\end{array}
\]

The rules for ``$\circledast$'' evaluation mirrors those for ``$\oplus$'', so we omit them.

We need a separate kind of transitions to propagate the sugnal for cutting the branches $\xrightarrow{\circ}_c$/$\xrightarrow{(\sigma, n)}_c$.

The signal itself is risen when a ``cut'' construct is encountered:

\[
\begin{array}{cr}
  \inbr{!, \sigma, n} \xrightarrow{(\sigma, n)}_c \Diamond &\ruleno{Cut} 
\end{array}
\]

When the signal is being propagated through ``$\oplus$'' and ``$\circledast$'' nodes, their right branches are cut out, and for ``$\circledast$'' the
propagation continues (see Fig.~\ref{cut-signal-propagation}). In case of ``$\otimes$'' nodes the signal is simply propagated; we omit the rules since they mirrors the regular ones.

For this semantics we can repeat the proof of soundness w.r.t. to the denotational semantics. There is, however, a little subtlety with our construction:
we cannot formally prove, that our semantics indeed encodes the conventional meaning of ``cut''. Nevertheless we can demonstrate a plausible behavior
using extracted reference interpreter.


\subsection{Reference Interpreters}

Using \textsc{Coq} extraction mechanism, we extracted two reference interpreters from out definitions and theorems: one for conventional
\textsc{miniKanren} and another for SLD search with cut. These interpreters can be used to practically investigate the behaviour
of specifications in unclear, complex or corner cases. Our experience has shown that these interpreters demonstrate the expected behavior
in all cases.

\begin{comment}
  
Мы не можем доказать, что поведение нашего cut в точности соответствует поведению cut в пролог, для этого была бы нужна детальная семантика
Prolog с cut.

Мы однако можем продемонстрировать, что обрезаются ровно те ветки поиска, которые должны, сравнив результаты исполнения Prolog и нашей семантики на
достаточно полном примере.

\begin{lstlisting}[language=Prolog,basicstyle=\footnotesize,numbers=left,stepnumber=1]
  a(0).
  a(1).
  b(2, 0).
  b(X, Y) :- a(X), !, a(Y).
  b(3, 0).
  c(4, 0, 0, 0).
  c(W, X, Y, Z) :- a(W), b(X, Y), a(Z).
  c(5, 0, 0, 0).
\end{lstlisting}


(Для простоты мы используем здесь целые числа, для соответствия нашему синтаксису нужно представить их в форме Пеано или просто заменить их на различные конструкторы.)

В момент вычисления cut создастся ``отрезающий'' сигнал, в результате чего отрежется выбор второй альтернативы для a(X) в строке 5 --- это будет дополнительная ветка, подвешенная к $\circledast$ --- и альтернатива в строке 6 --- эта ветка (содержащая дизъюнкцию всех альтернатив после пятой строчки) подвешенная к $\oplus$ --- на которой распространение сигнала остановится.

Таким образом из строк 8-10 ничего не отрезается, как и положено.

Мы получаем те же ответы, что и Prolog, и в том же порядке:

\[
\begin{array}{|cccc|}
\hline
W & X & Y & Z \\
\hline
4 & 0 & 0 & 0 \\
0 & 2 & 0 & 0 \\
0 & 2 & 0 & 1 \\
0 & 0 & 0 & 0 \\
0 & 0 & 0 & 1 \\
0 & 0 & 1 & 0 \\
0 & 0 & 1 & 1 \\
1 & 2 & 0 & 0 \\
1 & 2 & 0 & 1 \\
1 & 0 & 0 & 0 \\
1 & 0 & 0 & 1 \\
1 & 0 & 1 & 0 \\
1 & 0 & 1 & 1 \\
5 & 0 & 0 & 0 \\
\hline
\end{array}
\]
\end{comment}

\section{Related Work}

The study of formal semantics for logic programming languages, particularly \textsc{Prolog}, is a well-established research domain. Early
works~\cite{JonesMycroftSemantics,DebrayMishraSemantics} addressed the computational aspects of both pure \textsc{Prolog} and its extension
with the cut construct. Recently, the application of certified/mechanized approaches came into focus as well. In particular,
in one work~\cite{CertifiedPrologEquivalences} the equivalence of a few differently defined semantics
for pure \textsc{Prolog} is proven, and in another work~\cite{CeritfiedDenotationalCut} a denotational semantics for \textsc{Prolog} with cut is presented; both
works provide \textsc{Coq}-mechanized proofs. It is interesting that the former one also advocates the use of higher-order
abstract syntax. We are not aware of any prior work on certified semantics for \textsc{Prolog} which contributed a correct-by-construction
interpreter. Our certified description of SLD resolution with cut can be considered as a certified semantics for \textsc{Prolog} modulo
occurs check in unification (which \textsc{Prolog} does not have by default).

The implementation of first-order unification in dependently typed languages constitutes a well-known challenge with a number of
known solutions. The major difficulty comes from the non-structural recursivity of conventional unification algorithms, which
requires to provide a witness for convergence. The standard approach is to define a generally-recursive function and a well-founded order
for its arguments. This route is taken in a number of works~\cite{MGUinLCF,MGUinMLTT,IdempMGUinCoq,TextbookMGUinCoq}, where the descriptions of
unification algorithms are given in \textsc{Coq}, \textsc{LCF} and \textsc{Alf}. The well-founded used there is
lexicographically ordered tuples, containing the information about the number of different free variables and the sizes of
the arguments. We implement a similar approach, but we separate the test for the non-matching case into a dedicated
function. Thus, we make a recursive call only when the current substitution extension is guaranteed, which allows us to use the
number of different free variables as the well-founded order. An alternative approach suggested by McBride~\cite{StructuralMGU} gives a structurally recursive definition of
the unification algorithm; this is achieved by indexing the arguments with the numbers of their free variables.

The use of higher-order abstract syntax (HOAS) for dealing with language constructs in \textsc{Coq} was addressed in early work~\cite{HOASinCoq},
where it was employed to describe the lambda calculus. The inconsistency phenomenon of HOAS representation, mentioned in Section~\ref{specification}, is called
there ``exotic terms'' there and is handled using a dedicated inductive predicate ``\lstinline|Valid_v|''. The predicate has a non-trivial implementation based
on subtle observations on the behavior of bindings. Our case, however, is much simpler: there is not much variety in ``exotic terms'' (for example, we do not have
reductions in terms), and our consistency predicate can be considered as a limited version of ``\lstinline|Valid_v|'' for a smaller language.

The study of formal semantics for \textsc{miniKanren} is not a completely novel venture. Previously, a nondeterministic
small-step semantics was described~\cite{RelConversion}, as well as a big-step semantics for a finite number of answers~\cite{DivTest};
neither uses proof mechanization and in both works the interleaving is not addressed. 

The work of Kumar~\cite{MechanisingMiniKanren} can be considered as our direct predecessor. It also introduces both denotational and
operational semantics and presents a \textsc{HOL}-certified proof for the soundness of the latter w.r.t. the former. The denotational
semantics resembles ours but considers only queries with a single free variable (we do not see this restriction as important).
On the other hand, the operational semantics is non-deterministic, which makes it
impossible to express interleaving and extract the interpreter in a direct way. In addition, a specific form of ``executable semantics''
is introduced, but its connection to the other two is not established. Finally, no completeness result is presented.
We consider our completeness proof as an essential improvement. 

The most important property of interleaving search~--- completeness~--- was postulated in the introductory paper~\cite{Search}, and is delivered by
all major implementations. Hemann et al.~\cite{SmallEmbedding} give a proof of completeness for a specific implementation of \textsc{miniKanren};
however, the completeness is understood there as
preservation of all answers during the interleaving of answer streams, i.e. in a more narrow sense than in our work since no relation
to denotational semantics is established.

\section{Conclusion}
\label{conclusion}

We presented an approach for converting typed functional programs into relations. Relational conversion 
in many cases allows us to avoid tedious recoding of functional specifications into relational form and to 
concentrate on writing relational specifications only when their reconstruction from functions is impossible or 
undesirable. Our implementation works for the subset of OCaml; we evaluated it for a number of interesting 
examples and acquired some new relational solutions.

There is a number of directions for future research. First, a performance evaluation is desirable~--- at
present time we do not know, what slowdown factor is. Another problem is a development of an approach to
prove complete correctness (or refute this claim).



%%
%% Bibliography
%%

%% Please use bibtex,

%\renewcommand\bibliographytypesize{\small}
\bibliographystyle{abbrv} %{splncs04}
\bibliography{main}

\end{document}




