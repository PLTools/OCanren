\section{Applications}
\label{applications}

In this section, we consider some applications of the framework and results, described in the previous sections.

\subsection{Correctness of Transformations}

One important immediate corollary of the equivalence theorems we have proven is the justification of correctness for certain program transformations.
The completeness of interleaving search guarantees the correctness of any transformation that preserves the denotational semantics,
for example:

\begin{itemize}
\item changing the order of constituents in conjunctions and disjunctions;
\item distributing conjunctions over disjunctions and vice versa, for example, normalizing goals info CNF or DNF;
\item moving fresh variable introduction upwards/downwards, for example, transforming any relation into a top-level fresh
  construct with a freshless body.
\end{itemize}

Note that this way we can guarantee only the preservation of results as \emph{sets of ground terms}; the other aspects of program behavior,
such as termination, may be affected by some of these transformations.\footnote{Possible slowdown and loss of termination after reorderings in conjunction is a famous example of this phenomenon in \textsc{miniKanren}, known as conjunction non-commutativity~\cite{WillThesis}.}

\begin{comment}
Since in \textsc{Prolog} a rather limited form of goals (implicit conjunction of atoms) is used the conversion to \textsc{miniKanren} is easy:

\begin{itemize}
  \item the built-in constructors in \textsc{Prolog} terms (for example, for lists) are converted into \textsc{miniKanren} representation;
  \item each atom is converted into a corresponding relational invocation with the same parameters (modulo the conversion of terms);
  \item the list of atoms in the body of \textsc{Prolog} clause is converted into explicit conjunction;
  \item a number of fresh variables (one for each argument) are created
  \item the arguments are unified with the terms in the corresponding
    argument position in the head of the corresponding clause;
  \item different clauses for the same predicate are combined using disjunction.
\end{itemize}

For example, consider the result of conversion from \textsc{Prolog} definition for the list appending relation into \textsc{miniKanren}.
Like in \textsc{miniKanren}, the definition consists of two clauses. The first one is

\begin{lstlisting}
  append ([], X, X).
\end{lstlisting}

and the result of conversion is

\begin{lstlisting}
  append$^o$ = fun x y z .
    fresh X . x == Nil /\
              y == X   /\
              z == X
\end{lstlisting}

The second one is

\begin{lstlisting}
  append ([H$|$T], Y, [H$|$TY]) :- append (T, Y, TY).
\end{lstlisting}

which is converted into

\begin{lstlisting}
  append$^o$ = fun x y z .
    fresh H T Y TY . x == Cons (H, T)  /\
                     y == Y            /\
                     z == Cons (H, TY) /\
                     append$^o$ (T, Y, TY)
\end{lstlisting}

The overall result is not literally the same as what we've shown in Section~\ref{language}, but denotationally equivalent.

The conversion in the opposite direction involves the following steps:

\begin{itemize}
  \item converting between term representation;
  \item moving all ``\lstinline|fresh|'' constructs into the top-level;
  \item transforming the freshless body into DNF;
  \item replacing all unifications with calls for a specific predicate ``\lstinline|unify/2|'', defined as

\begin{lstlisting}
   unify (X, X).
\end{lstlisting}    

  \item splitting the top-level disjunctions into separate clauses with the same head.
\end{itemize}

The correctness of these, again, can be justified denotationally. For the append$^o$ relation in Section~\ref{language} the result
will be as follows:

\begin{lstlisting}
  append (X, Y, Z) :- unify (X, []), unify (Z, Y).
  append (X, Y, Z) :-
    unify (X, [H$|$T]),
    unify (Z, [H$|$TY]),
    append (T, Y, TY).
\end{lstlisting}

\end{comment}

One of the applications for these transformations is a conversion from/to \textsc{Prolog}. As both languages use essentially the same fragment of first-order logic,
their programs are mutually convertible. The conversion from \textsc{Prolog} to \textsc{miniKanren} is simpler as the latter admits a richer syntax of goals. The inverse
conversion involves the transformation into a DNF and splitting the disjunction into a number of separate clauses. This transformation, in particular, makes it possible to
reuse our approach to describe the semantics of \textsc{Prolog} as well. In the following sections we briefly address this problem.

\begin{comment}
We can use these substantiated transformations, for example, to convert an arbitrary \textsc{miniKanren} specification into a form analogous to specifications in \textsc{Prolog}.
With this conversion we can not only interpret \textsc{Prolog} specifications in interleaving semantics, but we also can, using the certified framework we developed, describe conventional \textsc{Prolog} search strategies.
\end{comment}

\subsection{SLD Semantics}
\label{sld}

The conventional \textsc{Prolog} SLD search differs from the interleaving one in just one aspect~--- it does not perform interleaving.
Thus, changing just two rules in the operational semantics converts interleaving search into the depth-first one:

\[
  \begin{array}{crcr}
    \dfrac{s_1 \xrightarrow{\circ} s'_1}{(s_1 \oplus s_2) \xrightarrow{\circ} (s'_1 \oplus s_2)} &\ruleno{DisjStep}&
    \dfrac{s_1 \xrightarrow{r} s'_1}{(s_1 \oplus s_2) \xrightarrow{r} (s'_1 \oplus s_2)} &\ruleno{DisjStepAns}
  \end{array}
\]
\vskip3mm

With this definition we can almost completely reuse the mechanized proof of soundness (with minor changes); the completeness, however,
can no longer be proven (as it does not hold anymore).

\subsection{Cut}
\label{cut}

Dealing with the ``cut'' construct is known to be a cornerstone feature in the study of operational semantics for \textsc{Prolog}. It turned out that
in our case the semantics of ``cut'' can be expressed naturally (but a bit verbosely). Unlike SLD-resolution, it does not amount to an incremental
change in semantics description. It also would work only for programs directly converted from \textsc{Prolog} specifications.

The key observation in dealing with the ``cut'' in our setting is that a state in our semantics, in fact, encodes the whole current
search tree (including all backtracking possibilities). This opens the opportunity to organize proper ``navigation'' through the tree
to reflect the effect of ``cut''. The details of the semantic description can be found in the extended version of the paper.

For this semantics, we can repeat the proof of soundness w.r.t. to the denotational semantics. There is, however, a little subtlety with our construction:
we cannot formally prove that our semantics indeed encodes the conventional meaning of ``cut'' (since we do not have other semantics of ``cut'' to compare with).
Nevertheless, we can demonstrate a plausible behavior using the extracted reference interpreter.

\subsection{Reference Interpreters}

Using the \textsc{Coq} extraction mechanism, we extracted two reference interpreters from our definitions and theorems: one for conventional
\textsc{miniKanren} with interleaving search and another one for SLD search with cut. These interpreters can be used to practically investigate the behavior
of specifications in unclear, complex, or corner cases. Our experience has shown that these interpreters demonstrate the expected behavior
in all cases.
