\documentclass [mathserif]{beamer}

\usepackage{listings}
\usepackage{color}
\usepackage{amssymb, amsmath}
\usepackage[all]{xy}
\usepackage{alltt}
\usepackage{pslatex}
\usepackage{epigraph}
\usepackage{verbatim}
\usepackage{graphicx}

\definecolor{shadecolor}{gray}{1.00}
\definecolor{darkgray}{gray}{0.30}

\newcommand{\set}[1]{\{#1\}}
\newcommand{\angled}[1]{\langle {#1} \rangle}
\newcommand{\fib}{\rightarrow_{\mathit{fib}}}
\newcommand{\fibm}{\Rightarrow_{\mathit{fib}}}

\setlength{\epigraphwidth}{.55\textwidth}

\definecolor{light-gray}{gray}{0.90}
\newcommand{\graybox}[1]{\colorbox{light-gray}{\texttt{#1}}}

\newcommand{\nredrule}[3]{
  \begin{array}{cl}
    \textsf{[{#1}]}& 
    \begin{array}{c}
      #2 \\
      \hline
      \raisebox{-1pt}{\ensuremath{#3}}
    \end{array}
  \end{array}}

\newcommand{\naxiom}[2]{
  \begin{array}{cl}
    \textsf{[{#1}]} & \raisebox{-1pt}{\ensuremath{#2}}
  \end{array}}

\lstdefinelanguage{ocaml}{
keywords={let, begin, end, in, match, type, and, fun, 
function, try, with, class, object, method, of, rec, repeat, until,
while, not, do, done, as, val, inherit, module, sig, @type, struct, 
if, then, else, open, virtual, new},
sensitive=true,
basicstyle=\small,
commentstyle=\scriptsize\rmfamily,
keywordstyle=\ttfamily\bfseries,
identifierstyle=\ttfamily,
basewidth={0.5em,0.5em},
columns=fixed,
fontadjust=true,
literate={->}{{$\to$}}1
}

\lstset{
basicstyle=\small,
identifierstyle=\ttfamily,
keywordstyle=\bfseries,
commentstyle=\scriptsize\rmfamily,
basewidth={0.5em,0.5em},
fontadjust=true,
escapechar=!,
language=ocaml
}

\setbeamertemplate{footline}[frame number]
\setbeamertemplate{navigation symbols}{}
\setbeamertemplate{blocks}[rounded][shadow=true] 
\beamertemplateballitem

\mode<presentation>{
  \usetheme{default}
}

\theoremstyle{definition}

\title{Typed embedding of relational programming language}

\author[Dmitrii Kosarev, Dmitrii Boulytchev]{Dmitrii Kosarev  (presenter), Dmitrii Boulytchev}
\date{22 Septemper 2016}

\date{
   \textbf{ML Workshop 2016}\\
   \small{%22 September 2016 \\
   Nara, Japan}
}

\institute[]{
%\textbf{Software Engineering Chair }\\ 
\textbf{Saint-Petersburg State University} \\
\texttt{Dmitrii.Kosarev@protonmail.ch}
}

\begin{document}
\begin{frame} 
  \titlepage
\end{frame}

\begin{frame}[fragile]{}
Example 1:
\begin{lstlisting}
let typecheck: _ -> Parsetree.expr -> Types.t
\end{lstlisting}
\pause

\begin{lstlisting}
let inhabitance: _ -> Types.t -> Parsetree.expr 
\end{lstlisting}
\pause

Example 2:
\begin{lstlisting}
let eval_match: Expr.t -> rule list -> Expr.t
\end{lstlisting}
\pause

\begin{lstlisting}
let check_exhaustiveness: Expr.t -> rule list -> Expr.t 
...
check_exhaustiveness (ExceptionExpr Not_exhaustive) rules
\end{lstlisting}
\pause

Reverse execution.
\end{frame}

\begin{frame}[fragile]{3rd example (1)}
\begin{lstlisting}
let rec append xs ys = match xs with
| [] -> ys
| h::tl -> h :: (append tl ys)
\end{lstlisting}

\pause
We are going to emulate \textit{ reverse execution} using \textit{relational programming}.

\begin{verbatim}
function: (xs,ys) -> result
relation: (xs, ys, result)
\end{verbatim}

\end{frame}

\begin{frame}[fragile]{3rd example  (2)}
\onslide<1->
\begin{verbatim}
let rec append xs ys = match xs with
| [] -> ys
| h::tl -> h :: (append tl ys)
\end{verbatim}

Going to rewrite as `mathy` formula:

\onslide<2->\begin{verbatim}
let rec appendo xs ys rez =
\end{verbatim}
\vspace{-.5cm}
\onslide<4->\begin{lstlisting}[mathescape,frame=none]
  ((xs $\equiv$ []) $\wedge$ (ys $\equiv$ rez))\end{lstlisting}
\vspace{-.2cm}
\onslide<3->$\vee$
\vspace{-.2cm}
\onslide<6->\begin{lstlisting}[mathescape,frame=none]
  $\exists$ h,tl,tmp: \end{lstlisting}
\vspace{-.4cm}
\onslide<5->\begin{lstlisting}[mathescape,frame=none]
    ((xs$\equiv$ h::tl) $\wedge$ (appendo tl ys tmp) $\wedge$ (rez$\equiv$ h::tmp))
\end{lstlisting}
\end{frame}

\begin{frame}[fragile]{}

\begin{lstlisting}
let rec appendo xs ys rez = 
  ((xs === llist_nil) &&& (ys===rez))
  |||
  Fresh.three (fun h tl tmp ->
     (xs === h % tl) &&& (appendo tl ys tmp) &&& (rez === x % tmp) 
  )
\end{lstlisting}
\begin{onlyenv}<2,3>
Let's try to execute it.
\begin{verbatim}
# ...give me q such that (appendo [1;2] [3;4] q)
[1;2;3;4]
\end{verbatim}
\end{onlyenv}

\begin{onlyenv}<3,4>
\begin{verbatim}
# ...give me q such that (appendo q [3;4] [1;2;3;4])
[1;2]
\end{verbatim}
\end{onlyenv}

\begin{onlyenv}<4,5>
%\uncover<4->{
%\only<4,5>{
\begin{verbatim}
# ...give me (q,r) such that (appendo q r [1;2])
([], [1;2])
([1], [2])
([1;2], [])
\end{verbatim}
\end{onlyenv}
%}

\begin{onlyenv}<5>
\begin{verbatim}
# ...give me (r,q) such that (appendo r q q)
([], _.0)
\end{verbatim}
\end{onlyenv}
\end{frame}
%-%%%%%%%%%%%%%%%%%%%%%%%%%%%%%%%%%%%%%%%%%%

\begin{frame}{Relational programming}

\begin{itemize}
 \item Relational programming $\equiv$ miniKanren
 \item \url{ http://minikanren.org/}
 \item A family of many languages: $alpha$-Kanren, $\mu$Kanren, constrained Kanren, etc.
 \item Implemented in many languages: various LISPS, OCaml, Haskell, F\#, Scala, etc.
\end{itemize}
 
\end{frame}

\begin{frame}{Contributions of this work}
\begin{itemize}
 \item 2nd implementation of miniKanren in OCaml (more type-safe than first one: 
   \url{https://github.com/lightyang/minikanren-ocaml}.
 \item Less verbose programming than other approaches
   \begin{itemize}
    \item No need for ad-hoc conversions to \texttt{term} type.
   \end{itemize}
  \item No typeclasses used.
  \item We are reusing OCaml specific features in \textit{unification} (algorithm behind $\equiv$).
\end{itemize}
\end{frame}

\begin{frame}[fragile]{About implementations with \texttt{term} type}
\begin{itemize}
 \item Special type for relation-related evaluations.
 \item \begin{lstlisting}
type term = Var of int
           | ConstString of string
           | ConstInt of int
           | ListNil
           | ListCons of term * term
           | Atom of string
           | etc...
 \end{lstlisting}
 \item Special convertion functions for every type \texttt{t}: \begin{lstlisting}
 val inject_t:  t -> term
 val project_t: term -> t
 \end{lstlisting}
 \item And \texttt{project} is partial by two reasons: 
  \begin{itemize}
    \item Value of type \texttt{term} can contain free logic variables. "
    \item The value \texttt{ListCons(ConstInt, ListCons(ConstString "a"),ListNil))} cannot be converted 
    neither to OCaml \texttt{int list} not to \texttt{string list}.
  \end{itemize}

\end{itemize}

 %\item Special type for 
%\end{lstlisting}
 
\end{frame}


% \begin{frame}[fragile]{}
% \begin{tabular}{||p{5cm}|p{5cm}||}
% \hline
% ADT & Object Representation \\
% \hline
% \hline
% Single type definition with multiple constructors & Multiple class definitions\\
% \hline
% Adding new transformation is easy & Adding new transformation is tedious\\
% \hline
% \textcolor{red}{Adding new constructor is tedious} & Adding new class is (rather) easy\\
% \hline
% \textcolor{red}{Joining types is tough} & Joining class hierarchies is tedious \\
% \hline
% \textcolor{red}{Modifying existing transformation is tough} & Modifying existing transformation is easy\\
% \hline
% \end{tabular}
% 
% \end{frame}
% 
% \begin{frame}[fragile]{Polymorphic Variants in OCaml}
% 
% ``The Expression Problem'' [\emph{Wadler, 1998}]
% 
% ``Code Reuse Through Polymorphic Variants'' [\emph{Garrigue, 2002}]
% \begin{lstlisting}
% type var = [`Var of string]
% 
% let evalVar s (`Var n) = s n
% 
% type 'a bin = [`Add of 'a * 'a | `Mul of 'a * 'a]
% 
% let evalBin eval = function
% | `Add (x, y) -> eval x + eval y
% | `Mul (x, y) -> eval x * eval y
% 
% type 'a expr = [var | 'a bin]
% 
% let evalExpr eval s = function
% | #var as e -> evalVar s e
% | #bin as e -> evalBin eval e
% 
% let rec eval s e = evalExpr (eval s) s e
% \end{lstlisting}
% \end{frame}
% 
% \begin{frame}[fragile]{Local Modification of Existing Transformation}
% 
% Artificial example: counting number of leaves in a tree:
% 
% \begin{lstlisting}
% type tree = Leaf of string | Node of tree * tree
% 
% let rec leaves = function
% | Leaf _ -> 1 
% | Node (l, r) -> leaves l + leaves r
% \end{lstlisting}
% 
% \pause
% Modification: do not count ``empty leaves'':
% \pause
% 
% \begin{lstlisting}
% let non_empty_leaves = function
% | Leaf "" -> 0
% | t       -> leaves t 
% \end{lstlisting}
% \pause
% In OO: visitors.
% 
% \end{frame}
% 
% \begin{frame}[fragile]{Object-Encoded Transformers}
% 
% \begin{itemize}
% \item Data representation is left untouched.
% \item Transformation is represented as object with method
% per each constructor.
% \item Traversal function to match against constructors and
% call appropriate methods. 
% \end{itemize}
% \end{frame}
% 
% \begin{frame}[fragile]{Object-Encoded Transformers}
% 
% Object-encoded transformer for counting leaves:
% \begin{lstlisting}
% class leaves = object
%   method c_Leaf l = 1
%   method c_Node l r = l+r
% end
% \end{lstlisting}
% \pause
% 
% Traversal function:
% \begin{lstlisting}
% let rec transform t = function
% | Leaf l -> t#c_Leaf l
% | Node (l, r) -> t#c_Node (transform t l) (transform t r)
% \end{lstlisting}
% \pause
% 
% Example:
% \begin{lstlisting}
% let leaves = transform (new leaves)
% let non_empty_leaves = transform 
%   object inherit leaves as super
%     method c_Leaf l = if l = "" then 0 else super#c_Leaf l
%   end
% \end{lstlisting}
% 
% \end{frame}
% 
% \begin{frame}[fragile]{More Elaborated Version}
% \begin{itemize}
% \item Transformation class: catamorphisms (folds/attribute grammar-defined 
% transformations).
% \item One traversal function per type.
% \item Arguments of per-constructor methods are augmented with
% transformation functions. 
% \item One abstract transformer (virtual class) to inherit from.
% \item Support for polymorphic ADT and polymorphic variant types.
% \item Syntax extension to generate all boilerplate code from type definitions.
% \end{itemize}
% \end{frame}
% 
% \begin{frame}[fragile]{Example: ``show'' and ``eval''}
% \begin{lstlisting}
% !\graybox{\bfseries @}!type expr = Var of string | Add of expr * expr | Const of int
% 
% class eval = object inherit [string -> int, int] !\graybox{@expr}!
%   method c_Const _ _ n = n
%   method c_Add s _ l r = !\graybox{l.fx s}! + !\graybox{r.fx s}!
%   method c_Var s _ x   = s x
% end
% 
% class show = object inherit [unit, string] @expr
%   method c_Const _ _ n = string_of_int n
%   method c_Add s _ l r = "Add (" ^ !\graybox{l.fx s}! ^ ", " ^ !\graybox{r.fx s}! ^ ")"
%   method c_Var _ _ x   = "Var " ^ x
% end
% 
% let eval = transform(expr) (new eval)
% let show = transform(expr) (new show) ()
% \end{lstlisting}
% \end{frame}
% 
% \begin{frame}[fragile]{Example: Expression Problem}
% \begin{lstlisting}
% !\graybox{\bfseries @}!type var = [`Var of string] 
% class ['v] var_eval = object inherit [string -> 'v, 'v] !\graybox{@var}!      
%   method c_Var s _ x = s x
% end
% 
% !\graybox{\bfseries @}!type 'a bin = [ `Add of 'a * 'a | `Mul of 'a * 'a] 
% class ['a, 'b] bin_eval = object inherit ['a, int, 'b, int] !\graybox{@bin}!
%   method c_Add s _ l r = !\graybox{l.fx s}! + !\graybox{r.fx s}!
%   method c_Mul s _ l r = !\graybox{l.fx s}! * !\graybox{r.fx s}!
% end
% 
% !\graybox{\bfseries @}!type 'a expr = [ var | 'a arith ] 
% class ['a] expr_eval = object
%   inherit ['a, int, string->int, int] !\graybox{@expr}!
%   inherit [int] var_eval
%   inherit ['a, string -> int] bin_eval
% end
% \end{lstlisting}
% \end{frame}
% 
% \begin{frame}[fragile]{Custom Traits/Plugins}
% 
% \begin{itemize}
% \item User-defined generators for concrete transformers.
% \item Dynamically loaded during syntax extension phase.
% \item Easy to implement for simple boilerplate transformations.
% \item Examples: \lstinline{show}, \lstinline{map}, \lstinline{fold}. 
% \end{itemize}
% \pause
% \begin{lstlisting}
% !\graybox{\bfseries @}!type tree = Leaf of string | Node of tree * tree !\graybox{{\bfseries with} foldl}!
% 
% let leaves = transform(tree) 
%   object inherit [int] !\graybox{@fold[tree]}! 
%     method c_Leaf a _ _ = a+1 
%   end 0
% 
% let non_empty_leaves = transform(tree)
%   object inherit [int] !\graybox{@fold[tree]}!
%     method c_Leaf a _ l = if l = "" then a else a+1
%   end 0 
% \end{lstlisting}
% \end{frame}
% 
% \begin{frame}[fragile]{Custom Traits/Plugins}
% \begin{lstlisting}
% !\graybox{\bfseries @}!type expr = Var   of string 
%               | Add   of expr * expr 
%               | Const of int !\graybox{{\bfseries with} map}!
% 
% class simplify = object inherit !\graybox{@map[expr]}!
%   method c_Add _ _ l r = match !\graybox{l.fx ()}!, !\graybox{r.fx ()}! with
%   | Add (Const x, y), Add (Const z, t) -> Add (Const (x+z), Add (y, t))
%   | Add (Const x, y), Const z -> Add (Const (x+z), y)
%   | Const x, Add (Const y, z) -> Add (Const (x+y), z) 
%   | Const x, Const y          -> Const (x+y)
%   | x      , (Const _ as y)   -> Add (y, x)
%   | x      , y                -> Add (x, y)  
% end
% \end{lstlisting}
% \end{frame}
% 
% \begin{frame}[fragile]{Example: Lambda Reductions}
% 
% ``Demonstrating Lambda Calculus Reduction'' [\emph{Sestoft, 2002}]:
% \vskip 3mm
% 
% \begin{itemize}
%   \item categorization of steps;
%   \item seven different reduction orders;
%   \item big-step operational semantics;
%   \item one-to-one correspondence between semantic
% specification and implementation code;
%   \item seven similarly looking pieces of code (\emph{actually, three}).
% \end{itemize}
% \end{frame}
% 
% \begin{frame}[fragile]{Example: Lambdas, Variables, Free Variables}
% \begin{lstlisting}
% !\graybox{\bfseries @}!type lam = 
% | Var of string 
% | App of lam * lam
% | Lam of string * lam !\graybox{{\bfseries with} show, foldl}!
% \end{lstlisting}
% 
% \pause
% 
% \begin{lstlisting}
% class var = object inherit [S.t] !\graybox{@foldl[lam]}!
%   method c_Var s _ x = S.add x s
% end
% 
% let vars = transform(lam) (new var) S.empty
% \end{lstlisting}
% 
% \pause
% 
% \begin{lstlisting}
% let fv = transform(lam) 
%   object inherit var   
%     method c_Lam s _ x l = S.union s (S.remove x (!\graybox{l.fx S.empty}!))
%   end S.empty 
% \end{lstlisting}
% \end{frame}
% 
% \begin{frame}[fragile]{Example: Object-Encoded Reducer}
% \begin{lstlisting}
% class virtual reducer = object inherit [unit, lam] !\graybox{@lam}!
%   method virtual arg  : (unit, lam, lam, < >) a -> lam
%   method head : (unit, lam, lam, < >) a -> lam = fun x -> !\graybox{x.fx ()}!
%   method c_Var _ x _ = !\graybox{\textasciitilde:x}!
% end
% \end{lstlisting}
% 
% \pause
% 
% \begin{lstlisting}
% let reduce r = transform(lam) r ()
% \end{lstlisting}
% 
% \end{frame}
% 
% \begin{frame}[fragile]{Example: Dealing With Abstractions and Arguments}
% 
% \pause
% 
% \begin{lstlisting}
% class virtual reduce_under_abstractions = 
%   object inherit reducer
%     method c_Lam _ _ x l = Lam (x, !\graybox{l.fx ()}!)
%   end
% 
% class virtual dont_reduce_under_abstractions = 
%   object inherit reducer
%     method c_Lam _ s _ _ = !\graybox{\textasciitilde:s}!
%   end
% \end{lstlisting}
% 
% \pause
% 
% \begin{lstlisting}
% class virtual reduce_arguments =
%   object inherit reducer
%     method arg x = !\graybox{x.fx ()}!
%   end
% 
% class virtual dont_reduce_arguments =
%   object inherit reducer
%     method arg x = !\graybox{\textasciitilde:x}!
%   end
% \end{lstlisting}
% \end{frame}
% 
% \begin{frame}[fragile]{Example: Strict vs. Non-strict}
% \pause
% 
% \begin{lstlisting}
% class virtual strict g =
%   object (this) inherit reducer
%     method c_App _ s l m = 
%       match this#head l with
%       | Lam (x, l') -> !\graybox{s.f ()}! (subst g x (!\graybox{m.fx ()}!) l')
%       | l'          -> let l'' = !\graybox{s.f () l'}! in 
%                        App (l'', this#arg m)
%   end
% \end{lstlisting}
% 
% \pause
% 
% \begin{lstlisting}
% class virtual non_strict g =
%   object (this) inherit reducer
%     method c_App _ s l m = 
%       match this#head l with
%       | Lam (x, l') -> !\graybox{s.f ()}! (subst g x !\graybox{\textasciitilde:m}! l')
%       | l'          -> let l'' = !\graybox{s.f () l'}! in 
%                        App (l'', this#arg m)
%   end
% \end{lstlisting}
% \end{frame}
% 
% \begin{frame}[fragile]{Example: Building Reduction Orders}
% \pause
% 
% Call-by-name:
% 
% \begin{lstlisting}
% class bn g = object 
%   inherit dont_reduce_under_abstractions
%   inherit dont_reduce_arguments      
%   inherit non_strict g
% end 
% 
% let bn l = reduce (new bn (context l)) l
% \end{lstlisting}
% 
% \pause
% 
% Call-by-value:
% \begin{lstlisting}
% class bv g = object
%   inherit dont_reduce_under_abstractions
%   inherit reduce_arguments
%   inherit strict g
% end
% 
% let bv l = reduce (new bv (context l)) l
% \end{lstlisting}
% \end{frame}
% 
% \begin{frame}[fragile]{Examlpe: Building Reduction Orders}
% \pause
% 
% Normal order:
% \begin{lstlisting}
% class nor g = object
%   inherit reduce_under_abstractions
%   inherit reduce_arguments
%   inherit non_strict g
%   method  head x = bn !\graybox{\textasciitilde:x}!
% end 
% 
% let nor l = reduce (new nor (context l)) l
% \end{lstlisting}
% 
% \pause
% 
% Applicative order:
% \begin{lstlisting}
% class ao c = object
%   inherit bv c
%   inherit reduce_under_abstractions
% end
% 
% let ao l = reduce (new ao (context l)) l
% \end{lstlisting}
% \end{frame}
% 
% \begin{frame}[fragile]{Examlpe: Building Reduction Orders}
% \pause
% Hybrid Applicative/Head Spine/Hybrid Normal Orders:
% 
% \begin{lstlisting}
% class ha c = object
%   inherit ao c
%   method head x = bv !\graybox{\textasciitilde:x}!
% end
% let ha l = reduce (new ha (context l)) l
% 
% class he c = object
%   inherit bn c
%   inherit reduce_under_abstractions
% end
% let he l = reduce (new he (context l)) l
% 
% class hn c = object
%   inherit nor c
%   method  head x = he !\graybox{\textasciitilde:x}!
% end
% let hn l = reduce (new hn (context l)) l
% \end{lstlisting}
% \end{frame}
% 
% \begin{frame}[fragile]{Conclusions}
% 
% By now:
% \begin{itemize}
% \item lighweight datatype-generic framework;
% \item one per-type traversal function, one abstract
% object-encoded transformer (generated automatically);
% \item a number of plugins to generate concrete transformers
% for some widespread transformations 
% (\lstinline{show}, \lstinline{map}, \lstinline{fold}, 
% \lstinline{eq}, \lstinline{compare},...);
% \item an ability to modify existing transformations;
% \item an ability to ``join'' transformations for
% ``joined'' types.
% \end{itemize}
% 
% Future work:
% \begin{itemize}
% \item find more applications;
% \item evaluate and address performance issues;
% \item implement more plugins (with contexts?).
% \end{itemize}
% 
% \end{frame}
% 
% \begin{frame}{Thank you!}
% \begin{itemize}
% \item Source code: \texttt{https://code.google.com/p/generic-transformers}
% \end{itemize}
% \end{frame}

\end{document}
