\documentclass{llncs}

\usepackage{makeidx}
\usepackage{amssymb}
\usepackage{mathtools}
\usepackage{multirow}
\usepackage{listings}
\usepackage{indentfirst}
\usepackage{verbatim}
\usepackage{amsmath, amssymb}
\usepackage{graphicx}
\usepackage{xcolor}
\usepackage{url}
\usepackage{stmaryrd}
\usepackage{xspace}
\usepackage{comment}
\usepackage{wrapfig}
\usepackage[caption=false]{subfig}
%\usepackage{subcaption}
\usepackage{placeins}
\usepackage{tabularx}
\usepackage{ragged2e}

\setlength{\parskip}{-1pt}

\def\transarrow{\xrightarrow}
\newcommand{\setarrow}[1]{\def\transarrow{#1}}

\newcommand{\trule}[2]{\frac{#1}{#2}}
\newcommand{\crule}[3]{\frac{#1}{#2},\;{#3}}
\newcommand{\withenv}[2]{{#1}\vdash{#2}}
\newcommand{\trans}[3]{{#1}\transarrow{#2}{#3}}
\newcommand{\ctrans}[4]{{#1}\transarrow{#2}{#3},\;{#4}}
\newcommand{\llang}[1]{\mbox{\lstinline[mathescape]|#1|}}
\newcommand{\pair}[2]{\inbr{{#1}\mid{#2}}}
\newcommand{\inbr}[1]{\left<{#1}\right>}
\newcommand{\highlight}[1]{\color{red}{#1}}
\newcommand{\ruleno}[1]{\eqno[\scriptsize\textsc{#1}]}
\newcommand{\rulename}[1]{\textsc{#1}}
\newcommand{\inmath}[1]{\mbox{$#1$}}
\newcommand{\lfp}[1]{fix_{#1}}
\newcommand{\gfp}[1]{Fix_{#1}}
\newcommand{\vsep}{\vspace{-2mm}}
\newcommand{\supp}[1]{\scriptsize{#1}}
\newcommand{\G}{\mathfrak G}
\newcommand{\sembr}[1]{\llbracket{#1}\rrbracket}
\newcommand{\cd}[1]{\texttt{#1}}
\newcommand{\miniKanren}{miniKanren\xspace}
\newcommand{\ocanren}{OCanren\xspace}
\newcommand{\free}[1]{\boxed{#1}}
\newcommand{\binds}{\;\mapsto\;}
\newcommand{\dbi}[1]{\mbox{\bf{#1}}}
\newcommand{\sv}[1]{\mbox{\textbf{#1}}}
\newcommand{\bnd}[2]{{#1}\mkern-9mu\binds\mkern-9mu{#2}}

\newcommand{\meta}[1]{{\cal{#1}}}
\renewcommand{\emptyset}{\varnothing}

\lstdefinelanguage{ocanren}{
keywords={fresh, let, in, match, with, when, class, type,
object, method, of, rec, repeat, until, while, not, do, done, as, val, inherit,
new, module, sig, deriving, datatype, struct, if, then, else, open, private, virtual, include, success, failure,
true, false},
sensitive=true,
commentstyle=\small\itshape\ttfamily,
keywordstyle=\ttfamily\underbar,
identifierstyle=\ttfamily,
basewidth={0.5em,0.5em},
columns=fixed,
fontadjust=true,
literate={fun}{{$\lambda$}}1 {->}{{$\to$}}3 {===}{{$\equiv$}}1 {=/=}{{$\not\equiv$}}1 {|>}{{$\triangleright$}}3 {\\/}{{$\vee$}}2 {/\\}{{$\wedge$}}2 {^}{{$\uparrow$}}1,
morecomment=[s]{(*}{*)}
}

\lstset{
mathescape=true,
basicstyle=\small,
identifierstyle=\ttfamily,
keywordstyle=\bfseries,
commentstyle=\scriptsize\rmfamily,
basewidth={0.5em,0.5em},
fontadjust=true,
language=ocanren
}

\usepackage{letltxmacro}
\newcommand*{\SavedLstInline}{}
\LetLtxMacro\SavedLstInline\lstinline
\DeclareRobustCommand*{\lstinline}{%
  \ifmmode
    \let\SavedBGroup\bgroup
    \def\bgroup{%
      \let\bgroup\SavedBGroup
      \hbox\bgroup
    }%
  \fi
  \SavedLstInline
}
%\addtolength{\parskip}{-2pt}

\begin{document}
\sloppy
\mainmatter

\title{Improving Refutational Completeness\\
of Relational Search via Divergence Test}

\author{
  Dmitri Rozplokhas\inst{1} \and Dmitri Boulytchev\inst{2}
}

\institute{
\email{rozplokhas@gmail.com}\\
St.Petersburg Academic University
\and
\email{dboulytchev@math.spbu.ru}\\
St.Petersburg State University\\
JetBrains Research
}

\maketitle

\begin{abstract}
We describe a search optimization technique for implementation of relational programming language
miniKanren which makes more queries to converge. Our technique is based on a certain 
divergence criterion, which we use to force a dynamic reordering of subgoals. We present a formal semantics of
miniKanren-like language, and prove, that our optimization does not compromise already
converging programs, thus being a proper improvement. We also present the prototype
implementation of improved search and demonstrate its application for a number of
useful specifications.
\end{abstract}

\section{Introduction}
\label{intro}

Relational programming is an attractive technique, based on the idea of constructing programs as relations.
While in general some relational effects can be reproduced with a number of languages for logic programming, such as
Prolog, Mercury\footnote{\url{https://mercurylang.org}}, or Curry\footnote{\url{http://www-ps.informatik.uni-kiel.de/currywiki}}, in
a narrow sense relational programming amounts to writing relational specifications in \miniKanren~\cite{TRS}. \miniKanren\footnote{\url{http://minikanren.org}},
initially designed as a small relational DSL, embedded in Scheme/Racket, was later implemented for a number of general-purpose host languages,
including Scala, Haskell, Standard ML and OCaml.

With the relational approach, it becomes possible to give simple and elegant solutions for the problems, otherwise
considered as tricky, tough, tedious, or boring~\cite{unified}. For example, relational interpreters can be used to derive
\emph{quines}~--- programs, which reduce to itself, as well as \emph{twines} or \emph{thrines} (pairs or triples of
programs, reducing to each other)~\cite{Untagged}; a straightforward relational description of
simply typed lambda calculus~\cite{Lambda} inference rules works both as type inferencer and inhabitation problem solver~\cite{WillThesis};
relational list sorting can be used to generate all permutations~\cite{ocanren}, etc. 

On the other hand, writing relational specifications can sometimes be a tricky and error-prone task. Fortunately, many 
specifications can be written systematically by ``generalizing'' a certain functional program. From the very beginning, 
the conversion from functional to relational form was considered as an element of relational programming thesaurus~\cite{TRS}. However,
the traditional approach~--- \emph{unnesting}~--- was formulated for an untyped case, worked only for specifically written
programs and was never implemented.

We present a generalized form of relational conversion, which can be applied to typed terms in general form. We study the relational conversion 
for a small ML-like language (essentially, a certain subset of OCaml), equipped with Hindley-Milner type system with let-polymorphism~\cite{Types}. 
We start from retelling the syntax, typing rules, and operational semantics, and then extend the source language with a conventional set of 
relational constructs. This set corresponds to existing typed embedding of \miniKanren into OCaml~\cite{ocanren}. We then present typing rules and 
develop operational semantics for this relational extension; to our knowledge, this is the first attempt to specify formal semantics for
\miniKanren. Next, we develop formal rules for relational conversion and prove, that these rules respect both typing and
semantics. Finally, we describe the implementation of a relational converter and demonstrate its application for a number of problems, for some
of which we present a relational solution for the first time.

\section{The Syntax and Semantics of Relational Language}
\label{language}

In this section we describe the syntax and semantics of the language, which is used in the rest of the paper. To some extent this description serves as
a short introduction to miniKanren. The main distinction between ``the real'' miniKanren and our version is that we give a proper semantics only to converging programs, 
which deliver a finite set of answers, while in the reality of relational programming the result is represented as an infinite stream, from which any number of answers can be requested, and the request of a non-existing answer can
lead to a divergence. Our semantics, thus, corresponds to the scenario, when \emph{all} answers are requested from the stream. On the other hand, we do not distinguish programs, calculating the infinite number 
of answers, from those diverging with no results at all.   
However, we consider the finite version of the semantics as an important case, which is justified by the evaluation, presented in Section~\ref{evaluation}.


\begin{figure}[t]
$$
\begin{array}{rcll}
\meta{C}    & = & \{C^k,\dots\}                                        & \;\;\mbox{\emph{(constructors)}}\\
\meta{T}(X) & = & x\in X \mid C^k\,(t_1,\dots,t_k),\,t_i\in \meta{T}(X) & \;\;\mbox{\emph{(terms)}}\\
\meta{V}    & = & \{x, y, z, \dots\}                                   & \;\;\mbox{\emph{(syntactic variables)}}\\
\meta{T_V}  & = & \meta{T}(\meta{V})                                   & \;\;\mbox{\emph{(syntactic terms)}}\\
\meta{R}    & = & \{r^k,\dots\}                                        & \;\;\mbox{\emph{(relational symbols)}}\\
\meta{G}    & = & t_1\equiv t_2,\,t_i\in\meta{T_V}                      & \;\;\mbox{\emph{(unification)}}\\ 
            &   & g_1\wedge g_2                                        & \;\;\mbox{\emph{(conjunction)}}\\
            &   & g_1\vee g_2                                          & \;\;\mbox{\emph{(disjunction)}}\\
            &   &\lstinline|fresh|\,(x)\;g                             & \;\;\mbox{\emph{(fresh variable introduction)}}\\
            &   &r^k\;t_1\dots t_k,\,t_i\in\meta{T_V}                   & \;\;\mbox{\emph{(relational reference)}}\\
\meta{D}    & = & r^k\binds\lambda x_1\dots x_k\,.\,g,\,x_i\in\meta{V}  & \;\;\mbox{\emph{(relational definition)}}\\
\meta{S}    & = & d_1,\dots,\,d_k; g                                    & \;\;\mbox{\emph{(specification)}}
\end{array}
$$
\caption{The syntax of source language}
\label{syntax}
\end{figure}

The syntax of our relational language is shown on Fig.~\ref{syntax}. First, we introduce the alphabet of constructors $\meta{C}$, each of which is equipped with a 
non-negative arity. Then we in a conventional fashion inductively define the set of all terms $\meta{T}(X)$, parameterized by the set of variables $X$. We need this parameterization 
since later we will be dealing with two sorts of variables~--- \emph{syntactic} and \emph{semantic}, and, therefore, two sorts of terms. Next, we choose the set of syntactic variables
$\meta{V}$ and the set of \emph{relational symbols} $\meta{R}$ with arities, which will be used as names for relational definitions. We also introduce a shortcut
$\meta{T_V}$ for the set of all terms over syntactic variables since it will be used in all other syntactic definitions.

The core syntax category in the language is a \emph{goal}. There are five types of goals: unification of two terms, conjunction and disjunction of two
goals, fresh variable introduction and a call of some relational definition. We stipulate, that the calls of relational definitions respect their arities; we 
will also use a shortcut form \lstinline|fresh ($x$ $y$ $z$ ...) ...| instead of \lstinline|fresh($x$) (fresh ($y$) (fresh ($z$) ...)| where needed.

Next, \emph{relational definition} $\meta{D}$ binds some relational symbol to a parameterized goal; the number of parameters corresponds to the arity of the symbol, and
we assume, that all parameter variables are pairwise distinct. Finally, the top-level syntax category is \emph{specification} $\meta{S}$~--- a goal in the context 
of some relational definitions. 

Note, we define here a language with first-order relations; in particular, we do not allow partial application. As we see later, our approach critically depends
on recursive call identification, which is a trivial task in the first-order case. Some existing frameworks for relational programming~\cite{OCanren,RelConversion}
do not impose such a limitation; extending our approach for the higher-order case is a subject of future research.

\setarrow{\xRightarrow}
\newcommand{\otrans}[4]{\withenv{#1}{\trans{#2}{\mbox{$#3$}}{#4}}}
\newcommand{\cotrans}[5]{\withenv{#1}{\ctrans{#2}{\mbox{$#3$}}{#4}{#5}}}
 
\begin{figure}[t]
$$
\cotrans{\Gamma,\,\iota}{(\sigma,\,\delta)}{t_1\equiv t_2}{\emptyset}{mgu\,(t_1\iota\sigma,\,t_2\iota\sigma) = \bot}\ruleno{UnifyFail}
$$

$$
\cotrans{\Gamma,\,\iota}{(\sigma,\,\delta)}{t_1\equiv t_2}{(\sigma\circ\Delta,\,\delta)}{mgu\,(t_1\iota\sigma,\,t_2\iota\sigma) = \Delta\ne\bot}\ruleno{UnifySuccess}
$$

$$
\trule{\otrans{\Gamma,\,\iota}{(\sigma,\,\delta)}{g_1}{S_1},\;\;\;\;
       \otrans{\Gamma,\,\iota}{(\sigma,\,\delta)}{g_2}{S_2}
      }
      {\otrans{\Gamma,\,\iota}{(\sigma,\,\delta)}{g_1\vee g_2}{S_1\cup S_2}}\ruleno{Disj}
$$

$$
\trule{\otrans{\Gamma,\,\iota}{(\sigma,\,\delta)}{g_1}{\{(\sigma_i,\,\delta_i)\}},\;\;\;\;
       \otrans{\Gamma,\,\iota}{(\sigma_i,\,\delta_i)}{g_2}{S_i}
      }
      {\otrans{\Gamma,\,\iota}{(\sigma,\,\delta)}{g_1\wedge g_2}{\bigcup_i S_i}}\ruleno{Conj}
$$

$$
\crule{\otrans{\Gamma,\,\iota[x\gets\alpha]}{(\sigma,\,\delta\cup\{\alpha\})}{g}{S}}
      {\otrans{\Gamma,\,\iota}{(\sigma,\,\delta)}{\lstinline|fresh($x$) $\;g$|}{S}}
      {\alpha\in\meta{W}\setminus\delta}\ruleno{Fresh}
$$

$$
\crule{\otrans{\Gamma,\,[x_i\gets v_i]}{(\epsilon,\,\delta)}{g}{\{(\sigma_i,\,\delta_i)\}}}
      {\otrans{\Gamma,\,\iota}{(\sigma,\,\delta)}{r^k\,t_1\dots t_k}{\{(\sigma\circ\sigma_i,\,\delta_i)\}}}
      {\Gamma(r^k)=\lambda x_1\dots x_k.g,\,v_i=t_i\iota\sigma}\ruleno{Invoke}
$$
\caption{Big-step operational semantics}
\label{semantics}
\end{figure}

\begin{figure}[t]
\arraycolsep=5pt
\def\arraystretch{2.2}
\subfloat[\label{appendo_eval_a}]{
$
\begin{array}{c|c}
  \multicolumn{2}{c}{\otrans{\bot}{(\epsilon,\,\emptyset)}{\lstinline|fresh ($q$)|\;...}{...}}\\
  \hline
  \multicolumn{2}{c}{\otrans{[\bnd{q}{\sv{0}}]}{(\epsilon,\,\{\sv{0}\})}{\lstinline|append$^o$ (Cons (A, Nil)) Nil $\;q$|}{...}}\\
  \hline
  \multicolumn{2}{c}{\otrans{[\bnd{x}{\lstinline|Cons(A, Nil)|},\,\bnd{y}{\lstinline|Nil|},\,\bnd{xy}{\sv{0}}]}{(\dots)}{\lstinline|(...) $\vee$ (...)|}{...}}\\
  \hline
  \otrans{\dots}{(\dots)}{\lstinline|($x\;$ === $\;$Nil) $\wedge$ (...)|}{\emptyset} & \multirow{2}*{Fig.~\ref{appendo_eval_b}}\\
  \cline{1-1}
  \otrans{\dots}{(\dots)}{\lstinline|$x\;$ === $\;$Nil|}{\emptyset} & 
\end{array}
$
}\\
\subfloat[\label{appendo_eval_b}]{
$
\begin{array}{c|c}
\multicolumn{2}{c}{\otrans{[\bnd{x}{\lstinline|Cons(A, Nil)|},\,\bnd{y}{\lstinline|Nil|},\,\bnd{xy}{\sv{0}}]}{(\epsilon,\,\{\sv{0}\})}{\lstinline|fresh($h\;t\;ty$) ...|}{...}}\\
\hline
\multicolumn{2}{c}{\otrans{[\dots,\,\bnd{h}{\sv{1}},\,\bnd{t}{\sv{2}},\,\bnd{ty}{\sv{3}}]}{(\epsilon,\,\{\sv{0}..\sv{3}\})}{\lstinline|($x\;$ === $\;$Cons ($h$, $\;t$)) $\wedge$ (...)|}{...}}\\
\hline
\otrans{\dots}{(\epsilon,\,\{\sv{0}..\sv{3}\})}{\lstinline|$x\;$ === $\;$Cons ($h$, $\;t$)|}{\{([\bnd{\sv{1}}{\lstinline|A|},\,\bnd{\sv{2}}{\lstinline|Nil|}],\,\{\sv{0}..\sv{3}\})\}} & \mbox{Fig.~\ref{appendo_eval_c}}
\end{array}
$
}\\
\subfloat[\label{appendo_eval_c}]{
$
\begin{array}{c|c|c}
\multicolumn{3}{c}{\otrans{\dots}{\{([\bnd{\sv{1}}{\lstinline|A|},\,\bnd{\sv{2}}{\lstinline|Nil|}],\,\{\sv{0}..\sv{3}\})\}}{\lstinline|(append$^o$ $\;t\;$ $y\;$ $ty$) $\wedge$ (...)|}{...}}
\\
\hline
\multicolumn{2}{c|}{\otrans{\dots}{(\dots)}{\lstinline|append$^o$ $\;t\;$ $y\;$ $ty$|}{...}} & 
\multirow{4}*{Fig.~\ref{appendo_eval_d}} \\
\cline{1-2}
\multicolumn{2}{c|}{\otrans{[\bnd{x}{\lstinline|Nil|},\,\bnd{y}{\lstinline|Nil|},\,\bnd{xy}{\sv{3}}]}{(\epsilon,\,\{\sv{0}..\sv{3}\})}{\lstinline|(...) $\vee$ (...)|}{...}} & \\
\cline{1-2}
\multicolumn{2}{c|}{\otrans{\dots}{(\dots)}{\lstinline|($x\;$ === $\;$Nil) $\wedge$ ($xy\;$ === $\;y$)|}{...}} &  \\
\cline{1-2}
\otrans{\dots}{(\dots)}{\lstinline|$x\;$ === $\;$Nil|}{(\dots)} & 
\otrans{\dots}{(\dots)}{\lstinline|$xy\;$ === $\;y$|}{\{([\bnd{\sv{3}}{\lstinline|Nil|}],\,\{\sv{0}..\sv{3}\})\}} &
\end{array}
$}\\
\subfloat[\label{appendo_eval_d}]{
$
\begin{array}{c}
\otrans{\dots}{([\bnd{\sv{1}}{\lstinline|A|},\,\bnd{\sv{2}}{\lstinline|Nil|},\,\bnd{\sv{3}}{\lstinline|Nil|}],\,\{\sv{0}..\sv{3}\})}{\lstinline|$xy\;$ === $\;$Cons ($h$, $\;ty$)|}{\{([\dots,\,\bnd{\sv{0}}{\lstinline|Cons (A, Nil)|}],\,\{\sv{0}..\sv{3}\})\}}
\end{array}
$}
\caption{An example of relational evaluation}
\label{appendo_eval}
\end{figure}

We describe the semantics of our language using a conventional big-step style inference system. First, we choose an infinite
set of \emph{semantic variables} $\meta{W}$. As we will see shortly, the \lstinline|fresh($x$)...| construct allocates a fresh variable, not being
used before, and associates it with the syntactic variable $x$. Thus, in the semantics we will need an infinite supply of fresh variables.

Next, we introduce the \emph{interpretation} of syntactic variables $\iota$ as a (partial) mapping

$$
\iota : \meta{V} \to \meta{T}(\meta{W})
$$

The role of the interpretation is twofold: first, it binds syntactic variables, used in the \lstinline|fresh| construct, to their semantic counterparts, and second, 
it binds relational parameters to their values~--- terms over semantic variables. For a syntactic term $t$ and an interpretation $\iota$ we denote
$t\iota$ the result of substitution of all syntactic variables in $t$ by their interpretations according to $\iota$; we assume $t\iota$ to be defined 
only when $\iota$ is defined for all variables in $t$. Thus, $t\iota$, if defined, is always an element of $\meta{T}(\meta{W})$.

Then, we borrow some conventional machinery from unification theory~\cite{Unification,UnificationRevisited}. Namely, we define a substitution $\sigma$ to
be a partial mapping between semantic variables and semantic terms:

$$
\sigma : \meta{W} \to \meta{T}(\meta{W})
$$

For any substitution $\sigma$ we assume, that the set of all free variables of all terms in the range of $\sigma$ has an empty intersection with the
domain of $\sigma$, and we denote by $\sigma\circ\theta$ the composition of substitutions, defined in a usual way. For arbitrary $t\in\meta{T}(\meta{W})$ and
a substitution $\sigma$ we denote the result of application of $\sigma$ to $t$ as $t\sigma$.

The basic inference relation for our semantics has the form

$$
\otrans{\Gamma,\iota}{(\sigma,\,\delta)}{g}{S}
$$

\noindent where $\Gamma$ is an environment, which binds relational symbols to their definitions, $\iota$~--- an interpretation, $\sigma$~--- a substitution, 
$\delta$~--- a set of allocated semantic variables, $g$~--- a goal, and $S$~--- a set of pairs $(\sigma^\prime,\,\delta^\prime)$, where $\sigma^\prime$ and
$\delta^\prime$~--- a substitution and a set of allocated semantic variables respectively. Informally speaking, we interpret a goal $g$ in the context of
relational definitions $\Gamma$, current interpretation $\iota$, current substitution $\sigma$ and current set of allocated semantic variables $\sigma$. As a 
result, we obtain a (possibly empty) set of answers. Each answer consists of a new substitution, accumulated through the execution of $g$, and a new set of
allocated semantic variables (note, in original miniKanren a goal can produce the same answer multiple number of times, but this property is not important
in our case).

The inference rules themselves are shown on Fig.~\ref{semantics}. The first two rules handle two possible outcomes of the unification. Note, we use here the most 
general unifier ($mgu$) of two semantic terms; we assume ``occurs check'' to be incorporated in the unification algorithm. Since the unification goal is built of 
syntactic terms, we have to interpret them first (by applying $\iota$), and take into account current substitution $\sigma$.

The rule for the disjunction first interprets the constituents of the disjunction in the same state and then combines the outcomes.
% using the multiset union operation ``$\biguplus$''. 

The rule for the conjunction threads the execution of its subgoals in a left-to-right successive manner: first the
left conjunct is evaluated, providing a set $\{(\sigma_i,\,\delta_i)\}$. Then the second conjunct is evaluated for each element of the set, and the
results are eventually combined. Note, the evaluation of both conjuncts is performed under the \emph{same} interpretation $\iota$ since both of them occur in the 
\emph{same} bounding context. The substitution and the set of allocated semantic variables, on the other hand, are inherited from left to right since the evaluation 
of the right conjunct has to be performed in the context of the results, provided by the left one. 

The rule for the \lstinline|fresh| construct allocates arbitrary semantic variable, not taken before, and evaluates the unique subgoal in the updated interpretation, which
associates the syntactic variable, bound in this \lstinline|fresh|, with taken semantic one.

Finally, the rule for relational definition invocation describes its evaluation in a few steps. First, the body of the definition is found, using the environment $\Gamma$. 
Then, the terms $t_i$, specified as the arguments of the invocation, are converted into their semantic forms $v_i$ using current interpretation $\iota$ and current 
substitution $\sigma$. Next, the body of the definition is evaluated in the context of \emph{empty} substitution $\epsilon$ and an interpretation, containing nothing
else, than the bindings for the formal parameters of the definition. This way of handling interpretation models the behavior of call stack in conventional languages
with no nested functions. Finally, the result substitutions are composed with the original one\footnote{We could use the original
substitution instead of the empty one without the need to use composition; however we found the approach we took more proof-friendly since each relational definition is evaluated
in initially empty substitution.}. 

Given this big-step evaluation relation for goals, we can describe the evaluation for the top-level specification $s=d_1,\dots,d_k;g$. First, we construct the associated environment
$\Gamma_s$, which properly binds all relational symbols in $s$ to their bodies. Then, we evaluate the top-level goal

$$
\otrans{\Gamma_s,\,\bot}{(\epsilon,\,\emptyset)}{g}{S_s}
$$

\noindent obtaining the set of results $S_s$; here we use empty (everywhere undefined) interpretation $\bot$ and empty substitution $\epsilon$ as a starting point. 
Finally, we choose all substitutions from $S_s$. 

Our semantics is almost deterministic~--- the only source of ambiguity is the rule for the \lstinline|fresh| construct, where we choose a new semantic variable
arbitrarily. If we fix the order, in which semantic variables are allocated, the semantics becomes completely deterministic. It is also easy to see, that if each
relational symbol is unambiguously defined in the specification and called with a proper number of parameters, and all goals in all relational definitions and the 
top-level goal are closed (i.e. each variable occurrence is bound either in some \lstinline|fresh| constructor in a parameter list of enclosing definition), 
then during the evaluation all syntactic variables are properly interpreted~--- in other words, the execution cannot break down halfway through and either diverges or 
finishes with some results.

We illustrate the evaluation, determined by this semantics, by the canonical example for relational programming~--- list concatenation relation \lstinline|append$^o$| 
(we respect here the convention to add the ``$^o$'' suffix to all names of relational entities):

\begin{lstlisting}  
   append$^o$ $\binds$ $\lambda\;x\;y\;xy$ . 
     (($x$ === $\;$Nil) /\ ($xy$ === $\;y$)) \/
     (fresh ($h$ $t$ $ty$)
        ($x$  === $\;$Cons ($h$, $t$)) /\
        (append$^o$ $t$ $y$ $ty$) /\
        ($xy$ === $\;$Cons ($h$, $ty$)) /\
     );
   fresh ($q$) (append$^o$ (Cons (A, Nil)) Nil $q$)
\end{lstlisting}

For the simplicity we omitted the arities of constructors \lstinline|A|, \lstinline|Cons|, \lstinline|Nil| and relational symbol \lstinline|append$^o$|. Since we require the 
top-level goal to be closed, from now on we conventionalize the use of the top-level \lstinline|fresh| construct as a binder for the variables whose values we are 
interested in (in this particular example $q$).

The evaluation is illustrated on Fig.~\ref{appendo_eval}; here we use numbers in bold font to denote semantic variables. For the sake of brevity and in order to
make the illustration observable we do not show the binding environment for relational definitions and as a rule denote by ellipses the inherited components
of derivation tree (the components in the left side of ``$\Rightarrow$'' are inherited top-down, in the right side~--- bottom-up).

\FloatBarrier
We start from the top-level goal and first apply the rule $\rulename{Fresh}$ (see Fig.~\ref{appendo_eval_a}). Since 
we did not use any semantic variables yet, we allocate the first one ($\sv{0}$), update the interpretation and the set of used semantic variables and continue. The next construct
is the call for \lstinline|append$^o$|, so we unfold its definition, replace the interpretation of syntactic variables by the bindings for the formal parameters, and 
evaluate the body w.r.t. the empty substitution (which has no difference from the current one yet). The body of \lstinline|append$^o$| definition is a disjunction, so we
take its left constituent, which is a conjunction, so we in turn take its left constituent, which is a unification \lstinline|$x\;$ === $\;$Nil|. This unification clearly fails, as
current interpretation binds $x$ to \lstinline|Cons (A, Nil)|. This completes the whole branch for the first disjunct of \lstinline|append$^o$| with empty result.

The evaluation of the second disjunct is shown on Fig.~\ref{appendo_eval_b}. Its top-level construct is \lstinline|fresh ($h\;t\;ty$)|, so we allocate three successive 
semantic variables $\sv{1}$, $\sv{2}$ and $\sv{3}$ and save the bindings in the interpretation. The next construct is a conjunction of three goals (assuming
``$\wedge$'' is right-associative in the concrete syntax), and we proceed with the first one, which is a unification \mbox{\lstinline|$x\;$ === $\;$Cons($h$, $\;t$)|}. 
Since $x$, $h$ and $t$ are free in current substitution and $x$ is bound to \lstinline|Cons(A, Nil)| by current interpretation, the unification succeeds with the 
substitution \mbox{$[\bnd{\sv{1}}{\lstinline{A}},\,\bnd{\sv{2}}{\lstinline|Nil|}]$}. The evaluation of remaining conjuncts is shown on Fig.~\ref{appendo_eval_c}.

\FloatBarrier
The first one is a recursive call to \lstinline|append$^o$|. We evaluate the actual parameters~--- $t$, $y$ and $ty$~--- in current interpretation and substitution, 
obtaining \lstinline|Nil|, \lstinline|Nil| and $\sv{3}$ respectively, update the interpretation to bind formal parameters to these values, and recurse to the body with
empty current substitution. Again, we have the disjunction, and the first disjunct is a conjunction \mbox{\lstinline|($x\;$ === $\;$Nil) $\wedge$ ($xy\;$ === $\;y$)|}.
Now \mbox{\lstinline|$x\;$ === $\;$Nil|} succeeds, as $x$ is already bound to \lstinline|Nil| by the interpretation, and \mbox{\lstinline|$xy\;$ === $\;y$|} succeeds
as well, providing a new substitution \mbox{$[\bnd{\sv{3}}{\lstinline|Nil|}]$}. We omit the detailed evaluation of the second top-level disjunct of \lstinline|append$^o$| since
it contains a unification \lstinline|$x\;$ === $\;$Cons (_, _)| which, clearly, does not contribute anything.

Finally, we return from the recursive call to \lstinline|append$^o$| and take the composition of substitutions~--- one before the call, and another after~--- which
gives us \mbox{$[\bnd{\sv{1}}{\lstinline|A|},\,\bnd{\sv{2}}{\lstinline|Nil|},\,\bnd{\sv{3}}{\lstinline|Nil|}]$} (see Fig.~\ref{appendo_eval_d}). We only need now to 
interpret the last conjunct of the second disjunct of \lstinline|append$^o$|~--- \mbox{\lstinline|$xy\;$ === $\;$Cons ($h$, $\;ty$)|}~--- which gives us the
final substitution \mbox{$[\bnd{\sv{1}}{\lstinline|A|},\,\bnd{\sv{2}}{\lstinline|Nil|},\,\bnd{\sv{3}}{\lstinline|Nil|},\,\bnd{\sv{0}}{\lstinline|Cons (A, Nil)|}]$}. Now, we
have to remember, that the topmost bound variable of the top-level goal is $q$, and corresponding semantic variable is $\sv{0}$. Thus, the answer is 
\mbox{$q\;=\;\lstinline|Cons (A, Nil)|$}, which is rather expected.
\section{Refutational (In)Completeness}
\label{incompleteness}

The language, defined in the previous section, is expected to allow defining computable relations in a 
very concise and declarative form. In particular, it is expected (and desirable), that a relational 
specification should exhibit the same behavior, regardless the order of conjunction/disjunction 
constituents. Regretfully, in general this is not true, and one of the most important
manifestations of this deficiency is \emph{refutational incompleteness}.  

In the context of relational programming, refutational completeness~\cite{WillThesis} is understood as 
a capability of a program to discover the absence of a solution and stop. At the first glance,
the divergence in the case of solution absence does not seem to be a severe problem. However, as
we see shortly, refutational incompleteness leads to many observable negative effects in numerous
practically important cases. 

We demonstrate the effect of refutational incompleteness on a very simple example. Let us take the
definition of \lstinline{append$^o$} from the previous section and try to evaluate the following query:

\begin{lstlisting}
   fresh ($p\;q$) (append$^o$ $p$ $q$ Nil)
\end{lstlisting}

We would expect this query to converge to the single answer \mbox{$p=\lstinline|Nil|$}, \mbox{$q=\lstinline|Nil|$};
however, in the reality the query diverges. We sketch here the explanation, omitting some non-essential technical
details, such as semantic variables allocation, etc.:

\begin{itemize}
\item First we evaluate the first disjunct of \lstinline|append$^o$|'s body and unify $p$ with \lstinline|Nil| (successfully)
and \lstinline|Nil| with $q$ (successfully), which gives us the first (expected) answer.

\item Then we proceed to the second disjunct, which is a conjunction of three simpler goals:

  \begin{itemize} 
     \item in the first one we unify $p$ with \lstinline|Cons ($h$, $t$)| (successfully);
     \item in the second we encounter a recursive call \lstinline|append$^o$ $t$ $q$ Nil|; since its arguments are merely renamings of the enclosing one, we repeat from the top and never stop.
  \end{itemize} 
\end{itemize}

The problem is, that the semantics of conjunction, in fact, is not commutative: when the first conjunct diverges and the second fails, the whole
conjunction diverges. We stress, that this is not a deviation of our semantics, but a well-known phenomenon, manifesting itself in all known
miniKanren implementations. In our example switching two last conjuncts in the definition of \lstinline|append$^o$| solves the problem~---
now the whole search stops after the unsuccessful attempt to unify \lstinline|Nil| and \lstinline|Cons ($h$, $ty$)| with no recursive call.
This, improved version of \lstinline|append$^o$|, is known to be refutationally complete. In fact, there is a conventional ``rule of thumb''
for miniKanren programming to place the recursive call as far as possible in a list of conjuncts. 

This convention, however, does not always help; to tell the truth, it often makes the things worse. Consider 
as an example yet another relation on lists:

\begin{lstlisting}
   revers$^o$ $\binds$ $\lambda\;x\;x_r$ . 
     (($x$ === $\;\;$Nil) /\ ($x_r$ === $\;\;$Nil)) \/
     (fresh ($h$ $t$ $t_r$)
        ($x$  === $\;$Cons ($h$, $t$)) /\
        (append$^o$ $t_r$ $h$ $x_r$) /\
        (revers$^o$ $t$ $t_r$)
     )
\end{lstlisting}

This relation corresponds to a relational list reversing; as we see, the recursive call is placed to
the end. However, the following query

\begin{lstlisting}
   fresh ($q$) (revers$^o$ (Cons (A, Nil)) $q$)
\end{lstlisting}

\noindent diverges, while

\begin{lstlisting}
   fresh ($q$) (revers$^o$ $q$ (Cons (A, Nil)))
\end{lstlisting}

\noindent converges to the expected results. If we switch two last conjuncts in the definition of
\lstinline|revers$^o$|, the situation reverses: the first query converges, while the second diverges. 
This example demonstrates, that the position of a recursive call (and, in general, the order of
conjuncts) depends on the direction, in which the relation of interest is evaluated.

There are, however, some cases, when the same relation is evaluated in both directions, regardless
the query. We can take as an example relational permutations, which can be implemented by running
relational list sorting in both directions:

\begin{lstlisting}
   sort$^o$ $\binds\lambda\;x\;x_s\;.\; \dots$
   perm$^o$ $\binds\lambda\;x\;x_p\;.$
     fresh ($x_s$) 
       (sort$^o$ $x$ $x_s$) $\wedge$ (sort$^o$ $x_p$ $x_s$) 
\end{lstlisting}

The concrete definition of a relational list sorting \lstinline|sort$^o$| is not important, so we
omit it due to the space considerations (an interested reader can refer to~\cite{OCanren}). The important part 
is that it is obviously recursive and not refutationally complete, and it is being evaluated 
in \emph{both} directions within the body of \lstinline|perm$^o$|. So, \lstinline|perm$^o$| is expected 
to perform poorly regardless the order of recursive calls in \lstinline|sort$^o$| implementation; it, 
indeed, does. First, if we request all solutions, both \lstinline|fresh ($q$) (perm$^o$ l $q$)| and \lstinline|fresh ($q$) (perm$^o$ $q$ l)| diverge for arbitrary non-empty list \lstinline|l| regardless the implementation of \lstinline|sort$^o$|; second, even if we request only a first few existing solutions, it does not provide any results in a reasonable time even for very small list lengths (4, 5, etc.). Interesting, that if we interested in all solutions,
we have to accurately precompute their number in order not to request more, than exists. For some problems,
it may be not so simple, as it looks at a first glance (for example, the number of all permutations is
not a factorial, but a number of permutations with repetitions). Finally, getting the number of solutions can 
itself be an objective for writing a relational specification (we provide some examples in Section~\ref{evaluation}),
and without refutational completeness requesting all solutions to calculate their number is out of
reach.
\section{Search Improvement}
\label{improvement}

As we've seen in the previous section, the non-commutativity of conjunction is one of the reasons for
refutational incompleteness (the other one is recursion). Switching arguments of a certain conjunction
can sometimes improve the results; there is, however, no certain static order,
beneficial in all cases. Thus, we can make the following observations:

\begin{itemize}
\item the conjunction to change has to be properly identified;
\item the order of conjuncts has to be a subject of a \emph{dynamic} choice.
\end{itemize}

Our improvement of the search is based on the idea of switching the order of conjuncts only when
the divergence of the first one is detected. More specifically: 

\begin{itemize}
\item during the search, we keep the track of all conjunctions being performed;
\item when we detect the divergence, we roll back to the nearest conjunction, for which 
we did not try all orders of constituents yet, switch its constituents, and rerun 
the search from that conjunction.
\end{itemize}

The important detail is the divergence test. Of course, due to the fundamental results in computability
theory, there is no hope to find a \emph{precise} computable test, which constitutes the necessary and 
sufficient condition of divergence. However, in our case a sufficient condition is sufficient. Indeed,  
a sufficient condition identifies a case, when the search, being continued, will lead to an incompleteness 
(since a divergence in our semantics always means incompleteness). Thus, it is no harm to try some other way. 

Our divergence test is based on the following notion:

\begin{definition}
\normalfont 
We say, that a vector of terms $\overline{a^{\phantom{x}}_i}$ is more general, than a vector of terms $\overline{b^{\phantom{x}}_i}$ (notation 
$\overline{a^{\phantom{x}}_i}\succeq\overline{b^{\phantom{x}}_i}$), if there is a substitution $\tau$, such that for $\forall i\;b_i = a_i \tau$.
\end{definition}

The idea of the divergence test is rather simple: it identifies a recursive call with more general arguments 
than (some) enclosing one. In the Appendix~\ref{appendix} we provide a formal proof, that, indeed,
this test constitutes a sufficient condition for divergence.

An efficient reordering of the conjuncts also requires a special treatment. First, we have to represent immediately nested conjunctions in a 
uniform way: indeed, simply switching any two operands of, for example, \mbox{$(g_1\wedge g_2)\wedge g_3$}, would not 
allow us to try \mbox{$(g_1\wedge g_3)\wedge g_2$}. Thus, we have to flatten each ``cluster'' of nested conjunctions into a list of conjuncts\mbox{$\bigwedge g_i$}, 
where none of the goals $g_i$ is a conjunction. Then, it may seem at the first glance, that the number of orderings to try 
is exponential on the number of conjuncts; we are going to show, that, fortunately, this is not the case. 

Indeed, consider a diverging conjunction \mbox{$g_1\wedge g_2\wedge g_3$}, and assume, that \mbox{$g_1\wedge g_2$} converges. 
Does it make any sense to switch the first two conjuncts? Obviously, \mbox{$g_2\wedge g_1$} either diverges, or converges with the same 
result as \mbox{$g_1\wedge g_2$} (up to the renaming of semantic variables). Anyway, switching the first two conjuncts is
unnecessary. This observation can be easily generalized: if we have a converging prefix $\omega$ in a list of conjuncts $\omega\rho$, making
any permutations inside $\omega$ is pointless.

Next, suppose we already have a list of conjuncts \mbox{$\omega\dots g_1\dots g_2\dots$}, where $\omega$ is a converging prefix, and
$g_1$ and $g_2$~--- two goals, which, being placed immediately after $\omega$, converge (if there are no such
goals, the whole list, obviously, diverges). Do we need to try both cases ($g_1$ or $g_2$ immediately after $\omega$)? 
It is rather easy to see, that \mbox{$\omega g_1$} delivers no less information, that \mbox{$\omega$}; since
$g_2$ converges immediately after $\omega$, it will converge after \mbox{$\omega g_1$}. Thus, we may apply a greedy approach: each
time we have a converging prefix of conjuncts (possibly empty), and some tail. We try to put each conjunct from the tail 
immediately after the prefix. If we find a converging conjunct, we attach it to the prefix and continue; if no, then the list of 
conjuncts diverges. Thus, we can find a converging order (if any) in a quadratic time.

% (Lemma~\ref{three} from Appendix~\ref{appendix}
%can be used to justify this claim). Thus, we do not to try two cases.


\begin{comment}
First, we need a way to detect 
a situation, when we give up on current conjunction and proceed to the next enclosing one (if any).
We represent a cluster of nested conjunctions as a state \mbox{$([p_i], n, [s_i])$}, where $[p_i]$, $[s_i]$ are lists of
non-conjunction goals, $n$~--- some natural number. We call $[p_i]$ a prefix, $[s_i]$~--- a suffix, and $n$~--- a position. 
Initially, the prefix is empty, $n=0$, and the suffix consists of the list of all conjuncts.

Let \mbox{$[p_i], n, [s_1,\dots]$} be a state; we try to evaluate $s_1$. There are three possible outcomes:

\begin{enumerate}
\item The evaluation converges; then we change the state into \mbox{$[p_i,s_1], 0, [s_2\dots]$} and continue.
\item The evaluation ends with a signal, that a divergence was detected inside the evaluation of $s_1$. We
change the state into \mbox{$[p_i], n+1, [s_{n+1},\dots,s_n=s_1,\dots]$} (in other words, we switch $s_1$ with 
$s_{n+1}$).
\item The evaluation diverges without detectiong. Nothing can be done in this case.
\end{enumerate}



We also need to pay a special attention to make it possible to enumerate all orders of all conjuncts in
compound conjunctions, not only immediate ones; thus, in \mbox{$(g_1\wedge g_2)\wedge g_3$} we should
be able to try not only \mbox{$(g_1\wedge g_2)\wedge g_3$} and \mbox{$g_3 \wedge(g_1\wedge g_2)$}, but, 
for example, \mbox{$(g_2\wedge g_1)\wedge g_3$}, etc. It can be easily achieved by flattening all nested
conjunctions to a \emph{cluster} $\bigwedge g_i$, where none of $g_i$ is a conjunction. 

In may seem at a first glance, that in the worst case an exponential number of conjunct orders have to
be tried. This is, actually, not true, because we do not need to try different orders in the
converging prefix of conjuncts. We present the following discipline of reordering:

\begin{itemize}
\item We associate a natural number $n$ with each cluster of conjuncts; initially this
number is $0$. Informally, the non-zero value $k$ says, that we reordered the cluster by
putting $k$-th conjuncts in the first place.
\item    
\end{itemize}



Moreover, all orders tried so far, have to be memorized. When we tried out every one, we need to 
proceed to the next enclosing conjunction (if any). Note, in this approach we modify the conjunctions 
according to their dynamic evaluation order, not static scoping. 


Now it is sufficient to present a query, which is not refutationally complete w.r.t. to the semantics, 
but becomes refutationally complete w.r.t. to the improvement. For such query we can take 
\mbox{\lstinline|fresh ($p\;q$) (append$^o\;p\;q\;$ Nil)|}~--- indeed, from Section~\ref{incompleteness} we
already know, that it diverges. We also remember that the reason of the divergence is the infinite
sequence of recursive calls with renamed arguments~--- but this means, that the divergence test is satisfied.
Finally, switching the recursive call to \lstinline|append$^o$| with the preceding conjunct makes the query converge~--- this is exactly, what the improvement does.
\end{comment}
%\begin{figure}[t]
%\centering
%\includegraphics{graph2.pdf}
%\caption{The Second Set of Benchmarks}
%\label{eval:second}
%\end{figure}

\section{Performance Evaluation}
\label{sec:evaluation}

One of our initial goals was to evaluate, what performance impact would choosing OCaml as a host language make. In addition we spent some 
efforts in order to implement \miniKanren in an efficient, tagless manner, and, of course, the outcome of this decision also has to be 
measured. For comparison we took faster-miniKanren\footnote{\url{https://github.com/webyrd/faster-miniKanren}}~--- a full-fledged 
\miniKanren implementation for Scheme/Racket. It turned out that faster-miniKanren implements a number of optimizations~\cite{WillThesis, Optimizations} 
to speedup the search; moreover, the search order in our implementation initially was a little bit different. In order to make the comparison
fair, we additionally implemented all these optimizations and adjusted the search order to exactly coincide with 
what faster-miniKanren does.

\begin{figure}[t]
\centering
\includegraphics[scale=0.4]{graph.png}
\caption{The Results of the Performance Evaluation}
\label{eval}
\end{figure}

\FloatBarrier 

For the set of benchmarks we took the following problems:

\begin{itemize}
%\item \textbf{sorto, permo}~--- sorting and permutation for lists of Peano numbers (shown as example in Section~\ref{sec:examples}).
%The concrete tests are the sorting of the list \lstinline{[3; 2; 1; 0]} and taking all permutations of the list \lstinline{[0; 1; 2; 3; 4; 5; 6; 7]}.
\item \textbf{pow, logo}~--- exponentiation and logarithm for integers in binary form. The concrete tests relationally computed
$3^5$ (which in 243) and $log_3 243$ (which is, conversely, 5).
\item \textbf{quines, twines, trines}~--- self/co-evaluating program synthesis problems from~\cite{Untagged}. The
concrete tests took the first 100, 15 and 3 answers for these problems respectively.
\end{itemize}

%Since the last bundle of benchmarks uses disequality constraints (and, hence, $\mu$Kanren is ruled out) we
%split all benchmarks into two sets.

The evaluation was performed on a desktop computer with Intel Core i7-4790K CPU @ 4.00GHz processor and 16GB of memory.
For OCanren \mbox{ocaml-4.04.0+frame_pointer+flambda} was used, for faster-miniKanren~--- Chez~Scheme~9.4.1.
All benchmarks were ran in the natively compiled mode ten times, then average user time was taken. The results of the evaluation
are shown on Figure~\ref{eval}. The whole evaluation repository with all scripts and detailed description is accessible 
from \lstinline{GitHub}\footnote{\url{https://github.com/Kakadu/ocanren-perf}}.

The first conclusion, which is rather easy to derive from the results, is that the tagless approach indeed matters. Our initial
implementation did not show essential speedup in comparison even with $\mu$Kanren (and was even \emph{slower} on the logarithm
and permutations benchmarks). The situation was improved drastically, however, when we switched to the tagless version.

Yet, in comparison with faster-miniKanren our implementation is still lagging behind. We can conclude, that the optimizations, 
used in Scheme/Racket version, have a different impact in the OCaml case; we save this problem for future research.


\section{Conclusion}
\label{conclusion}

We presented an improvement of a search strategy for relational programming, which is aimed at
improving refutationally completeness. We've proven, that in the case of a finite number of 
answers our modification is a strict improvement over the original strategy. Our evaluation 
shows, that w.r.t. the improved search many practically important refutationally incomplete 
queries became refutationally complete; in addition in a number of cases the performance was greatly 
improved since our modification, as a side effect, causes the search to choose more 
``optimistic'' branches. 

We can identify the following directions for future work. 

First, we believe, that our result on refutational improvement for a finite number of answers 
can be extended to the general case as well (note, in our current development we did not make 
any use of the \emph{completeness} property of miniKanren search). For this, we would also need 
another, more general, semantics. 

Another direction is extending the language with disequality constraints. Our evaluation has 
shown, that disequality constraints do not compromise our improvement in all user benchmarks, 
but we do not have a proof, that they are indeed harmless.

Next, we are working on a certified proof of the main theorem in Coq.

Finally, our practical evaluation is performed only for a prototype. 
We consider the embedding of our improvement in a full-fledged implementation to be
an important task.

\begin{thebibliography}{99}
\bibitem{TRS}
Daniel P. Friedman, William E.Byrd, Oleg Kiselyov. The Reasoned Schemer. The MIT
Press, 2005.

\bibitem{MicroKanren}
Jason Hemann, Daniel P. Friedman. $\mu$Kanren: A Minimal Core for Relational Programming //
Proceedings of the 2013 Workshop on Scheme and Functional Programming (Scheme '13).

%\bibitem{alphaKanren}
%William E. Byrd, Daniel P. Friedman. alphaKanren: A Fresh Name in Nominal Logic Programming //
%Proceedings of the 2007 Workshop on Scheme and Functional Programming (Scheme '07).

\bibitem{CKanren}
Claire E. Alvis, Jeremiah J. Willcock, Kyle M. Carter, William E. Byrd, Daniel P. Friedman.
cKanren: miniKanren with Constraints //
Proceedings of the 2011 Workshop on Scheme and Functional Programming (Scheme '11).

\bibitem{Untagged}
William E. Byrd, Eric Holk, Daniel P. Friedman.
miniKanren, Live and Untagged: Quine Generation via Relational Interpreters (Programming Pearl) //
Proceedings of the 2012 Workshop on Scheme and Functional Programming (Scheme '12).

%\bibitem{Kumar}
%Ramana Kumar. Mechanising Aspects of miniKanren in HOL. Bachelor Thesis, The Australian National University, 2010.

\bibitem{Unification}
Franz Baader, Wayne Snyder. Unification theory. In John Alan Robinson and Andrei Voronkov, editors,
Handbook of Automated Reasoning. Elsevier and MIT Press, 2001.

\bibitem{UnificationRevisited}
J.-L. Lassez, M.J. Maher, K. Marriott. Unification Revisited // Foundations of Deductive Databases and Logic Programming, 
Morgan Kaufmann Publishers Inc., 1988.

%\bibitem{Lambda}
%Henk Barendregt. Lambda Calculi with Types, Handbook of Logic in Computer Science (Vol.~2), 1992.

\bibitem{WillThesis}
William E. Byrd. Relational Programming in miniKanren: Techniques, Applications, and Implementations. PhD Thesis,
Indiana University, Bloomington, IN, September 30, 2009.

\bibitem{OCanren}
Dmitry Kosarev, Dmitry Boulytchev. Typed Embedding of a Relational Language in OCaml // International Workshop on ML, 2016.

\bibitem{RelConversion}
Petr Lozov, Andrei Vyatkin, Dmitry Boulytchev. Typed Relational Conversion // International Symposium on Trends in Functional
Programming, 2017.

%\bibitem{Types}
%Benjamin Pierce. Types and Programming Languages. MIT Press, 2002.

%\bibitem{Felleisen}
%Andrew Wright, Matthias Felleisen. A Syntactic Approach to Type Soundness // Information and Computation, Vol.~115, No.~1, 1994.

%\bibitem{cardelli}
%Luca Cardelli, Peter Wegner. On Understanding Types, Data Abstraction, and Polymorphism // ACM Computing Surveys, Vol.~17, No.~4, 1985.

\bibitem{unified}
William E. Byrd, Michael Ballantyne, Gregory Rosenblatt, Matthew Might. A Unified Approach to Solving Seven Programming Problems // 
Proceedings of the International Conference on Functional Programming, 2017.

%\bibitem{WillOnHM}
%William E. Byrd. Personal communications.

\bibitem{Guided}
Cameron Swords, Daniel P. Friedman. rKanren: Guided Search in miniKanren //
Proceedings of the 2013 Workshop on Scheme and Functional Programming (Scheme '13). 

\bibitem{AlphaKanren}
William E. Byrd, Daniel P. Friedman. alphaKanren: A Fresh Name in Nominal Logic Programming //
Proceedings of the 2007 Workshop on Scheme and Functional Programming (Scheme '07).

\bibitem{2016}
Jason Hemann, Daniel P. Friedman, William E. Byrd, Matthew Might.
A Small Embedding of Logic Programming with a Simple Complete Search //
Proceedings of the 12th Symposium on Dynamic Languages (DLS 2016).

\bibitem{CKanren1}
Jason Hemann, Daniel P. Friedman. A Framework for Extending microKanren with Constraints //
Proceedings of the 2015 Workshop on Scheme and Functional Programming (Scheme '15).

\bibitem{Search}
Oleg Kiselyov, Chung-chieh Shan, Daniel P. Friedman, Amr Sabry. Backtracking, Interleaving, and Terminating Monad Transformers (functional pearl) //
Proceedings of the 10th ACM SIGPLAN International Conference on Functional Programming (ICFP '05).

\bibitem{KiselyovArithmetic}
Oleg Kiselyov, William E. Byrd, Daniel P. Friedman, Chung-chieh Shan.
Pure, declarative, and constructive arithmetic relations (declarative pearl) //
Proceedings of the 9th International Symposium on Functional and Logic Programming, 2008.

\end{thebibliography}

\clearpage
\appendix
\section{Appendix}
\label{appendix}

In this appendix we present a proof of partial semantic correctness of relational conversion, or, to be precise, 
a number of observations, definitions, and claims, which, we believe, are sufficient to reconstruct
the complete proof. 

We remind, that our goal is to prove the following statement:

\begin{theorem} 
\normalfont For arbitrary functional program $p$ of a ground type $t$, arbitrary value $v$, and
arbitrary variable $x$

$$
\begin{array}{c}
p\leadsto^f v \Rightarrow \lstinline|fresh ($x$) ($\sembr{p}^c x$)| \leadsto^r (\theta, \emptyset)\\
\mbox{and}\\
\theta(\mathfrak{s})=v
\end{array}
$$

\noindent where $\mathfrak{s}$ is a semantic variable, associated with
$x$ on the first step of the relational evaluation.
\end{theorem}
  
We first comment on the empty set as the set of negative substitutions. A disequality constraint can
come only from polymorphic equality, which is applied when both its operands are reduced to
values. In the relational counterpart, being run in a forward direction, this corresponds to the evaluation of disequality constraints for
closed terms only, which, in turn, means, that they will immediately succeed or fail. Both cases
add nothing to the set of negative substitutions, which is initially empty. 

Next, we cannot prove the theorem, using an induction by a derivation length, since in the case of
application, for example, the type of the term in the head position is not ground. This 
obstacle could be lifted, if we could prove the following generalization:

$$
p\leadsto^f f \Rightarrow \sembr{p}^c\leadsto^r\sembr{f}^c
$$ 

\noindent for arbitrary $p$ of any type. This claim, however, turned out to be false~--- a term
\lstinline|C ((fun x.x) A)| can be taken as an example.  

The origin of the problem is that we \emph{functionalize} the constructors, \lstinline|match| and
equality expressions, and, hence, change the order of reductions in the relational counterpart in 
comparison with the original functional program. Thus, we need to take this change into account.

First, we develop a modified functional semantics, which corresponds better to the reduction
order in the relational case. We call this semantics \emph{deferred}, as it defers the evaluation
of constructors, \lstinline|match|, and equality expressions. This semantics can be acquired in
two steps: first, we consider a reduced version of the original functional semantics, in which
we treat arbitrary constructor, \lstinline|match|, and equality expressions as values. Then, the
deferred semantics is just an iterative application of the reduced version to the arguments 
of these new values (arguments of constructors or equality operator, or scrutinees of \lstinline|match| 
expressions).

Next, we claim, that if a term of some ground type is reduced to some value by the original semantics,
then it as well is reduced to the same value by the deferred one. This claim is based on the following
observations:

\begin{itemize}
\item progress and type preservation properties for both semantics (which can be proven in a standard
way);
\item Church-Rosser property for lambda-calculus;
\item the fact, that the reduced semantics applies a proper subset of rules of the original one.
\end{itemize}

Now, we are going to prove the theorem by a simulation between the deferred semantics for the original program
and the relational one for the relationally converted. Before that, we formulate the number of lemmas and 
definitions.

\begin{lemma}
\label{stack_split}
\normalfont Let us separate all the contexts into two disjoint kinds: 

\begin{itemize}
\item functional

$$
C_f = \Box\;e\mid v\;\Box\mid\lstinline|let $x$ = $\Box$ in $e$|
$$

\item ground

$$
C_g = \lstinline|match $\;\Box\;$ with $\{p_i$->$e_i\}$|\mid C^n(\bar{v},\Box,\bar{e})\mid\Box\lstinline|=e|\mid\lstinline|v=|\Box
$$

Let $\left<{\mathcal S},\,e\right>$ be an arbitrary state in a derivation sequence w.r.t. the deferred
semantics. Then $\mathcal S=C_f^*C_g^*$.
\end{itemize}

In other words, during the evaluation w.r.t. the deferred semantics, the stack of contexts is separated into the two
(possibly empty) segments: all ground contexts reside below all functional. The proof is by the induction on the
length of derivation sequence.
\end{lemma}

\begin{definition}
\normalfont
We as well separate all terms of the source language into the two disjoint kinds:

\begin{itemize}
\item functional

$$
e_1\,e_2\mid \lambda x.e \mid \mu f.\lambda x.e \mid \lstinline|let $x$ = $e_1$ in $e_2$| \mid \lstinline|let rec $f$ = $\lambda x.e_1$ in $e_2$|
$$

\item ground

$$
e_1 = e_2 \mid \lstinline|match $e$ with {$p_i$ -> $e_i$ }| \mid \lstinline|C$^k$ ($e_1\dots e_k$)|
$$

\end{itemize}

\end{definition}

\begin{definition}
\normalfont Augmented conversion of a term w.r.t. to a substitution $\sembr{\bullet}_\theta$ is defined as follows: 

$$
\begin{array}{rcl}
\sembr{p}_\theta&=&\sembr{p}^c\\
\sembr{v}_\theta&=&(\lambda x.x\equiv\mathfrak{s}),\,\mbox{if}\;\;\theta(\mathfrak s)=v
\end{array}
$$

Here $\theta$ is a substitution, $p$~--- arbitrary functional term, $v$~--- arbitrary value of a
ground type in the sense of the original semantics (i.e. the composition of constructors). Note, the
cases in this definition are not disjoint, and in the second case there can be more, than one
variable with the requested property, so augmented conversion defines a set of relational terms.
\end{definition}

\begin{lemma}
\label{substitution}
\normalfont Let $f$, $e$ be two arbitrary terms of the source language, $\theta$~--- arbitrary
substitution. Then

$$
\sembr{f[x\gets e]}_\theta=\sembr{f}_\theta[x\gets\sembr{e}_\theta]
$$

The equality here is understood in a set-theoretic sense. The proof is by structural 
induction.
\end{lemma}

\begin{definition}
\normalfont For arbitrary substitution $\theta$ define a conversion of a functional context  
$\sembr{\bullet}_\theta$ as follows:

$$
\begin{array}{rcl}
\sembr{\Box\,e}_\theta&=&\Box\,\sembr{e}_\theta\\
\sembr{v\,\Box}_\theta&=&\sembr{v}_\theta\,\Box\\
\sembr{\lstinline|let $\;x\; = \;\Box\;$ in $\;e$|}_\theta&=&\lstinline|let $\;x\; = \;\Box\;$ in $\;\sembr{e}_\theta$|
\end{array}
$$

Here $e$ is an arbitrary functional term, $v$~--- abstraction. This conversion is an extension of augmented
conversion for functional contexts, hence the same denotation.
\end{definition}

\begin{definition}
\normalfont For arbitrary semantic variables $q_1$, $q_2$ and arbitrary substitution $\theta$ 
define a conversion of ground context $\sembr{\bullet}^{q_1q_2}_\theta$ as follows:

$$ 
\begin{array}{rcl}
\sembr{C^k(v_1, \ldots, v_{i-1}, \Box, e_{i+1}, \ldots, e_k)}^{q_1q_2}_\theta&=&\Box \; \wedge \\
       & & (\sembr{e_{i+1}}_\theta \; x_{i+1}) \; \wedge \\
       & & \ldots  \\
       & & (\sembr{e_k}_\theta \; x_k) \; \wedge \\
       & & (q_2 \equiv\; \uparrow C^k(x_1, \ldots, x_{i-1}, q_1, x_{i+1}, \ldots, x_k)),\,\mbox{if}\;\theta(x_j)=v_j,\,j<i
\end{array}
$$

$$
\begin{array}{rcl}
\sembr{\Box = e}^{q_1q_2}_\theta&=&\Box\, \wedge \\
 & & (\sembr{e}_\theta\; x) \wedge \\
 & & (((q_1 \equiv x) \wedge (q_2 \equiv \lstinline|^true|))\, \vee \\ 
 & & ((q_1 \not \equiv x) \wedge (q_2 \equiv \lstinline|^false|))) 
\end{array}
$$

$$
\begin{array}{rcl}
\sembr{v = \Box}^{q_1q_2}_\theta&=&\Box\,\wedge \\
 & & (((x \equiv q_1) \wedge (q_2 \equiv \lstinline|^true|))\, \vee \\ 
 & & ((x \not \equiv q_1) \wedge (q_2 \equiv \lstinline|^false|))),\,\mbox{if}\;\theta(x)=v 
\end{array}
$$

$$
\begin{array}{rcl}
\sembr{\lstinline|match $\;\Box\;$ with \{$C^{n_i}_i$($y^i_1$, ..., $y^i_{n_i}$) -> $\;e_i$\}|}^{q_1q_2}_\theta&=&\Box \; \wedge \bigvee_i\\
& &(\lstinline|fresh ($s^i_1 \ldots s^i_{n_i}$)| \\
& &\qquad(q_1 \equiv \;\uparrow C_i^{n_i}(s^i_1, \ldots, s^i_{n_i})) \\
& &\qquad(\lambda y^i_1. \ldots \lambda  y^i_{n_i}. \sembr{e_i}_\theta) \; (\equiv s^i_1) \ldots (\equiv s^i_{n_i})\;q_2)
\end{array}
$$

Here we assume $x_i$ to be arbitrary semantic variables, $v_i$~--- arbitrary values w.r.t. the original 
functional semantics, $e_i$~--- arbitrary terms of the source language. We also claim, that $\theta$ is
undefined for all mentioned semantic variables, unless the opposite is specified explicitly.

\end{definition}

\begin{definition}
\normalfont For arbitrary substitution $\theta$, arbitrary semantic variable $q_m$ and a functional 
term $e$ define a conversion of a stack $\sembr{\bullet}^{e,q_m}_\theta$ as follows:

$$
\def\arraystretch{1.5}
\sembr{f_n\dots f_1g_m\dots g_1}^{e,q_m}_\theta=\left\{
\begin{array}{lcl}
\sembr{g_m}^{q_mq_{m-1}}_\theta\dots\sembr{g_1}^{q_1q_0}_\theta&,&n=0\;\;\mbox{and $e$~--- ground}\\
\sembr{f_n}_\theta\dots\sembr{f_1}_\theta(\Box\,q_m)\sembr{g_m}^{q_mq_{m-1}}_\theta\dots\sembr{g_1}^{q_1q_0}_\theta&,&\mbox{otherwise}
\end{array}
\right.
$$

Here $q_0\dots q_{m-1}$ designate arbitrary distinct semantic variables.
\end{definition}

\begin{definition}
\normalfont For arbitrary substitution $\theta$ and arbitrary semantic variable $q_m$ define a simulation
conversion $\sembr{\bullet}^{q_m}_\theta$ of the source language term as follows:

$$
\begin{array}{rcl}
\sembr{e_1 = e_2}^{q_m}_\theta&=& (\sembr{e_1}_\theta\; x_1) \wedge \\
                           & & (\sembr{e_2}_\theta\; x_2) \wedge \\
                           & & (((x_1 \equiv x_2) \wedge (q_m \equiv \lstinline|^true|))\, \vee \\ 
                           & & ((x_1 \not \equiv x_2) \wedge (q_m \equiv \lstinline|^false|)))
\end{array}
$$

$$
\begin{array}{rcl}
\sembr{v = e}^{q_m}_\theta&=& (\sembr{e}_\theta\; x_2) \wedge \\
                        & & (((x_1 \equiv x_2) \wedge (q_m \equiv \lstinline|^true|))\, \vee \\ 
                        & & ((x_1 \not \equiv x_2) \wedge (q_m \equiv \lstinline|^false|))),\,\mbox{if}\;\theta(x_1)=v
\end{array}
$$

$$
\begin{array}{rcl}
\sembr{v_1 = v_2}^{q_m}_\theta&=& (((x_1 \equiv x_2) \wedge (q_m \equiv \lstinline|^true|))\, \vee \\ 
                           & & ((x_1 \not \equiv x_2) \wedge (q_m \equiv \lstinline|^false|))),\,\mbox{if}\;\theta(x_j)=v_j
\end{array}
$$

$$ 
\begin{array}{rcl}
\sembr{C^k(v_1, \ldots, v_{i-1}, e_i, \ldots, e_k)}^{q_m}_\theta&=&(\sembr{e_i}_\theta \; x_i) \; \wedge \\
       & & \ldots  \\
       & & (\sembr{e_k}_\theta \; x_k) \; \wedge \\
       & & (q_m \equiv\; \uparrow C^k(x_1, \ldots, x_k)),\,\mbox{if}\;\theta(x_j)=v_j,\,j<i
\end{array}
$$

$$ 
\sembr{C^k(v_1, \ldots, v_k)}^{q_m}_\theta = (q_m \equiv\; \uparrow C^k(x_1, \ldots, x_k)),\,\mbox{if}\;\theta(x_j)=v_j
$$

$$ 
\sembr{C^k(v_1, \ldots, v_k)}^{q_m}_\theta = (q_m \equiv\; q),\;\mbox{if}\;\theta(q)=C^k(v_1, \ldots, v_k)
$$

$$
\begin{array}{rcl}
\sembr{\lstinline|match $\;e\;$ with \{$C^{n_i}_i$($y^i_1$, ..., $y^i_{n_i}$) -> $\;e_i$\}|}^{q_m}_\theta&=&\sembr{e}_\theta\;q\;\wedge\;\bigvee_i\\
& &(\lstinline|fresh ($s^i_1 \ldots s^i_{n_i}$)| \\
& &\qquad(q \equiv \;\uparrow C_i^{n_i}(s^i_1, \ldots, s^i_{n_i})) \\
& &\qquad(\lambda y^i_1. \ldots \lambda  y^i_{n_i}. \sembr{e_i}_\theta) \; (\equiv s^i_1) \ldots (\equiv s^i_{n_i})\;q_m)
\end{array}
$$

$$
\begin{array}{rcl}
\sembr{\lstinline|match $\;v\;$ with \{$C^{n_i}_i$($y^i_1$, ..., $y^i_{n_i}$) -> $\;e_i$\}|}^{q_m}_\theta&=&\bigvee_i\\
& &(\lstinline|fresh ($s^i_1 \ldots s^i_{n_i}$)| \\
& &\qquad(q \equiv \;\uparrow C_i^{n_i}(s^i_1, \ldots, s^i_{n_i})) \\
& &\qquad(\lambda y^i_1. \ldots \lambda  y^i_{n_i}. \sembr{e_i}_\theta) \; (\equiv s^i_1) \ldots (\equiv s^i_{n_i})\;q_m),\,\mbox{if}\;\theta(q)=v
\end{array}
$$

Here all $x_i$ designate arbitrary semantic variables, $e$~--- arbitrary term, $v$~--- arbitrary value w.r.t. the
original semantics. We also claim, that $\theta$ is undefined for all mentioned semantic variables, unless the opposite is specified explicitly.
\end{definition}

\begin{definition}
\normalfont Let 
\begin{itemize}
\item \mbox{$\left<\mathcal S,\,e\right>$}~--- a state w.r.t. the deferred semantics;
\item \mbox{$\left<\Sigma, \hat{\mathcal S}, \hat{e}, (\theta, \emptyset)\right>$}~--- a state w.r.t. the
relational semantics.
\end{itemize} 

We say, that these states are connected, if there exists a semantic variable $q_m$, such, that:\vspace{1mm}

\begin{enumerate}
\item \mbox{$\hat{\mathcal S}\in\sembr{\mathcal S}^{e,q_m}_\theta$}\vspace{1mm}
\item \mbox{$\hat{e}\in\left\{
                          \begin{array}{lcl}
                            \sembr{e}^{q_m}_\theta&,&e\mbox{~--- ground and }\mathcal S\mbox{ does not contain functional contexts}\\[1mm]
                            \sembr{e}_\theta&,&\mbox{otherwise}
                          \end{array}
                       \right.
            $} 
\item $\Sigma$ contains all semantic variables from $\hat{e}$, $\hat{\mathcal S}$, and $\theta$.
\end{enumerate}

\end{definition}

\begin{lemma}
\label{constructor}
\normalfont Let $v=\lstinline|C$^k$($v_1$,...,$v_k$)|$ be a value. Then
for arbitrary $\Sigma$, $\mathcal S$, $\theta$, $\hat{v}\in \sembr{v}_\theta$, and 
semantic variable $q$, such, that $q\not\in dom(\theta)$ either

$$
\left<\Sigma,\,\mathcal S, (\hat{v}\,q),\, (\theta,\,\emptyset)\right>\leadsto^*\left<\Sigma^\prime,\,\mathcal S,\,q\equiv\lstinline|C$^k$($q_1$,...,$q_k$)|,\,(\theta^\prime,\,\emptyset)\right>\;\mbox{and}\;\theta^\prime(q_i)=v_i
$$

or

$$
\left<\Sigma,\,\mathcal S, (\hat{v}\,q),\, (\theta,\,\emptyset)\right>\leadsto\left<\Sigma,\,\mathcal S,\,q\equiv q_0,\,(\theta,\,\emptyset)\right>\;\mbox{and}\;\theta(q_0)=v
$$
 
The proof is by induction on the height of $v$.
\end{lemma}

\begin{lemma}
\label{evaluation_lemma}
\normalfont Let $s=\left<\mathcal S=g_m\dots g_1,\,e\right>$ be a state w.r.t. the deferred semantics, 
$g_i$~--- ground contexts, $e$~--- expression of a ground type, $\theta$~--- some substitution,
$q_m$~--- some semantic variable, \mbox{$\hat{\mathcal{S}}\in\sembr{\mathcal S}^{e,\,q_m}_\theta$}, 
\mbox{$\hat{e} \in \sembr{e}_\theta$}. Then there is a sequence of steps w.r.t. the relational
semantics, such, that

$$
\left<\Sigma, \hat{\mathcal S}, (\hat{e} \, q_m), (\theta,\,\emptyset) \right>\leadsto^*\hat{s}
$$

\noindent and $s$ and $\hat{s}$ are connected. Here we assume $\Sigma$ to contain all semantic variables from
$\hat{\mathcal S}$ and $\theta$. The proof is by case analysis on $e$, using Lemma~\ref{constructor}.
\end{lemma}

\begin{lemma} 
\label{connection}
\normalfont Let \mbox{$s_1 \to s_2$}~--- a single evaluation step w.r.t. the deferred semantics,
$\hat{s_1}$~--- a state of the relational semantics, such, that $s_1$ and $\hat{s_1}$ are connected, then
there exists a sequence of steps in the relational semantics \mbox{$\hat{s_1}\leadsto^*\hat{s_2}$}, such, 
that $s_2$ and $\hat{s_2}$ are connected. The proof is by case analysis and definition of connection
relation, using Lemmas~\ref{substitution},~\ref{constructor},~\ref{evaluation_lemma}. 
\end{lemma}

\begin{lemma}
\label{prefix}
\normalfont Let $s_0=\left<\emptyset,\,\epsilon,\,\lstinline|fresh ($x$) $(\sembr{e}^c\;x)$|,\,\iota\right>$ be an
initial state of evaluation w.r.t. the relational semantics. Then there is a sequence of steps
\mbox{$s_0\leadsto^*\hat{s}$}, such, that \mbox{$\left<\epsilon,\,e\right>$} (an initial state of
evaluation of $e$ w.r.t. the deferred semantics) and $\hat{s}$ are connected. Immediately follows from
Lemma~\ref{evaluation_lemma}.
\end{lemma}

Now we can prove the partial correctness theorem. Let us have a term $e$ of a ground type in the source language, which
reduces to a value $v=\lstinline|C$^k$($v_1$,...,$v_k$)|$ w.r.t. the original call-by-value semantics. Then it reduces to the same value w.r.t. the
deferred semantics: 

$$
\left<\epsilon,\,e\right>\to^*\left<\epsilon,\,v\right>
$$

By Lemma~\ref{prefix} 

$$
\left<\emptyset,\,\epsilon,\lstinline|fresh ($x$) $(\sembr{e}^c\;x)$|,\iota\right>\leadsto^*\hat{s}
$$

\noindent where \mbox{$\left<\epsilon,\,e\right>$} and $\hat{s}$ are connected. By Lemma~\ref{connection}, there is
a state $\hat{s^\prime}$ w.r.t. the relational semantics, such, that

$$
\hat{s}\leadsto^*\hat{s^\prime}
$$

\noindent and \mbox{$\left<\epsilon,\,v\right>$} and $\hat{s^\prime}$ are connected. By the definition of
connection relation, $\hat{s^\prime}$ has one of the following forms:

$$
\left<\Sigma,\,\epsilon,\,q_o\equiv\lstinline|C$^k$($x_1$,...,$x_k$)|,\,(\theta,\,\emptyset)\right>,\,\theta(x_i)=v_i
$$

\noindent or

$$
\left<\Sigma,\,\epsilon,\,q_o\equiv q,\,(\theta,\,\emptyset)\right>,\,\theta(q)=v
$$

\noindent where $q_0$ is the first semantic variable, added to $\Sigma$ and \mbox{$q_0\not\in dom(\theta)$}. In
both cases, we can make the one last step in the relational semantics, which completes the proof. 


\end{document}