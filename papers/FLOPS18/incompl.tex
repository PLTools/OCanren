\section{Refutational (In)Completeness}

The language, defined in the previous section, is expected to allow defining computable relations in a 
very concise and declarative form. In particular, it is expected (and desirable), that a relational 
specification should exibit the same behavior, regardless the order of conjunction/disjunction 
constituents. Regretfully, in general this is not true, and one of the most important
manifestations of this deficience is \emph{refutational incompleteness}.  

In the context of relational programming, refutational completeness~\cite{WillThesis} is understood as 
a capability of a program to discover the absence of a solution and stop. At the first glance
the divergence in the case of solution absense does not seem to be a severe problem. However, as
we see shortly, refutational incompleteness leads to many observable negative effects in numerous
practically important cases. 

We demonstrate the effect of refutational incompleteness on the very simple example. Let us take the
definition of \lstinline{append$^o$} from the previous section and try to evaluate the following query:

\begin{lstlisting}
   fresh ($p\;q$) (append$^o$ $p$ $q$ Nil)
\end{lstlisting}

We would expect this query to converge to the single answer \mbox{$p=\lstinline|Nil|$}, \mbox{$q=\lstinline|Nil|$};
however, in the reality the query diverges. We sketch here the explanation, omitting some non-essential technical
details, such as semantic variables allocation, etc.:

\begin{itemize}
\item First we evaluate the first disjunct of \lstinline|append$^o$|'s body and unify $p$ with \lstinline|Nil| (successfully)
and \lstinline|Nil| with $q$ (successfully), which gives us the first (expected) answer.

\item Then we proceed to the second disjunct, which is a conjunction of three simpler goals:

  \begin{itemize} 
     \item in the first one we unify $p$ with \lstinline|Cons ($h$, $t$)| (successfully);
     \item in the second we encounter a recursive call \lstinline|append$^o$ $t$ $q$ Nil|; since this call is merely a renaming of
the enclosing one, we repeat from the top and never get any results.
  \end{itemize} 
\end{itemize}

The problem is, that the semantics of conjunction in fact is not commutative: when the first conjunct diverges and the second fails, the whole
conjunction diverges. We stress, that this is not a deviation of our semantics, but a well-known phenomenon, manifesting itself in all known
miniKanren implementations. In our example switching two last conjuncts in the definition of \lstinline|append$^o$| solves the problem~---
now the whole search stops after the unsuccessfull attempt to unify \lstinline|Nil| and \lstinline|Cons ($h$, $ty$)| with no recursive call.
This, improved version of \lstinline|append$^o$|, is known to be refutationally complete. In fact, there is a conventional ``rule of thumb''
for miniKanren programming to place the recursive call as far as possible in a list of conjuncts. 

This convention, however, does not always help; to tell the truth, it often makes the things worse. Consider 
as an example yet another relation on lists:

\begin{lstlisting}
   revers$^o$ $\binds$ $\lambda\;x\;x_r$ . 
     (($x$ === $\;\;$Nil) /\ ($x_r$ === $\;\;$Nil)) \/
     (fresh ($h$ $t$ $t_r$)
        ($x$  === $\;$Cons ($h$, $t$)) /\
        (append$^o$ $t_r$ $h$ $x_r$) /\
        (revers$^o$ $t$ $t_r$)
     )
\end{lstlisting}

This relation corresponds to a relational list reversing; as we see, the recursive call is placed to
the end. However, the following query

\begin{lstlisting}
   fresh ($q$) (revers$^o$ (Cons (A, Nil)) $q$)
\end{lstlisting}

\noindent diverges, while

\begin{lstlisting}
   fresh ($q$) (revers$^o$ $q$ (Cons (A, Nil)))
\end{lstlisting}

\noindent converges to the expected results. If we switch two last conjuncts in the definition of
\lstinline|revers$^o$|, the situation reverses: the first query converges, while the second diverges. 
This example demonstrates, that the position of a recursive call (and, in general, the order of
conjuncts) depends on the direction, in which the relation of interest is evaluated.

There are, however, some cases, when the same relation is evaluated in both directions, regardless
the query. We can take as an example relational permutations, which can be implemented by running
relational list sorting in both directions:

\begin{lstlisting}
   sort$^o$ $\binds\lambda\;x\;x_s\;.\; \dots$
   perm$^o$ $\binds\lambda\;x\;x_p\;.$
     fresh ($x_s$) 
       (sort$^o$ $x$ $x_s$) $\wedge$ (sort$^o$ $x_p$ $x_s$) 
\end{lstlisting}

The concrete definition of a relational list sorting \lstinline|sort$^o$| is not important, so we
omit it due to the space considerations (an interested reader can refer to~\cite{OCanren}). The important part 
is that it is obviously recursive and not refutationally complete, and it is being evaluated 
in \emph{both} directions within the body of \lstinline|perm$^o$|. So, \lstinline|perm$^o$| is expected 
to perform poorly regardless the order of recursive calls in \lstinline|sort$^o$| implementation; it, 
indeed, does. First, if we request all solutions, both \lstinline|fresh ($q$) (perm$^o$ l $q$)| and \lstinline|fresh ($q$) (perm$^o$ $q$ l)| diverge for arbitrary non-empty list \lstinline|l| regardless the implementation of \lstinline|sort$^o$|; second, even if we request only a first few existing solutions, it does not provide any results in a reasonable time even for very small list lengths (4, 5, etc.).

Finally, refutational incompleteness prevents us from discovering the number of solutions, which can 
itself be an objective for writing a relational specification (we provide some examples in Section~\ref{evaluation}).


