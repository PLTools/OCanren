\section{Refutational Incompleteness}

We demonstrate the effect of refutational incompleteness on the very simple example. Let us take the
definition of \lstinline{append$^o$} from the previous section and try to evaluate the following query:

\begin{lstlisting}
   fresh ($p\;q$) (append$^o$ $p$ $q$ Nil)
\end{lstlisting}

We would expect this query to converge to the single answer \mbox{$p=\lstinline|Nil|$}, \mbox{$q=\lstinline|Nil|$};
however, in the reality the query diverges. We sketch here the explanation, omitting some non-essential technical
details such as semantic variables allocation etc.:

\begin{itemize}
\item First we evaluate the first disjunct of \lstinline|append$^o$|'s body and unify $p$ with \lstinline|Nil| (successfully)
and \lstinline|Nil| with $q$ (successfully), which gives us the first (expected) answer.

\item Then we proceed to the second disjunct, which is a conjunction of three simpler goals:

  \begin{itemize} 
     \item in the first one we unify $p$ with \lstinline|Cons ($h$, $t$)| (successfully);
     \item in the second we encounter a recursive call \lstinline|append$^o$ $t$ $q$ Nil|; since this call is merely a renaming of
the enclosing one, we repeat from the top and never get any results.
  \end{itemize} 
\end{itemize}

The problem is that the semantics of conjunction in fact is not commutative: when the first conjunct diverges and the second fails, the whole
conjunction diverges. We stress, that this is not a deviation of our semantics, but a well-known phenomenon, manifesting itself in all known
miniKanren implementations. In our example switching two last conjuncts in the definition of \lstinline|append$^o$| solves the problem~---
now the whole search stops after the unsuccessfull attempt to unify \lstinline|Nil| and \lstinline|Cons ($h$, $ty$)| with no recursive call.
This, improved version of \lstinline|append$^o$|, is known to be refutationally complete. In fact, there is a conventional ``rule of thumb''
for miniKanren programming to place the recursive call as far as possible in a list of conjuncts. 

This convention, however, does not always help; to tell the truth, it often makes the things even worse. Consider 
as an example yet another relational implementation of a list-processing function:

\begin{lstlisting}
\end{lstlisting}

This convention, however, cannot be always respected, since there can be more than one recursive 
call in a conjunction. 




