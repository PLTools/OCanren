\section{Appendix}

The proof of divergence in this case is based on the following definitions and
lemmas.

\begin{definition}
\normalfont
A semantic variable $v$ is \emph{reachable} w.r.t. the intrepretation $\iota$ and substitution $\sigma$, if there exists 
a syntactic variable $x$, such that \mbox{$v \in FV(\iota(x) \sigma)$}.
\end{definition}

\begin{definition}
\normalfont
A semantic statement 

$$
\otrans{\Gamma,\iota}{(\sigma,\,\delta)}{g}{S}
$$ 

\noindent is \emph{well-formed}, if \mbox{$dom(\sigma) \subseteq \delta$}, and any semantic variable, reachable w.r.t. $\iota$ and $\sigma$, belongs to $\delta$.  
\end{definition}

\begin{lemma}
\label{one}
\normalfont
 For a well-formed semantic statement, every statement in its derivation tree is also well-formed.
\end{lemma}

\begin{lemma}
\label{two}
\normalfont
For a well-formed semantic statement 

$$
\otrans{\Gamma,\iota}{(\sigma,\,\delta)}{g}{S}
$$ 

\noindent and any result \mbox{$(\sigma_r,\,\delta_r) \in S$}, there exists a substitution $\Delta$, such that:
  \begin{enumerate}
    \item \mbox{$\sigma_r = \sigma\circ\Delta$};
    \item any semantic variable \mbox{$v\in dom(\Delta)\cup ran(\Delta)$} either is reachable w.r.t. $\iota$ and $\sigma$,
 or does not belong to $\delta$.
  \end{enumerate}   
\end{lemma}

\begin{lemma}
\label{three}
\normalfont
Let 

$$
\otrans{\Gamma,\iota}{(\sigma,\,\delta)}{g}{S}
$$ 

and 

$$\otrans{\Gamma,\iota^\prime}{(\sigma^\prime,\,\delta^\prime)}{g}{S^\prime}
$$

\noindent are two well-formed semantic statements, and there exists a substitution $\tau$, such that 
for any syntactic variable $x$ \mbox{$\iota^\prime(x) \sigma^\prime = \iota(x) \sigma \tau$}. Then the 
derivation tree of the first statement has greater or equal height, then the derivation 
tree of the second statement.
\end{lemma}
