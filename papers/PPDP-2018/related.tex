\section{Related Works}
\label{sec:related_works}

The non-commutativity of conjunction evaluation in miniKanren is a well-known problem. In~\cite{WillThesis} some language extensions are discussed, which,
presumably, can be used to provide the commutativity. They include both simple enumeration of conjunct orders and more advanced techniques, based on a combination
of tabling, parallel goal evaluation, and continuations. However, by now none of these proposals were implemented or evaluated. The tabling technique,
described in the same work, can indeed be used to provide the convergence of some queries, but it deals with the problems, orthogonal to the non-commutativity, and,
thus, does not heal the queries, which we do (but heals some other cases, like divergence of path-finding queries for graphs with cycles, which we do not).

For a number of problems some \emph{ad-hoc} refutationally complete solutions were already presented before. For example, in~\cite{TRS} a number of relations for
binary arithmetics, implemented using the idea of bounding the sizes of terms, are presented. In a follow-up paper~\cite{KiselyovArithmetic} this technique is
explained in details, and the proof of refutational completeness is given. Unfortunately, the specifications, written using this technique, are verbose and
hard to understand, and the implementation requires insight. Our improvement, on the other hand, makes it possible to stick with the simplest definitions, and
althought we do not provide a proof of refutational completeness for each case, for the majority of realistic queries they converge and demonstrate the same
performance, as those, implemented with advanced methods.
