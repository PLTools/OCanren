\section{Introduction}

Relational programming is an approach, based on the idea of describing programs not as functions, but 
as relations, without distinguishing between the arguments and the result value. This technique makes it 
possible to ``query'' programs in various ways, for example, to execute them ``backwards'', finding
all sets of arguments for a given result. Relational behavior can be reproduced using a number of
logic programming languages, such as Prolog, Mercury~\cite{Mercury}, 
or Curry~\cite{Curry}. 
There is also a family of embedded DSLs, specifically designed for writing declarative relational
programs that originates from \miniKanren~\cite{TRS}. \miniKanren is a minimalistic 
declarative language, initially developed for Scheme/Racket. The smallest implementation of \miniKanren 
is reported to comprise of only 40 LOC~\cite{MicroKanren,2016}; there are also more elaborate versions, including
\miniKanren with constraints~\cite{CKanren,CKanren1}, user-assisted search~\cite{Guided}, nominal unification~\cite{AlphaKanren},
etc. Due to its simplicity, \miniKanren was implemented for more than 50 other languages, such as
Haskell, Go, Smalltalk, and OCaml. In a nutshell, miniKanren introduces a minimalistic set of constructs to describe
relations over a set of syntactic terms, thus providing the same expressivity as a pure core of conventional
logic programming\footnote{A detailed miniKanren to Prolog comparison can be found at \url{http://minikanren.org/minikanren-and-prolog.html}}. 

\miniKanren has proven to be a useful tool to provide elegant solutions for various problems, otherwise considered as
non-trivial~\cite{WillThesis}. One of the most promising areas of application for \miniKanren is the implementation of
\emph{relational interpreters}. Such interpreters are capable not only to interpret programs in various directions, but also
to infer programs on the basis of expected input-output specification~\cite{Untagged}.

Being quite simple and easy to use by design, in implementation \miniKanren introduces some subtleties. Under the hood, \miniKanren 
uses a complete interleaving search~\cite{Search}. This search is guaranteed to find all existing solutions; however, it can diverge, when no 
solution exists. In reality, this amounts to divergence in a number of important cases~--- for example, when a program is asked to 
return \emph{all} existing solutions, or when the number of requested solutions exceeds the number of existing ones.
It is often possible to refactor the specification of a concrete query to avoid the divergence, but this has to be done for every execution
``direction'' of interest that compromises the idea of fully declarative relational programming.

The specifications that do not diverge even when no solutions exist, are called \emph{refutationally complete}~\cite{WillThesis}. Writing 
refutationally complete relational specifications nowadays requires knowledge of \miniKanren implementation intrinsics, and is not always
possible due to the undecidability of the fundamental computability problems. However, by developing a more advanced search it is possible
to make more specifications refutationally complete.

In this work we address one particular problem that often leads to refutational incompleteness~--- the non-commutativity of
conjunction. We present an optimization technique that is based on a certain non-termination test. Our optimization
is \emph{online} (performed during the search), \emph{non-intrusive} (does not introduce new constructs and does not require
any changes to be made to the existing specifications), and \emph{conservative} (applied only when the divergence
is detected). We prove that for the queries that return a finite number of answers, our optimization preserves convergence. 
We also demonstrate the application of the optimization for a number of interesting and important problems.

We express our gratitude to William Byrd and the reviwers of this paper for their constructive remarks and suggestions.

