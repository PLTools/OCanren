%%%%%%%%%%%%%%%%%%%%%%%%%%%%%%%%%%%%%%%%%
% Jacobs Portrait Poster
% LaTeX Template
% Version 1.0 (31/08/2015)
% (Based on Version 1.0 (29/03/13) of the landscape template
%
% Created by:
% Computational Physics and Biophysics Group, Jacobs University
% https://teamwork.jacobs-university.de:8443/confluence/display/CoPandBiG/LaTeX+Poster
% 
% Further modified by:
% Nathaniel Johnston (nathaniel@njohnston.ca)
%
% Portrait version by:
% John Hammersley
%
% The landscape version of this template was downloaded from:
% http://www.LaTeXTemplates.com
%
% License:
% CC BY-NC-SA 3.0 (http://creativecommons.org/licenses/by-nc-sa/3.0/)
%
%%%%%%%%%%%%%%%%%%%%%%%%%%%%%%%%%%%%%%%%%

%----------------------------------------------------------------------------------------
%	PACKAGES AND OTHER DOCUMENT CONFIGURATIONS
%----------------------------------------------------------------------------------------

\documentclass[final]{beamer}

\usepackage[orientation=portrait,size=a0]{beamerposter} % Use the beamerposter package for laying out the poster
\usepackage{wrapfig}
\usepackage{listings}
\usepackage{filecontents}
\usepackage{hhline}
\usepackage{tikz}
\usetikzlibrary{backgrounds,tikzmark}
\usepackage{color}

%\usepackage[backend=bibtex,style=numeric,maxnames=2,firstinits=true]{biblatex}

\usetheme{confposter} % Use the confposter theme supplied with this template

%% \setbeamercolor{titlelike}         {bg=jblue,fg=white}
\setbeamercolor{block title}{fg=jblue,bg=white} % Colors of the block titles
\setbeamercolor{block body}{fg=black,bg=white} % Colors of the body of blocks
\setbeamercolor{block alerted title}{fg=white,bg=dblue!70} % Colors of the highlighted block titles
\setbeamercolor{block alerted body}{fg=black,bg=dblue!10} % Colors of the body of highlighted blocks
% Many more colors are available for use in beamerthemeconfposter.sty

%-----------------------------------------------------------
% Define the column widths and overall poster size
% To set effective sepwid, onecolwid and twocolwid values, first choose how many columns you want and how much separation you want between columns
% In this template, the separation width chosen is 0.024 of the paper width and a 4-column layout
% onecolwid should therefore be (1-(# of columns+1)*sepwid)/# of columns e.g. (1-(4+1)*0.024)/4 = 0.22
% Set twocolwid to be (2*onecolwid)+sepwid = 0.464
% Set threecolwid to be (3*onecolwid)+2*sepwid = 0.708

\newlength{\sepwid}
\newlength{\onecolwid}
\newlength{\twocolwid}
\newlength{\threecolwid}

%% \setlength{\paperwidth}{33.1in} % A0 width: 46.8in
%% \setlength{\paperheight}{46.8in} % A0 height: 33.1in

\setlength{\sepwid}{0.024\paperwidth} % Separation width (white space) between columns
\setlength{\onecolwid}{0.301\paperwidth} % Width of one column
\setlength{\twocolwid}{0.626\paperwidth} % Width of two columns
\setlength{\threecolwid}{0.952\paperwidth} % Width of three columns
\setlength{\topmargin}{-0.5in} % Reduce the top margin size
%-----------------------------------------------------------

\usepackage{graphicx}  % Required for including images

\usepackage{booktabs} % Top and bottom rules for tables

\usepackage{mathtools}
\usepackage{listings}
\usepackage{wasysym}

\usepackage{verbatim}

\lstdefinelanguage{ocanren}{
keywords={fresh, let, in, match, with, when, class, type,
object, method, of, rec, repeat, until, while, not, do, done, as, val, inherit,
new, module, sig, deriving, datatype, struct, if, then, else, open, private, virtual, include, success, failure},
sensitive=true,
commentstyle=\small\itshape\ttfamily,
keywordstyle=\ttfamily\underbar,
identifierstyle=\ttfamily,
basewidth={0.5em,0.5em},
columns=fixed,
fontadjust=true,
literate={fun}{{$\lambda$}}1 {->}{{$\to$}}3 {===}{{$\equiv$}}1 {=/=}{{$\not\equiv$}}1 {|>}{{$\triangleright$}}3 {/\\}{{$\wedge$}}2 {\\/}{{$\vee$}}2 {^}{{$\uparrow$}}1,
morecomment=[s]{(*}{*)}
}

\lstset{
mathescape=true,
basicstyle=\small,
identifierstyle=\ttfamily,
keywordstyle=\bfseries,
commentstyle=\scriptsize\rmfamily,
basewidth={0.5em,0.5em},
fontadjust=true,
language=ocanren
}

%----------------------------------------------------------------------------------------
%	TITLE SECTION 
%----------------------------------------------------------------------------------------

\title{???} % Poster title

\author{???} % Author(s)

\institute{???} % Institution(s)

%----------------------------------------------------------------------------------------

\begin{document}

\addtobeamertemplate{block end}{}{\vspace*{2ex}} % White space under blocks
\addtobeamertemplate{block alerted end}{}{\vspace*{2ex}} % White space under highlighted (alert) blocks

\setlength{\belowcaptionskip}{2ex} % White space under figures
\setlength\belowdisplayshortskip{2ex} % White space under equations

\begin{frame}[fragile] % The whole poster is enclosed in one beamer frame

\begin{columns}[t] % The whole poster consists of three major columns, the second of which is split into two columns twice - the [t] option aligns each column's content to the top


\begin{column}{\sepwid}\end{column} % Empty spacer column


\begin{column}{\onecolwid} % The first column

\begin{block}{Relational programming}

\begin{itemize}

\item Construction of functions as relations.

\item Allows execution in various directions \linebreak (e.g. get arguments by result value).

\item Provides elegant solutions to non-trivial problems.

\begin{figure}
  \begin{tabular}{ l c l }
    \textbf{\textcolor{blue}{Sorting function}} & for & \textbf{\textcolor{blue}{generation of permutations}}  \\
    \textbf{\textcolor{blue}{Type checker}}     & for & \textbf{\textcolor{blue}{type inhabitation problem}}     \\
    \textbf{\textcolor{blue}{Interpreter}}          & for & \textbf{\textcolor{blue}{generation of quines}}                      
  \end{tabular}
\end{figure}

\end{itemize}

\end{block}


\begin{block}{MiniKanren}

MiniKanren is a family of embedded relational DSLs.

Host languages include:

\begin{itemize}
  \item Scheme (original implementation)
  \item Closure
  \item Haskell
  \item Go
  \item OCaml (implementation we use)
\end{itemize}

Specification constructors:
\begin{itemize}
  \item unification ($\equiv$) of two terms
  \item conjunction ($\wedge$) and disjunction ($\vee$)
  \item fresh variable introduction
\end{itemize}

\begin{alertblock}{Example 1}

Relational specification of list concatenation.

\begin{lstlisting}
  let rec append$^o$ x y xy =
    (x === [] /\ y === xy) \/
    (fresh (h t ty) (
      (x === h :: t) /\
      (append$^o$ t y ty) /\
      (xy === h :: ty)
    ))
\end{lstlisting}

\end{alertblock}

\end{block}


\begin{block}{Problem}

\begin{itemize}
    \item[\smiley] Interleaving search is guaranteed to find all answers. 
    \item[\frownie] But it can diverge when no answers left.

\begin{figure}
\begin{lstlisting}
(fun q -> append$^o$ [1; 2] [3] q) $\leadsto$ {q=[1; 2; 3]}

(fun q r -> append$^o$ q r [1; 2; 3]) $\leadsto$ $\bot$
\end{lstlisting}
\end{figure}

\end{itemize}

\textbf{Reason:} non-commutativity of conjunction.

\begin{figure}

\Large{Information passing}\\

\begin{tikzpicture}
  \node[text width=6cm] at (0,0) 
  	{\huge $g_1 \wedge g_2$};
  \draw [->, line width=9, blue] (-7,0) -- (-3.5,0);
  \draw [->, line width=9, blue] (3.5,0) -- (7,0); 
  % \draw [->, line width=9, blue] (-2,1.5) -- (2,1.5); 
  \draw [->, line width=9, blue] (-2,1) .. controls (-1,3) and (1,3) .. (2,1);
  \draw [->, line width=9, blue] (2,-1) .. controls (1,-3) and (-1,-3) .. (-2,-1);
  \draw [line width=9, red] (-0.75,-3.25) -- (0.75,-1.75); 
  \draw [line width=9, red] (-0.75,-1.75) -- (0.75,-3.25);
\end{tikzpicture}

\end{figure}


Search in $g_1$ diverges $\Rightarrow$ search in conjunction diverges

\bigskip

\bigskip

Specification is \textbf{Refutationally Complete}, \\ if search by it always terminates when no answers left. 

Shift of recursive call to the end makes {\ttfamily append$^o$} RC.

It doesn't work in more complex cases.

\end{block}


\begin{alertblock}{Goal}

Make it easier to write refutationally complete specifications.

\end{alertblock}


\begin{block}{Possible solutions}

\begin{enumerate}
  \item Advanced technics of writing specifications, \\ such as bounding the sizes of terms
  \item Simulation of commutative conjunction
  \item Reordering of conjuncts during execution \textcolor{red}{ $\leftarrow$ our approach }
\end{enumerate}

\end{block}

\end{column} % End of the first column


\begin{column}{\sepwid}\end{column} % Empty spacer column


\begin{column}{\onecolwid} % Begin a column which is two columns wide (column 2)

\begin{block}{Conjuncts reordering approach}

\textbf{Idea:} 

\begin{enumerate}
\item testing current search proccess for divergence
\item try different order if it was detected
\end{enumerate}

This search extension is:

\begin{itemize}
  \item online
  \item non-intrusive
  \item conservative
\end{itemize}

\end{block}

\begin{block}{Examples}

\begin{alertblock}{Example 2}

In relational sorting

\begin{lstlisting}
  let rec sort$^o$ xs ys=
    (xs === [] /\ ys === []) \/
    (fresh (s xst yst) (
      (ys === s :: yst) /\
      (smallest$^o$ xs s xst) /\  (*1*)
      (sort$^o$ xst yst)           (*2*)
    ))
\end{lstlisting}

for termination we need

\begin{itemize}
\item order $1 \to 2$ for direct execution
\item order $2 \to 1$ for reverse execution
\end{itemize}

\bigskip

Permutation relstion based on it

\begin{lstlisting}
  let rec perm$^o$ xs ys =
    (fresh (ts) (
      (sort$^o$ xs ts) /\ (sort$^o$ ys ts)
    ))
\end{lstlisting}

requires execution in both directions.

\bigskip

Original search diverges with any order

+ search is too inefficient on lengths more than $3$.

\end{alertblock}

Extended search reconstructs (experimentally) the way \\ of information propagation for each call.

Therefore multidirectional calls (like in {\ttfamily perm$^o$}) converge.

\bigskip
\bigskip

This aproach allows to write RC specifications naively.

\begin{alertblock}{Example 3}

For division with remainder in binary arithmetics:

\[ n = m \cdot q + r, \quad r < m \]

instead of this sophisticated solution

\begin{lstlisting}
  let rec div$^o$ n m q r =
    (r === n /\ [] === q /\ plus$^o$ r m n /\ lt$^o$ r m) \/
    ([1] === q /\ eql$^o$ n m /\ plus$^o$ r m n /\ lt$^o$ r m) \/
    ((ltl$^o$ m n) /\ (lt$^o$ r m) /\ (pos$^o$ q) /\
     (fresh (nh nl qh ql qlm qlmr rr rh) (
       (split$^o$ n r nl nh) /\
       (split$^o$ q r ql qh) /\
       ((([] === nh) /\ ([] === qh) /\ 
         (minus$^o$ nl r qlm) /\ (mult$^o$ ql m qlm)) \/
        ((pos$^o$ nh) /\ (mult$^o$ ql m qlm) /\
         (plus$^o$ qlm r qlmr) /\ (minus$^o$ qlmr nl rr) /\
         (split$^o$ rr r [] rh) /\ (div$^o$ nh m qh rh))
       )
     ))
    )
\end{lstlisting}

we can just write down definition

\begin{lstlisting}
  let rec div$^o$ n m q r =
    (fresh (mq) (
      (mult$^o$ m q mq) /\ (plus$^o$ mq r n) /\ (lt$^o$ r m)
    ))
\end{lstlisting}

and it will be RC under extended search.

\end{alertblock}

\end{block}

\end{column}


\begin{column}{\sepwid}\end{column}


\begin{column}{\onecolwid}

\begin{block}{MiniKanren semantics}

\end{block}

\begin{block}{Non-termination test}

\end{block}

\begin{block}{Implementation details?}

\end{block}

\begin{block}{References?}

\end{block}

\begin{block}{Acknowledgements?}

\end{block}

\end{column}

\end{columns} % End of all the columns in the poster

\end{frame}

\end{document}
